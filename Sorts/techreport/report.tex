\documentclass{report}

\title{SORTS Tech Report}
\author{James Irizarry, Sam Wintermute, Joseph Xu}

\begin{document}
\maketitle

\section{ORTS Overview}

The Open Real Time Strategy software is a highly configurable game
engine used to play real time strategy (RTS) games \cite{ORTS}. The main purpose for
ORTS is to serve as an open source, open interface RTS game engine for
RTS AI tournaments. ORTS is undergoing active development as of July
2006 at the University of Alberta under the direction of Michael Buro.

There are several reasons why ORTS is especially suitable for use in
AI tournaments. It has a (relatively) straightforward C++ API, making
interfacing with your favorite AI system easy. All the specific game
mechanics, ranging from types of units, actions, and physics, are
specified via C++ style scripts called blueprints. This means that ORTS
can be easily configured to simulate a wide range of environments,
from arbitrarily simple ones like Wumpus World to complex ones like
Starcraft. Finally, ORTS has a client/server architecture in which
the server maintains the state of the world and only report to the
clients information they are supposed to have for a fair game. This
is in contrast to most commercial RTS games, in which each client
maintains the entire world state and prevents the player from accessing
forbidden information such as other players' locations only by hiding
them from the GUI. The result is that ORTS is impervious to "memory
hack" cheats that are widespread in commercial RTS games. This feature
is particularly important if tournaments are to be run across the
Internet.

\section{SORTS Overview}

SORTS is a piece of software that allows the Soar cognitive architecture
to act as a client to the ORTS game server, so that ORTS game playing
agents can be written in and executed on Soar. SORTS is much more than
an interface bridge in that it intentionally constrains the space of
possible Soar agents in important ways and also handles aspects of
low-level game control that are not suitable for Soar. In particular,
SORTS heavily abstracts the world state information obtained from
the ORTS API before feeding them into Soar as perceptions, and also
interprets and executes high-level commands from Soar as low-level
actions sent to the ORTS API.

\section{Soar IO Description}

\subsection{The SORTS input-link}

There are five top-level attributes on the SORTS input link, "groups", "game-info", "feature-maps", "vision-info", and "query-results". The groups, feature-maps, and vision-info structures are all part of the main visual system (see XXX), while game-info contains higher-level information about the game world, and query-results is used to communicate the results of specialized queries from Soar to the middleware.

The exact data structures are as follows:

\begin{center}
\begin{tabular}{|l|p{3.5in}|}
\hline
\multicolumn{2}{|c|}{\textbf{Attributes of io.input-link}}\\ 
\hline
attribute  &  description\\
\hline \hline
vision-info & Contains information on the current state of the vision system. \\
\hline
vision-info.center-x & The coordinate of the center of the region in view. \\
vision-info.center-y & \\
\hline
vision-info.focus-x & The coordinate of the center of focus (spotlight of attention). \\
vision-info.focus-y & \\
\hline
vision-info.num-objects-visible & The maximum number of objects (groups) present on the input-link. All other objects within the view window are present in feature maps. \\
\hline
vision-info.grouping-radius & All objects of the same type (except as below) and owner within this distance of each other are in the same group (set to 0 for individuals). \\
\hline
vision-info.owner-grouping & Ignore type when grouping, only group by owner (1 if enabled, 0 if disabled). \\
\hline
groups & The set of groups being attended to. \\
\hline
groups.group & Multi-valued, one instance for each group. Detailed below. \\
\hline
\end{tabular}

\begin{tabular}{|l|l|p{3.5in}|}
\hline
\multicolumn{3}{|c|}{\textbf{Attributes of io.input-link.groups.group objects}}\\ 
\hline
attribute  & which groups &  description\\
\hline \hline
num-members & all groups &how many individuals comprise the group. \\
\hline
type & all groups &the type of the group (ex: worker, mineral). \\
\hline
x-pos & all groups &the x,y location of the center of gravity of the group.\\
y-pos & \\
\hline
x-min & all groups &the bounding box of the group.\\
x-max & \\
y-min & \\
y-max & \\
\hline
health & all groups &the sum of the health of all units in the group.\\
\hline
taking-damage & all groups &the number of members of the group currently taking damage (under attack). \\
\hline
shooting & all groups &the number of members of the group currently attacking an enemy. \\
\hline
speed & all groups &the average speed of the group. \\
\hline
heading & all groups &the average heading of the group. \\
\hline
dist-to-focus & all groups &the distance from the center of gravity of the group to the attentional focus point.\\
\hline
dist-to-query & all groups &the distance from the center of gravity of the group to the last query location.\\
\hline
owner & all groups &the player number of the group's owner.\\
\hline
enemy & all groups &1 if the group belongs to an enemy player, 0 otherwise.\\
\hline
sticky & friendly groups & 1 if the group is sticky- sticky groups remain together even if they are no longer spatially close.\\
\hline
command & friendly groups & The last command issued to the group ("none" if no command has been issued).\\
\hline
command-running & friendly groups & The number of members of the group currently executing a command.\\
\hline
command-success & friendly groups & The number of members of the group that successfully completed the last command.\\
\hline
command-failure & friendly groups & The number of members of the group that unsuccessfully completed the last command.\\
\hline
\end{tabular}
\end{center}

\bibliographystyle{plain}
\bibliography{report}

\end{document}
