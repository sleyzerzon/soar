\subsection{\soarb{chunk-name-format}}
\label{chunk-name-format}
\index{chunk-name-format}
Specify format of the name to use for new chunks. 
 Priority: 4; Status: Complete, EvilBackDoor
\subsubsection*{Synopsis}
\begin{verbatim}
chunk-name-format [-sl] -p [<prefix>]
chunk-name-format [-sl] -c [<count>]
\end{verbatim}
\subsubsection*{Options}
\begin{tabular}{|l|l|}
\hline 
 -c, --count  & \\
 \hline 
 count  & \\
 \hline 
 -l, --long  & \\
 \hline 
 -p, --prefix  & \\
 \hline 
 prefix  & \\
 \hline 
 -s, --short  & \\
 \hline 
\end{tabular}
\subsubsection*{Description}
\subsubsection*{Structured Output:}
\paragraph*{On Query}
\begin{itemize}
\item  kParamChunkLongFormat, kTypeBoolean, true if using long format, false for short format. 
\item  kParamChunkCount, kTypeInt, integer representing current chunk count. 
\item  kParamChunkNamePrefix, kTypeString, the current chunk prefix. 
\end{itemize}
\paragraph*{On Success}
\paragraph*{Notes}
\subsubsection*{Error Values:}
\paragraph*{During Parsing}
 kIntegerExpected, kIntegerMustBeNonNegative, kMissingOptionArg, kUnrecognizedOption, kGetOptError, kTooManyArgs
\paragraph*{During Execution}
 kCountGreaterThanMaxChunks, kCountLessThanChunks, kInvalidPrefix
