\subsection{\soarb{clog}}
\label{clog}
\index{clog}
Record all user-interface input and output to a file. 
\subsubsection*{Synopsis}
\begin{verbatim}
clog -[Ae] filename
clog –a string
clog [–cdoq]
\end{verbatim}
\subsubsection*{Options}
\begin{tabular}{|l|l|}
\hline
\soar{ filename } & Open filename and begin logging.  \\
\hline
\soar{ -c, --close, -o, --off, -d, --disable } & Stop logging, close the file.  \\
\hline
\soar{ -a, --add string } & Add the given string to the open log file.  \\
\hline
\soar{ -q, --query } & Returns \emph{open}
 if logging is active or \emph{closed}
 if logging is not active.  \\
\hline
\soar{ -A, --append, -e, --existing } & Opens existing log file named filename and logging is added at the end of the file.  \\
\hline
\end{tabular}
\subsubsection*{Description}
 The \textbf{clog}
 command allows users to save all user-interface input and output to a file. When Soar is logging to a file, everything typed by the user and everything printed by Soar is written to the file (in addition to the screen). 
 Invoke \textbf{clog}
 with no arguments (or with \textbf{-q}
) to query the current logging status. Pass a filename to start logging to that file (relative to the command line interface's home directory (see the home command)). Use the \textbf{close}
 option to stop logging. 
\subsubsection*{Examples}
 To initiate logging and place the record in foo.log: \begin{verbatim}
clog foo.log
\end{verbatim}
 To append log data to an existing foo.log file: \begin{verbatim}
clog -A foo.log
\end{verbatim}
 To terminate logging and close the open log file: \begin{verbatim}
clog -c
\end{verbatim}
\subsubsection*{Known Issues}
 Does not log everything when structured output is selected. 
\subsubsection*{See also}
\hyperref[command-to-file]{command-to-file} 