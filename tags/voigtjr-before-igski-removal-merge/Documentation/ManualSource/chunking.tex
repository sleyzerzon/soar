% ----------------------------------------------------------------------------
\typeout{--------------- learning (CHUNKING) --------------------------------}
\chapter{Learning}
\label{CHUNKING}
\index{chunking}
\index{learning}
\index{result}
\index{subgoal}

\nocomment{

	This chapter really needs a general explanation of learning in
	Soar before it dives into WHEN chunks are and are not formed,
	and HOW ... cf. Chpater 2-level discussions

	I'd like to add two figures to this chapter: \\
	1. an illustration of the backtracing process \\
	2. an illustration of how learning fits into the decision cycle.

	Also, I'm pretty sure this chapter hasn't changed much since
	justifications were added to Soar. It would be nice to connect
	justifications to chunks.
	}


\nocomment{
	\begin{figure}
	%\insertfigure{learn}{7.75in}
	\insertcaption{Will somehow be an illustration of learning in
		the decision cycle.} 
	\label{fig:learn}
	\end{figure}
	}

Chunking is Soar's learning mechanism, the sole learning mechanism in Soar.
Chunking creates productions, called \emph{chunks}, that summarize the
processing required to produce the results of subgoals. When a chunk is built,
it is added to production memory, where it will be matched in similar
situations, avoiding the need for the subgoal. Chunks are created only when
results are formed in subgoals; since most Soar programs are continuously
subgoaling and returning results to higher-level states, chunks are typically
created continuously as Soar runs.

This chapter begins with a discussion of when chunks are built (Section
\ref{CHUNKING-creation} below), followed by a detailed discussion of
how Soar determines a chunk's conditions and actions (Section
\ref{CHUNKING-determining}). Sections \ref{CHUNKING-variablizing} through
\ref{CHUNKING-ordering} examine the construction of chunks in further
detail. Section \ref{CHUNKING-inhibition} explains how and why chunks are
prevented from matching with the WME's that led to their creation. Section
\ref{CHUNKING-problems} reviews the problem of overgeneral chunks.


% ----------------------------------------------------------------------------
\section{Chunk Creation}
\label{CHUNKING-creation}
\index{chunking!creation}

Several factors govern when chunks are built. Soar chunks the results of every
subgoal, \emph{unless} one of the following conditions is true:
\index{chunking!when active}

\index{learn}
\begin{enumerate}
\item Learning is \soar{off}. (See Section \ref{learn} on page \pageref{learn}
	for details of \soar{learn} used to turn learning off.) 

	Learning can be set to \soar{on} or \soar{off}.
	When \soar{learn} is \soar{on} chunks are built.  
	When \soar{learn} is \soar{off}, chunks are not built. 

\item Learning is set to \soar{bottom-up} and a chunk has already 
	been built for a subgoal of the state that generated the results. 
	(See Section \ref{learn} on page \pageref{learn} for details of 
	\soar{learn} used to set learning to bottom-up.) 

	With bottom-up learning, chunks are learned only in states in which no
	subgoal has yet generated a chunk. In this mode, chunks are learned
	only for the ``bottom'' of the subgoal hierarchy and not the
	intermediate levels. With experience, the subgoals at the bottom will
	be replaced by the chunks, allowing higher level subgoals to be
	chunked.\footnote{For some tasks, bottom-up chunking facilitates
	modelling power-law speedups, although its long-term theoretical
	status is problematic.}
	\index{bottom-up chunking}
	\index{chunking!bottom-up}
	
	\nocomment{this would be the appropriate place in the manual to discuss
		the rationale behind the existence of bottom-up chunking. I
		don't believe it's explained anywhere. 
		}

\item The chunk duplicates a production or chunk already in production memory.
	In some rare cases, a duplicate production will not be detected because the
	order of the conditions or actions is not the same as an existing production.  
	\index{chunking!duplicate chunks}

\item The augmentation, \soar{\carat quiescence t}, of the substate that
	produced the result is backtraced through.

	This mechanism is motivated by the \emph{chunking from exhaustion}
	problem, where the results of a subgoal are dependent on the
	exhaustion of alternatives (see Section \ref{CHUNKING-problems} on page
	\pageref{CHUNKING-problems}). If this substate augmentation is
	encountered when determining the conditions of a chunk, then no chunk
	will be built for the currently considered action. This is recursive, 
	so that if an un-chunked result is relevant to a second result, no 
	chunk will be built for the second result. This does not prevent the
	creation of a chunk that would include \soar{\carat quiescence t} as a
	condition.  
	\index{quiescence t (augmentation)} \index{exhaustion}
      
\item Learning has been temporarily turned off via a call to the
	\soar{dont-learn} production action (described on page
	\pageref{SYNTAX-pm-actions-learning} in Section 
	\ref{SYNTAX-pm-actions-learning}).

	This capability is provided for debugging and system development, and
	it is not part of the theory of Soar.
\end{enumerate}

If a result is to be chunked, Soar builds the chunk \emph{as soon as the
result is created}, rather than waiting until subgoal termination.
\index{result}
\index{subgoal!result}

	\nocomment{CBC: As soon as it's identified as a result, I assume.
		E.g., for the case where a ``result'' is created
		first, and not linked to the superstate until later.
	
		BobD: it doesn't become a result until it's linked.}

% ----------------------------------------------------------------------------
\section{Determining Conditions and Actions}
\label{CHUNKING-determining}

Chunking is an experience-based learning mechanism that summarizes  as 
productions the problem solving that occurs within a state. In order to 
maintain a
history of the processing to be used for chunking, Soar builds a 
\emph{trace} of the productions that fire in the subgoals. This section
describes how the relevant actions are determined, how information is 
stored in a trace, and finally, how the trace and the actions together 
determine the conditions for the chunk.

In order for the chunk to apply at the appropriate time, its conditions must
test exactly those working memory elements that were necessary to produce the
results of the subgoal. 
Soar computes a chunk's conditions based on the
productions that fire in the subgoal, beginning with the results of
the subgoal,
and then \emph{backtracing} through the productions that created 
each result.  It recursively backtraces through the working memory
elements that matched the conditions of the productions, finding the
actions that led to the WME's creation, etc., until conditions are
found that test elements that are linked to a superstate.
\index{backtracing}

\index{working memory!trace}
\index{trace!memory}

\nocomment{  This is what is used to say...
Soar computes a chunk's conditions based on the productions that 
fire in the subgoal. Chunking begins with the results of the subgoal,
and then \emph{backtraces} through the productions that created the preference
for each result. It then recursively backtraces through the working memory
elements that matched the conditions of the productions, finding the
acceptable preferences that led to their creation, etc., until conditions are
found that test elements that are linked to a superstate.}

% ----------------------------------------------------------------------------
\subsection{Determining a chunk's actions}
\index{result}
\index{subgoal result}
\index{chunking!determining actions}

A chunk's actions are built from the results of a subgoal.  A \emph{result} is
any working memory element created in the substate that is linked to a 
superstate.  A working memory element
is linked if its identifier is either the value of a superstate
WME, or the value of an augmentation  for an object that is linked to a
superstate.

\index{linked!chunk action}
\index{chunking!actions}

The results produced by a single production firing are the basis for creating
the actions of a chunk. A new result can lead to other results by linking a
superstate to a WME in the substate. This WME may in turn link
other WMEs in the substate to the superstate, making them results.
Therefore, the creation of a single WME that is linked to a superstate
can lead to the creation of a large number of results. All of the newly
created results become the basis of the chunk's actions.

% ----------------------------------------------------------------------------
\subsection{Tracing the creation and reference of working memory elements} 

Soar automatically maintains information on the creation of each 
working memory element in every state.  When a production fires, a
trace of the production is saved with the appropriate state. A \emph{trace} is
a list of the working memory elements matched by the production's conditions,
together with the actions created by the production.  The appropriate state
is the most recently created state (i.e., the state \emph{lowest} in the
subgoal hierarchy) that occurs in the production's matched working memory
elements.
\index{trace!memory}

Recall that when a subgoal is created, the \carat item augmentation lists all
values that lead to the impasse.
Chunking is complicated by the fact that the \soar{\carat item} augmentation
of the substate is created by the architecture and not by productions.
Backtracing cannot determine the cause of these substate augmentations in the
same way as other working memory elements. To overcome this, Soar maps these
augmentations onto the acceptable preferences for the operators in the 
\soar{\carat item} augmentations.


\subsubsection*{Negated conditions}
\index{negated conditions}
\index{chunking!negated conditions}

Negated conditions are included in a trace in the following way: when a
production fires, its negated conditions are fully instantiated with its
variables' appropriate values. This instantiation is based on the working
memory elements that matched the production's positive conditions. If the
variable is not used in any positive conditions, such as in a conjunctive
negation, a dummy variable is used that will later become a variable in a
chunk.

If the identifier used to instantiate a negated condition's identifier field
is linked to the superstate, then the instantiated negated condition is
added to the trace as a negated condition. In all other cases, the negated
condition is ignored because the system cannot determine why a working memory
element \emph{was not} produced in the subgoal and thus allowed the production
to fire. Ignoring these negations of conditions internal to the subgoal may
lead to overgeneralization in chunking (see Section \ref{CHUNKING-problems} on
page \pageref{CHUNKING-problems}). 
\index{overgeneral chunk}
     
% ----------------------------------------------------------------------------
\subsection{Determining a chunk's conditions}
\index{chunking!determining conditions}

The conditions of a chunk are determined by a dependency analysis of
production traces --- a process called \emph{backtracing}.  For each
instantiated production that creates a subgoal result, backtracing examines
the production trace to determine which working memory elements were matched.
If a matched working memory element is linked to a superstate, it is included
in the chunk's conditions. If it is not linked to a superstate, then
backtracing recursively examines the trace of the production that created the
working memory element. Thus, backtracing begins with a subgoal result, traces
backwards through all working memory elements that were used to produce that
result, and collects all of the working memory elements that are linked to a
superstate. This method ignores when the working memory elements were created,
thus allowing the conditions of one chunk to test the results of a chunk
learned earlier in the subgoal. The user can observe the backtracing process
by setting setting backtracing on, using the watch command: \soar{watch
backtracing -on} (see Section \ref{watch} on page \pageref{watch}). 
This prints out a trace of the conditions as they are collected.
\index{backtracing}
\index{chunking!conditions}

\index{desirability preference} 
Certain productions do not participate in backtracing. If a production creates
only a \soar{reject} preference or a desirability preference (\soar{better},
\soar{worse}, \soar{indifferent}, or \soar{parallel}), then neither the
preference nor the objects that led to its creation will be included in the
chunk. (The exception to this is that if the desirability or \soar{reject}
preference is a {\em result} of a subgoal, it will be in the chunk's actions.)
Desirability and reject preferences should be used only as search control for
choosing between legal alternatives and should not be used to guarantee the
correctness of the problem solving. The argument is that such preferences
should affect only the \emph{efficiency} and not the \emph{correctness} of
problem solving, and therefore are not necessary to produce the results.
Necessity preferences (\soar{require} or \soar{prohibit}) should be used to
enforce the correctness of problem solving; the productions that create these
preferences will be included in backtracing.
\index{preference!require}
\index{preference!prohibit}
\index{require preference}
\index{prohibit preference}

Given that results can be created at any point during a subgoal, it is
possible for one result to be relevant to another result. Whether or not the 
first result is included in the chunk for the second result depends on the
links that were used to match the first result in the subgoal. If the elements
are linked to the superstate, they are included as conditions. If the
elements are not linked to the superstate, then the result is traced through.
In some cases, there may be more than one set of links, so it is possible for
a result to be both backtraced through, and included as a condition.


% ----------------------------------------------------------------------------
\section{Variablizing Identifiers}
\label{CHUNKING-variablizing}
\index{identifier!variablization of}
\index{chunking!variablization}
\index{variablization}

Chunks are constructed by examining the traces, which include working memory
elements and operator preferences. To achieve any useful generality in chunks,
identifiers of actual objects must be replaced by variables when the chunk is
created; otherwise chunks will only ever fire when the exact same objects
are matched.  However, a constant value is never variablized; the actual 
value always appears directly in the chunk.

When a chunk is built, all occurrences of the same identifier are replaced
with the same variable. This can lead to an overspecific chunk, when two
variables are forced to be the same in the chunk, even though distinct
variables in the original productions just happened to match the same
identifier.

A chunk's conditions are also constrained by any not-equal (\soar{<>}) tests
for pairs of indentifiers used in the conditions of productions that are
included in the chunk. These tests are saved in the production traces and then
added in to the chunk.
\index{chunking!conditions}

% ----------------------------------------------------------------------------
\section{Ordering Conditions}
\label{CHUNKING-ordering}
\index{matcher}
\index{ordering chunk conditions}

	\nocomment{I think we need an actual section (earlier in the
		manual), describing the rete matcher and the
		reordering of conditions (which also happens 
		internally for user-defined productions). Then this
		section would mention that it's the same reordering process.}

Since the efficiency of the Rete matcher  \cite{Forg81} depends
heavily upon the order of a production's conditions, the chunking mechanism
attempts to write the chunk's conditions in the most favorable order. At each
stage, the condition-ordering algorithm tries to determine which eligible
condition, if placed next, will lead to the fewest number of partial
instantiations when the chunk is matched. A condition that matches an object
with a multi-valued attribute will lead to multiple partial instantiations, so
it is generally more efficient to place these conditions later in the
ordering.
\index{chunking!ordering conditions}
\index{multi-valued attribute}

This is the same process that internally reorders the conditions in
user-defined productions, as mentioned briefly in Section \ref{ARCH-pm-structure}. 


% ----------------------------------------------------------------------------
\section{Inhibition of Chunks}
\label{CHUNKING-inhibition}
\index{chunking!refractory inhibition}
\index{refractory inhibition of chunks}

When a chunk is built, it may be able to match immediately with the same
working memory elements that participated in its creation. If the production's
actions include preferences for new operators, the production would immediately
fire and create a preference for a new operator, which duplicates the 
operator preference
that was the original result of the subgoal. To prevent this,
\emph{inhibition} is used. This means that each production that is built 
during chunking is considered to have already fired with the instantiation of
the exact set of working memory elements used to create it. This does not
prevent a newly learned chunk from matching other working memory elements
that are present and firing with those values.

	\nocomment{any insights to why its called ``refractory''?}

% ----------------------------------------------------------------------------
\section{Problems that May Arise with Chunking}
\label{CHUNKING-problems}
\index{chunking!overgeneral}
\index{chunking!incorrect chunks}
\index{incorrect chunks}
\index{overgeneral chunk}

\nocomment{Moved from chapter 2: If there are no variables in justifications, I
	don't quite understand how overgeneralization can occur. \\
        RD: can still be overgeneral due to lack of conditions, e.g.,
	chunking from exhaustion (from testing a negative condition in the
	subgoal) \\
	JEL: it's from losing the local negations}

\nocomment{BobD: there are more problems with chunking than this. See last two
	slides from ``guts of chunking'', 11th soar workshop (and talk to Bob
	and/or John for an explanation
	}

One of the weaknesses of Soar is that chunking can create overgeneral productions
that apply in inappropriate situations, or overspecific productions that will
never fire. These problems arise when chunking cannot accurately summarize the
processing that led to the creation of a result. Below is a description of
three known problems in chunking.

\subsection{Using search control to determine correctness}
\index{desirability preference}

Overgeneral chunks can be created if a result of problem solving in a subgoal
is dependent on search-control knowledge. Recall that desirability
preferences, such as \soar{better}, \soar{best}, and \soar{worst}, are not
included in the traces of problem solving used in chunking (Section
\ref{CHUNKING-determining} on page \pageref{CHUNKING-determining}). In theory,
these preferences do not affect the validity of search. In practice, however,
a Soar program can be written so that search control \emph{does} affect the
correctness of search. Here are two examples:\vspace{-12pt}

\begin{enumerate} 
\item Some of the tests for correctness of a result are included in
	productions that prefer operators that will produce correct results.
  	The system will work correctly only when those productions are loaded.\vspace{-8pt}
\item An operator is given a worst preference, indicating that it
  	should be used only when all other options have been exhausted.
  	Because of the semantics of worst, this operator will be selected
  	after all other operators; however, if this operator then produces a
  	result that is dependent on the operator occurring after all others,
  	this fact will not be captured in the conditions of the chunk.
\end{enumerate}
\index{necessity preference} 
\index{preference!require} \index{preference!prohibit}
\index{require preference} \index{prohibit preference}

In both of these cases, part of the test for producing a result is {\em
implicit} in search control productions. This move allows the explicit state
test to be simpler because any state to which the test is applied is
guaranteed to satisfy some of the requirements for success. However, chunks
created in such a problem space will be overgeneral because the implicit parts
of the state test do not appear as conditions. 

\textbf{Solution:} To avoid this problem, necessity preferences
(\soar{require} and \soar{prohibit}) should be used whenever a control
decision is being made that also incorporates goal-attainment knowledge.  The
necessity preferences are included in the backtrace by chunking, thereby
avoiding overgenerality.

\subsection{Testing for local negated conditions}

Overgeneral chunks can be created when negated conditions test for the absence
of a working memory element that, if it existed, would be local to the
substate.  Chunking has no mechanism for determining \textit{why} a given
working memory element does not exist, and thus a condition that occurred in a
production in the subgoal is not included in the chunk. For example, if a
production tests for the absence of a local flag, and that flag is copied down
to the substate from a superstate, then the chunk should include a test that
the flag in the superstate does not exist. 
Unfortunately, it is computationally expensive to determine why a given
working memory element does not exist. Chunking only includes negated tests if
they test for the absence of superstate working memory elements. 

\textbf{Solution:} To avoid using negated conditions for local data, the local
data can be made a result by attaching it to the superstate. This increases
the number of chunks learned, but a negated condition for the superstate can
be used that leads to correct chunks.
\index{negated!conditions}
\index{chunking!negated conditions}

\subsection{Testing for the substate}

Overgeneral chunks can be created if a result of a subgoal is dependent on the
creation of an impasse within the substate. For example, processing in a
subgoal may consist of exhaustively applying all the operators in the problem
space. If so, then a convenient way to recognize that all operators have
applied and processing is complete is to wait for a state no-change impasse to
occur. When the impasse occurs, a production can test for the resulting
substate and create a result for the original subgoal. This form of state test
builds overgeneral chunks because no pre-existing structure is relevant to the
result that terminates the subgoal. The result is dependent only on the
existence of the substate within a substate.
\index{quiescence t (augmentation)}
\index{exhaustion}

\textbf{Solution:} The current solution to this problem is to allow the
problem solving to signal the architecture that the test for a substate is
being made.  The signal used by Soar is a test for the \soar{\carat quiescence
t} augmentation of the subgoal.  The chunking mechanism recognizes this test
and does not build a chunk when it is found in a backtrace of a subgoal.  The
history of this test is maintained, so that if the result of the substate is
then used to produce further results for a superstate, no higher chunks will
be built.  However, if the result is used as search control (it is a
desirability preference), then it does not prevent the creation of chunks
because the original result is not included in the backtrace.  If the
\soar{\carat quiescence t} being tested is connected to a superstate, it will
not inhibit chunking and it will be included in the conditions of the chunk.



\nocomment{
	\subsection{Overuse of predicates}

	Moved from chapter 2:

	All of the predicate tests are lost in the chunk, and only the
	exact value is included. If the predicate is explicitly
	represented as a relation between two objects in working
	memory, chunking will capture that abstract relationship 
	and create a much more general chunk.

	(also needs clarification about the use of predicates in the
	blocks world task, where we have to say  that the block that's
	being moved is not the same as the block that's 
        being moved to)

	}
