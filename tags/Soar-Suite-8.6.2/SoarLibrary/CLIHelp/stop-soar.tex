\subsection{\soarb{stop-soar}}
\label{stop-soar}
\index{stop-soar}
Pause Soar. 
\subsubsection*{Synopsis}
\begin{verbatim}
stop-soar [-s] [reason string]
\end{verbatim}
\subsubsection*{Options}
\begin{tabular}{|l|l|}
\hline
\soar{ -s, --self } & Stop only the soar agent where the command is issued. All other agents continue running as previously specified.  \\
\hline
\soar{ reason\_string } & An optional string which will be printed when Soar is stopped, to indicate why it was stopped. If left blank, no message will be printed when Soar is stopped.  \\
\hline
\end{tabular}
\subsubsection*{Description}
 The \textbf{stop-soar}
 command stops any running Soar agents. It sets a flag in the Soar kernel so that Soar will stop running at a ``safe'' point and return control to the user. This command is usually not issued at the command line prompt - a more common use of this command would be, for instance, as a side-effect of pressing a button on a Graphical User Interface (GUI). 
\subsubsection*{Default Aliases}
\begin{tabular}{|l|l|}
\hline
\soar{ Alias } & Maps to  \\
\hline
\soar{ interrupt } & stop-soar  \\
\hline
\soar{ ss } & stop-soar  \\
\hline
\soar{ stop } & stop-soar  \\
\hline
\end{tabular}
\subsubsection*{See Also}
\hyperref[run]{run} \subsubsection*{Warnings}
 If the graphical interface doesn't periodically do an ``update'' of flush the pending I/O, then it may not be possible to interrupt a Soar agent from the command line. 
