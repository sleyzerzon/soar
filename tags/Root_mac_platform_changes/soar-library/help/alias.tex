\subsection{\soarb{alias}}
\label{alias}
\index{alias}
Define a new alias, or command, using existing commands and arguments. 
 Complete
\subsubsection*{Synopsis}
  \begin{verbatim}
alias name [cmd <args>]
alias -d name
alias
\end{verbatim}
\subsubsection*{Options}
\begin{tabular}{|l|l|}
\hline 
 -d, --disable, --off  & Remove the named alias.  \\
 \hline 
 name  & The name of the alias, i.e. the new command.  \\
 \hline 
 cmd  & An existing command that will be invoked when the alias is entered on the commandline.  \\
 \hline 
 args  & Valid arguments to the cmd (optional \& optional number).  \\
 \hline 
\end{tabular}
\subsubsection*{Description}
 This command defines new aliases by creating Soar procedures with the given name. The new procedure can then take an arbitrary number of arguments which are post-pended to the given definition and then that entire string is executed as a command. The definition must be a single command, multiple commands are not allowed. The \textbf{alias}
 procedure checks to see if the name already exists, and does not destroy existing procedures or aliases by the same name. Existing aliases can be removed by using the \textbf{-d}
 flag. With no arguments, \textbf{alias}
 returns the list of defined aliases. With only the name given, \textbf{alias}
 returns the current definition. 
\subsubsection*{Examples}
 The alias \emph{wmes}
 is defined as: \begin{verbatim}
alias wmes print -i
\end{verbatim}
 If the user executes a command such as: \begin{verbatim}
wmes {(* ^superstate nil)}
\end{verbatim}
 it is as if the user had typed this command: \begin{verbatim}
print -i {(* ^superstate nil)}
\end{verbatim}
 To check what a specific alias is defined as, you would type \begin{verbatim}
alias wmes
\end{verbatim}
\subsubsection*{Default Aliases}
\begin{tabular}{|l|l|}
\hline 
 Alias  & Maps to  \\
 \hline 
 unalias  & alias -d  \\
 \hline 
\end{tabular}
\subsubsection*{See Also}
\hyperref[unalias]{unalias} 