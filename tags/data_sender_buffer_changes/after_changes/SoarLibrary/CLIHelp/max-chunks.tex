\subsection{\soarb{max-chunks}}
\label{max-chunks}
\index{max-chunks}
Limit the number of chunks created during a decision cycle. 
\subsubsection*{Synopsis}
max-chunks [n]
\end{verbatim}
\subsubsection*{Options}
\hline
\soar{\soar{\soar{ n }}} & Maximum number of chunks allowed during a decision cycle.  \\
\hline
\end{tabular}
\subsubsection*{Description}
 The \textbf{max-chunks}
 command is used to limit the maximum number of chunks that may be created during a decision cycle. The initial value of this variable is 50; allowable settings are any integer greater than 0. 
 The chunking process will end after \textbf{max-chunks}
 chunks have been created, \emph{even if there are more results that have not been backtraced through to create chunks}
, and Soar will proceed to the next phase. A warning message is printed to notify the user that the limit has been reached. 
 This limit is included in Soar to prevent getting stuck in an infinite loop during the chunking process. This could conceivably happen because newly-built chunks may match immediately and are fired immediately when this happens; this can in turn lead to additional chunks being formed, etc. If you see this warning, something is seriously wrong; Soar is unable to guarantee consistency of its internal structures. You should not continue execution of the Soar program in this situation; stop and determine whether your program needs to build more chunks or whether you've discovered a bug (in your program or in Soar itself). 
