\subsection{\soarb{add-wme}}
\label{add-wme}
\index{add-wme}
Manually add an element to working memory. 
\subsubsection*{Synopsis}
\begin{verbatim}
add-wme id [^]attribute value [+]
\end{verbatim}
\subsubsection*{Options}
\begin{tabular}{|l|l|}
\hline
\soar{ id } & Must be an existing identifier.  \\
\hline
\soar{ \^{} } & Leading \^{} on attribute is optional.  \\
\hline
\soar{ attribute } & Attribute can be any Soar symbol. Use * to have Soar create a new identifier.  \\
\hline
\soar{ value } & Value can be any soar symbol. Use * to have Soar create a new identifier.  \\
\hline
\soar{ + } & If the optional preference is specified, its value must be + (acceptable).  \\
\hline
\end{tabular}
\subsubsection*{Description}
 Manually add an element to working memory. \textbf{add-wme}
 is often used by an input function to update Soar's information about the state of the external world. 
 \textbf{add-wme}
 adds a new wme with the given id, attribute, value and optional preference. The given id must be an existing identifier. The attribute and value fields can be any Soar symbol. If * is given in the attribute or value field, Soar creates a new identifier (symbol) for that field. If the preference is given, it can only have the value + to indicate that an acceptable preference should be created for this wme. 
 Note that because the id must already exist in working memory, the WME that you are adding will be attached (directly or indirectly) to the top-level state. As with other WME's, any WME added via a call to \textbf{add-wme}
 will automatically be removed from working memory once it is no longer attached to the top-level state. 
\subsubsection*{Examples}
 This example adds the attribute/value pair ``message-status received'' to the identifier (symbol) S1: \begin{verbatim}
add-wme S1 ^message-status received
\end{verbatim}
 This example adds an attribute/value pair with an acceptable preference to the identifier (symbol) Z2. The attribute is ``message'' and the value is a unique identifier generated by Soar. Note that since the \^{} is optional, it has been left off in this case. \begin{verbatim}
add-wme Z2 message * +
\end{verbatim}
\subsubsection*{Default Aliases}
\begin{tabular}{|l|l|}
\hline
\soar{ Alias } & Maps to  \\
\hline
\soar{ aw } & add-wme  \\
\hline
\end{tabular}
\subsubsection*{Warnings}
 Be careful how you use this command. It may have weird side effects (possibly even including system crashes). For example, the chunker can't backtrace through wmes created via \textbf{add-wme}
, nor will such wmes ever be removed thru Soar's garbage collection. Manually removing context/impasse wmes may have unexpected side effects. 
\subsubsection*{See Also}
\hyperref[remove-wme]{remove-wme} 