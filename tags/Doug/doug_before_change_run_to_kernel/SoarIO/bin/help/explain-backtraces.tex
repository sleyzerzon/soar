\documentclass[10pt]{article}
\usepackage{fullpage, graphicx, url}
\title{Explain-backtraces - Soar Wiki}
\begin{document}
\section*{Explain-backtraces}
\subsubsection*{From Soar Wiki}


 This is part of the Soar Command Line Interface. 
\section*{ Name }


 \textbf{explain-backtraces}
 - Print information about chunk and justification backtraces. 


 Priority: 3; Status: Incomplete, EvilBackDoor\\ 
Result generated by kernel.--Jonathan 18:16, 25 Feb 2005 (EST) 
\section*{ Synopsis }
\begin{verbatim}
explain-backtraces -f prod_name
explain-backtraces [-c <n>] prod_name

\end{verbatim}
\section*{ Options }


\begin{tabular}{|p{1in}|p{5in}|}
\hline 
 prod\_name  & List all conditions and grounds for the chunk or justification.  \\
 \hline 
 -c, --condition  & Explain why condition number \emph{n}
 is in the chunk or justification.  \\
 \hline 
 -f, --full  &�?  \\
 \hline 

\end{tabular}



 \\ 

\section*{ Description }


 This command provides some interpretation of backtraces generated during chunking. If no option is given, then a list of all chunks and justifications is printed. 


 The two most useful variants are: \begin{verbatim}
explain-backtraces prodname 
explain-backtraces name n

\end{verbatim}



 The first variant lists all of the conditions for the named chunk or justification, and the ground which resulted in inclusion in the chunk/justification. A ground is a working memory element (WME) which was tested in the supergoal. Just knowing which WME was tested may be enough to explain why the chunk/justification exists. If not, the conditions can be listed with an integer value. This value can be used in explain-backtraces name n to obtain a list of the productions which fired to obtain this condition in the chunk/justification (and the crucial WMEs tested along the way). Why use an integer value to specify the condition? To save a big parsing job. 


 save\_backtraces mode must be on when a chunk or justification is created or no explanation will be available. 
\section*{ Structured Output }
\subsection*{ On Success }
\subsection*{ Notes }
\section*{ Error Values }
\subsection*{ During Parsing }


 kNotImplemented
\subsection*{ During Execution }

\end{document}
