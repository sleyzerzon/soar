\documentclass[10pt]{article}
\usepackage{fullpage, graphicx, url}
\title{Log - Soar Wiki}
\begin{document}
\section*{Log}
\subsubsection*{From Soar Wiki}


 This is part of the Soar Command Line Interface. 
\section*{ Name }


 \textbf{log}
 - Record all user-interface input and output to a file. 


 Status: Complete
\section*{ Synopsis }
\begin{verbatim}
log –Ae filename
log –a  string
log [–cdoq]

\end{verbatim}
\section*{ Options }


\begin{tabular}{|p{1in}|p{5in}|}
\hline 
 filename  & Open filename and begin logging.  \\
 \hline 
 -c, --close, -o, --off, -d, --disable  & Stop logging, close the file.  \\
 \hline 
 -a, --add string  & Add the given string to the open log file.  \\
 \hline 
 -q, --query  & Returns \emph{open}
 if logging is active or \emph{closed}
if logging is not active.  \\
 \hline 
 -A, --append, -e, --existing  & Opens existing log file named filename and logging is added at the end of the file.  \\
 \hline 

\end{tabular}



 \\ 

\section*{ Description }


 Invoked log with no arguments (or with -q) to query the current logging status. Pass a filename to start logging to that file (relative to the command line interface's home directory (see the home command)). Use the close option to stop logging. 


 \\ 

\section*{ Examples }


 To initiate logging and places the record in foo.log: \begin{verbatim}
log -new foo.log

\end{verbatim}



 To append log data to an existing foo.log file: \begin{verbatim}
log -existing foo.log

\end{verbatim}



 To terminate logging and closes the open log file: \begin{verbatim}
log -off


\end{verbatim}

\section*{ See Also }


 \\ 

\section*{ Structured Output }
\subsection*{ On Success }
\begin{verbatim}
<result>
  <arg param="log_setting" type="boolean">setting</arg>
  <arg param="filename" type="string">log_filename</arg>
</result>

\end{verbatim}
\subsection*{ Notes }
\begin{itemize}
\item  setting is true (log file open) or false (closed). 
\item  log\_filename is only included if the log is open. 
\item  currently prompts and input are not logged because \^a��log\^a�� simply registers on the gSKI print callback. Need to implement fully. 

\end{itemize}
\section*{ Error Values }
\subsection*{ During Parsing }


 kUnrecognizedOption, kGetOptError, kTooFewArgs, kTooManyArgs, kInvalidOperation
\subsection*{ During Execution }


 kLogAlreadyOpen, kLogOpenFailure, kLogNotOpen, kInvalidOperation

\end{document}
