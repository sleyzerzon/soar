\documentclass[10pt]{article}
\usepackage{fullpage, graphicx, url}
\title{Alias - Soar Wiki}
\begin{document}
\section*{Alias}
\subsubsection*{From Soar Wiki}


 This is part of the Soar Command Line Interface. 
\section*{ Name }


 \textbf{alias}
 - Define a new alias, or command, using existing commands and arguments.\\ 
 Status: Incomplete\\ 
Structured output.--Jonathan 15:37, 25 Mar 2005 (EST) 
\section*{ Synopsis }


  \begin{verbatim}
alias name [cmd <args>]
alias -d name
alias

\end{verbatim}



 
\section*{ Options }


\begin{tabular}{|p{1in}|p{5in}|}
\hline 
 -d, --disable, --off  & Remove the named alias.  \\
 \hline 
 name  & The name of the alias, i.e. the new command.  \\
 \hline 
 cmd  & An existing command that will be invoked when the alias is entered on the commandline.  \\
 \hline 
 args  & Valid arguments to the cmd (optional \& optional number).  \\
 \hline 

\end{tabular}



 \\ 

\section*{ Description }


 This command defines new aliases by creating Soar procedures with the given name. The new procedure can then take an arbitrary number of arguments which are post-pended to the given definition and then that entire string is executed as a command. The definition must be a single command,multiple commands are not allowed. The alias procedure checks to see if the name already exists, and does not destroy existing procedures or aliases by the same name. Existing aliases can be removed by using the -d flag. With no arguments, alias returns the list of defined aliases. With only the name given, alias returns the current definition. 
\section*{ Examples }


 The alias \emph{wmes}
 is defined as: \begin{verbatim}
alias wmes print -i

\end{verbatim}



 If the user executes a command such as: \begin{verbatim}
wmes {(* ^superstate nil)}

\end{verbatim}



 it is as if the user had typed this command: \begin{verbatim}
print -i {(* ^superstate nil)}

\end{verbatim}



 To check what a specific alias is defined as, you would type \begin{verbatim}
alias wmes

\end{verbatim}

\section*{ See Also }


 unalias
\section*{ Structured Output }
\subsection*{ On Success }
\begin{verbatim}
<result>
  <arg name="alias" type="string">alias</arg>
  <arg name="aliasedcommand" type="string">aliased_command</arg>
</result>

\end{verbatim}
\subsection*{ Notes }
\begin{itemize}
\item  When listing the current alias database, each alias will have the alias/aliased command pair returned. 
\item  An example pair for step (run -s 1) would be: 

\end{itemize}
\begin{verbatim}
<arg name="alias" type="string">step</arg>
<arg name="aliasedcommand" type="string">run -s 1</arg>

\end{verbatim}
\section*{ Error Values }
\subsection*{ During Parsing }


 kMissingOptionArg, kUnrecognizedOption, kGetOptError, kTooManyArgs, kTooFewArgs
\subsection*{ During Execution }


 kAliasNotFound, kAliasExists

\end{document}
