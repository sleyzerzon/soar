\documentclass[10pt]{article}
\usepackage{fullpage, graphicx, url}
\title{O-support-mode - Soar Wiki}
\begin{document}
\section*{O-support-mode}
\subsubsection*{From Soar Wiki}


 This is part of the Soar Command Line Interface. 
\section*{ Name }


 \textbf{o-support-mode}
 - Choose experimental variations of o-support. 


 Priority: 3; Status: Complete
\section*{ Synopsis }
\begin{verbatim}
o-support-mode [0|1|2|3|4]

\end{verbatim}
\section*{ Options }


\begin{tabular}{|p{1in}|p{5in}|}
\hline 
 0  & Mode 0 is the base mode. O-support is calculated based on the structure of working memory that is tested and modified. Testing an operator or operator acceptable preference results in state or operator augmentations being o-supported. The support computation is very complex (see soar manual).  \\
 \hline 
 1  & Not available through gSKI.  \\
 \hline 
 2  & Mode 2 is the same as mode 0 except that all support is calculated the production structure, not from working memory structure. Augmentations of operators are still o-supported.  \\
 \hline 
 3  & Mode 3 is the same as mode 2 except that operator elaborations (adding attributes to operators) now get i-support even though you have to test the operator to elaborate an operator.  \\
 \hline 
 4  & Mode 4 is the default.  \\
 \hline 

\end{tabular}



 \\ 

\section*{ Description }


 Running o-support-mode with no arguments prints out the current o-support-mode. 
\section*{ Structured Output }
\subsection*{ On Query }
\begin{verbatim}
<result>
  <arg param="value" type="int">mode</arg>
</result>

\end{verbatim}
\subsection*{ Otherwise }


 When setting o-support, if successful, true is returned. 
\section*{ Error Values }
\subsection*{ During Parsing }


 kTooManyArgs, kIntegerOutOfRange
\subsection*{ During Execution }


 kAgentRequired, kInvalidOSupportMode

\end{document}
