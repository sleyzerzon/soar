\subsection{\soarb{preferences}}
\label{preferences}
\index{preferences}
Examine details about the preferences that support the specified \emph{id}
 and \emph{attribute}
. 
\subsubsection*{Synopsis}
preferences [-0123nNtw] [[id] [[^]attribute]]
\end{verbatim}
\subsubsection*{Options}
\hline
\soar{\soar{\soar{ -0, -n, --none }}} & Print just the preferences themselves  \\
\hline
\soar{\soar{\soar{ -1, -N, --names }}} & Print the preferences and the names of the productions that generated them  \\
\hline
\soar{\soar{\soar{ -2, -t, --timetags }}} & Print the information for the --names option above plus the timetags of the wmes matched by the LHS of the indicated productions  \\
\hline
\soar{\soar{\soar{ -3, -w, --wmes }}} & Print the information for the --timetags option above plus the entire wme matched on the LHS.  \\
\hline
\soar{\soar{\soar{ -o, --object }}} & Print the support for all the wmes that comprise the object (the specified ID).  \\
\hline
\soar{\soar{\soar{id}}} & Must be an existing Soar object identifier.  \\
\hline
\soar{\soar{\soar{attribute}}} & Must be an existing \emph{\^{}attribute}
 of the specified identifier.  \\
\hline
\end{tabular}
\subsubsection*{Description}
 The \textbf{preferences}
 command prints all the preferences for the given object id and attribute. If \emph{id}
 and \emph{attribute}
 are not specified, they default to the current state and the current operator. The '\^{}' is optional when specifying the attribute. The optional arguments indicates the level of detail to print about each preference. 
 This command is useful for examining which candidate operators have been proposed and what relationships, if any, exist among them. If a preference has O-support, the string, ``:O'' will also be printed. 
 When only the ID is specified on the commandline, if the ID is a state, Soar uses the default attribute \^{}operator. If the ID is not a state, Soar prints the support information for all WMEs whose $<$value$>$ is the ID. 
 When an ID and the --object flag are specified, Soar prints the preferences / wme support for all WMEs comprising the specified ID. 
\subsection*{Note}
 For the time being, \textbf{numeric-indifferent}
 preferences are listed under the heading ``binary indifferents:''. 
\subsubsection*{Examples}
soar> preferences S1 operator --names
Preferences for S1 ^operator:
acceptables:
 O2 (fill) +
   From waterjug*propose*fill
 O3 (fill) +
   From waterjug*propose*fill
unary indifferents:
 O2 (fill) =
   From waterjug*propose*fill
 O3 (fill) =
   From waterjug*propose*fill
\end{verbatim}
 preferences -n
\end{verbatim}
soar> preferences s1 jug
Preferences for S1 ^jug:
  
acceptables:
  (S1 ^jug I4) �:O 
  (S1 ^jug J1) �:O 
\end{verbatim}
soar> pref J1 -1
 Support for (31: O3 ^jug J1)
   (O3 ^jug J1) 
     From water-jug*propose*fill
 Support for (11: S1 ^jug J1)
   (S1 ^jug J1) �:O 
     From water-jug*apply*initialize-water-jug-look-ahead
\end{verbatim}
 soar> pref -o s1
 Support for S1 ^problem-space:
   (S1 ^problem-space P1) 
 Support for S1 ^name:
   (S1 ^name water-jug) �:O 
 Support for S1 ^jug:
   (S1 ^jug I4) �:O 
   (S1 ^jug J1) �:O 
 Support for S1 ^desired:
   (S1 ^desired D1) �:O 
 Support for S1 ^superstate-set:
   (S1 ^superstate-set nil) 
 Preferences for S1 ^operator:
 acceptables:
   O2 (fill) +
   O3 (fill) +
 Arch-created wmes for S1�:
 (2: S1 ^superstate nil)
 (1: S1 ^type state)
 Input (IO) wmes for S1�:
 (3: S1 ^io I1)
\end{verbatim}
\subsubsection*{See Also}
