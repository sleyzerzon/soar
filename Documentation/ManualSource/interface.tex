% ----------------------------------------------------------------------------
\typeout{--------------- The Soar User INTERFACE -----------------------------}
\chapter{The Soar User Interface}
\label{INTERFACE}
\index{interface}
%\index{user interface}
%\index{function definitions}

\nocomment{for each command, use the 'funsum' command with a brief
	description. This writes to the manual.glo file which can be edited
	into the funtion summary and index (see that file for more
	instructions). This is a bit tedious, but the reason I've set it up
	this way is that the command set is in flux right now -- this lessens
	the chance that a command will be inadvertently omitted from the
	function summary (or that a defunct command will be inadvertently
	included). 
	}

\nocomment{\begin{figure}[h]
\psfig{figure=dilbert-living.ps,height=2.2in} \vspace{12pt}
\end{figure}
}
% ----------------------------------------------------------------------------


This chapter describes the set of user interface commands for Soar. All commands and examples are presented as 
if they are being entered at the Soar command prompt.

This chapter is organized into 7 sections:
\begin{enumerate}
\item Basic Commands for Running Soar
\item Examining Memory
\item Configuring Trace Information and Debugging
\item Configuring Soar's Run-Time Parameters
\item File System I/O Commands
\item Soar I/O commands
\item Miscellaneous Commands
\end{enumerate}

Each section begins with a summary description of the commands covered
in that section, including the role of the command and its importance
to the user.  Command syntax and usage are then described fully, in
alphabetical order.

The following pages were automatically generated from the wiki version
at

\hspace{2em}\soar{\htmladdnormallink{http://code.google.com/p/soar/wiki/CommandIndex}{http://code.google.com/p/soar/wiki/CommandIndex}}


on the date listed on the title page of this manual.  Please consult
the wiki directly for the most accurate and up-to-date information.

For a concise overview of the Soar interface functions, see the Function
Summary and Index on page \pageref{func-sum}. This index is intended to be a
quick reference into the commands described in this chapter.

\subsubsection*{Notation}

\nocomment{check for all commands that I've got the notation current}

The notation used to denote the syntax for each user-interface command follows
some general conventions:\vspace{-12pt}
\begin{itemize}
\item The command name itself is given in a \soarb{bold} font.\vspace{-8pt}
\item Optional command arguments are enclosed within square brackets,
	\soar{[} and \soar{]}.\vspace{-8pt}
\item A vertical bar, \soar{|}, separates alternatives.\vspace{-8pt}
\item Curly braces, \soar{\{\}}, are used to group arguments when at least
one argument from the set is required.
\item The commandline prompt that is printed by Soar, is normally
the agent name, followed by '\soar{>}'.  In the examples in this manual, 
we use ``\soar{soar>}''.
\item Comments in the examples are preceded by
a '\soar{\#}', and in-line comments are preceded by '\soar{;\#}'.
\end{itemize}

For many commands, there is some flexibility in the order in which the
arguments may be given. (See the online help for each command for more
information.)  We have not incorporated this flexible ordering into the syntax
specified for each command because doing so complicates the specification of
the command.  When the order of arguments will affect the output
produced by a command, the reader will be alerted.

% ----------------------------------------------------------------------------
\section{Basic Commands for Running Soar}
\label{BASIC}

This section describes the commands used to start, run and stop a Soar 
program; to invoke on-line help information; and to create and 
delete Soar productions.  The specific commands described in this
section are:

\paragraph{Summary}
\begin{quote}
\begin{description}
%\item[d] - Run the Soar program for one decision cycle.
%\item[e] - Run the Soar program for one elaboration cycle.
\item[excise] - Delete Soar productions from production memory.
%\item[exit] - Terminate Soar and return to the operating system.
\item[gp] - Define a pattern used to generate and source a set of Soar productions.
\item[gp-max] - Set the upper-limit to the number of productions generated by the gp command.
\item[help] - Provide formatted, on-line information about Soar commands.
\item[init-soar] - Reinitialize Soar so a program can be rerun from scratch.
\item[run] - Begin Soar's execution cycle.
\item[sp] - Create a production and add it to production memory.
\item[stop-soar] - Interrupt a running Soar program.
\end{description}
\end{quote}
These commands are all frequently used anytime Soar is run.

\subsection{\soarb{excise}}
\label{excise}
\index{excise}
Delete Soar productions from production memory. 
\subsubsection*{Synopsis}
\begin{verbatim}
excise production_name [production_name ...]
excise -[acdtu]
\end{verbatim}
\subsubsection*{Options}
\begin{tabular}{|l|l|}
\hline 
 -a, --all  & Remove all productions from memory and perform an init-soar command  \\
 \hline 
 -c, --chunks  & Remove all chunks (learned productions) and justifications from memory  \\
 \hline 
 -d, --default  & Remove all default productions (:default) from memory  \\
 \hline 
 -t, --task  & Remove chunks, justifications, and user productions from memory  \\
 \hline 
 -u, --user  & Remove all user productions (but not chunks or default rules) from memory  \\
 \hline 
production\_name & Remove the specific production with this name.  \\
 \hline 
\end{tabular}
\subsubsection*{Description}
 This command removes productions from Soar's memory. The command must be called with either a specific production name or with a flag that indicates a particular group of productions to be removed. Using the flag \textbf{-a}
 or \textbf{--all}
 also causes an init-soar. 
\subsubsection*{Examples}
 This command removes the production my*first*production and all chunks: \begin{verbatim}
excise my*first*production --chunks
\end{verbatim}
 This removes all productions and does an init-soar: \begin{verbatim}
excise --all
\end{verbatim}
\subsubsection*{Default Aliases}
\begin{tabular}{|l|l|}
\hline 
 Alias  & Maps to  \\
 \hline 
 ex  & excise  \\
 \hline 
\end{tabular}
\subsubsection*{See Also}
\hyperref[init-soar]{init-soar} 
\input{wikicmd/tex/gp}
\input{wikicmd/tex/gp-max}
\subsection{\soarb{help}}
\label{help}
\index{help}
Provide formatted usage information about Soar commands. 
 Status: Incomplete\\ 
Currently uses working directory to find command-names and help/ subdir.--Jonathan 14:02, 25 Mar 2005 (EST) 
\subsubsection*{Synopsis}
\begin{verbatim}
help [command_name]
\end{verbatim}
\subsubsection*{Options}
\begin{tabular}{|l|l|}
\hline 
 command\_name  & Print usage syntax for the command.  \\
 \hline 
\end{tabular}
\subsubsection*{Description}
 This command prints formatted help for the given command name. 
\subsubsection*{Examples}
 To see the syntax for the \emph{excise}
 command: \begin{verbatim}
help excise
\end{verbatim}
 To see what commands help is available for: \begin{verbatim}
help
\end{verbatim}
\subsubsection*{Default Aliases}
\begin{tabular}{|l|l|}
\hline 
 Alias  & Maps to  \\
 \hline 
�?  & help  \\
 \hline 
 man  & help  \\
 \hline 
\end{tabular}
\subsubsection*{See Also}
\hyperref[helpex]{helpex} 
\subsection{\soarb{init-soar}}
\label{init-soar}
\index{init-soar}
empties working memory and resets run-time statistics. 
 Status: Complete
\subsubsection*{Synopsis}
\begin{verbatim}
init-soar
\end{verbatim}
\subsubsection*{Options}
 No options. 
\subsubsection*{Description}
 The init-soar command initializes Soar. It removes all elements from working memory, wiping out the goal stack, and resets all runtime statistics. The firing counts for all productions is reset to zero. The init-soar command allows a Soar program that has been halted to be reset and start its execution from the beginning. 
 init-soar does not remove any productions from production memory; to do this, use the excise command. Note however, that all justifications will be removed because they will no longer be supported. 
\subsubsection*{Examples}
\subsubsection*{See Also}
 excise
\subsubsection*{Structured Output:}
\paragraph*{On Success}
\begin{verbatim}
<result output="raw">true</result>
\end{verbatim}
\subsubsection*{Error Values:}
\paragraph*{During Parsing}
 No errors, all arguments ignored. 
\paragraph*{During Execution}
 kAgentRequired

\subsection{\soarb{run}}
\label{run}
\index{run}
Begin Soar\^a��s execution cycle. 
 Complete Complete, except --output may work incorrectly due to gSKI--Jonathan 14:07, 18 Feb 2005 (EST) 
\subsubsection*{Synopsis}
\begin{verbatim}
run [count]
run -[d|e|p|o][fs] [count]
\end{verbatim}
\subsubsection*{Options}
\begin{tabular}{|l|l|}
\hline 
 -d, --decision  & Run Soar for count decision cycles.  \\
 \hline 
 -e, --elaboration  & Run Soar for count elaboration cycles.  \\
 \hline 
 -f, --forever  & Run until halted by problem-solving completion or until stopped by an interrupt.  \\
 \hline 
 -o, --output  & Run Soar until the nth time output is generated by the agent. Limited by the value of max-nil-output-cycles.  \\
 \hline 
 -p, --phase  & Run Soar by phases. A phase is either an input phase, proposal phase, decision phase, apply phase, or output phase.  \\
 \hline 
 -s, --self  & If other agents exist within the kernel, do not run them at this time.  \\
 \hline 
 count  & A single integer which specifies the number of cycles to run Soar.  \\
 \hline 
\end{tabular}
\paragraph*{Deprecated Options}
 These may be reimplemented in the future. 
\begin{tabular}{|l|l|}
\hline 
 --operator  & Run Soar until the nth time an operator is selected.  \\
 \hline 
 --state  & Run Soar until the nth time a state is selected.  \\
 \hline 
\end{tabular}
\subsubsection*{Description}
 The \textbf{run}
 command starts the Soar execution cycle or continues any execution that was temporarily stopped. The default behavior of \textbf{run}
, with no arguments, is to cause Soar to execute until it is halted or interrupted by an action of a production, or until an external interrupt is issued by the user. The \textbf{run}
 command can also specify that Soar should run only for a specific number of Soar cycles or phases (which may also be prematurely stopped by a production action or a control-C). This is helpful for debugging sessions, where users may want to pay careful attention to the specific productions that are firing and retracting. 
 The \textbf{run}
 command takes two optional arguments: an integer, \emph{count}
, which specifies how many units to run; and a \emph{units}
 flag indicating what steps or increments to use. If \emph{count}
 is specified, but no \emph{units}
 are specified, then Soar is run by decision cycles. If \emph{units}
 are specified, but \emph{count}
 is unpecified, then \emph{count}
 defaults to '1'. 
 If there are multiple Soar agents that exist in the same Soar process, then issuing a \textbf{run}
 command in any agent will cause all agents to run with the same set of parameters, unless the flag \textbf{--self}
 is specified, in which case only that agent will execute. 
\paragraph*{Note}
 If Soar has been stopped due to a \textbf{halt}
 action, an \textbf{init-soar}
 command must be issued before Soar can be restarted with the \textbf{run}
 command. 
\subsubsection*{Default Aliases}
\begin{tabular}{|l|l|}
\hline 
 Alias  & Maps to  \\
 \hline 
 d  & run -d 1  \\
 \hline 
 e  & run -e 1  \\
 \hline 
 step  & run 1  \\
 \hline 
\end{tabular}

\subsection{\soarb{sp}}
\label{sp}
\index{sp}
Define a Soar production. 
 Status: Complete
\subsubsection*{Synopsis}
\begin{verbatim}
sp {production_body}
\end{verbatim}
\subsubsection*{Options}
\begin{tabular}{|l|l|}
\hline 
 production\_body  & A Soar production.  \\
 \hline 
\end{tabular}
\subsubsection*{Description}
 This command defines a new Soar production. rule is a single argument parsed by the Soar kernel, so it should be enclosed in curly braces to avoid being parsed by other scripting languages that might be in the same proces. The overall syntax of a rule is as follows: \begin{verbatim}
  name 
      ["documentation-string"] 
      [FLAG*]
      LHS
      -->
      RHS
\end{verbatim}
 The first element of a rule is its name. Conventions for names are given in the Soar Users Manual. If given, the documentation-string must be enclosed in double quotes. Optional flags define the type of rule and the form of support its right-hand side assertions will receive. The specific flags are listed in a separate section below. The LHS defines the left-hand side of the production and specifies the conditions under which the rule can be fired. Its syntax is given in detail in a subsequent section. The --$>$ symbol serves to separate the LHS and RHS portions. The RHS defines the right-hand side of the production and specifies the assertions to be made and the actions to be performed when the rule fires. The syntax of the allowable right-hand side actions are given in a later section. The Soar Users Manual gives an elaborate discussion of the design and coding of productions. Please see that reference for tutorial information about productions. 
  More complex productions can be formed by surrounding the rule with double quotes instead of curly braces. This enables variable and command result substitutions in productions. If another production with the same name already exists, it is excised, and the new production is loaded. 
 \textbf{RULE FLAGS}
\\ 
 The optional FLAGs are given below. Note that these switches are preceeded by a colon instead of a dash -- this is a Soar parser convention. \begin{verbatim}
:o-support      specifies that all the RHS actions are to be given
                o-support when the production fires 
\end{verbatim}
 \begin{verbatim}
:no-support     specifies that all the RHS actions are only to be given
                i-support when the production fires 
\end{verbatim}
 \begin{verbatim}
:default        specifies that this production is a default production 
                (this matters for excise -task and watch task) 
\end{verbatim}
 \begin{verbatim}
:chunk          specifies that this production is a chunk 
                (this matters for learn trace)
\end{verbatim}
\subsubsection*{Examples}
 There are many examples in the Soar Users Manual and the demos subdirectory. Here is a simple production to create a problem space. It comes from the critter-world demo (see the file critter.tcl): \begin{verbatim}
sp {critter*create*space*critter
   "Formulate the initial problem space"
   (state <state> ^superstate nil)
   -->
   (<state> ^name move-around ^problem-space <p1>)
   (<p1> ^name critter)}
\end{verbatim}
 The production above has the name critter*create*space*critter. It has a documentation string that is surrounded by double quotes. The LHS is (state $<$state$>$ \^{}superstate nil) and indicates that this rule will match whenever there is a state object that has the attribute-value pair \^{}superstate nil. The --$>$ arrow separates the left and right-hand sides. The RHS consists of two lines. The first asserts that the state object is to be augmented with the name move-around and a problem space should be created. The second line of the RHS indicates that this problem space should be named critter. 
  New for Soar 8, is right-hand-side dot notation. So this production could also be written: \begin{verbatim}
sp {critter*create*space*critter
   "Formulate the initial problem space"
   (state <state> ^superstate nil)
   -->
   (<state> ^name move-around ^problem-space.name critter)}
\end{verbatim}
 Here is a variant of the above example using double quotes instead of curly braces. Double quotes are needed in order to imbed the value of the Tcl variable soar\_agent\_name in the production. The value of this variable is used to name the problem-space created. \begin{verbatim}
sp "critter*create*space*critter
   (state <state> ^superstate nil)
  -->
  (<state> ^name move-around ^problem-space <p1>)
  (<p1> ^name $soar_agent_name)"
\end{verbatim}
 \textbf{ the rest of this may no longer apply, depending on parsing...}
\\ 
 The primary change in the rule is the last clause of the RHS. In that clause, the scripting (Tcl) variable soar\_agent\_name is expanded. If this rule is given in an interpreter which has the variable soar\_agent\_name set to fred, then the RHS would expand to the following before being sent to the Soar kernel to be parsed: \begin{verbatim}
 (<p1> ^name fred)
\end{verbatim}
 Please be aware that when using double quotes, both the dollar sign (variable expansion) and square brackets (command result substitution) could be interpreted by a scripting language such as Tcl, if loaded into the process that is running Soar. If these characters (\$, [, and ]) are to be passed to the Soar production parser, they must be escaped (using a backslash) to avoid interpretation by the scripting language. 
\subsubsection*{See Also}
 excise learn watch
\subsubsection*{Structured Output:}
\paragraph*{On Success}
\begin{verbatim}
<result output="raw">true</result>
\end{verbatim}
\subsubsection*{Error Values:}
\paragraph*{During Parsing}
 kTooFewArgs, kTooManyArgs, kInvalidProduction
\paragraph*{During Execution}
 kAgentRequired, kgSKIError

\subsection{\soarb{stop-soar}}
\label{stop-soar}
\index{stop-soar}
Pause Soar. 
 Status: Complete\\ 
Reason for stopping currently ignored, not sure what this is for/why this is here.--Jonathan 13:59, 4 Feb 2005 (EST) 
\subsubsection*{Synopsis}
\begin{verbatim}
stop-soar [-s] [reason string]
\end{verbatim}
\subsubsection*{Options}
\begin{tabular}{|l|l|}
\hline 
 -s, --self  & Stop only the soar agent where the command is issued. All other agents continue running as previously specified.  \\
 \hline 
 reason\_string  & An optional string which will be printed when Soar is stopped, to indicate why it was stopped. If left blank, no message will be printed when Soar is stopped.  \\
 \hline 
\end{tabular}
\subsubsection*{Description}
 The \textbf{stop-soar}
 command stops any running Soar agents. It sets a flag in the Soar kernel so that Soar will stop running at a ``safe'' point and return control to the user. This command is usually not issued at the command line prompt - a more common use of this command would be, for instance, as a side-effect of pressing a button on a Graphical User Interface (GUI). 
\subsubsection*{Default Aliases}
\begin{tabular}{|l|l|}
\hline 
 Alias  & Maps to  \\
 \hline 
 stop  & stop-soar  \\
 \hline 
 interrupt  & stop-soar  \\
 \hline 
\end{tabular}
\subsubsection*{See Also}
\hyperref[run]{run} \subsubsection*{Warnings}
 If the graphical interface doesn't periodically do an ``update'' of flush the pending I/O, then it may not be possible to interrupt a Soar agent from the command line. 


\section{Examining Memory}
\label{MEMORY}

This section describes the commands used to inspect production memory,
working memory, and preference memory; to see what productions will 
match and fire in the next Propose or Apply phase;  and to examine the 
goal dependency set.  These commands are particularly useful when
running or debugging Soar, as they let users see what Soar is ``thinking.''
The specific commands described in this section are:

\paragraph{Summary}
\begin{quote}
\begin{description}
\item[default-wme-depth] - Set the level of detail used to print WMEs.
\item[gds-print] - Print the WMEs in the goal dependency set for each goal.
\item[internal-symbols] - Print information about the Soar symbol table.
\item[matches] - Print information about the match set and partial matches.
\item[memories] - Print memory usage for production matches.
\item[preferences] - Examine items in preference memory.
\item[print] - Print items in working memory or production memory.
\item[production-find] - Find productions that contain a given pattern.
%\item[wmes] - An alias for the print command; prints items in working memory.
\end{description}
\end{quote}

Of these commands, \textbf{print} is the most often used (and the most
complex) followed by \textbf{matches} and \textbf{memories}.  \textbf{preferences}
is used to examine which candidate operators have been proposed.
\textbf{production-find} is especially useful when the number of
productions loaded is high.  \textbf{gds-print}
is useful for examining the goal dependecy set when subgoals seem to
be disappearing unexpectedly.  \textbf{default-wme-depth} is related to the \textbf{print} command.
\textbf{internal-symbols} is not often used but is helpful when debugging Soar extensions or
trying to locate memory leaks.

\subsection{\soarb{default-wme-depth}}
\label{default-wme-depth}
\index{default-wme-depth}
Set the level of detail used to print WME\^a��s. 
\subsubsection*{Synopsis}
\begin{verbatim}
default-wme-depth [depth]
\end{verbatim}
\subsubsection*{Options}
\begin{tabular}{|l|l|}
\hline
\soar{ depth } & A non-negative integer.  \\
\hline
\end{tabular}
\subsubsection*{Description}
 The \textbf{default-wme-depth}
 command reflects the default depth used when working memory elements are printed (using the \textbf{print}
 command or \textbf{wmes}
 alias). The default value is 1. When the command is issued with no arguments, \textbf{default-wme-depth}
 returns the current value of the default depth. When followed by an integer value, \textbf{default-wme-depth}
 sets the default depth to the specified value. This default depth can be overridden on any particular call to the \textbf{print}
 or \textbf{wmes}
 command by explicitly using the \textbf{--depth}
 flag, e.g.,\textbf{print --depth 10 \emph{args}
}
. 
 By default, the \textbf{print}
 command prints \emph{objects}
 in working memory, not just the individual working memory element. To limit the output to individual working memory elements, the \textbf{--internal}
 flag must also be specified in the \textbf{print}
 command. Thus when the print depth is \textbf{0}
, by default Soar prints the entire object, which is the same behavior as when the print depth is \textbf{1}
. But if \textbf{--internal}
 is also specified, then a depth of \textbf{0}
 prints just the individual WME, while a depth of \textbf{1}
 prints all WMEs which share that same identifier. This is true when printing timetags, identifiers or WME patterns. 
 When the depth is greater than \textbf{1}
, the identifier links from the specified WME's will be followed, so that additional substructure is printed. For example, a depth of \textbf{2}
 means that the object specified by the identifier, wme-pattern, or timetag will be printed, along with all other objects whose identifiers appear as values of the first object. This may result in multiple copies of the same object being printed out. If \textbf{--internal}
 is also specified, then individuals WMEs and their timetags will be printed instead of the full objects. 
\subsubsection*{Default Aliases}
\begin{tabular}{|l|l|}
\hline
\soar{ Alias } & Maps to  \\
\hline
\soar{ set-default-depth } & default-wme-depth  \\
\hline
\end{tabular}
\subsubsection*{See Also}
\hyperref[print]{print} 
\subsection{\soarb{gds-print}}
\label{gds-print}
\index{gds-print}
Print the WMEs in the goal dependency set for each goal. 
\subsubsection*{Synopsis}
\begin{verbatim}
gds-print
\end{verbatim}
\subsubsection*{Options}
 No options. 
\subsubsection*{Description}
 The Goal Dependency Set (GDS) is described in an appendix of the Soar manual. This command is a debugging command for examining the GDS for each goal in the stack. First it steps through all the working memory elements in the rete, looking for any that are included in \emph{any}
 goal dependency set, and prints each one. Then it also lists each goal in the stack and prints the wmes in the goal dependency set for that particular goal. This command is useful when trying to determine why subgoals are disappearing unexpectedly: often something has changed in the goal dependency set, causing a subgoal to be regenerated prior to producing a result. 
\subsubsection*{Warnings}
 gds-print is horribly inefficient and should not generally be used except when something is going wrong and you need to examine the Goal Dependency Set. 
\subsubsection*{Default Aliases}
\begin{tabular}{|l|l|}
\hline
\soar{ Alias } & Maps to  \\
\hline
\soar{ gds\_print } & gds-print  \\
\hline
\end{tabular}
 Categories: Command Line Interface

\subsection{\soarb{internal-symbols}}
\label{internal-symbols}
\index{internal-symbols}
Print information about the Soar symbol table. 
 Priority: 4; Status: Incomplete, EvilBackDoor\\ 
Result generated by kernel.--Jonathan 16:16, 23 Feb 2005 (EST) 
\subsubsection*{Synopsis}
\begin{verbatim}
internal-symbols
\end{verbatim}
\subsubsection*{Options}
 No options. 
\subsubsection*{Description}
\subsubsection*{Structured Output:}
\paragraph*{On Success}
 Returns the output from the kernel in a string message. 
\paragraph*{Notes}
\subsubsection*{Error Values:}
 No errors. 
\paragraph*{During Parsing}
\paragraph*{During Execution}

\subsection{\soarb{matches}}
\label{matches}
\index{matches}
Prints information about partial matches and the match set. 
 Priority: 1�; Status: Incomplete, EvilBackDoor\\ 
Result generated by kernel.--Jonathan 12:18, 7 Feb 2005 (EST) 
\subsubsection*{Synopsis}
\begin{verbatim}
matches [-nc0t1w2] production name
matches -[a|r] [-nc0t1w2]
\end{verbatim}
\subsubsection*{Options}
\begin{tabular}{|l|l|}
\hline 
production\_name & Print partial match information for the named production.  \\
 \hline 
 -0, -n, --names, -c, --count  & For the match set, print only the names of the productions that are about to fire or retract (the default). If printing partial matches for a production, just list the partial match counts.  \\
 \hline 
 -1, -t, --timetags  & Also print the timetags of the wmes at the first failing condition  \\
 \hline 
 -2, -w, --wmes  & Also print the full wmes, not just the timetags, at the first failing condition.  \\
 \hline 
 -a, --assertions  & List only productions about to fire.  \\
 \hline 
 -r, --retractions  & List only productions about to retract.  \\
 \hline 
\end{tabular}
\subsubsection*{Description}
 The matches command prints a list of productions that have instantiations in the match set, i.e., those productions that will retract or fire in the next Propose or Apply phase. It also will preint partial match information for a single, named production. 
\subsection*{Printing the match set}
 When printing the match set (i.e., no production name is specified), the default action prints only the names of the productions which are about to fire or retract. If there are multiple instantiations of a production, the total number of instantiations of that production is printed after the production name, unless \textbf{--timetags|1}
 or \textbf{--wmes|2}
 are specified, in which case each instantiation is printed on a separate line. 
 When printing the match set, the \textbf{--assertions}
 and \textbf{--retractions}
 arguments may be specified to restrict the output to print only the assertions or retractions. 
\subsection*{Printing partial matches for productions}
. The pointer \textbf{$>$$>$$>$$>$}
 before a condition indicates that this is the first condition that failed to match. 
 When printing partial matches, the default action is to print only the counts of the number of WME's that match, and is a handy tool for determining which condition failed to match for a production that you thought should have fired. At levels \textbf{1}
 and \textbf{2}
 (or \textbf{--timetags}
 and \textbf{--wmes}
\subsection*{Notes}
 In Soar 8, the execution cycle (decision cycle) is input, propose, decide, apply output; it no longer stops for user input after the decision phase when running by decision cycles (\textbf{run -d 1}
). If a user wishes to print the match set immediately after the decision phase and before the apply phase, then the user must run Soar by \emph{phases}
 (\textbf{run -p 1}
). 
\subsubsection*{Examples}
 (fix this) - output? This example prints the productions which are about to fire and the wmes that match the productions on their left-hand sides: \begin{verbatim}
matches --assertions --wmes
\end{verbatim}
 This example prints the wme timetags for a single production. \begin{verbatim}
matches -t my*first*production</code.
\end{verbatim}
\subsubsection*{See Also}
 monitor

\subsection{\soarb{memories}}
\label{memories}
\index{memories}
Print memory usage for partial matches. 
 Status: Complete
\subsubsection*{Synopsis}
\begin{verbatim}
memories [-cdju] [\emph{n}
]
memories production_name 
\end{verbatim}
\subsubsection*{Options}
\begin{tabular}{|l|l|}
\hline 
 -c, --chunks  & Print memory usage of chunks.  \\
 \hline 
 -d, --default  & Print memory usage of default productions.  \\
 \hline 
 -j, --justifications  & Print memory usage of justifications.  \\
 \hline 
 -u, --user  & Print memory usage of user-defined productions.  \\
 \hline 
production\_name & Print memory usage for a specific production.  \\
 \hline 
\emph{n}
 & Number of productions to print, sorted by those that use the most memory.  \\
 \hline 
\end{tabular}
\subsubsection*{Description}
 is given, only \emph{n}
 productions will be printed: the \emph{n}
 productions that use the most memory. Output may be restricted to print memory usage for particular types of productions using the command options. 
 Memory usage is recorded according to the tokens that are allocated in the rete network for the given production(s). This number is a function of the number of elements in working memory that match each production. Therefore, this command will not provide useful information at the beginning of a Soar run (when working memory is empty) and should be called in the middle (or at the end) of a Soar run. 
 As a rule of thumb, numbers less than 100 mean that the production is using a small amount of memory, numbers above 1000 mean that the production is using a large amount of memory, and numbers above 10,000 mean that the production is using a \emph{very}
 large amount of memory. 
\subsubsection*{Examples}
 To show how to use the command in context, do this: \begin{verbatim}
command --option arg
\end{verbatim}
 and possibly explain the results. 
\subsubsection*{See Also}
 matches

\documentclass[10pt]{article}
\usepackage{fullpage, graphicx, url}
\setlength{\parskip}{1ex}
\setlength{\parindent}{0ex}
\title{Preferences - Soar Wiki}
\begin{document}
\section*{Preferences}
\subsubsection*{From Soar Wiki}


 This is part of the Soar Command Line Interface. 
\section*{ Name }


 \textbf{preferences}
 - Examine details about the preferences that support the specified \emph{id}
 and \emph{attribute}
. 


 Priority: 2; Status: Incomplete, EvilBackDoor\\ 
Result generated by kernel.--Jonathan 15:45, 18 Feb 2005 (EST) 
\section*{ Synopsis }
\begin{verbatim}
preferences [-0123nNtw] [id] [[^]attribute]

\end{verbatim}
\section*{ Options }


\begin{tabular}{|c|c|}
\hline 
 -0, -n, --none  & Print just the preferences themselves  \\
 \hline 
 -1, -N, --names  & Print the preferences and the names of the productions that generated them  \\
 \hline 
 -2, -t, --timetags  & Print the information for the --names option above plus the timetags of the wmes matched by the indicated productions  \\
 \hline 
 -3, -w, --wmes  & Print the information for the --timetags option above plus the entire wme.  \\
 \hline 
id & Must be an existing Soar object identifier.  \\
 \hline 
attribute & Must be an existing \emph{\^{}attribute}
 of the specified identifier.  \\
 \hline 

\end{tabular}



 \\ 

\section*{ Description }


 This command prints all the preferences for the given object id and attribute. If \emph{id}
 and \emph{attribute}
 are not specified, they default to the current state and the current operator. The '\^{}' is optional when specifying the attribute. The optional arguments indicates the level of detail to print about each preference. 
\section*{ Examples }


 This example prints the preferences on the (S1 \^{}operator) and the production names which created the preferences: \begin{verbatim}
preferences S1 operator --names

\end{verbatim}



 if the current state is S1, then the above syntax is equivalent to: \begin{verbatim}
 preferences -n

\end{verbatim}

\section*{ See Also }
\section*{ Structured Output }


 preferences returns formatted output in a string, this needs to be re-done.--Jonathan 15:44, 18 Feb 2005 (EST) 
\subsection*{ On Success }
\begin{verbatim}
<result>
  <arg param="message" type="string">output_string</arg>
</result>

\end{verbatim}
\subsection*{ Notes }
\section*{ Error Values }
\subsection*{ During Parsing }


 kUnrecognizedOption, kGetOptError, kTooManyArgs
\subsection*{ During Execution }


 kAgentRequired, kKernelRequired, kgSKIError Retrieved from ``\url{http://winter.eecs.umich.edu/soarwiki/Preferences}``

\end{document}

\subsection{\soarb{print}}
\label{print}
\index{print}
Print items in working memory or production memory. 
 1 Incomplete EvilBackDoor ResultByKernel
\subsubsection*{Synopsis}
\begin{verbatim}
print [-fFin] production_name
print -[a|c|D|j|u][fFin]
print [-i] [-d <depth>] \emph{identifier}
|\emph{timetag}
|\emph{pattern}
print -s[oS]
\end{verbatim}
\subsubsection*{Options}
\subsection*{Printing items in production memory}
\begin{tabular}{|l|l|}
\hline 
 -a, --all  & print the names of all productions currently loaded  \\
 \hline 
 -c, --chunks  & print the names of all chunks currently loaded  \\
 \hline 
 -D, --defaults  & print the names of all default productions currently loaded  \\
 \hline 
 -f, --full  & When printing productions, print the whole production. This is the default when printing a named production.  \\
 \hline 
 -F, --filename  & also prints the name of the file that contains the production.  \\
 \hline 
 -i, --internal  & items should be printed in their internal form. For productions, this means leaving conditions in their reordered (rete net) form.  \\
 \hline 
 -j, --justifications  & print the names of all justifications currently loaded.  \\
 \hline 
 -n, --name  & When printing productions, print only the name and not the whole production. This is the default when printing any category of productions, as opposed to a named production.  \\
 \hline 
 -u, --user  & print the names of all user productions currently loaded  \\
 \hline 
production\_name & print the production named production-name \\
 \hline 
\end{tabular}
\subsection*{Printing items in working memory}
\begin{tabular}{|l|l|}
\hline 
 -d, --depth \emph{n}
 & This option overrides the default printing depth (see the default-wme-depth command for more detail).  \\
 \hline 
 -i, --internal  & items should be printed in their internal form. For working memory, this means printing the individual elements with their timetags, rather than the objects.  \\
 \hline 
\emph{identifier}
 & print the object \emph{identifier}
. \emph{identifier}
 must be a valid Soar symbol such as \textbf{S1 }
 \hline 
\emph{pattern}
 & print the object whose working memory elements matching the given pattern. See Description for more information on printing objects matching a specific pattern.  \\
 \hline 
\emph{timetag}
 & print the object in working memory with the given \emph{timetag}
 \hline 
\end{tabular}
\subsection*{Printing the current subgoal stack}
\begin{tabular}{|l|l|}
\hline 
 -s, --stack  & Specifies that the Soar goal stack should be printed. By default this includes both states and operators.  \\
 \hline 
 -o, --operators  & When printing the stack, print only \textbf{operators}
.  \\
 \hline 
 -S, --states  & When printing the stack, print only \textbf{states}
.  \\
 \hline 
\end{tabular}
\subsubsection*{Description}
 The \textbf{print}
 command is used to print items from production memory or working memory. It can take several kinds of arguments. When printing items from working memory, the Soar objects are printed unless the --internal flag is used, in which case the wmes themselves are printed. \begin{verbatim}
(\emph{identifier}
 ^\emph{attribute value}
 [+])
\end{verbatim}
 The pattern is surrounded by parentheses. The \emph{identifier}
, \emph{attribute}
, and \emph{value}
 must be valid Soar symbols or the wildcard symbol * which matches all occurences. The optional \textbf{+}
 symbol restricts pattern matches to acceptable preferences. 
\subsubsection*{Examples}
 Print the working memory elements (and their timetags) which have the identifier s1 as object and v2 as value: \begin{verbatim}
print --internal (s1 ^* v2)
\end{verbatim}
 Print the Soar stack which includes states and operators: \begin{verbatim}
print --stack
\end{verbatim}
 Print the named production in its RETE form: \begin{verbatim}
print -if prodname
\end{verbatim}
 Print the names of all user productions currently loaded: \begin{verbatim}
print -u
\end{verbatim}
\subsubsection*{Default Aliases}
\begin{tabular}{|l|l|}
\hline 
 Alias  & Maps to  \\
 \hline 
 p  & print  \\
 \hline 
 wmes  & print -i  \\
 \hline 
\end{tabular}
\subsubsection*{See Also}
\hyperref[default-wme-depth]{default-wme-depth} \hyperref[predefined-aliases]{predefined-aliases} 
\subsection{\soarb{production-find}}
\label{production-find}
\index{production-find}
\subsubsection*{Synopsis}
production-find [-lrs[n|c]] \emph{pattern}
\end{verbatim}
\subsubsection*{Options}
\hline
\soar{\soar{ -c, --chunks }} & Look \emph{only}
 for chunks that match the pattern.  \\
\hline
\soar{\soar{ -l, --lhs }} & Match pattern only against the conditions (left-hand side) of productions (default).  \\
\hline
\soar{\soar{ -n, --nochunks }} &\emph{Disregard}
 chunks when looking for the pattern.  \\
\hline
\soar{\soar{ -r, --rhs }} & Match pattern against the actions (right-hand side) of productions.  \\
\hline
\soar{\soar{ -s, --show-bindings }} & Show the bindings associated with a wildcard pattern.  \\
\hline
\soar{\soar{ pattern }} & Any pattern that can appear in productions.  \\
\hline
\end{tabular}
\subsubsection*{Description}
 The production-find command is used to find productions in production memory that include conditions or actions that match a given \emph{pattern}
. The pattern given specifies one or more condition elements on the left hand side of productions (or negated conditions), or one or more actions on the right-hand side of productions. Any pattern that can appear in productions can be used in this command. In addition, the asterisk symbol, *, can be used as a wildcard for an attribute or value. It is important to note that the whole pattern, including the parenthesis, must be enclosed in curly braces for it to be parsed properly. 
 The variable names used in a call to production-find do not have to match the variable names used in the productions being retrieved. 
 The production-find command can also be restricted to apply to only certain types of productions, or to look only at the conditions or only at the actions of productions by using the flags. 
\subsubsection*{Examples}
 Find productions that test that some object \emph{gumby}
 has an attribute \emph{alive}
 with value \emph{t}
. In addition, limit the rules to only those that test an operator named \emph{foo}
production-find (<state> ^gumby <gv> ^operator.name foo)(<gv> ^alive t)
\end{verbatim}
 Note that in the above command, $<$state$>$ does not have to match the exact variable name used in the production. 
 Find productions that propose the operator \emph{foo}
production-find --rhs (<x> ^operator <op> +)(<op> ^name foo)
\end{verbatim}
production-find --chunks (<x> ^pokey *)
\end{verbatim}
source demos/water-jug/water-jug.soar
production-find (<s> ^name *)(<j> ^volume *)
production-find (<s> ^name *)(<j> ^volume 3)
production-find --rhs (<j> ^* <volume>)
\end{verbatim}
\subsubsection*{See Also}
\hyperref[sp]{sp} 

% ****************************************************************************
% ----------------------------------------------------------------------------
\section{Configuring Trace Information and Debugging}
\label{DEBUG}

This section describes the commands used primarily for debugging or
to configure the trace output printed by Soar as it runs.  Users may:
specify the content of the runtime trace output; ask that
they be alerted when specific productions fire and retract; 
or request details on Soar's performance.

The specific commands described in this section are:


\paragraph{Summary}
\begin{quote}
\begin{description}
\item[chunk-name-format] - Specify format of the name to use for new chunks.
\item[firing-counts] - Print the number of times productions have fired.
%\item[format-watch] - Change the trace output that's printed as Soar runs.
%\item[interrupt] - Add \& remove pre-firing interrupts on specific productions.
%\item[monitor] - Manage attachment of Tcl scripts to Soar events.
\item[pwatch] - Trace firings and retractions of specific productions.
\item[stats] - Print information on Soar's runtime statistics.
\item[verbose] -  Control detailed information printed as Soar runs.
\item[warnings] - Toggle whether or not warnings are printed.
\item[watch] - Control the information printed as Soar runs.
\item[watch-wmes] -  Print information about wmes that match a certain pattern as they are added and removed
\end{description}
\end{quote}

Of these commands, \soar{watch} is the most often used (and the most 
complex). \soar{pwatch} is related to \soar{watch}, but applies only 
to specific, named productions. \soar{firing-counts} and \soar{stats} 
are useful for understanding how much work Soar is doing. \soar{chunk-name-format} is less-frequently
used, but allows for detailed control of Soar's chunk naming.

\subsection{\soarb{chunk-name-format}}
\label{chunk-name-format}
\index{chunk-name-format}
Specify format of the name to use for new chunks. 
 Priority: 4; Status: Complete, EvilBackDoor
\subsubsection*{Synopsis}
\begin{verbatim}
chunk-name-format [-sl] -p [<prefix>]
chunk-name-format [-sl] -c [<count>]
\end{verbatim}
\subsubsection*{Options}
\begin{tabular}{|l|l|}
\hline 
 -s, --short  & Use the short format for naming chunks  \\
 \hline 
 -l, --long  & Use the long format for naming chunks (default)  \\
 \hline 
 -p, --prefix [$<$prefix$>$]  & If $<$prefix$>$ is given, use $<$prefix$>$ as the prefix for naming chunks. Otherwise, return the current \emph{prefix}
. (defaults to ``\textbf{chunk}
``)  \\
 \hline 
 -c, --count [$<$count$>$]  & If $<$count$>$ is given, set the chunk counter for naming chunks to $<$count$>$. Otherwise, return the current value of the chunk counter.  \\
 \hline 
\end{tabular}
\subsubsection*{Description}
 The short format for naming newly-created chunks is: 
 \emph{prefixChunknum}
 The long (default) format for naming chunks is: 
 \emph{prefix-Chunknum}
*d\emph{dc}
*\emph{impassetype}
*\emph{dcChunknum}
 where: 
 \emph{prefix}
 is a user-definable prefix string; \emph{prefix}
 defaults to ``\textbf{chunk}
`` when unspecified by the user. It many not contain the character *, 
 \emph{Chunknum}
 is $<$count$>$ for the first chunk created, $<$count$>$+1 for the second chunk created, etc. 
 \emph{dc}
 is the number of the decision cycle in which the chunk was formed, 
 \emph{impassetype}
 is one of \textbf{[tie | conflict | cfailure | snochange | opnochange]}
, 
 \emph{dcChunknum}
 is the number of the chunk within that specific decision cycle. 

\subsection{\soarb{firing-counts}}
\label{firing-counts}
\index{firing-counts}
Print the number of times each production has fired. 
 Status: Complete
\subsubsection*{Synopsis}
\begin{verbatim}
firing-counts [\emph{n}
]
firing-counts \emph{production_names}
\end{verbatim}
\subsubsection*{Options}
 If given, an option can take one of two forms -- an integer or a list of production names: 
\begin{tabular}{|l|l|}
\hline 
\emph{n}
 & List the top \emph{n}
 productions. If \emph{n}
 is 0, only the productions which haven't fired are listed  \\
 \hline 
 production\_name  & For each production in production\_names, print how many times the production has fired  \\
 \hline 
\end{tabular}
\subsubsection*{Description}
, is given, only the top \emph{n}
 productions are listed. If \textbf{n}
 is zero (0), only the productions that haven't fired at all are listed. If one or more production names are given as arguments, only firing counts for these productions are printed. 
 Note that firing counts are reset by a call to \textbf{init-soar}
. 
\subsubsection*{Examples}
 This example prints the 10 productions which have fired the most times along with their firing counts: \begin{verbatim}
firing-counts 10
\end{verbatim}
 This example prints the firing counts of productions my*first*production and my*second*production: \begin{verbatim}
firing-counts my*first*production my*second*production
\end{verbatim}
\subsubsection*{Warnings}
 Firing-counts are reset to zero after an init-soar. \\ 
 NB: This command is slow, because the sorting takes time O(n*log n) 
\subsubsection*{Default Aliases}
\begin{tabular}{|l|l|}
\hline 
 Alias  & Maps to  \\
 \hline 
 fc  & firing-counts  \\
 \hline 
\end{tabular}
\subsubsection*{See Also}
 init-soar

\subsection{\soarb{pwatch}}
\label{pwatch}
\index{pwatch}
Trace firings and retractions of specific productions. 
 Complete EvilBackDoor
\subsubsection*{Synopsis}
\begin{verbatim}
pwatch [-d|e] [production name]
\end{verbatim}
\subsubsection*{Options}
\begin{tabular}{|l|l|}
\hline 
 -d, --disable, --off  & Turn production watching off for the specified production. If no production is specified, turn production watching off for all productions.  \\
 \hline 
 -e, --enable, --on  & Turn production watching on for the specified production. The use of this flag is optional, so this is pwatch's default behavior. If no production is specified, all productions currently being watched are listed.  \\
 \hline 
production name & The name of the production to watch.  \\
 \hline 
\end{tabular}
\subsubsection*{Description}
 The \textbf{pwatch}
 command enables and disables the tracing of the firings and retractions of individual productions. This is a companion command to \textbf{watch}
, which cannot specify individual productions by name. 
 With no arguments, \textbf{pwatch}
 lists the productions currently being traced. With one production-name argument, \textbf{pwatch}
 enables tracing the production; \textbf{--enable}
 can be explicitly stated, but it is the default action. 
 If \textbf{--disable}
 is specified followed by a production-name, tracing is turned off for the production. When no production-name is specified, \textbf{pwatch --enable}
 lists all productions currently being traced, and \textbf{pwatch --disable}
 disables tracing of all productions. 
 Note that \textbf{pwatch}
 now only takes one production per command. Use multiple times to watch multiple functions. 
\subsubsection*{See Also}
\hyperref[watch]{watch} 
\documentclass[10pt]{article}
\usepackage{fullpage, graphicx, url}
\title{Stats - Soar Wiki}
\begin{document}
\section*{Stats}
\subsubsection*{From Soar Wiki}


 This is part of the Soar Command Line Interface. 
\section*{ Name }


 \textbf{stats}
 - Print information on Soar\^a��s runtime statistics. 


 Priority: 1; Status: Incomplete\\ 
Memory pool and rete stats not implemented with structured output.--Jonathan 16:04, 8 Mar 2005 (EST) \\ 
stats -r (raw output) looks like trash output.--Jonathan 16:01, 8 Mar 2005 (EST) 
\section*{ Synopsis }
\subsection*{ Structured Output }
\begin{verbatim}
stats

\end{verbatim}
\subsection*{ Raw Output }
\begin{verbatim}
stats [-s|-m|-r]

\end{verbatim}
\section*{ Options }


\begin{tabular}{|p{1in}|p{5in}|}
\hline 
 -m, --memory  & report usage for Soar's memory pools  \\
 \hline 
 -r, --rete  & report statistics about the rete structure  \\
 \hline 
 -s, --system  & report the system (agent) statistics. This is the default if no args are specified.  \\
 \hline 

\end{tabular}



 \\ 

\section*{ Description }


 This command prints Soar internal statistics. The module indicates the component of interest. If specified, module must be one of --system, --memory, or --rete. All statistics are listed for that module. 
\section*{ Examples }


 This prints all statistics in the --system module: \begin{verbatim}
stats --system

\end{verbatim}

\section*{ See Also }


 timers


 \\ 

\section*{ A Note on Timers }


 The current implementation of Soar uses a number of timers to provide time-based statistics for use in the stats command calculations. These timers are: \\ 
 total CPU time total kernel time phase kernel time (per phase) phase callbacks time (per phase) input function time output function time \\ 
 Total CPU time is calculated from the time a decision cycle (or number of decision cycles) is initiated until stopped. Kernel time is the time spent in core Soar functions. In this case, kernel time is defined as the all functions other than the execution of callbacks and the input and output functions. The total kernel timer is only stopped for these functions. The phase timers (for the kernel and callbacks) track the execution time for individual phases of the decision cycle (i.e., input phase, preference phase, working memory phase, output phase, and decision phase). Because there is overhead associated with turning these timers on and off, the actual kernel time will always be greater than the derived kernel time (i.e., the sum of all the phase kernel timers). Similarly, the total CPU time will always be greater than the derived total (the sum of the other timers) because the overhead of turning these timers on and off is included in the total CPU time. In general, the times reported by the single timers should always be greater than than the corresponding derived time. Additionally, as execution time increases, the difference between these two values will also increase. For those concerned about the performance cost of the timers, all the run time timing calculations can be compiled out of the code by defining NO\_TIMING\_STUFF (in soarkernel.h) before compilation. 


 \\ 

\section*{ Structured Output }


 The following arg parameters are returned: \begin{verbatim}
kParamStatsProductionCountDefault, kTypeInt
kParamStatsProductionCountUser, kTypeInt
kParamStatsProductionCountChunk, kTypeInt
kParamStatsProductionCountJustification, kTypeInt
kParamStatsCycleCountDecision, kTypeInt
kParamStatsCycleCountElaboration, kTypeInt
kParamStatsProductionFiringCount, kTypeInt
kParamStatsWmeCountAddition, kTypeInt
kParamStatsWmeCountRemoval, kTypeInt
kParamStatsWmeCount, kTypeInt
kParamStatsWmeCountAverage, kTypeDouble
kParamStatsWmeCountMax, kTypeInt
kParamStatsKernelTimeTotal, kTypeDouble
kParamStatsMatchTimeInputPhase, kTypeDouble
kParamStatsMatchTimeDetermineLevelPhase, kTypeDouble
kParamStatsMatchTimePreferencePhase, kTypeDouble
kParamStatsMatchTimeWorkingMemoryPhase, kTypeDouble
kParamStatsMatchTimeOutputPhase, kTypeDouble
kParamStatsMatchTimeDecisionPhase, kTypeDouble
kParamStatsOwnershipTimeInputPhase, kTypeDouble
kParamStatsOwnershipTimeDetermineLevelPhase, kTypeDouble
kParamStatsOwnershipTimePreferencePhase, kTypeDouble
kParamStatsOwnershipTimeWorkingMemoryPhase, kTypeDouble
kParamStatsOwnershipTimeOutputPhase, kTypeDouble
kParamStatsOwnershipTimeDecisionPhase, kTypeDouble
kParamStatsChunkingTimeInputPhase, kTypeDouble
kParamStatsChunkingTimeDetermineLevelPhase, kTypeDouble
kParamStatsChunkingTimePreferencePhase, kTypeDouble
kParamStatsChunkingTimeWorkingMemoryPhase, kTypeDouble
kParamStatsChunkingTimeOutputPhase, kTypeDouble
kParamStatsChunkingTimeDecisionPhase, kTypeDouble
kParamStatsMemoryUsageMiscellaneous, kTypeInt
kParamStatsMemoryUsageHash, kTypeInt
kParamStatsMemoryUsageString, kTypeInt
kParamStatsMemoryUsagePool, kTypeInt
kParamStatsMemoryUsageStatsOverhead, kTypeInt

\end{verbatim}

\section*{ Error Values }
\subsection*{ During Parsing }


 kTooManyArgs, kUnrecognizedOption, kGetOptError
\subsection*{ During Execution }


 kAgentRequired, kgSKIError

\end{document}

\subsection{\soarb{verbose}}
\label{verbose}
\index{verbose}
Control detailed information printed as Soar runs. 
\subsubsection*{Synopsis}
verbose [-ed]
\end{verbatim}
\subsubsection*{Options}
\hline
\soar{\soar{\soar{ -d, --disable, --off }}} & Turn verbosity off.  \\
\hline
\soar{\soar{\soar{ -e, --enable, --on }}} & Turn verbosity on.  \\
\hline
\end{tabular}
\subsubsection*{Description}
 Invoke with no arguments to query. (fix this) - More about what this command does? 

\subsection{\soarb{warnings}}
\label{warnings}
\index{warnings}
 Complete EvilBackDoor
\subsubsection*{Synopsis}
\begin{verbatim}
warnings -[e|d]
\end{verbatim}
\subsubsection*{Options}
\begin{tabular}{|l|l|}
\hline 
 -e, --enable, --on  & Default. Print all warning messages from the kernel.  \\
 \hline 
 -d, --disable, --off  & Disable all, except most critical, warning messages.  \\
 \hline 
\end{tabular}
\subsubsection*{Description}
 Enables and disables the printing of warning messages. If an argument is specified, then the warnings are set to that state. If no argument is given, then the current warnings status is printed. At startup, warnings are initially enabled. If warnings are disabled using this command, then some warnings may still be printed, since some are considered too important to ignore. 
 The warnings that are printed apply to the syntax of the productions, to notify the user when they are not in the correct syntax. When a lefthand side error is discovered (such as conditions that are not linked to a common state or impasse object), the production is generally loaded into production memory anyway, although this production may never match or may seriously slow down the matching process. In this case, a warning would be printed only if \textbf{warnings}
 were \textbf{--on}
. Righthand side errors, such as preferences that are not linked to the state, usually result in the production not being loaded, and a warning regardless of the \textbf{warnings}
 setting. 
\subsubsection*{Examples}
\subsubsection*{See Also}

\documentclass[10pt]{article}
\usepackage{fullpage, graphicx, url}
\setlength{\parskip}{1ex}
\setlength{\parindent}{0ex}
\title{Watch - Soar Wiki}
\begin{document}
\section*{Watch}
\subsubsection*{From Soar Wiki}


 This is part of the Soar Command Line Interface. 
\section*{ Name }


 \textbf{watch}
 - Control the run-time tracing of Soar. 


 Status: Complete, EvilBackDoor
\section*{ Synopsis }
\begin{verbatim}
watch
watch [--level] [0|1|2|3|4|5]
watch -N
watch -[dpPwrDujcbi] [<remove>] -[n|t|f]
watch --learning [<print|noprint|fullprint>]

\end{verbatim}
\section*{ Options }


\begin{tabular}{|c|c|c|}
\hline 
\emph{Option Flag}
 &\emph{Argument to Option}
 &\emph{Description}
 \\
 \hline 
 -l, --level  & 0 to 5 (see \textbf{Watch Levels}
 below)  & This flag is optional but recommended. Set a specific watch level using an integer 0 to 5, this is an inclusive operation  \\
 \hline 
 -N, --none  & No argument  & Turns off all printing about Soar's internals, equivalent to --level 0  \\
 \hline 
 -d, --decisions  & remove (optional, see \textbf{Remove}
 below)  & Controls whether state and operator decisions are printed as they are made  \\
 \hline 
 -p, --phases  & remove (optional, see \textbf{Remove}
 below)  & Controls whether decisions cycle phase names are printed as Soar executes  \\
 \hline 
 -P, --productions  & remove (optional, see \textbf{Remove}
 below)  & Controls whether the names of productions are printed as they fire and retract, equivalent to -Dujc  \\
 \hline 
 -w, --wmes  & remove (optional, see \textbf{Remove}
 below)  & Controls the printing of working memory elements that are added and deleted as productions are fired and retracted  \\
 \hline 
 -r, --preferences  & remove (optional, see \textbf{Remove}
 below)  & Controls whether the preferences generated by the traced productions are printed when those productions fire or retract  \\
 \hline 
 -D, --default  & remove (optional, see \textbf{Remove}
 below)  & Control only default-productions as they fire and retract  \\
 \hline 
 -u, --user  & remove (optional, see \textbf{Remove}
 below)  & Control only user-productions as they fire and retract  \\
 \hline 
 -c, --chunks  & remove (optional, see \textbf{Remove}
 below)  & Control only chunks as they fire and retract  \\
 \hline 
 -j, --justifications  & remove (optional, see \textbf{Remove}
 below)  & Control only justifications as they fire and retract  \\
 \hline 
 -n, --nowmes  & No argument  & When watching productions, do not print any information about wmes as they are added or retracted  \\
 \hline 
 -t, --timetags  & No argument  & When watching productions, print only the timetags for wmes as they are added or retracted  \\
 \hline 
 -f, --fullwmes  & No argument  & When watching productions, print the full wmes as they are added or retracted  \\
 \hline 
 -b, --backtracing  & remove (optional, see \textbf{Remove}
 below)  & Controls the printing of backtracing information when a chunk or justification is created  \\
 \hline 
 -i, --indifferent-selection  & remove (optional, see \textbf{Remove}
 below)  & Controls the printing of the scores for tied operators in random indifferent selection mode  \\
 \hline 
 -L, --learning  & noprint, print, or fullprint (see \textbf{Learning}
 below)  & Controls the printing of chunks/justifications as they are created  \\
 \hline 

\end{tabular}

\subsection*{ Watch Levels }


 Use of the --level (-l) flag is optional but recommended. 

\begin{tabular}{|c|c|}
\hline 
 0  & watch nothing; equivalent to \^a��N  \\
 \hline 
 1  & watch decisions; equivalent to -d  \\
 \hline 
 2  & watch phases and decisions; equivalent to -dp  \\
 \hline 
 3  & watch productions, phases, and decisions; equivalent to -dpP  \\
 \hline 
 4  & watch wmes, productions, phases, and decisions; equivalent to -dpPw  \\
 \hline 
 5  & watch preferences, wmes, productions, phases, and decisions; equivalent to -dpPwr  \\
 \hline 

\end{tabular}




 \\ 
 It is important to note that watch level 0 turns off ALL watch options, including backtracing, indifferent selection and learning. However, the other watch levels do not change these settings. That is, if any of these settings is changed from its default, it will retain its new setting until it is either explicitly changed again or the watch level is set to 0. 
\subsection*{ Remove }


 The remove argument has a numeric alias; you can use 0 for remove. A mix of formats is acceptable, even in the same command line. 

\begin{tabular}{|c|c|c|}
\hline 
 remove  & 0  & Turn watching off only for the specified option  \\
 \hline 

\end{tabular}


\subsection*{ Learning }


 The learning options have numeric aliases; you can use 0 for noprint, 1 for print, and 2 for fullprint. A mix of formats is acceptable, even in the same command line. 

\begin{tabular}{|c|c|c|}
\hline 
 noprint  & 0  & Print nothing about new chunks or justifications (default)  \\
 \hline 
 print  & 1  & Print the names of new chunks and justifications when created  \\
 \hline 
 fullprint  & 2  & Print entire chunks and justifications when created  \\
 \hline 

\end{tabular}




 \\ 

\section*{ Description }


 The watch command controls run-time tracing of Soar. With no arguments, this command prints out the current watch status. The various levels are used to modify the current watch settings. Each level can be indicated with either a number or a series of flags as follows: \begin{verbatim}
0 or --none
1 or --decisions
2 or --decisions --phases
3 or --decisions --phases --productions
4 or --decisions --phases --productions --wmes
5 or --decisions --phases --productions --wmes --preferences

\end{verbatim}



 The numerical arguments \emph{inclusively}
 turn on all levels up to the number specified. To use numerical arguments to turn off a level, specify a number which is less than the level to be turned off. For instance, to turn off watching of productions, specify ``watch --level 2'' (or 1 or 0). Numerical arguments are provided for shorthand convenience. For more detailed control over the watch settings, the named arguments should be used. 


 For the named arguments, including the named argument turns on only that setting. To turn off a specific setting, follow the named argument with \emph{remove}
 or \emph{0}
. 


 The named argument --productions is shorthand for the four arguments --default, --user, --justifications, and --chunks. 


 The pwatch command is used to watch individual productions specified by name rather than watch a type of productions, such as --user. 
\section*{ Examples }


 The most common uses of watch are by using the numeric arguments which indicate watch levels. To turn off all printing of Soar internals, do any one of the following (not all possibilities listed): \begin{verbatim}
watch --level 0
watch -l 0
watch -N

\end{verbatim}



 Although the --level flag is optional, its use is recommended: \begin{verbatim}
watch --level 5 \emph{... OK}

watch 5         \emph{... OK, but try to avoid}


\end{verbatim}



 Be careful of where the level is on the command line, for example, if you want level 2 and preferences: \begin{verbatim}
watch -r -l 2 \emph{... Incorrect: -r flag ignored, level 2 parsed after it and overrides the setting}

watch -r 2    \emph{... Syntax error: 0 or remove expected as optional argument to -r}

watch -r -l 2 \emph{... Incorrect: -r flag ignored, level 2 parsed after it and overrides the setting}

watch 2 -r    \emph{... OK, but try to avoid}

watch -l 2 -r \emph{... OK}


\end{verbatim}



 To turn on printing of decisions, phases and productions, do any one of the following (not all possibilities listed): \begin{verbatim}
watch --level 3
watch -l 3
watch --decisions --phases --productions
watch -d -p -P

\end{verbatim}



 Individual options can be changed as well. To turn on printing of decisions and wmes, but not phases and productions, do any one of the following (not all possibilities listed): \begin{verbatim}
watch --level 1 --wmes
watch -l 1 -w
watch --decisions --wmes
watch -d --wmes
watch -w --decisions
watch -w -d

\end{verbatim}



 To turn on printing of decisions, productions and wmes, and turns phases off, do any one of the following (not all possibilities listed): \begin{verbatim}
watch --level 4 --phases remove
watch -l 4 -p remove
watch -l 4 -p 0
watch -d -P -w -p remove

\end{verbatim}



 To watch the firing and retraction of decisions and \emph{only}
 user productions, do any one of the following (not all possibilities listed): \begin{verbatim}
watch -l 1 -u
watch -d -u

\end{verbatim}



 To watch decisions, phases and all productions \emph{except}
 user productions and justifications, and to see full wmes, do any one of the following (not all possibilities listed): \begin{verbatim}
watch --decisions --phases --productions --user remove --justifications remove --fullwmes
watch -d -p -P -f -u remove -j 0 
watch -f -l 3 -u 0 -j 0

\end{verbatim}

\section*{ See Also }


 pwatch print run watch-wmes
\section*{ Structured Output }
\subsection*{ On Query }


 The following arg parameters are returned: \begin{verbatim}
kParamWatchDecisions, kTypeBoolean
kParamWatchPhases, kTypeBoolean
kParamWatchProductionDefault, kTypeBoolean
kParamWatchProductionUser, kTypeBoolean
kParamWatchProductionChunks, kTypeBoolean
kParamWatchProductionJustifications, kTypeBoolean
kParamWatchWMEDetail, kTypeInt
kParamWatchWorkingMemoryChanges, kTypeBoolean
kParamWatchPreferences, kTypeBoolean
kParamWatchLearning, kTypeInt
kParamWatchBacktracing, kTypeBoolean
kParamWatchIndifferentSelection, kTypeBoolean

\end{verbatim}

\subsection*{ Otherwise }
\begin{verbatim}
<result output="raw">true</result>

\end{verbatim}


 \\ 

\section*{ Error Values }
\subsection*{ During Parsing }


 kMissingOptionArg, kUnrecognizedOption, kGetOptError, kTooManyArgs, kIntegerExpected, kIntegerMustBeNonNegative, kIntegerOutOfRange, kInvalidLearnSetting, kRemoveOrZeroExpected
\subsection*{ During Execution }


 kAgentRequired, kKernelRequired Retrieved from ``\url{http://winter.eecs.umich.edu/soarwiki/Watch}``

\end{document}

\subsection{\soarb{watch-wmes}}
\label{watch-wmes}
\index{watch-wmes}
\subsubsection*{Synopsis}
\begin{verbatim}
watch-wmes -[a|r]  -t <type>  pattern
watch-wmes -[l|R] [-t <type>]
\end{verbatim}
\subsubsection*{Options}
\begin{tabular}{|l|l|}
\hline
\soar{ -a, --add-filter } & Add a filter to print wmes that meet the type and pattern criteria.  \\
\hline
\soar{ -r, --remove-filter } & Delete filters for printing wmes that match the type and pattern criteria.  \\
\hline
\soar{ -l, --list-filter } & List the filters of this type currently in use. Does not use the pattern argument.  \\
\hline
\soar{ -R, --reset-filter } & Delete all filters of this type. Does not use pattern arg.  \\
\hline
\soar{ -t, --type } & Follow with a type of wme filter, see below.  \\
\hline
\end{tabular}
\paragraph*{Pattern}
 The pattern is an id-attribute-value triplet: \begin{verbatim}
\emph{id}
 \emph{attribute}
 \emph{value}
\end{verbatim}
 Note that \textbf{*}
 can be used in place of the id, attribute or value as a wildcard that maches any string. Note that braces are not used anymore. 
\paragraph*{Types}
 When using the -t flag, it must be followed by one of the following: 
\begin{tabular}{|l|l|}
\hline
\soar{ adds } & Print info when a wme is \emph{added}
.  \\
\hline
\soar{ removes } & Print info when a wme is \emph{retracted}
.  \\
\hline
\soar{ both } & Print info when a wme is added \emph{or}
 retracted.  \\
\hline
\end{tabular}
 When issuing a \textbf{-R}
 or \textbf{-l}
, the \textbf{-t}
 flag is optional. Its absence is equivalent to \textbf{-t both}
. 
\subsubsection*{Description}
 This commands allows users to improve state tracing by issuing filter-options that are applied when watching wmes. Users can selectively define which \emph{object-attribute-value}
 triplets are monitored and whether they are monitored for addition, removal or both, as they go in and out of working memory. 
 \textbf{Note:}
 The functionality of \textbf{watch-wmes}
 resided in the \textbf{watch}
 command prior to Soar 8.6. 
\subsubsection*{Examples}
 Users can \textbf{watch}
 an \emph{attribute}
 of a particular object (as long as that object already exists):  \begin{verbatim}
soar> watch-wmes --add-filter -t both D1 speed *
\end{verbatim}
 or print WMEs that retract in a specific state (provided the \textbf{state}
 already exists):  \begin{verbatim}
soar> watch-wmes --add-filter -t removes S3 * *
\end{verbatim}
  or watch any relationship between objects:  \begin{verbatim}
soar> watch-wmes --add-filter -t both * ontop *
\end{verbatim}


% ----------------------------------------------------------------------------
\section{Configuring Soar's Runtime Parameters}
\label{RUNTIME}

This section describes the commands that control Soar's Runtime Parameters.
Many of these commands provide options that simplify or restrict 
runtime behavior to enable easier and more localized debugging.
Others allow users to select alternative algorithms or methodologies.
Users can configure Soar's learning mechanism; examine the
backtracing information that supports chunks and justifications;
provide hints that could improve the efficiency of the Rete matcher;
limit runaway chunking and production firing;
choose an alternative algorithm for determining whether a working memory
element receives O-support;  and 
configure options for selecting between mutually indifferent operators.

The specific commands described in this section are:

\paragraph{Summary}
\begin{quote}
\begin{description}
\item[epmem] - Get/Set episodic memory parameters and statistics
\item[explain-backtraces] - Print information about chunk and justification backtraces.
\item[indifferent-selection] -  Controls indifferent preference arbitration.
\item[learn] - Set the parameters for chunking, Soar's learning mechanism.
\item[max-chunks] - Limit the number of chunks created during a decision cycle.
\item[max-dc-time] - Set a wall-clock time limit such that the agent will be interrupted when a single decision cycle exceeds this limit.
\item[max-elaborations] - Limit the maximum number of elaboration cycles in a given phase.
\item[max-goal-depth] - Limit the sub-state stack depth.
\item[max-memory-usage] - Set the number of bytes that when exceeded by an agent, will trigger the memory usage exceeded event. 
\item[max-nil-output-cycles] - Limit the maximum number of decision cycles executed without producing output. 
\item[multi-attributes] - Declare multi-attributes so as to increase Rete matching efficiency.
\item[numeric-indifferent-mode] - Select method for combining numeric preferences.
\item[o-support-mode] - Choose experimental variations of o-support.
\item[predict] - Predict the next selected operator 
\item[rl] - Get/Set RL parameters and statistics 
\item[save-backtraces] - Save trace information to explain chunks and justifications.
\item[select] - Force the next selected operator 
\item[set-stop-phase] -  Controls the phase where agents stop when running by decision.
\item[smem] - Get/Set semantic memory parameters and statistics
\item[timers] - Toggle on or off the internal timers used to profile Soar.
\item[waitsnc] - Generate a wait state rather than a state-no-change impasse.
\item[wma] - Get/Set working memory activation parameters
\end{description}
\end{quote}

% ----------------------------------------------------------------------------
\chapter{Episodic Memory}
\label{EPMEM}
\index{episodic memory}
\index{epmem}


\subsection{\soarb{explain-backtraces}}
\label{explain-backtraces}
\index{explain-backtraces}
Print information about chunk and justification backtraces. 
 Priority: 3; Status: Incomplete, EvilBackDoor\\ 
Result generated by kernel.--Jonathan 18:16, 25 Feb 2005 (EST) 
\subsubsection*{Synopsis}
\begin{verbatim}
explain-backtraces -f prod_name
explain-backtraces [-c <n>] prod_name
\end{verbatim}
\subsubsection*{Options}
\begin{tabular}{|l|l|}
\hline 
 prod\_name  & List all conditions and grounds for the chunk or justification.  \\
 \hline 
 -c, --condition  & Explain why condition number \emph{n}
 is in the chunk or justification.  \\
 \hline 
 -f, --full  &�?  \\
 \hline 
\end{tabular}
\subsubsection*{Description}
 This command provides some interpretation of backtraces generated during chunking. If no option is given, then a list of all chunks and justifications is printed. 
 The two most useful variants are: \begin{verbatim}
explain-backtraces prodname 
explain-backtraces name n
\end{verbatim}
 The first variant lists all of the conditions for the named chunk or justification, and the ground which resulted in inclusion in the chunk/justification. A ground is a working memory element (WME) which was tested in the supergoal. Just knowing which WME was tested may be enough to explain why the chunk/justification exists. If not, the conditions can be listed with an integer value. This value can be used in explain-backtraces name n to obtain a list of the productions which fired to obtain this condition in the chunk/justification (and the crucial WMEs tested along the way). Why use an integer value to specify the condition? To save a big parsing job. 
 save\_backtraces mode must be on when a chunk or justification is created or no explanation will be available. 
\subsubsection*{Structured Output:}
\paragraph*{On Success}
\paragraph*{Notes}
\subsubsection*{Error Values:}
\paragraph*{During Parsing}
 kNotImplemented
\paragraph*{During Execution}

\subsection{\soarb{indifferent-selection}}
\label{indifferent-selection}
\index{indifferent-selection}
Controls indifferent preference arbitration. 
\subsubsection*{Synopsis}
\begin{verbatim}
indifferent-selection [-aflr]
\end{verbatim}
\subsubsection*{Options}
\begin{tabular}{|l|l|}
\hline
\soar{ -a, --ask } & Ask the user to choose. Not implemented. \\
\hline
\soar{ -f, --first } & Select the first indifferent object from Soar's internal list.  \\
\hline
\soar{ -l, --last } & Select the last indifferent object from Soar's internal list.  \\
\hline
\soar{ -r, --random } & Select randomly (default).  \\
\hline
\end{tabular}
\subsubsection*{Description}
 The \textbf{indifferent-selection}
 command allows the user to set which option should be used to select between operator proposals that are mutally indifferent in preference memory. 
 The default option is \textbf{--random}
 which chooses an operator at random from the set of mutually indifferent proposals, with the selection biased by any existing numeric preferences. For repeatable results, the user may choose the \textbf{--first}
 or \textbf{--last}
 option. ``First'' refers to the list of operator augmentations internal to Soar; the ordering of the augmentations is arbitrary but deterministic, so that if you run Soar repeatedly, \textbf{--first}
 will always make the same decision. Similarly, \textbf{--last}
 chooses the last of the tied objects from the internal list. For complete control over the decision process, the \textbf{--ask}
 option prompts the user to select the next operator from a list of the tied operators. 
 If no argument is provided, \textbf{indifferent-selection}
 will display the current setting. 
\subsubsection*{Default Aliases}
\begin{tabular}{|l|l|}
\hline
\soar{ Alias } & Maps to  \\
\hline
\soar{ inds } & indifferent-selection  \\
\hline
\end{tabular}
\subsubsection*{See Also}
\hyperref[numeric-indifferent-mode]{numeric-indifferent-mode} 
\documentclass[10pt]{article}
\usepackage{fullpage, graphicx, url}
\setlength{\parskip}{1ex}
\setlength{\parindent}{0ex}
\title{Learn - Soar Wiki}
\begin{document}
\section*{Learn}
\subsubsection*{From Soar Wiki}


 This is part of the Soar Command Line Interface. 
\section*{ Name }


 \textbf{learn}
 - Set the parameters for chunking, Soar\^a��s learning mechanism. 


 Status: Complete, EvilBackDoor
\section*{ Synopsis }
\begin{verbatim}
learn [-l]
learn -[d|E|o]
learn -e [ab]

\end{verbatim}
\section*{ Options }


\begin{tabular}{|c|c|}
\hline 
 -a, --all-levels  & Build chunks whenever a subgoal returns a result. Learning must be --enabled.  \\
 \hline 
 -b, --bottom-up  & Build chunks only for subgoals that have not yet had any subgoals with chunks built. Learning must be --enabled.  \\
 \hline 
 -d, --disable, --off  & Turn all chunking off.  \\
 \hline 
 -e, --enable, --on  & Turn chunking on. Can be modified by -a or -b  \\
 \hline 
 -E, --except  & Learning is on, except as specified by RHS \emph{dont-learn}
 actions.  \\
 \hline 
 -l, --list  & Prints listings of dont-learn and force-learn states.  \\
 \hline 
 -o, --only  & Chunking is on only as specified by RHS \emph{force-learn}
 actions.  \\
 \hline 

\end{tabular}



 \\ 

\section*{ Description }


 The learn command controls the parameters for chunking (Soar's learning mechanism). With no arguments, this command prints out the current learning environment status. If arguments are provided, they will alter the learning environment as described in the options and arguments table. The watch command can be used to provide various levels of detail when productions are learned. Learning is \textbf{disabled}
 by default. 


 \\ 

\section*{ Examples }


 To enable learning only at the lowest subgoal level: \begin{verbatim}
learn -e b 

\end{verbatim}



 To see all the \emph{force-learn}
 and \emph{dont-learn}
 states registered by RHS actions \begin{verbatim}
learn -l

\end{verbatim}

\section*{ See Also }
\begin{description}
watch, explain-backtraces, save-backtraces

\end{description}


 \\ 

\section*{ Structured Output }
\subsection*{ On Query }


 If learning is on: \begin{verbatim}
<result>
  <arg param="learnsetting" type="boolean">true</arg>
  <arg param="learnonlysetting" type="boolean">setting</arg>
  <arg param="learnexceptsetting" type="boolean">setting</arg>
  <arg param="learnalllevelssetting" type="boolean">setting</arg>
</result>

\end{verbatim}



 If learning is off: \begin{verbatim}
<result>
  <arg param="learnsetting" type="boolean">false</arg>
</result>

\end{verbatim}

\subsection*{ On List }


 When the list flag is issued, the results of a query are returned plus the following two arg tags: \begin{verbatim}
<arg param="learnforcelearnstates" type="string">string</arg>
<arg param="learndontlearnstates" type="string">string</arg>

\end{verbatim}

\subsection*{ Otherwise }
\begin{verbatim}
<result output="raw">true</result>

\end{verbatim}
\subsection*{ Notes }
\begin{itemize}
\item  Setting is true (learning enabled) or false (disabled). 
\item  learnalllevelssetting true means all-levels enabled, false means bottom-up 

\end{itemize}
\section*{ Error Values }
\subsection*{ During Parsing }


 kUnrecognizedOption, kGetOptError, kTooManyArgs
\subsection*{ During Execution }


 kAgentRequired Retrieved from ``\url{http://winter.eecs.umich.edu/soarwiki/Learn}``

\end{document}

\subsection{\soarb{max-chunks}}
\label{max-chunks}
\index{max-chunks}
Limit the number of chunks created during a decision cycle. 
\subsubsection*{Synopsis}
\begin{verbatim}
max-chunks [n]
\end{verbatim}
\subsubsection*{Options}
\begin{tabular}{|l|l|}
\hline
\soar{ n } & Maximum number of chunks allowed during a decision cycle.  \\
\hline
\end{tabular}
\subsubsection*{Description}
 The \textbf{max-chunks}
 command is used to limit the maximum number of chunks that may be created during a decision cycle. The initial value of this variable is 50; allowable settings are any integer greater than 0. 
 The chunking process will end after \textbf{max-chunks}
 chunks have been created, \emph{even if there are more results that have not been backtraced through to create chunks}
, and Soar will proceed to the next phase. A warning message is printed to notify the user that the limit has been reached. 
 This limit is included in Soar to prevent getting stuck in an infinite loop during the chunking process. This could conceivably happen because newly-built chunks may match immediately and are fired immediately when this happens; this can in turn lead to additional chunks being formed, etc. If you see this warning, something is seriously wrong; Soar is unable to guarantee consistency of its internal structures. You should not continue execution of the Soar program in this situation; stop and determine whether your program needs to build more chunks or whether you've discovered a bug (in your program or in Soar itself). 

\input{wikicmd/tex/max-dc-time}
\subsection{\soarb{max-elaborations}}
\label{max-elaborations}
\index{max-elaborations}
Limit the maximum number of elaboration cycles in a given phase. Print a warning message if the limit is reached during a run. 
\subsubsection*{Synopsis}
\begin{verbatim}
max-elaborations [n]
\end{verbatim}
\subsubsection*{Options}
\begin{tabular}{|l|l|}
\hline 
\emph{n}
 & Maximum allowed elaboration cycles, must be a positive integer.  \\
 \hline 
\end{tabular}
\subsubsection*{Description}
 This command sets and prints the maximum number of elaboration cycles allowed. If \emph{n}
 is given, it must be a positive integer and is used to reset the number of allowed elaboration cycles. The default value is 100. \textbf{max-elaborations}
 with no arguments prints the current value. 
 \textbf{max-elaborations}
 controls the maximum number of elaborations allowed in a single decision cycle. The elaboration phase will end after \emph{max-elaboration}
 cycles have completed, even if there are more productions eligible to fire or retract; and Soar will proceed to the next phase after a warning message is printed to notify the user. This limits the total number of cycles of parallel production firing but does not limit the total number of productions that can fire during elaboration. 
 This limit is included in Soar to prevent getting stuck in infinite loops (such as a production that repeatedly fires in one elaboration cycle and retracts in the next); if you see the warning message, it may be a signal that you have a bug your code. However some Soar programs are designed to require a large number of elaboration cycles, so rather than a bug, you may need to increase the value of \emph{max-elaborations}
. 
 In Soar8, \emph{max-elaborations}
 is checked during both the Propose Phase and the Apply Phase. If Soar8 runs more than the max-elaborations limit in either of these phases, Soar8 proceeds to the next phase (either Decision or Output) even if quiescence has not been reached. 
\subsubsection*{Examples}
 The command issued with no arguments, returns the max elaborations allowed: \begin{verbatim}
max-elaborations 
\end{verbatim}
 to set the maximum number of elaborations in one phase to 50: \begin{verbatim}
max-elaborations 50
\end{verbatim}

\input{wikicmd/tex/max-goal-depth}
\subsection{\soarb{max-memory-usage}}
\label{max-memory-usage}
\index{max-memory-usage}
Set the amount of bytes necessary to trigger the memory usage exceeded event. 
\subsubsection*{Synopsis}
max-memory-usage [n]
\end{verbatim}
\subsubsection*{Options}
\hline
\soar{\soar{\soar{ n }}} & Size of limit in bytes.  \\
\hline
\end{tabular}
\subsubsection*{Description}
 The \textbf{max-memory-usage}
 command is used to trigger the memory usage exceeded event. The initial value of this is 100MB (100,000,000); allowable settings are any integer greater than 0. 
 Using the command with no arguments displays the current limit. 

\subsection{\soarb{max-nil-output-cycles}}
\label{max-nil-output-cycles}
\index{max-nil-output-cycles}
Limit the maximum number of decision cycles that are executed without producing output when run is invoked with run-til-output args. 
\subsubsection*{Synopsis}
\begin{verbatim}
max-nil-output-cycles [n]
\end{verbatim}
\subsubsection*{Options}
\begin{tabular}{|l|l|}
\hline
\emph{n}
 & Maximum number of consecutive output cycles allowed without producing output. Must be a positive integer.  \\
\hline
\end{tabular}
\subsubsection*{Description}
 This command sets and prints the maximum number of nil output cycles (output cycles that put nothing on the output link) allowed when running using run-til-output (run --output). If \emph{n}
 is not given, this command prints the current number of nil-output-cycles allowed. If \emph{n}
 is given, it must be a positive integer and is used to reset the maximum number of allowed nil output cycles. 
 \textbf{max-nil-output-cycles}
 controls the maximum number of output cycles that generate no output allowed when a \textbf{run --out}
 command is issued. After this limit has been reached, Soar stops. The default initial setting of \emph{n}
 is 15. 
\subsubsection*{Examples}
 The command issued with no arguments, returns the max empty output cycles allowed: \begin{verbatim}
max-nil-output-cycles 
\end{verbatim}
 to set the maximum number of empty output cycles in one phase to 25: \begin{verbatim}
max-nil-output-cycles 25 
\end{verbatim}
\subsubsection*{See Also}
\hyperref[run]{run}  Categories: Command Line Interface

\subsection{\soarb{multi-attributes}}
\label{multi-attributes}
\index{multi-attributes}
Declare a symbol to be multi-attributed. 
 Complete
\subsubsection*{Synopsis}
\begin{verbatim}
multi-attributes [symbol [\emph{n}
]] 
\end{verbatim}
\subsubsection*{Options}
\begin{tabular}{|l|l|}
\hline 
symbol & Any Soar attribute.  \\
 \hline 
\emph{n}
 & Integer $>$ 1, estimate of degree of simultaneous values for attribute.  \\
 \hline 
\end{tabular}
\subsubsection*{Description}
 This command declares the given symbol to be an attribute which can take on multiple values. The optional \emph{n}
 is an integer ($>$1) indicating an upper limit on the number of expected values that will appear for an attribute. If \emph{n}
 is not specified, the value 10 is used for each declared multi-attribute. More informed values will tend to result in greater efficiency. This command is used only to provide hints to the production condition reorderer so it can produce better condition orderings. Better orderings enable the rete network to run faster. This command has no effect on the actual contents of working memory and most users needn't use this at all. 
 Note that multi-attributes declarations must be made before productions are loaded into soar or this command will have no effect. 
\subsubsection*{Examples}
 Declare the symbol ``thing'' to be an attribute likely to take more than 1 but no more than 4 values: \begin{verbatim}
 multi-attributes thing 4 
\end{verbatim}

\subsection{\soarb{numeric-indifferent-mode}}
\label{numeric-indifferent-mode}
\index{numeric-indifferent-mode}
Select method for combining numeric preferences. 
 Status: Complete
\subsubsection*{Synopsis}
\begin{verbatim}
numeric-indifferent-mode [-as]
\end{verbatim}
\subsubsection*{Options}
\begin{tabular}{|l|l|}
\hline 
 -a, --avg, --average  & Use average mode (default).  \\
 \hline 
 -s, --sum  & Use sum mode.  \\
 \hline 
\end{tabular}
\subsubsection*{Description}
 The numeric-indifferent-mode command is used to select the method for combining numeric preferences. This command is only meaningful in indifferent-selection --random  mode. 
 The default procedure is \textbf{-avg}
 (average) which assigns a final value to an operator according to the rule: \begin{itemize}
\item  If the operator has at least one numeric preference, assign it the value that is the average of all of its numeric preferences. 
\item  If the operator has no numeric preferences (but has been included in the indifferent selection through some combination of non-numeric preferences), assign it the value 50. 
\end{itemize}
 The intended range of numeric-preference values for \textbf{-avg}
 mode is 0-100. 
 The other combination option \textbf{-sum}
 assigns a final value according to the rule: \begin{itemize}
\item  Add together any numeric preferences for the operator (defaulting to 0 if there are none). 
\item  Assign the operator the value \textbf{Failed to parse (Missing texvc executable; please see math/README to configure.): e\^{}\{PreferenceSum / AgentTemperature\}}
\end{itemize}
 , where AgentTemperature is a compile-time constant currently set at 25.0. 
 Any real-numbered preference may be used in \textbf{-sum}
 mode. 
 Once a value has been computed for each operator, the next operator is selected probabilistically, with each candidate operator's chance weighted by its computed value. 

\subsection{\soarb{o-support-mode}}
\label{o-support-mode}
\index{o-support-mode}
Choose experimental variations of o-support. 
\subsubsection*{Synopsis}
\begin{verbatim}
o-support-mode [0|1|2|3|4]
\end{verbatim}
\subsubsection*{Options}
\begin{tabular}{|l|l|}
\hline 
 0  & Mode 0 is the base mode. O-support is calculated based on the structure of working memory that is tested and modified. Testing an operator or operator acceptable preference results in state or operator augmentations being o-supported. The support computation is very complex (see soar manual).  \\
 \hline 
 1  & Not available through gSKI.  \\
 \hline 
 2  & Mode 2 is the same as mode 0 except that all support is calculated the production structure, not from working memory structure. Augmentations of operators are still o-supported.  \\
 \hline 
 3  & Mode 3 is the same as mode 2 except that operator elaborations (adding attributes to operators) now get i-support even though you have to test the operator to elaborate an operator.  \\
 \hline 
 4  & Mode 4 is the default.  \\
 \hline 
\end{tabular}
\subsubsection*{Description}
 The \textbf{o-support-mode}
 command is used to control the way that o-support is determined for preferences. Only o-support modes 3 \& 4 can be considered current to Soar8, and o-support mode 4 should be considered an improved version of mode 3. The default o-support mode is mode 4. 
 In o-support modes 3 \& 4, support is given production by production; that is, all preferences generated by the RHS of a single instantiated production will have the same support. The difference between the two modes is in how they handle productions with both operator and non-operator augmentations on the RHS. For more information on o-support calculations, see the relevant appendix in the Soar manual. 
 Running o-support-mode with no arguments prints out the current o-support-mode. 

\input{wikicmd/tex/predict}
\chapter{Reinforcement Learning}
\label{RL}
\index{reinforcement learning}
\index{preference!numeric-indifferent}
\index{rl}

Soar has a reinforcement learning (RL) mechanism that tunes operator selection knowledge based on a given reward function.
This chapter describes the RL mechanism and how it is integrated with production memory, the decision cycle, and the state stack.
We assume that the reader is familiar with basic reinforcement learning concepts and notation. If not, we recommend first reading \emph{Reinforcement Learning: An Introduction} (1998) by Richard S. Sutton and Andrew G. Barto.
The detailed behavior of the RL mechanism is determined by numerous parameters that can be controlled and configured via the \soarb{rl} command.
Please refer to the documentation for that command in section \ref{rl} on page \pageref{rl}.

\section{RL Rules}
\label{RL-rules}

Soar's RL mechanism learns Q-values for state-operator\footnote{In this context, the term ``state'' refers to the state of the task or environment, not a state identifier.
For the rest of this chapter, bold capital letter names such as \soarb{S1} will refer to identifiers and italic lowercase names such as $s_1$ will refer to task states.} pairs.
Q-values are stored as numeric indifferent preferences asserted by specially formulated productions called \emph{RL rules}.
RL rules are identified by syntax.
A production is a RL rule if and only if its left hand side tests for a proposed operator, its right hand side asserts a single numeric indifferent preference, and it is not a template rule (see \ref{RL-templates}).
These constraints ease the technical requirements of identifying/updating RL rules and makes it easy for the agent programmer to add/maintain RL capabilities within an agent.

The following is an RL rule:

\begin{verbatim}
sp {rl*3*12*left
   (state <s> ^name task-name
              ^x 3
              ^y 12
	          ^operator <o> +)
   (<o> ^name move
	    ^direction left)
-->
   (<s> ^operator <o> = 1.5)
}
\end{verbatim}

Note that the LHS of the rule can test for anything as long as it contains a test for a proposed operator.
The RHS is constrained to exactly one action: asserting a numeric indifferent preference for the proposed operator.

The following are not RL rules:

\begin{verbatim}
sp {multiple*preferences
   (state <s> ^operator <o> +)
-->
   (<s> ^operator <o> = 5, >)
}
\end{verbatim}  \vspace{12pt}

\begin{verbatim}
sp {variable*binding
    (state <s> ^operator <o> +
               ^value <v>)
-->
    (<s> ^operator <o> = <v>)
}
\end{verbatim}

The first rule proposes multiple preferences for the proposed operator and thus does not comply with the rule format.
The second rule does not comply because it does not provide a \emph{constant} for the numeric indifferent preference value.

In the typical RL use case, the agent should learn to choose the optimal operator in each possible state of the environment.
The most straightforward way to achieve this is to give the agent a set of RL rules, each matching exactly one possible state-operator pair.
This approach is equivalent to a table-based RL algorithm, where the Q-value of each state-operator pair corresponds to the numeric indifferent preference asserted by exactly one RL rule.

In the more general case, multiple RL rules can match a single state-operator pair, and a single RL rule can match multiple state-operator pairs.
Assuming that the value of \soarb{numeric-indifferent-mode} is set to \soarb{sum} (see page \pageref{numeric-indifferent-mode}), all numeric indifferent preferences for an operator are summed when calculating the operator's Q-value.
In this context, RL rules can be interpreted more generally as binary features in a linear approximator of each state-operator pair's Q-value, and their numeric indifferent preference values their weights.
In other words,
$$Q(s, a) = w_1 \phi_2 (s, a) + w_2 \phi_2 (s, a) + \ldots + w_n \phi_n (s, a)$$
where all RL rules in production memory are numbered $1 \dots n$, $Q(s, a)$ is the Q-value of the state-operator pair $(s, a)$, $w_i$ is the numeric indifferent preference value of RL rule $i$, $\phi_i (s, a) = 0$ if RL rule $i$ does not match $(s, a)$, and $\phi_i (s, a) = 1$ if it does.
This interpretation allows RL rules to simulate a number of popular function approximation schemes used in RL such as tile coding and sparse coarse coding.

\section{Reward Representation}
\label{RL-reward}

RL updates are driven by reward signals.
In Soar, these reward signals are fed to the RL mechanism through a working memory link called the \soarb{reward-link}.
Each state in Soar's goal stack is automatically populated with a \soarb{reward-link} structure upon creation.
Soar will check this structure for a numeric reward signal for the last operator executed in the associated state at the beginning of every decision phase.
Reward is also counted when the agent is halted or a substate is retracted.
% What happens when an agent with multiple states is halted? Do the rewards in the substates get counted?

In order to be recognized, the reward signal must follow this pattern:

\begin{verbatim}
(<r1> ^reward <r2>)
(<r2> ^value [val])
\end{verbatim}

where \verb=<r1>= is the \soarb{reward-link} identifier, \verb=<r2>= is some intermediate identifier, and \verb=[val]= is any constant numeric value.
Any structure that does not match this pattern are ignored.
If there are multiple matching WMEs, their values are summed into a single reward signal.

As an example, consider the following state:

\begin{verbatim}
(S1 ^reward-link R1)
  (R1 ^reward R2)
    (R2 ^value 1.0)
    (R2 ^source environment)
  (R1 ^reward R3)
    (R3 ^value -0.2)
    (R3 ^source intrinsic)
\end{verbatim}  

In this state, there are two reward signals with values 1.0 and -0.2.
They will be summed together for a total reward of 0.8 and this will be the value given to the RL update algorithm.
The \verb=(R2 ^source environment)= and \verb=(R3 ^source intrinsic)= WMEs are not counted as rewards or special in any way, but were added by the agent to keep track of where the rewards came from.

Note that the \soarb{reward-link} is not part of the \soarb{io} structure and is not modified directly by the environment.
Reward information from the environment should be copied, via rules, from the \soarb{input-link} to the \soarb{reward-link}.
Also note that when counting rewards, Soar simply scans the \soarb{reward-link} and sums the values of all valid reward WMEs.
The WMEs are not modified and no bookkeeping is done to keep track of previously seen WMEs.
This means that reward WMEs that exist for multiple decision cycles such as o-supported WMEs will be counted multiple times.

\section{Updating RL Rule Values}
\label{RL-algo}

Soar's RL mechanism is integrated naturally with the decision cycle and performs online updates of RL rules.
Whenever an operator supported by RL rules is selected, the values of those RL rules are updated.
The update can be on-policy (Sarsa) or off-policy (Q-Learning), as controlled by the \soarb{learning-policy} parameter of the \soarb{rl} command.
For Sarsa, the update is
$$ \delta_t = \alpha \left[ r_{t+1} + \gamma Q(s_{t+1}, a_{t+1}) - Q(s_t, a_t) \right] $$
where 
\begin{itemize}
\item $Q(s_t, a_t)$ is the Q-value of the state and chosen operator in decision cycle $t$.
\item $Q(s_{t+1}, a_{t+1})$ is the Q-value of the state and chosen operator in the next decision cycle.
\item $r_{t+1}$ is the total reward counted in the next decision cycle.
\item $\alpha$ and $\gamma$ are the settings of the \soarb{learning-rate} and \soarb{discount-rate} parameters of the \soarb{rl} command, respectively.
\end{itemize}

For Q-Learning, the update is
$$ \delta_t = \alpha \left[ r_{t+1} + \gamma \underset{a \in A_{t+1}}{\max} Q(s_{t+1}, a) - Q(s_t, a_t) \right] $$
where $A_{t+1}$ is the set of operators proposed in the next decision cycle.

Finally, $\delta_t$ is divided by the number of RL rules comprising the Q-value for the operator and the numeric indifferent values for each RL rule is updated by that amount.

An example walkthrough of a Sarsa update with $\alpha = 0.3$ and $\gamma = 0.9$ follows.

\begin{enumerate}

\item In decision cycle $t$, an operator \soarb{O1} is proposed, and RL rules \soarb{rl-1} and \soarb{rl-2} assert the following numeric indifferent preferences for it:
\begin{verbatim}
   rl-1: (S1 ^operator O1 = 2.3)
   rl-2: (S1 ^operator O1 =  -1)
\end{verbatim}  
	The Q-value for \soarb{O1} is $Q(s_t, \soarb{O1}) = 2.3 - 1 = 1.3$.
	 
\item \soarb{O1} is selected and executed, so $Q(s_t, a_t) = Q(s_t, \soarb{O1}) = 1.3$.

\item In decision cycle $t+1$, a total reward of 1.0 is counted on the \soarb{reward-link}, an operator \soarb{O2} is proposed, and another RL rule \soarb{rl-3} asserts the following numeric indifferent preference for it:
\begin{verbatim}
	rl-3: (S1 ^operator O2 = 0.5)
\end{verbatim}
	So $Q(s_{t+1}, \soarb{O2}) = 0.5$.

\item \soarb{O2} is selected, so $Q(s_{t+1}, a_{t+1}) = Q(s_{t+1}, \soarb{O2}) = 0.5$
	Therefore, 
	$$\delta_t = \alpha \left[r_{t+1} + \gamma Q(s_{t+1}, a_{t+1}) - Q(s_t, a_t) \right] = 0.3 \times [ 1.0 + 0.9 \times 0.5 - 1.3 ] = 0.045$$
	Since \soarb{rl-1} and \soarb{rl-2} both contributed to the Q-value of \soarb{O1}, $\delta_t$ is evenly divided amongst them, resulting in updated values of
\begin{verbatim}
   rl-1: (<s> ^operator <o> = 2.3225)
   rl-2: (<s> ^operator <o> = -0.9775)
\end{verbatim}

\end{enumerate}

\subsection{Gaps in Rule Coverage}
\label{RL-gaps}

Call an operator with numeric indifferent preferences an RL operator.
The previous description had assumed that RL operators were selected in both decision cycles $t$ and $t+1$.
If the operator selected in $t+1$ is not an RL operator, then $Q(s_{t+1}, a_{t+1})$ would not be defined, and an update for the RL operator selected at time $t$ will be undefined.
% This is true for Sarsa, but what about Q-Learning?
We will call a sequence of one or more decision cycles in which RL operators are not selected between two decision cycles in which RL operators are selected a \emph{gap}.
Conceptually, it is desirable to use the temporal difference information from the RL operator after the gap to update the Q-value of the RL operator before the gap.
There are just no intermediate storage locations for these updates.
Requiring that RL rules support operators at every decision can be difficult for agent programmers, particularly when operators are required that do not represent steps in a task, but instead perform generic maintenance functions, such as cleaning processed output-link structures.

To address this issue, Soar's RL mechanism supports automatic propagation of updates over gaps.
For a gap of length $n$, the Sarsa update is
$$\delta_t = \alpha \left[ \sum_{i=t}^{t+n}{\gamma^{i-t} r_i} + \gamma^{n+1} Q(s_{t+n+1}, a_{t+n+1}) - Q(s_t, a_t) \right]$$
and the Q-Learning update is
$$\delta_t = \alpha \left[ \sum_{i=t}^{t+n}{\gamma^{i-t} r_i} + \gamma^{n+1} \underset{a \in A_{t+n+1}}{\max} Q(s_{t+n+1}, a) - Q(s_t, a_t) \right]$$

Note that rewards will still be counted during the gap, but they are discounted based on the number of decisions removed they are from the initial RL operator.

Gap propagation can be disabled by setting the \soarb{temporal-extension} parameter of the \soarb{rl} command to \soarb{off}.
When gap propagation is disabled, the RL rules supporting an operator that is followed by a gap are simply not updated.
The \soarb{rl} setting of the \soarb{watch} command (see Section \ref{watch} on page \pageref{watch}) is useful in identifying gaps.


\subsection{RL and Substates}
\label{RL-substates}

When an agent has multiple states in its state stack, the RL mechanism will treat each substate independently.
As mentioned previously, each state has its own \soarb{reward-link}.
When an RL operator is selected in a state \soarb{S}, the RL updates for that operator are only affected by the rewards counted on the \soarb{reward-link} for \soarb{S} and the Q-values of subsequent RL operators selected in \soarb{S}.

The only exception to this independence is when a selected RL operator forces an operator-no-change impasse.
When this occurs, the number of decision cycles the RL operator at the superstate remains selected is dependent upon the processing in the impasse state.
Consider the operator trace in Figure \ref{fig:rl-optrace}.

\begin{itemize}
\item At decision cycle 1, RL operator \soarb{O1} is selected in \soarb{S1} and causes an operator-no-change impass for three decision cycles.
\item In the substate \soarb{S2}, operators \soarb{O2}, \soarb{O3}, and \soarb{O4} are selected and applied sequentially.
\item Meanwhile in \soarb{S1}, reward values $r_2$, $r_3$, and $r_4$ are put on the \soarb{reward-link} sequentially.
\item Finally, the impasse is resolved by \soarb{O4}, the proposal for \soarb{O1} is retracted, and RL operator \soarb{O5} is selected in \soarb{S1}.
\end{itemize}

\begin{figure}
\insertfigure{Figures/rl-optrace}{1.5in}
\insertcaption{Example Soar subgoal operator trace.}
\label{fig:rl-optrace}
\end{figure}

In this scenario, only the RL update for $Q(s_1, \soarb{O1})$ will be different from the ordinary case.
Its value depends on the setting of the \soarb{hrl-discount} parameter of the \soarb{rl} command.
When this parameter is set to the default value \soarb{on}, the rewards on \soarb{S1} and the Q-value of \soarb{O5} are discounted by the number of decision cycles they are removed from the selection of \soarb{O1}.
In this case the update for $Q(s_1, \soarb{O1})$ is
$$\delta_1 = \alpha \left[ r_2 + \gamma r_3 + \gamma^2 r_4 + \gamma^3 Q(s_5, \soarb{O5}) \right]$$
which is equivalent to having a three decision gap separating \soarb{O1} and \soarb{O5}.

When \soarb{hrl-discount} is set to \soarb{off}, the number of cycles \soarb{O1} has been impassed will be ignored.
Thus the update would be
$$\delta_1 = \alpha \left[ r_2 + r_3 + r_4 + \gamma Q(s_5, \soarb{O5}) \right]$$

For impasses other than operator no-change, RL acts as if the impasse hadn't occurred.
If \soarb{O1} is the last RL operator selected before the impasse, $r_2$ the reward received in the decision cycle immediately following, and \soarb{On} the first operator selected after the impasse, then \soarb{O1} is updated with 
$$\delta_1 = \alpha \left[ r_2 + \gamma Q(s_n, \soarb{On}) \right]$$

Soar's automatic subgoaling and RL mechanisms can be combined to naturally implement hierarchical reinforcement learning algorithms such as MAXQ and options.

\subsection{Eligibility Traces}
\label{RL-et}
The RL mechanism supports eligibility traces, which can improve the speed of learning by updating RL rules across multiple sequential steps.
The \soarb{eligibility-trace-decay-rate} and \soarb{eligibility-trace-tolerance} parameters control this mechanism.
By setting \soarb{eligibility-trace-decay-rate} to \soarb{0} (default), eligibility traces are in effect disabled.
When eligibility traces are enabled, the particular algorithm used is dependent upon the learning policy.
For Sarsa, the eligibility trace implementation is \emph{Sarsa($\lambda$)}. 
For Q-Learning, the eligibility trace implementation is \emph{Watkin's Q($\lambda$)}.

\subsubsection{Exploration}

When operator selection is decided on the basis of numeric preferences, the decision mechanism should usually choose the operator with the highest numeric preferences, that is, to exploit the present operator selection knowledge.
However, for reinforcement learning to discover the optimal policy, it is necessary that the agent sometimes choose an action that does not have the maximum predicted value.
Such exploration is necessary because actions may be undervalued.
This situation can occur both during the initial learning of a task and as a result of change in the dynamics or reward structure of a task.

The exploration policy is selected and configured using the \soarb{indifferent-selection} command (see Section \ref{indifferent-selection} on page \pageref{indifferent-selection}).
In an effort to maintain backwards compatibility, the default exploration policy is \soarb{softmax}.
However, the first time that the reinforcement learning mechanism is enabled, the architecture changes this policy to \soarb{episilon-greedy} (a more suitable default for RL agents) and issues a message to the trace.

\section{Automatic Generation of RL Rules}

The number of RL rules required for an agent to accurately approximate operator Q-values is usually infeasibly large to write by hand, even for small domains.
Therefore, several methods exist to automate this.

\subsection{The gp Command}
The \soar{gp} command can be used to generate productions based on simple patterns.
This is useful if the states and operators of the environment can be distinguished by a fixed number of dimensions with finite domains.
An example is a grid world where the states are described by integer row/column coordinates, and the available operators are to move north, south, east, or west.
In this case, a single \soar{gp} command will generate all necessary RL rules:
	
\begin{verbatim}
gp {gen*rl*rules
   (state <s> ^name gridworld
              ^operator <o> +
              ^row [ 1 2 3 4 ]
              ^col [ 1 2 3 4 ])
   (<o> ^name move
        ^direction [ north south east west ])
-->
   (<s> ^operator <o> = 0.0)
}
\end{verbatim}
	
For more information see the documentation for this command on page \pageref{gp}.

\subsection{Rule Templates}
\label{RL-templates}

Rule templates allow Soar to dynamically generate new RL rules based on a predefined pattern as the agent encounters novel states.
This is useful when either the domains of environment dimensions are not known ahead of time, or when the enumerable state space of the environment is too large to capture in its entirety using \soar{gp}, but the agent will only encounter a small fraction of that space during its execution.
For example, consider the grid world example with 1000 rows and columns.
Attempting to generate RL rules for each grid cell and action a priori will result in $1000 \times 1000 \times 4 = 4 \times 10^6$ productions.
However, if most of those cells are unreachable due to walls, then the agent will never fire or update most of those productions.
Templates give the programmer the convenience of the \soar{gp} command without filling production memory with unnecessary rules.

Rule templates have variables that are filled in to generate RL rules as the agent encounters novel combinations of variable values.
A rule template is valid if and only if it is marked with the \soarb{:template} flag and, in all other respects, adheres to the format of an RL rule.
However, whereas an RL rule may only use constants as the numeric indifference preference value, a rule template may use a variable.
Consider the following rule template:

\begin{verbatim}
sp {sample*rule*template
    :template
    (state <s> ^operator <o> +
               ^value <v>)
-->
    (<s> ^operator <o> = <v>)
}
\end{verbatim}

During agent execution, this rule template will match working memory and fire like any other rule.
However, the rule firing will not create the numeric indifferent preference on the RHS.
Instead, a new production is created by substituting all variables in the rule template that matched against constant values with the values themselves.
Suppose that the LHS of the rule template matched against the state

\begin{verbatim}
(S1 ^value 3.2)
(S1 ^operator O1 +)
\end{verbatim}

Then the following production will be added to production memory:

\begin{verbatim}
sp {rl*sample*rule*template*1
    (state <s> ^operator <o> +
               ^value 3.2)
-->
    (<s> ^operator <o> = 3.2)
}
\end{verbatim}

The variable \soar{<v>} is replaced by \soar{3.2} on both the LHS and the RHS, but \soar{<s>} and \soar{<o>} are not replaced because they matches against identifiers (\soar{S1} and \soar{O1}).
As with other RL rules, the value of \soar{3.2} on the RHS of this rule may be updated later by reinforcement learning, whereas the value of \soar{3.2} on the LHS will remain unchanged.
If \soar{<v>} had matched against a non-numeric constant, it will be replaced by that constant on the LHS, but the RHS numeric indifference preference value will be set to zero to make the new rule valid.

The new production's name adheres to the following pattern:
\soarb{rl*template-name*id}, where \soarb{template-name} is the name of the originating rule template and \soarb{id} is the smallest positive integer such that the new production's name is unique.

If an identical production already exists in production memory, then the newly generate production is discarded.
It should be noted that the current process of identifying unique template match instances can become quite expensive in long agent runs.
Therefore, it is recommended to generate all necessary RL rules using the \soar{gp} command or via custom scripting when possible.

\subsection{Chunking}
Since RL rules are regular productions, they can be learned by chunking just like any other production.
This method is more general than using the \soar{gp} command or rule templates, and is useful if the environment state consists of arbitrarily complex relational structures that cannot be enumerated.

\subsection{\soarb{save-backtraces}}
\label{save-backtraces}
\index{save-backtraces}
Save trace information to explain chunks and justifications. 
 Priority: 3; Status: Complete, EvilBackDoor
\subsubsection*{Synopsis}
\begin{verbatim}
save-backtraces [-ed]
\end{verbatim}
\subsubsection*{Options}
\begin{tabular}{|l|l|}
\hline 
 -e, --enable, --on  & Turn explain sysparam on.  \\
 \hline 
 -d, --disable, --off  & Turn explain sysparam off.  \\
 \hline 
\end{tabular}
\subsubsection*{Description}
, backtracing information can be retrieved by using the explain-backtraces command. Saving backtracing information may slow down the execution of your Soar program, but it can be a very useful tool in understanding how chunks are formed. 
\subsubsection*{See Also}
\hyperref[explain-backtraces]{explain-backtraces} 
\input{wikicmd/tex/select}
\subsection{\soarb{set-stop-phase}}
\label{set-stop-phase}
\index{set-stop-phase}
Controls the phase where agents stop when running by decision. 
\subsubsection*{Synopsis}
set-stop-phase -[ABadiop] 
\end{verbatim}
\subsubsection*{Options}
 Options -A and -B are optional and mutually exclusive. If not specified, the default is -B. 
 Only one of -a, -d, -i, -o, -p must be selected. 
 With no options, reports the current stop phase. 
\hline
\soar{\soar{ -A, --after }} & Stop after specified phase.  \\
\hline
\soar{\soar{ -B, --before }} & Stop before specified phase (the default).  \\
\hline
\soar{\soar{ -a, --apply }} & Select the apply phase.  \\
\hline
\soar{\soar{ -d, --decision }} & Select the decision phase.  \\
\hline
\soar{\soar{ -i, --input }} & Select the input phase.  \\
\hline
\soar{\soar{ -o, --output }} & Select the output phase.  \\
\hline
\soar{\soar{ -p, --proposal }} & Select the proposal phase.  \\
\hline
\end{tabular}
\subsubsection*{Description}
 When running by decision cycle it can be helpful to have agents stop at a particular point in its execution cycle. This command allows the user to control which phase Soar stops in. The precise definition is that \emph{running for $<$n$>$ decisions and stopping before phase $<$ph$>$}
 means to run until the decision cycle counter has increased by $<$n$>$ and then stop when the next phase is $<$ph$>$. The phase sequence (as of this writing) is: input, proposal, decision, apply, output. Stopping after one phase is exactly equivalent to stopping before the next phase. 
 On initialization Soar defaults to stopping before the input phase (or after the output phase, however you like to think of it). 
 Setting the stop phase applies to all agents. 
\subsubsection*{Examples}
set-stop-phase -Bi                 // stop before input phase
set-stop-phase -Ad                 // stop after decision phase (before apply phase)
set-stop-phase -d                  // stop before decision phase
set-stop-phase --after --output    // stop after output phase
set-stop-phase                     // reports the current stop phase
\end{verbatim}
\subsubsection*{See Also}

\chapter{Semantic Memory}
\label{SMEM}
\index{semantic memory}
\index{smem}


\subsection{\soarb{timers}}
\label{timers}
\index{timers}
Toggle on or off the internal timers used to profile Soar. 
 Status: Complete, EvilBackDoor
\subsubsection*{Synopsis}
\begin{verbatim}
timers [-ed]
\end{verbatim}
\subsubsection*{Options}
\begin{tabular}{|l|l|}
\hline 
 -d, --disable, --off  & Disable all timers.  \\
 \hline 
 -e, --enable, --on  & Enable timers as compiled.  \\
 \hline 
\end{tabular}
\subsubsection*{Description}
 This command is used to control the timers that collect internal profiling information while Soar is running. With no arguments, this command prints out the current timer status. Timers are ENABLED by default. The default compilation flags for soar enable the basic timers and disable the detailed timers. The timers command can only enable or disable timers that have already been enabled with compiler directives. See the stats command for more info on the Soar timing system. 
\subsubsection*{Examples}
 To show how to use the command in context, do this: \begin{verbatim}
command --option arg
\end{verbatim}
 and possibly explain the results. 
\subsubsection*{See Also}
 stats
\subsubsection*{Structured Output:}
\paragraph*{On Query}
\begin{verbatim}
<result>
  <arg name="timers" type="boolean">setting</arg>
</result>
\end{verbatim}
\paragraph*{Otherwise}
\begin{verbatim}
<result output="raw">true</result>
\end{verbatim}
\subsubsection*{Error Values:}
\paragraph*{During Parsing}
 kUnrecognizedOption, kGetOptError, kTooManyArgs
\paragraph*{During Execution}
 kAgentRequired, kKernelRequired

\subsection{\soarb{waitsnc}}
\label{waitsnc}
\index{waitsnc}
\subsubsection*{Synopsis}
\begin{verbatim}
wait -[e|d]
\end{verbatim}
\subsubsection*{Options}
\begin{tabular}{|l|l|}
\hline
\soar{ -e, --enable, --on } & Turns a state-no-change into a \emph{wait}
 state.  \\
\hline
\soar{ -d, --disable, --off } & Default. A state-no-change generates an impasse.  \\
\hline
\end{tabular}
\subsubsection*{Description}
 In some systems, espcially those that model expert (fully chunked) knowledge, a state-no-change may represent a \emph{wait state}
 rather than an impasse. The waitsnc command allows the user to switch to a mode where a state-no-change that would normally generate an impasse (and subgoaling), instead generates a \emph{wait}
 state. At a \emph{wait}
 state, the decision cycle will repeat (and the decision cycle count is incremented) but no state-no-change impasse (and therefore no substate) will be generated. 
 When issued with no arguments, waitsnc returns its current setting. 
 Categories: Command Line Interface

\input{wikicmd/tex/wma}

% ----------------------------------------------------------------------------

\section{File System I/O Commands}
\label{FILE-IO}

This section describes commands which interact in one way or another
with operating system input and output, or file I/O.  Users can
save/retrieve information to/from files, redirect the information
printed by Soar as it runs, and save and load the binary representation
of productions.
The specific commands described in this section are:

\paragraph{Summary}
\begin{quote}
\begin{description}
%\item[command-to-file] - Evaluate a command and print its results to a file.
%\item[\emph{directory functions}] - \soar{cd, dirs, popd, pushd, pwd}
\item[cd] - Change directory.
\item[clog] - Record all user-interface input and output to a file. \emph{(was \soar{log})}
\item[command-to-file] - Dump the printed output and results of a command to a file. 
\item[dirs] - List the directory stack.
\item[echo] -  Print a string to the current output device.
\item[ls] - List the contents of the current working directory.
\item[popd] - Pop the current working directory off the stack and change to the next directory on the stack.
\item[pushd] - Push a directory onto the directory stack, changing to it.
\item[pwd] - Print the current working directory.
\item[rete-net] - Save the current Rete net, or restore a previous one.
\item[set-library-location] - Set the top level directory containing demos/help/etc.
%\item[output-strings-destination] - Redirect the Soar output stream.
\item[source] - Load and evaluate the contents of a file.
\end{description}
\end{quote}

The \textbf{source} command is used for nearly every Soar program.  The
directory functions are important to understand so that users can
navigate directories/folders to load/save the files of interest.  
Soar applications that include a graphical interface or other
simulation environment will often require the use of \textbf{echo}  .


\documentclass[10pt]{article}
\usepackage{fullpage, graphicx, url}
\setlength{\parskip}{1ex}
\setlength{\parindent}{0ex}
\title{Cd - Soar Wiki}
\begin{document}
\section*{Cd}
\subsubsection*{From Soar Wiki}


 This is part of the Soar Command Line Interface. 
\section*{ Name }


 \textbf{cd}
 - Change directory. 


 Status: Complete
\section*{ Synopsis }
\begin{verbatim}
cd [directory]

\end{verbatim}
\section*{ Options }


\begin{tabular}{|c|c|}
\hline 
 directory  & The directory to change to, can be relative or full path.  \\
 \hline 

\end{tabular}



 \\ 

\section*{ Description }


 Change the current working directory. If run with no arguments, returns to the directory that the command line interface was started in, often referred to as the \emph{home}
 directory. 
\section*{ Examples }


 To move to the relative directory named ../home/soar/agents \begin{verbatim}
cd ../home/soar/agents


\end{verbatim}

\section*{ See Also }
\begin{description}
dirs home ls pushd popd source topd

\end{description}
\section*{ Structured Output }
\subsection*{ On Success }
\begin{verbatim}
<result output="raw">true</result>

\end{verbatim}
\subsection*{ Notes }
\section*{ Error Values }
\subsection*{ During Parsing }


 kTooManyArgs
\subsection*{ During Execution }


 kchdirFail Retrieved from ``\url{http://winter.eecs.umich.edu/soarwiki/Cd}``

\end{document}

\subsection{\soarb{clog}}
\label{clog}
\index{clog}
Record all user-interface input and output to a file. 
\subsubsection*{Synopsis}
\begin{verbatim}
clog -[Ae] filename
clog –a string
clog [–cdoq]
\end{verbatim}
\subsubsection*{Options}
\begin{tabular}{|l|l|}
\hline
\soar{ filename } & Open filename and begin logging.  \\
\hline
\soar{ -c, --close, -o, --off, -d, --disable } & Stop logging, close the file.  \\
\hline
\soar{ -a, --add string } & Add the given string to the open log file.  \\
\hline
\soar{ -q, --query } & Returns \emph{open}
 if logging is active or \emph{closed}
 if logging is not active.  \\
\hline
\soar{ -A, --append, -e, --existing } & Opens existing log file named filename and logging is added at the end of the file.  \\
\hline
\end{tabular}
\subsubsection*{Description}
 The \textbf{clog}
 command allows users to save all user-interface input and output to a file. When Soar is logging to a file, everything typed by the user and everything printed by Soar is written to the file (in addition to the screen). 
 Invoke \textbf{clog}
 with no arguments (or with \textbf{-q}
) to query the current logging status. Pass a filename to start logging to that file (relative to the command line interface's home directory (see the home command)). Use the \textbf{close}
 option to stop logging. 
\subsubsection*{Examples}
 To initiate logging and place the record in foo.log: \begin{verbatim}
clog foo.log
\end{verbatim}
 To append log data to an existing foo.log file: \begin{verbatim}
clog -A foo.log
\end{verbatim}
 To terminate logging and close the open log file: \begin{verbatim}
clog -c
\end{verbatim}
\subsubsection*{Known Issues}
 Does not log everything when structured output is selected. 
\subsubsection*{See also}
\hyperref[command-to-file]{command-to-file} 
\subsection{\soarb{command-to-file}}
\label{command-to-file}
\index{command-to-file}
Dump the printed output and results of a command to a file. 
\subsubsection*{Synopsis}
\begin{verbatim}
command-to-file [-a] filename command [args]
\end{verbatim}
\subsubsection*{Options}
\begin{tabular}{|l|l|}
\hline
\soar{ -a, --append } & Append if file exists.  \\
\hline
\soar{ filename } & The file to log the results of the command to  \\
\hline
\soar{ command } & The command to log  \\
\hline
\soar{ args } & Arguments for command  \\
\hline
\end{tabular}
\subsubsection*{Description}
 This command logs a single command. It is almost equivalent to opening a log using clog, running the command, then closing the log, the only difference is that input isn't recorded. 
 Running this command while a log is open is an error. There is currently not support for multiple logs in the command line interface, and this would be an instance of multiple logs. 
 This command echos output both to the screen and to a file, just like clog. 
\subsubsection*{See also}
\hyperref[clog]{clog}  Categories: Command Line Interface

\subsection{\soarb{dirs}}
\label{dirs}
\index{dirs}
List the directory stack 
\subsubsection*{Synopsis}
\begin{verbatim}
dirs
\end{verbatim}
\subsubsection*{Options}
 No options. 
\subsubsection*{Description}
 This command lists the directory stack. Agents can move through a directory structure by pushing and popping directory names. The \textbf{dirs}
 command returns the stack. 
 The command \textbf{pushd}
 places a new ``agent current directory'' on top of the directory stack and cd's to it. The command \textbf{popd}
 removes the directory at the top of the directory stack and cd's to the previous directory which now appears at the top of the stack. 
\subsubsection*{See Also}
\hyperref[cd]{cd} \hyperref[home]{home} \hyperref[ls]{ls} \hyperref[pushd]{pushd} \hyperref[popd]{popd} \hyperref[source]{source} \hyperref[topd]{topd}  Categories: Command Line Interface

\documentclass[10pt]{article}
\usepackage{fullpage, graphicx, url}
\title{Echo - Soar Wiki}
\begin{document}
\section*{Echo}
\subsubsection*{From Soar Wiki}


 This is part of the Soar Command Line Interface. 
\section*{ Name }


 \textbf{echo}
 - Print a string to the current output device. 


 Status: Complete
\section*{ Synopsis }
\begin{verbatim}
echo string

\end{verbatim}
\section*{ Options }


\begin{tabular}{|p{1in}|p{5in}|}
\hline 
 string  & The string to print.  \\
 \hline 

\end{tabular}



 \\ 

\section*{ Description }


 This command echos the args to the current output stream. This is normally stdout but can be set to a variety of channels. If an arg is -nonewline then no newline is printed at the end of the printed strings. Otherwise a newline is printed after printing all the given args. Echo is the easiest way to add user comments or identification strings in a log file. 
\section*{ Examples }


 This example will add these comments to the screen and any open log file. \begin{verbatim}
echo This is the first run with disks = 12

\end{verbatim}

\section*{ See Also }
\begin{description}
log

\end{description}


 \\ 

\section*{ Structured Output }
\subsection*{ On Success }
\begin{verbatim}
<result>
  <arg param="message" type="string">message</arg>
</result>

\end{verbatim}
\subsection*{ Notes }
\section*{ Error Values }
\subsection*{ During Parsing }
\subsection*{ During Execution }


 No errors. 

\end{document}

\subsection{\soarb{ls}}
\label{ls}
\index{ls}
List the contents of the current working directory. 
\subsubsection*{Synopsis}
\begin{verbatim}
ls
\end{verbatim}
\subsubsection*{Options}
 No options. 
\subsubsection*{Description}
 List the contents of the working directory. 
\subsubsection*{Default Aliases}
\begin{tabular}{|l|l|}
\hline
\soar{ Alias } & Maps to  \\
\hline
\soar{ dir } & ls  \\
\hline
\end{tabular}
\subsubsection*{See Also}
\hyperref[cd]{cd} \hyperref[dirs]{dirs} \hyperref[home]{home} \hyperref[pushd]{pushd} \hyperref[popd]{popd} \hyperref[source]{source} \hyperref[topd]{topd}  Categories: Command Line Interface

\documentclass[10pt]{article}
\usepackage{fullpage, graphicx, url}
\title{Popd - Soar Wiki}
\begin{document}
\section*{Popd}
\subsubsection*{From Soar Wiki}


 This is part of the Soar Command Line Interface. 
\section*{ Name }


 \textbf{popd}
 - Pop the current working directory off the stack and change to the next directory on the stack. Can be relative pathname or fully specified path. 


 Status: Complete
\section*{ Synopsis }
\begin{verbatim}
popd

\end{verbatim}
\section*{ Options }


 No options. 
\section*{ Description }
\section*{ Soar8.3 Description }


 This command pops a directory off of the directory stack and cd's to it. See the dirs command for an explanation of the directory stack. 
\section*{ Examples }


 \\ 

\section*{ See Also }
\begin{description}
cd dirs home ls pushd \textbf{popd}
 source topd

\end{description}
\section*{ Structured Output }
\subsection*{ On Success }
\begin{verbatim}
<result output="raw">true</result>

\end{verbatim}
\section*{ Error Values }
\subsection*{ During Parsing }


 kTooManyArgs
\subsection*{ During Execution }


 kDirectoryStackEmpty

\end{document}

\subsection{\soarb{pushd}}
\label{pushd}
\index{pushd}
Push a directory onto the directory stack, changing to it. 
 Complete
\subsubsection*{Synopsis}
\begin{verbatim}
pushd directory
\end{verbatim}
\subsubsection*{Options}
\begin{tabular}{|l|l|}
\hline 
 directory  & Directory to change to, saving the current directory on to the stack.  \\
 \hline 
\end{tabular}
\subsubsection*{Description}
 Maintain a stack of working directories and push the directory on to the stack. Can be relative path name or fully specified. 
\subsubsection*{See Also}
\hyperref[cd]{cd} \hyperref[dirs]{dirs} \hyperref[home]{home} \hyperref[ls]{ls} \hyperref[popd]{popd} \hyperref[source]{source} \hyperref[topd]{topd} 
\subsection{\soarb{pwd}}
\label{pwd}
\index{pwd}
Print the current working directory. 
\subsubsection*{Synopsis}
pwd
\end{verbatim}
\subsubsection*{Options}
 No options. 
\subsubsection*{Description}
 Prints the current working directory of Soar. 
\subsubsection*{Default Aliases}
\hline
\soar{\soar{ Alias }} & Maps to  \\
\hline
\soar{\soar{ topd }} & pwd  \\
\hline
\end{tabular}

\subsection{\soarb{rete-net}}
\label{rete-net}
\index{rete-net}
Save the current Rete net, or restore a previous one. 
 Status: Complete
\subsubsection*{Synopsis}
\begin{verbatim}
rete-net -s|l filename
\end{verbatim}
\subsubsection*{Options}
\begin{tabular}{|l|l|}
\hline 
 -s, --save  & Save the Rete net in the named file. Cannot be saved when there are justifications present. Use excise -j \\
 \hline 
 -l, -r, --load, --restore  & Load the named file into the Rete network. working memory and production memory must both be empty. Use excise -a \\
 \hline 
filename & The name of the file to save or load.  \\
 \hline 
\end{tabular}
\subsubsection*{Description}
 The rete-net command saves the current Rete net to a file or restores a Rete net previously saved. The Rete net is Soar's internal representation of production and working memory; the conditions of productions are reordered and common substructures are shared across different productions. This command provides a fast method of saving and loading productions since a special format is used and no parsing is necessary. Rete-net files are portable across platforms that support Soar. 
 Normally users wish to save only production memory. Note that \emph{justifications}
 cannot be present when saving the Rete net. Issuing an init-soar before saving a Rete net will remove all justifications and working memory elements. \\ 
 If the filename contains a suffix of ``.Z'', then the file is compressed automatically when it is saved and uncompressed when it is loaded. Compressed files may not be portable to another platform if that platform does not support the same uncompress utility. 
\subsubsection*{See Also}
\hyperref[excise]{excise} \hyperref[init-soar]{init-soar} 
\documentclass[10pt]{article}
\usepackage{fullpage, graphicx, url}
\title{Set-library-location - Soar Wiki}
\begin{document}
\section*{Set-library-location}
\subsubsection*{From Soar Wiki}


 This is part of the Soar Command Line Interface. 
\section*{ Name }


 \textbf{set-library-location}
 - Set the top level directory containing demos/help/etc.\\ 
 Status: Complete
\section*{ Synopsis }
\begin{verbatim}
set-library-location [directory] 

\end{verbatim}
\section*{ Options }


\begin{tabular}{|p{1in}|p{5in}|}
\hline 
 directory  & The new desired library location.  \\
 \hline 

\end{tabular}



 \\ 

\section*{ Description }


 Invoke with no arguments to query what the current library location is. The library location should contain at least the help/ subdirectory and the command-names file for help to work. 
\section*{ Examples }
\begin{verbatim}
directory "c:\Documents and Settings\User\My Documents\Soar\SoarIO\bin"
directory /usr/local/share/soar/library

\end{verbatim}
\section*{ See Also }


 help
\section*{ Structured Output }
\subsection*{ On Success }


 Returns true on success, or kParamDirectory, kTypeString element with the directory on query. 
\section*{ Error Values }
\subsection*{ During Parsing }


 kTooManyArgs
\subsection*{ During Execution }


 No errors. 

\end{document}

\subsection{\soarb{source}}
\label{source}
\index{source}
Load and evaluate the contents of a file. 
 Status: Complete
\subsubsection*{Synopsis}
\begin{verbatim}
source filename
\end{verbatim}
\subsubsection*{Options}
\begin{tabular}{|l|l|}
\hline 
filename & The file of Soar productions and commands to load.  \\
 \hline 
\end{tabular}
\subsubsection*{Description}
 Load and evaluate the contents of a file. The \emph{filename}
\subsubsection*{See Also}


% ***************************************************************************
% ----------------------------------------------------------------------------
\section{Soar I/O Commands}
\label{SOAR-IO}

This section describes the commands used to manage Soar's Input/Output
(I/O) system, which provides a mechanism for allowing Soar to interact 
with external systems, such as a computer game environment or a robot.  

Soar I/O functions make calls to \soar{add-wme} and \soar{remove-wme}
to add and remove elements to the \textbf{io} structure of Soar's working
memory. 
 
The specific commands described in this section are:

\paragraph{Summary}
\begin{quote}
\begin{description}
\item[add-wme] - Manually add an element to working memory.
\item[capture-input] - Store the input wmes in a file for reloading later.
\item[remove-wme] - Manually remove an element from working memory.
\item[replay-input] - Load input wmes for each decision cycle from a file.
\end{description}
\end{quote}

These commands are used mainly  when Soar needs to interact with an
external environment.  Users might take advantage of these commands when
debugging agents, but care should be used in adding and removing wmes this
way as they do not fall under Soar's truth maintenance system.

\subsection{\soarb{add-wme}}
\label{add-wme}
\index{add-wme}
Manually add an element to working memory. 
 Status: Complete, EvilBackDoor
\subsubsection*{Synopsis}
  \begin{verbatim}
add-wme id [^]attribute value [+]
\end{verbatim}
\subsubsection*{Options}
\begin{tabular}{|l|l|}
\hline 
 id  & Must be an existing identifier.  \\
 \hline 
 \^{}  & Leading \^{} on attribute is optional.  \\
 \hline 
 attribute  & Attribute can be any Soar symbol. Use * to have Soar create a new identifier.  \\
 \hline 
 value  & Value can be any soar symbol. Use * to have Soar create a new identifier.  \\
 \hline 
 +  & If the optional preference is specified, its value must be + (acceptable).  \\
 \hline 
\end{tabular}
\subsubsection*{Description}
\subsubsection*{Examples}
 This example adds the attribute/value pair ``message-status received'' to the identifier (symbol) S1: \begin{verbatim}
 add-wme S1 ^message-status received
\end{verbatim}
 This example adds an attribute/value pair with an acceptable preference to the identifier (symbol) Z2. The attribute is ``message'' and the value is a unique identifier generated by Soar. Note that since the \^{} is optional, it has been left off in this case. \begin{verbatim}
 add-wme Z2 message * + 
\end{verbatim}
\subsubsection*{Warnings}
\subsubsection*{See Also}
\hyperref[remove-wme]{remove-wme} 
\input{wikicmd/tex/capture-input}
\subsection{\soarb{remove-wme}}
\label{remove-wme}
\index{remove-wme}
Manually remove an element from working memory. 
 Complete
\subsubsection*{Synopsis}
\begin{verbatim}
remove-wme \emph{timetag}
\end{verbatim}
\subsubsection*{Options}
\begin{tabular}{|l|l|}
\hline 
 timetag  & A positive integer matching the timetag of an existing working memory element.  \\
 \hline 
\end{tabular}
\subsubsection*{Description}
 The remove-wme command removes the working memory element with the given timetag. This command is provided primarily for use in Soar input functions; although there is no programming enforcement, remove-wme should only be called from registered input functions to delete working memory elements on Soar's input link. 
 Beware of weird side effects, including system crashes. 
\subsubsection*{See Also}
\hyperref[add-wme]{add-wme} \subsubsection*{Warnings}
 remove-wme should never be called from the RHS: if you try to match a wme on the LHS of a production, and then remove the matched wme on the RHS, Soar will crash. 
 If used other than by input and output functions interfaced with Soar, this command may have weird side effects (possibly even including system crashes). Removing input wmes or context/impasse wmes may have unexpected side effects. You've been warned. 

\input{wikicmd/tex/replay-input}

% ***************************************************************************
% ----------------------------------------------------------------------------
\section{Miscellaneous}
\label{MISC}


\comment{this section still needs to be rewritten...}

\nocomment{This section describes the commands used to inspect production memory,
working memory, and preference memory; to see what productions will 
match and fire in the next Propose or Apply phase;  and to examine the 
goal dependency set.  These commands are particularly useful when
running or debugging Soar, as they let users see what Soar is ``thinking.''}
The specific commands described in this section are:


\paragraph{Summary}
\begin{quote}
\begin{description}
\item[alias] - Define a new alias, or command, using existing commands and arguments.
\item[allocate] - Allocate additional 32 kilobyte blocks of memory for a specified memory pool without running Soar.
\item[echo-commands] - Set whether or not commands are echoed to other connected debuggers. 
\item[edit-production] - Fire event to Move focus in an open editor to this production.
\item[load-library] - Load a shared library into the local client
\item[port] - Returns the port the kernel instance is listening on.
\item[rand] - Generate a random number.
\item[soarnews] - Prints information about the current release.
\item[srand] -  Seed the random number generator.
\item[time] - Uses a default system clock timer to record the wall time required while executing a command.
\item[unalias] - Remove an existing alias.
\item[version] - Returns version number of Soar kernel.
\end{description}
\end{quote}

\subsection{\soarb{alias}}
\label{alias}
\index{alias}
Define a new alias, or command, using existing commands and arguments. 
\subsubsection*{Synopsis}
\begin{verbatim}
alias name [cmd <args>]
alias -d name
alias
\end{verbatim}
\subsubsection*{Options}
\begin{tabular}{|l|l|}
\hline
\soar{ -d, --disable, --off } & Remove the named alias.  \\
\hline
\soar{ name } & The name of the alias, i.e. the new command.  \\
\hline
\soar{ cmd } & An existing command that will be invoked when the alias is entered on the commandline.  \\
\hline
\soar{ args } & Valid arguments to the cmd (optional \& optional number).  \\
\hline
\end{tabular}
\subsubsection*{Description}
 This command defines new aliases by creating Soar procedures with the given name. The new procedure can then take an arbitrary number of arguments which are post-pended to the given definition and then that entire string is executed as a command. The definition must be a single command, multiple commands are not allowed. The \textbf{alias}
 procedure checks to see if the name already exists, and does not destroy existing procedures or aliases by the same name. Existing aliases can be removed by using the \textbf{-d}
 flag. With no arguments, \textbf{alias}
 returns the list of defined aliases. With only the name given, \textbf{alias}
 returns the current definition. 
\subsubsection*{Examples}
 The alias \emph{wmes}
 is defined as: \begin{verbatim}
alias wmes print -i
\end{verbatim}
 If the user executes a command such as: \begin{verbatim}
wmes {(* ^superstate nil)}
\end{verbatim}
 it is as if the user had typed this command: \begin{verbatim}
print -i {(* ^superstate nil)}
\end{verbatim}
 To check what a specific alias is defined as, you would type \begin{verbatim}
alias wmes
\end{verbatim}
\subsubsection*{Default Aliases}
\begin{tabular}{|l|l|}
\hline
\soar{ Alias } & Maps to  \\
\hline
\soar{ a } & alias  \\
\hline
\soar{ un } & alias -d  \\
\hline
\soar{ unalias } & alias -d  \\
\hline
\end{tabular}
\subsubsection*{See Also}
\hyperref[unalias]{unalias} 
\input{wikicmd/tex/allocate}
\input{wikicmd/tex/echo-commands}
\subsection{\soarb{edit-production}}
\label{edit-production}
\index{edit-production}
Move focus in an editor to this production. 
\subsubsection*{Synopsis}
edit-production production_name
\end{verbatim}
\subsubsection*{Options}
 production\_name The name of the production to edit. 
\subsubsection*{Description}
 If an editor (currently limited to Visual Soar) is open and connected to Soar, this command causes the editor to open the file containing this production and move the cursor to the start of the production. If there is no editor connected to Soar, the command does nothing. In order to connect Visual Soar to Soar, launch Visual Soar and choose Connect from the Soar Runtime menu. Then open the Visual Soar project that you're working on. At that point, you're set up and edit-production will start to work. 
\subsubsection*{Examples}
edit-production my*production*name
\end{verbatim}
\subsubsection*{See Also}
\hyperref[sp]{sp} 
\input{wikicmd/tex/load-library}
\input{wikicmd/tex/port}
\input{wikicmd/tex/rand}
\subsection{\soarb{srand}}
\label{srand}
\index{srand}
Seed the random number generator. 
\subsubsection*{Synopsis}
srand [seed]
\end{verbatim}
\subsubsection*{Options}
\hline
\soar{\soar{\soar{ seed }}} & Random number generator seed.  \\
\hline
\end{tabular}
\subsubsection*{Description}
 Seeds the random number generator with the passed seed. Calling srand without providing a seed will seed the generator based on the contents of /dev/urandom (if available) or else based on time() and clock() values. 
\subsubsection*{Examples}
srand 0
\end{verbatim}
\subsubsection*{See Also}

\subsection{\soarb{soarnews}}
\label{soarnews}
\index{soarnews}
 Priority: 4; Status: Incomplete\\ 
Decide what to put in the news and where to read it from.--Jonathan 15:42, 23 Feb 2005 (EST) 
\subsubsection*{Synopsis}
\subsubsection*{Options}
\begin{tabular}{|l|l|}
\hline 
 & \\
 \hline 
 & \\
 \hline 
\end{tabular}
\subsubsection*{Description}

\subsection{\soarb{time}}
\label{time}
\index{time}
Use a default system clock timer to record the wall time required while executing a command. 
\subsubsection*{Synopsis}
time command [arguments]
\end{verbatim}
\subsubsection*{Options}
\hline
\soar{\soar{ command }} & The command to execute.  \\
\hline
\soar{\soar{ arguments }} & Optional command arguments.  \\
\hline
\end{tabular}
\subsubsection*{Description}

\subsection{\soarb{unalias}}
\label{unalias}
\index{unalias}
Undefine an existing alias 
\subsubsection*{Synopsis}
unalias name
\end{verbatim}
\subsubsection*{Options}
 No options. 
\subsubsection*{Description}
 This command undefines a previously created alias. This command takes exactly one argument: the name of the alias to remove. Use the alias command by itself to list all defined aliases. 
\subsubsection*{Examples}
unalias varprint
\end{verbatim}
\subsubsection*{Default Aliases}
\hline
\soar{\soar{ Alias }} & Maps to  \\
\hline
\soar{\soar{un}} &\textbf{unalias}
\hline
\end{tabular}
\subsubsection*{See Also}
\hyperref[alias]{alias} 
\subsection{\soarb{version}}
\label{version}
\index{version}
\subsubsection*{Synopsis}
\begin{verbatim}
 version
\end{verbatim}
\subsubsection*{Options}
 No options 
\subsubsection*{Description}
 This command gives version information about the current Soar kernel. It returns the version number and build date which can then be stored by the agent or the application. 
 Categories: Command Line Interface

