\documentclass[10pt]{article}
\usepackage{fullpage, graphicx, url}
\title{Learn - Soar Wiki}
\begin{document}
\section*{Learn}
\subsubsection*{From Soar Wiki}


 This is part of the Soar Command Line Interface. 
\section*{ Name }


 \textbf{learn}
 - Set the parameters for chunking, Soar\^a��s learning mechanism. 


 Status: Complete, EvilBackDoor
\section*{ Synopsis }
\begin{verbatim}
learn [-l]
learn -[d|E|o]
learn -e [ab]

\end{verbatim}
\section*{ Options }


\begin{tabular}{|p{1in}|p{5in}|}
\hline 
 -a, --all-levels  & Build chunks whenever a subgoal returns a result. Learning must be --enabled.  \\
 \hline 
 -b, --bottom-up  & Build chunks only for subgoals that have not yet had any subgoals with chunks built. Learning must be --enabled.  \\
 \hline 
 -d, --disable, --off  & Turn all chunking off.  \\
 \hline 
 -e, --enable, --on  & Turn chunking on. Can be modified by -a or -b  \\
 \hline 
 -E, --except  & Learning is on, except as specified by RHS \emph{dont-learn}
 actions.  \\
 \hline 
 -l, --list  & Prints listings of dont-learn and force-learn states.  \\
 \hline 
 -o, --only  & Chunking is on only as specified by RHS \emph{force-learn}
 actions.  \\
 \hline 

\end{tabular}



 \\ 

\section*{ Description }


 The learn command controls the parameters for chunking (Soar's learning mechanism). With no arguments, this command prints out the current learning environment status. If arguments are provided, they will alter the learning environment as described in the options and arguments table. The watch command can be used to provide various levels of detail when productions are learned. Learning is \textbf{disabled}
 by default. 


 \\ 

\section*{ Examples }


 To enable learning only at the lowest subgoal level: \begin{verbatim}
learn -e b 

\end{verbatim}



 To see all the \emph{force-learn}
 and \emph{dont-learn}
 states registered by RHS actions \begin{verbatim}
learn -l

\end{verbatim}

\section*{ See Also }
\begin{description}
watch, explain-backtraces, save-backtraces

\end{description}


 \\ 

\section*{ Structured Output }
\subsection*{ On Query }


 If learning is on: \begin{verbatim}
<result>
  <arg param="learnsetting" type="boolean">true</arg>
  <arg param="learnonlysetting" type="boolean">setting</arg>
  <arg param="learnexceptsetting" type="boolean">setting</arg>
  <arg param="learnalllevelssetting" type="boolean">setting</arg>
</result>

\end{verbatim}



 If learning is off: \begin{verbatim}
<result>
  <arg param="learnsetting" type="boolean">false</arg>
</result>

\end{verbatim}

\subsection*{ On List }


 When the list flag is issued, the results of a query are returned plus the following two arg tags: \begin{verbatim}
<arg param="learnforcelearnstates" type="string">string</arg>
<arg param="learndontlearnstates" type="string">string</arg>

\end{verbatim}

\subsection*{ Otherwise }
\begin{verbatim}
<result output="raw">true</result>

\end{verbatim}
\subsection*{ Notes }
\begin{itemize}
\item  Setting is true (learning enabled) or false (disabled). 
\item  learnalllevelssetting true means all-levels enabled, false means bottom-up 

\end{itemize}
\section*{ Error Values }
\subsection*{ During Parsing }


 kUnrecognizedOption, kGetOptError, kTooManyArgs
\subsection*{ During Execution }


 kAgentRequired

\end{document}
