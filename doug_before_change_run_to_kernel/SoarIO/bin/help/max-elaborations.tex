\documentclass[10pt]{article}
\usepackage{fullpage, graphicx, url}
\title{Max-elaborations - Soar Wiki}
\begin{document}
\section*{Max-elaborations}
\subsubsection*{From Soar Wiki}


 This is part of the Soar Command Line Interface. 
\section*{ Name }


 \textbf{max-elaborations}
 - Limit the maximum number of elaboration cycles in a given phase. Print a warning message if the limit is reached during a run. 


 Status: Complete
\section*{ Synopsis }
\begin{verbatim}
max-elaborations [n]

\end{verbatim}
\section*{ Options }


\begin{tabular}{|p{1in}|p{5in}|}
\hline 
\emph{n}
 & Maximum allowed elaboration cycles, must be a positive integer.  \\
 \hline 

\end{tabular}



 \\ 

\section*{ Description }


 This command sets and prints the maximum number of elaboration cycles allowed. If \emph{n}
 is given, it must be a positive integer and is used to reset the number of allowed elaboration cycles. The default value is 100. max-elaborations with no arguments prints the current value. 


 max-elaborations controls the maximum number of elaborations allowed in a single decision cycle. The elaboration phase will end after \emph{max-elaboration}
 cycles have completed, even if there are more productions eligible to fire or retract; and Soar will proceed to the next phase after a warning message is printed to notify the user. 


 In Soar8, max-elaborations is checked during both the Propose Phase and the Apply Phase. If Soar8 runs more than the max-elaborations limit in either of these phases, Soar8 proceeds to the next phase (either Decision or Output) even if quiescence has not been reached. 
\section*{ Examples }


 The command issued with no arguments, returns the max elaborations allowed: \begin{verbatim}
max-elaborations 

\end{verbatim}



 to set the maximum number of elaborations in one phase to 50: \begin{verbatim}
max-elaborations 50

\end{verbatim}

\section*{ See Also }


 Use the Ab button above the editing window to make local links to other commands. \\ 

\section*{ Structured Output }
\subsection*{ On Query }
\begin{verbatim}
<result>
  <arg name="value" type="int">max_elaborations</arg>
</result>

\end{verbatim}
\subsection*{ Otherwise }
\begin{verbatim}
<result output="raw">true</result>

\end{verbatim}
\section*{ Error Values }
\subsection*{ During Parsing }


 kTooManyArgs, kIntegerMustBePositive
\subsection*{ During Execution }


 kAgentRequired

\end{document}
