% ----------------------------------------------------------------------------
\typeout{--------------- appendix: evaluation of PREFERENCES -----------------}
\chapter{The Resolution of Operator Preferences}
\label{PREFERENCES}
\index{preferences}
% This is a technical discussion of the filtering done to evaluate preferences;
% it might belong in a different version of the manual, but not 492

\comment{what's not clear in the following discussion is what happens in the
	usual case, that is, when there's a single acceptable preference.}

During the decision phase, operator preferences are evaluated in a sequence 
of eight steps, in an effort to select a single operator. 
Each step handles a specific type of preference, as illustrated in Figure 
\ref{fig:prefsem}. (The figure should be read starting at the top
where all the operator preferences are collected and passed into the procedure. At
each step, the procedure either exits through a arrow to the right, or passes to 
the next step through an arrow to the left.)

Input to the procedure are the set of current operator preferences, and the output
consists of:
\begin{enumerate}
\item a subset of the candidate operators, either the empty set, a set consisting of a single, 
winning candidate, or a larger set of candidates that may be conflicting,
tied, or indifferent.
\item an impasse-type, possibly NONE\_IMPASSE\_TYPE.
\end{enumerate}
The procedure has several potential exit points. Some occur when the procedure
has detected a particular type of impasse. The others occur when the number of
candidates has been reduced to 
one (necessarily the winner) or zero (a no-change impasse).

\nocomment{
There are nine filter-like operations involved in evaluating the preferences
available for a particular identifier and attribute. These filters are
executed in a specific order to determine the correct values for the working
memory augmentation, as illustrated in Figure \ref{fig:prefsem}. (The figure
should be read starting at the top left where all the values for an attribute
are collected and passed to the first filter.) Each filter reduces the number
of preferences that need to be considered. If a conflict is found, then an
impasse is generated and the filtering process is halted. The impasse
generation is handled as a special exit from a filter and is indicated with a
grey line in Figure \ref{fig:prefsem}.

The preference semantics module takes as input one or more preferences for a
given identifier and attribute; its output includes: \vspace{-10pt}
\begin{enumerate}
\item a possibly empty set of candidate augmentations that may be conflicting,
	indifferent, or parallel\vspace{-10pt}
\item possibly, a flag, \soar{number\_of\_winners}, created only if there is
	not an impasse (the candidates are not conflicting)\vspace{-10pt}
\item possibly, the creation of an impasse object in working memory (if the
	candidates are conflicting)
\end{enumerate}

If a single winner is chosen or there are a set of mutually indifferent
winners, number\_of\_winners is 1. If all of the winners are mutually
parallel, then number\_of\_winners is All.  This allows the decision procedure
to distinguish a set of mutually parallel candidates that should all be
installed in working memory from a set of mutually indifferent candidates that
should have only one value installed or maintained.
}

\index{decision!procedure}

\begin{figure}
\insertfigure{newprefsem}{7in}
\insertcaption{An illustration of the preference resolution process. There are eight
	steps; only five of these provide exits from the  resolution process.}
\label{fig:prefsem}
\end{figure}

Each step in Figure \ref{fig:prefsem} is described below:

\index{preference!require}
\index{require preference}
\index{"!}
\index{constraint-failure impasse}
\begin{description}
\item[RequireTest (!)]
This test checks for required candidates in preference memory and
also constraint-failure impasses involving require preferences (see
Section \ref{ARCH-impasses} on page \pageref{ARCH-impasses}).

\begin{itemize}
\item If there is exactly one candidate operator with a require preference and
	that candidate does not have a prohibit preference, then that candidate
	is the winner and preference semantics terminates.
\item Otherwise ---
	If there is more than one required candidate, then a constraint-
	failure impasse is recognized and preference semantics terminates 
	by returning the set of required candidates.
\item Otherwise ---
	If there exists a required candidate that is also prohibited, a
	constraint-failure impasse with the required/prohibited value is
	recognized and preference semantics terminates.
\item Otherwise ---
	The candidates are passed to AcceptableCollect.
\end{itemize}

\item[AcceptableCollect (+) ] This operation builds a list of operators
	for which there is an acceptable preference in preference memory.
	This list of candidate operators is passed to the ProhibitFilter.\index{+}
\nocomment{
\begin{itemize}
\item If there are no acceptable preferences in memory for the value of an
	attribute then exit preference semantics with no items picked. 
	(This is an efficiency termination, and does not apply to other filters.)
\item Otherwise ---
	The candidates are passed to the ProhibitFilter.
\end{itemize}
}
\index{preference!acceptable}
\index{acceptable preference}


\item[ProhibitFilter ($\sim$) ] This filter removes the candidates that
	have prohibit preferences in memory. The rest of the candidates are passed to
	the RejectFilter.
\index{preference!prohibit}
\index{prohibit preference}
\index{~}

\item[RejectFilter ($-$) ] This filter removes the candidates that have
	reject preferences in memory. 
	\begin{itemize}
	\item At this point, if the set of remaining candidates is either empty or has one
	member, preference semantics terminates and this set is returned.
	\item Otherwise, the remaining candidates are passed to the
	BetterWorseFilter.
	\end{itemize}
\index{preference!reject}
\index{reject preference}
\index{-}

\item[BetterWorseFilter ($>$), ($<$) ] This filter checks for better
	worse conflicts, and otherwise filters out candidates based on
	better and worse preferences.
\begin{itemize}
\item A better/worse conflict occurs when one candidate has both better and
	worse preferences with respect to another operator (i.e., \textbf{A $<$ B} 
	\& \textbf{B $<$ A}). Since preferences are not transitive, the situation 
	\textbf{A $<$ B $<$ C $<$ A} is not a conflict. If there are better/worse conflicts, 
	preference semantics terminates by declaring a conflict impasse and returning the
	set of conflicted items.
\item Otherwise ---
	Filter out of the candidates the ones that have
	another candidate that is better, or are worse than another candidate. 
	The resulting candidates are passed to the BestFilter.
\end{itemize}
\index{preference!worse}
\index{worse preference}
\index{preference!better}
\index{better preference}
\index{<}
\index{>}


\item[BestFilter ($>$) ] If some remaining candidate has a best preference,
	this filter removes any candidates that do not have
	a best preference. If there are no best preferences for any of the current
	candidates, the filter has no effect. The remaining candidates are passed
	to the WorstFilter.
\index{preference!best}
\index{best preference}

\item[WorstFilter ($<$) ] If all remaining candidates have worst preferences, this filter
	has no effect. Otherwise, the filter removes any candidates that have
	a worst preference.
	\begin{itemize}
	\item Once again, if the set of remaining candidates is either empty or has one
	member, preference semantics terminates and this set is returned.
	\item Otherwise, the remaining candidates are passed to the
	IndifferentTest.
	\end{itemize}
\index{preference!worst}
\index{worst preference}

\index{=}
\item[IndifferentTest (=) ] This operation traverses the remaining candidates and marks 
	each candidate for which one of the following is true:
	\begin{itemize}
	\item the candidate has a unary indifferent preference
	\item the candidate has a numeric indifferent preference
	\item the candidate is binary indifferent to all of the remaining candidate operators
	\end{itemize}
	If some candidate is left unmarked, then the procedure signals a tie impasse and returns 
	the complete set of candidates that passed into the IndifferentTest. Otherwise, the candidates
	are mutually indifferent, in which case an operator is chosen according to the method set
	by the \textbf{indifferent-selection} command, described on page \pageref{indifferent-selection}.
\nocomment{
This filter returns them as indifferent. 
	If they are not, and this is an operator, a tie impasse is
	declared and preference semantics terminates.
\index{preference!indifferent}
\index{indifferent preference}
	\comment{that makes no sense....also, "calling routine" below}
\begin{itemize}
\item If the candidates are all mutually indifferent, terminate preference
	semantics with all of the candidates in the item set and set
	number\_of\_winners to one. The calling routine then selects one
	of the indifferent candidates.
\item Otherwise ---
	If the non-mutually indifferent candidates are for operators, generate
	a tie impasse, with all of the candidates as items, not just those
	that are not mutually indifferent.
\item Otherwise ---
	The candidates are passed to the ParallelFilter.
\end{itemize}


\index{&}
\item[ParallelFilter (\&) ] If all of the remaining candidates are mutually
	parallel, they're returned as mutually parallel. Otherwise, a tie
	impasse is returned.  A candidate is mutually parallel to the set of
	other candidates if it has a unary parallel preference, or if it has
	either of the two possible 
	relative parallels between each candidate.
\index{preference!parallel}
\index{parallel preference}
\begin{itemize}
\item If all of the candidates are mutually parallel
	terminate preference semantics with all of 
	the candidates in the item set and set number\_of\_winners to All.
\item Otherwise ---
	Generate a tie impasse with all the non-mutually parallel
	candidates in the item set.
\end{itemize}
}
\end{description}

