\subsection{\soarb{firing-counts}}
\label{firing-counts}
\index{firing-counts}
Print the number of times each production has fired. 
\subsubsection*{Synopsis}
firing-counts [n]
firing-counts production_names
\end{verbatim}
\subsubsection*{Options}
 If given, an option can take one of two forms -- an integer or a list of production names: 
\hline
\emph{n}
 & List the top \emph{n}
 productions. If \emph{n}
 is 0, only the productions which haven't fired are listed  \\
\hline
\soar{\soar{\soar{\soar{ production\_name }}}} & For each production in production\_names, print how many times the production has fired  \\
\hline
\end{tabular}
\subsubsection*{Description}
 The \textbf{firing-counts}
 command prints the number of times each production has fired; production names are given from most requently fired to least frequently fired. With no arguments, it lists all productions. If an integer argument, \textbf{n}
, is given, only the top \emph{n}
 productions are listed. If \textbf{n}
 is zero (0), only the productions that haven't fired at all are listed. If one or more production names are given as arguments, only firing counts for these productions are printed. 
 Note that firing counts are reset by a call to \textbf{init-soar}
. 
\subsubsection*{Examples}
firing-counts 10
\end{verbatim}
firing-counts my*first*production my*second*production
\end{verbatim}
\subsubsection*{Warnings}
 Firing-counts are reset to zero after an init-soar. 
 NB: This command is slow, because the sorting takes time O(n*log n) 
\subsubsection*{Default Aliases}
\hline
\soar{\soar{\soar{\soar{ Alias }}}} & Maps to  \\
\hline
\soar{\soar{\soar{\soar{ fc }}}} & firing-counts  \\
\hline
\end{tabular}
\subsubsection*{See Also}
\hyperref[init-soar]{init-soar} 