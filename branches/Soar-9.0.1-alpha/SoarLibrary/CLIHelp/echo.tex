\subsection{\soarb{echo}}
\label{echo}
\index{echo}
Print a string to the current output device. 
\subsubsection*{Synopsis}
\begin{verbatim}
echo string
\end{verbatim}
\subsubsection*{Options}
\begin{tabular}{|l|l|}
\hline
\soar{ string } & The string to print.  \\
\hline
\end{tabular}
\subsubsection*{Description}
 This command echos the args to the current output stream. This is normally stdout but can be set to a variety of channels. If an arg is --nonewline then no newline is printed at the end of the printed strings. Otherwise a newline is printed after printing all the given args. Echo is the easiest way to add user comments or identification strings in a log file. 
\subsubsection*{Examples}
 This example will add these comments to the screen and any open log file. \begin{verbatim}
echo This is the first run with disks = 12
\end{verbatim}
\subsubsection*{See Also}
\hyperref[clog]{clog} 