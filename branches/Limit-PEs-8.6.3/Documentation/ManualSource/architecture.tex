% ----------------------------------------------------------------------------
\typeout{--------------- The Soar ARCHitecture ------------------------------}
\chapter{The Soar Architecture}
\label{ARCH}

This chapter describes the Soar architecture.  It covers all aspects of Soar
except for the specific syntax of Soar's memories and descriptions of the
Soar user-interface commands.

This chapter gives an abstract description of Soar.  It starts by giving
an overview of Soar and then goes into more detail for each of Soar's
main memories (working memory, production memory, and preference memory)
and processes (the decision procedure, learning, and input and output).

% ----------------------------------------------------------------------------
\section{An Overview of Soar}
\label{ARCH-overview}

The design of Soar is based on the hypothesis that all deliberate
\textit{goal}-oriented behavior can be cast as the selection and application
of \textit{operators} to a \textit{state}. A state is a representation of the
current problem-solving situation; an operator transforms a state (makes
changes to the representation); and a goal is a desired outcome of the
problem-solving activity.

As Soar runs, it is continually trying to apply the current operator and
select the next operator (a state can have only one operator at a time),
until the goal has been achieved. The selection and application of
operators is illustrated in Figure \ref{fig:select-apply}. 

\begin{figure}
\insertfigure{select-apply}{1.5in}
\insertcaption{Soar is continually trying to select and apply operators.}
\label{fig:select-apply}
\end{figure}

Soar has separate memories (and different representations) for
descriptions of its current situation and its long-term knowledge.  In
Soar, the current situation, including data from sensors, results of
intermediate inferences, active goals, and active operators is held in
\emph{working memory}.  Working memory is organized as
\emph{objects}. Objects are described in terms of their
\emph{attributes}; the values of the attributes may correspond to
sub-objects, so the description of the state can have a hierarchical
organization. (This need not be a strict hierarchy; for example, there's
nothing to prevent two objects from being ``substructure'' of each
other.)

The long-term knowledge, which specifies how to respond to different
situations in working memory, can be thought of as the program for Soar.
The Soar architecture cannot solve any problems without the addition of
long-term knowledge.  (Note the distinction between the ``Soar
architecture'' and the ``Soar program'': The former refers to the system
described in this manual, common to all users, and the latter refers to
knowledge added to the architecture.)

A Soar program contains the knowledge to be used for solving a specific
task (or set of tasks), including information about how to select and
apply operators to transform the states of the problem, and a means of
recognizing that the goal has been achieved.  

\subsection{Problem-Solving Functions in Soar}
\label{ARCH-functions}
\index{problem solving!functions}

\label{LIST:5functions}
All of Soar's long-term knowledge is organized around the functions of operator 
selection and operator application, which are organized into  four distinct types of
knowledge:\vspace{-10pt} 
\begin{description}
\item \textbf{Knowledge to select an operator} \\ \vspace{-20pt}
  \begin{enumerate}
  \item \textit{Operator Proposal:} Knowledge that an operator is appropriate for the current situation. 
  \item \textit{Operator Comparison:} Knowledge to compare candidate operators.
  \item 	\textit{Operator Selection:} Knowledge to select a single operator, based on the comparisons.
  \end{enumerate}
\item \textbf{Knowledge to apply an operator} \\  \vspace{-20pt}
  \begin{enumerate} \setcounter{enumi}{3}
  \item \textit{Operator Application:}
Knowledge of how a specific operator modifies the state.
  \end{enumerate}
\end{description}

In addition, there is a fifth type of knowledge in Soar that is
indirectly connected to both operator selection and operator application:\vspace{-10pt}
\begin{description}
\begin{enumerate}
\setcounter{enumi}{4}
\item Knowledge of monotonic inferences that can be made about the state
(\textit{state elaboration}). 
\end{enumerate}
\end{description}
State elaborations indirectly affect operator selection and application
by creating new descriptions of the current situation that can cue
the selection and application of operators.

\index{production}
\index{production!match}
\index{match|see{production!match}}
\index{fire|see{production!firing}}
\index{retract|see{production!retraction}}

These problem-solving functions are the primitives for generating behavior
in Soar.  Four of the functions require retrieving long-term knowledge
that is relevant to the current situation: elaborating the state,
proposing candidate operators, comparing the candidates, and applying
the operator by modifying the state. These functions are driven by the
knowledge encoded in a Soar program.  Soar represents that knowledge as
\textit{production} rules.  Production rules are similar to ``if-then''
statements in conventional programming languages. (For example, a
production might say something like ``if there are two blocks on the
table, then suggest an operator to move one block ontop of the other
block'').  The ``if'' part of the production is called its
\textit{conditions} and the ``then'' part of the production is called
its \textit{actions}. When the conditions are met in the current
situation as defined by working memory, the production is \emph{matched}
and it will \emph{fire}, which means that its actions are executed,
making changes to working memory. 

The other function, selecting the current operator, involves making a
decision once sufficient knowledge has been retrieved.  This is
performed by Soar's \emph{decision procedure}, which is a fixed
procedure that interprets \emph{preferences} that have been created by
the retrieval functions. The knowledge-retrieval and decision-making
functions combine to form Soar's \emph{decision cycle}.

\index{decision procedure}
\index{decision cycle}
\index{impasse}

When the knowledge to perform the problem-solving functions is not
directly available in productions, Soar is unable to make progress and
reaches an \textit{impasse}.  There are three types of possible impasses
in Soar:
\begin{enumerate}
\item An operator cannot be selected because none are proposed.\vspace{-4pt}
\item An operator cannot be selected because multiple operators are
        proposed and the comparisons are insufficient to determine which
        one should be selected.\vspace{-4pt}
\item An operator has been selected, but there is insufficient knowledge
        to apply it.\vspace{-4pt}
\end{enumerate}
In response to an impasse, the Soar architecture creates a
\textit{substate} in which operators can be selected and applied to
generate or deliberately retrieve the knowledge that was not directly
available; the goal in the substate is to resolve the impasse. For
example, in a substate, a Soar program may do a lookahead search to
compare candidate operators if comparison knowledge is not directly
available.  Impasses and substates are described in more detail in
Section \ref{ARCH-impasses}.


% ----------------------------------------------------------------------------
\subsection{An Example Task: The Blocks-World}

We will use a task called the blocks-world as an example throughout this
manual. In the blocks-world task, the initial state has three blocks named
\soar{A}, \soar{B}, and \soar{C} on a table; the operators move one block at a
time to another location (on top of another block or onto the table); and the
goal is to build a tower with \soar{A} on top, \soar{B} in the middle, and
\soar{C} on the bottom. The initial state and the goal are illustrated in
Figure \ref{fig:blocks}.

The Soar code for this task is included in Appendix \ref{BLOCKSCODE}. You do not
need to look at the code at this point.

\begin{figure}
\insertfigure{blocks}{2in}
\insertcaption{The initial state and goal of the ``blocks-world'' task.}
\label{fig:blocks}
\end{figure}

The operators in this task move a single block from its current location to a
new location; each operator is represented with the following information: 
\vspace{-12pt}
\begin{itemize}
\item the name of the block being moved \vspace{-9pt}
\item the current location of the block (the ``thing'' it is on top of) \vspace{-9pt}
\item the destination of the block (the ``thing'' it will be on top of) 
\vspace{-9pt}
\end{itemize}

The goal in this task is to stack the blocks so that \soar{C} is on the
table, with block \soar{B} on block \soar{C}, and block \soar{A} on
top of block \soar{B}.

% ----------------------------------------------------------------------------
\subsection{Representation of States, Operators, and Goals}
\label{OVERVIEW-ps-representation}
\index{state!representation}
\index{operator!representation}
\index{goal!representation}
\index{problem space!representation}
\index{attribute}

The initial state in our blocks-world task --- before any operators have been
proposed or selected --- is illustrated in Figure \ref{fig:ab-wmem}.

\begin{figure}
\insertfigure{ab-wmem}{3.5in}
\insertcaption{An abstract illustration of the initial state of the blocks
	world as working memory objects. At this stage of problem solving, no
	operators have been proposed or selected.}
\label{fig:ab-wmem}
\end{figure}

A state can have only one operator at a time, and the operator is represented
as substructure of the state. A state may also have as substructure a number
of \emph{potential} operators that are in consideration; however, these
suggested operators should not be confused with the current operator. 

Figure \ref{fig:ab-wmem2} illustrates working memory after the first operator
has been selected. There are six operators proposed, and only one of
these is actually selected.

\begin{figure}
\insertfigure{ab-wmem2}{4.25in}
\insertcaption{An abstract illustration of working memory in the blocks world
	after the first operator has been selected.}
\label{fig:ab-wmem2}
\end{figure}

Goals are either represented explicitly as substructure of the state
with general rules that recognize when the goal is achieved, or are
implicitly represented in the Soar program by goal-specific rules that
test the state for specific features and recognize when the goal is
achieved.  The point is that sometimes a description of the goal will be
available in the state for focusing the problem solving, whereas other
times it may not.  Although representing a goal explicitly has many advantages,
some goals are difficult to explicitly represent on the state.

The goal in our blocks-world task is represented implicitly in the Soar
program. A single production rule monitors the state for completion of the goal and
halts Soar when the goal is achieved.

% ----------------------------------------------------------------------------
\subsection{Proposing candidate operators}
\index{operator!proposal}

As a first step in selecting an operator, one or more candidate
operators are proposed.  Operators are proposed by rules that test
features of the current state.  When the blocks-world task is run, the
Soar program will propose six distinct (but similar) operators for the
initial state as illustrated in Figure \ref{fig:proposal}. These
operators correspond to the six different actions that are possible
given the initial state.

\begin{figure}
\insertfigure{blocks-proposal}{2.5in}
\insertcaption{The six operators proposed for the initial state of the blocks
	world each move one block to a new location.}
\label{fig:proposal}
\end{figure}


% ----------------------------------------------------------------------------
\subsection{Comparing candidate operators: Preferences}
\index{operator!comparison}

The second step Soar takes in selecting an operator is to evaluate or
compare the candidate operators. In Soar, this is done via rules that
test the proposed operators and the current state, and then create
\emph{preferences}.  Preferences assert the relative or absolute merits of the
candidate operators. For example, a preference may say that operator A
is a ``better'' choice than operator B at this particular time, or a
preference may say that operator A is the ``best'' thing to do at this
particular time.

% ----------------------------------------------------------------------------
\subsection{Selecting a single operator}
\index{operator!selection}

Soar attempts to select a single operator based on the preferences available
for the candidate operators. There are four different situations that may
arise:\vspace{-14pt}
\begin{enumerate}
\item The available preferences unambiguously prefer a single operator.\vspace{-
6pt}
\item The available preferences suggest multiple operators, and 
       prefer a subset that can be selected from randomly.\vspace{-6pt}
\item The available preferences suggest multiple operators,but neither case
       1 or 2 above hold.\vspace{-6pt}
\item The available preferences do not suggest any operators.
\end{enumerate}

In the first case, the preferred operator is selected.  In the second
case, one of the subset is selected randomly. In the third and fourth
cases, Soar has reached an ``impasse'' in problem solving, and a new
substate is created.  Impasses are discussed in Section
\ref{ARCH-impasses}.

In our blocks-world example, the second case holds, and Soar can select one of
the operators randomly.

% ----------------------------------------------------------------------------
\subsection{Applying the operator}
\index{operator!application}

An operator applies by making changes to the state; the specific changes
that are appropriate depend on the operator and the current state.

\index{I/O}
\index{problem solving!internal}
\index{problem solving!external}
There are two primary approaches to modifying the state: indirect and direct.
\emph{Indirect} changes are used in Soar programs that interact with
an external environment: The Soar program sends motor commands to the
external environment and monitors the external environment for
changes. The changes are reflected in an updated state description,
garnered from sensors. Soar may also make \emph{direct} changes to the
state; these correspond to Soar doing problem solving ``in its
head''. Soar programs that do not interact with an external environment
can make only direct changes to the state.

Internal and external problem solving should not be viewed as mutually
exclusive activities in Soar. Soar programs that interact with an
external environment will generally have operators that make direct and
indirect changes to the state: The motor command is represented as
substructure of the state \emph{and} it is a command to the environment. Also, a Soar program may maintain an internal
model of how it expects an external operator will modify the world; if
so, the operator must update the internal model (which is substructure
of the state).

When Soar is doing internal problem solving, it must know how to modify
the state descriptions appropriately when an operator is being
applied. If it is solving the problem in an external environment, it
must know what possible motor commands it can issue in order to affect
its environment.

The example blocks-world task described here does not interact with an external
environment. Therefore, the Soar program directly makes changes to the state
when operators are applied. There are four changes that may need to be made
when a block is moved in our task: \vspace{-14pt}
\label{LIST:blocks-app}
\begin{enumerate}
\item The block that is being moved is no longer where it was (it is no longer
   	``on top'' of the same thing).\vspace{-6pt}
\item The block that is being moved is now in a new location (it is ``on top''
	of a new thing).\vspace{-6pt}
\item The place that the block used to be is now clear.\vspace{-6pt}
\item The place that the block is moving to is no longer clear --- unless it
	is the table, which is always considered ``clear''\footnote{In this
	blocks-world task, the table always has room for another block, so it
	is represented as always being ``clear''.}.
\end{enumerate}

The blocks-world task could also be implemented using an external simulator. In this case,
the Soar program does not update all the ``on top'' and ``clear'' relations;
the updated state description comes from the simulator.

% ----------------------------------------------------------------------------
\subsection{Making inferences about the state}

Making monotonic inferences about the state is the other role that Soar
long-term knowledge may fulfill. Such elaboration knowledge can simplify
the encoding of operators because entailments of a set of core features
of a state do not have to be explicitly included in application of the
operator.  In Soar, these inferences will be automatically retracted
when the situation changes such that the inference no longer holds.

For instance, our example blocks-world task uses an elaboration to keep track
of whether or not a block is ``clear''. The elaboration tests for the absence
of a block that is ``on top'' of a particular block; if there is no such ``on top'',
the block is ``clear''. When an operator application creates a new ``on top'', the
corresponding elaboration retracts, and the block is no longer ``clear''.


% ----------------------------------------------------------------------------
\subsection{Problem Spaces}
\label{ARCH-functions-ps}

\index{problem space}
If we were to construct a Soar system that worked on a large number of
different types of problems, we would need to include large numbers of
operators in our Soar program. For a specific problem and a
particular stage in problem solving, only a subset of all possible operators
are actually relevant. For example, if our goal is to \textit{count} the
blocks on the table, operators having to do with moving blocks are probably
not important, although they may still be ``legal''. The operators that are
relevant to current problem-solving activity define the space of possible
states that might be considered in solving a problem, that is, they define the
\emph{problem space}.

Soar programs are implicitly organized in terms of problem spaces
because the conditions for proposing operators will restrict an operator
to be considered only when it is relevant.  The complete problem space
for the blocks world is show in Figure \ref{fig:blocks-ps}.  Typically,
when Soar solves a problem in this problem space, it does not explicitly
generate all of the states, examine them, and then create a path.
Instead, Soar is \emph{in} a specific state at a given time (represented
in working memory), attempting to select an operator that will move it
to a new state.  It uses whatever knowledge it has about selecting
operators given the current situation, and if its knowledge is
sufficient, it will move toward its goal.
\begin{figure}
\insertfigure{blocks-ps}{4.9in}
\insertcaption{The problem space in the blocks-world includes all operators
	that move blocks from one location to another and all possible
	configurations of the three blocks.}
\label{fig:blocks-ps}
\end{figure}
The same problem could be recast in Soar as a planning problem, where
the goal is to develop a plan to solve the problem, instead of just
solving the problem.  In that case, a state in Soar would consist of a
plan, which in turn would have representations of Blocks World states
and operators from the original space.  The operators would perform
editing operations on the plan, such as adding new Blocks World
operators, simulating those operators, etc.  In both formulations of the
problem, Soar is still applying operators to generate new states, it is
just that the states and operators have different content.

The remaining sections in this chapter describe the memories and processes of Soar:
working memory, production memory, preference memory, Soar's execution cycle (the decision
procedure), learning, and how  input and output fit in.
% ----------------------------------------------------------------------------
\section{Working memory: The Current Situation} 
\label{ARCH-wm}
\index{working memory|textbf}

\index{working memory}
Soar represents the current problem-solving situation in its \emph{working
memory}. Thus, working memory holds the current state and operator and is Soar's
``short-term'' knowledge, reflecting the current knowledge of the world and
the status in problem solving.

\index{working memory element}
\index{WME|see{working memory element}}
\index{identifier}
\index{attribute}
\index{value}
Working memory contains elements called working memory elements, or WME's for
short. Each WME contains a very specific piece of information; for example, a WME
might say that ``B1 is a block''. 
Several WME's collectively may provide more information about the same
\textit{object}, for example, ``B1 is a block'', ``B1 is named A'', ``B1 is on
the table'', etc. These WME's are related because they are all contributing to
the description of something that is internally known to Soar as ``B1''. B1 is
called an \textit{identifier}; the group of WME's that share this identifier
are referred to as an \textit{object} in working memory. 
Each WME describes a different \textit{attribute} of the object, for example,
its name or type or location; each \textit{attribute} has a \textit{value} associated
with it, for example, the name is A, the type is block, and the position is on
the table. Therefore, each WME is an identifier-attribute-value triple, and
all WME's with the same identifier are part of the same object.

\index{working memory!object}
\index{identifier}
\index{attribute}
\index{value}
\index{link}
Objects in working memory are \emph{linked} to other objects: The value of one
WME may be an identifier of another object. For example, a WME might say that
``B1 is ontop of T1'', and another collection of WME's might describe the
object T1: ``T1 is a table'', ``T1 is brown'', and ``T1 is ontop of F1''. And
still another collection of WME's might describe the object F1: ``F1 is a
floor'', etc. All objects in working memory must be linked to a state, either
directly or indirectly (through other objects). Objects that are not linked to
a state will be automatically removed from working memory by the Soar
architecture. 

\index{augmentation|see{working memory element}}
WME's are also often called \textit{augmentations} because they
``augment'' the object, providing more detail about it. While these two
terms are somewhat redundant, WME is a term that is used more often to
refer to the contents of working memory, while augmentation is a term
that is used more often to refer to the description of an object.
Working memory is illustrated at an abstract level in Figure
\ref{fig:ab-wmem} on page \pageref{fig:ab-wmem}. 


The attribute of an augmentation is usually a constant, such as \soar{name} or
\soar{type}, because in a sense, the attribute is just a label used to
distinguish one link in working memory from another.\footnote{In order to
allow these links to have some substructure, the attribute name may be an
identifier, which means that the attribute may itself have attributes and
values, as specified by additional working memory elements.}

The value of an augmentation may be either a constant, such as \soar{red}, or
an identifier, such as \soar{06}. When the value is an identifier, it refers
to an object in working memory that may have additional substructure. In
semantic net terms, if a value is a constant, then it is a terminal node with
no links; if it is an identifier it is a nonterminal node.

\index{multi-valued attribute}
\index{multi-attribute|see{multi-valued attribute}}
One key concept of Soar is that working memory is a set, which means that there can never be two elements in
working memory at the same time that have the same identifier-attribute-value
triple (this is prevented by the architecture). However, it is possible to have
multiple working memory elements that have the same identifier and attribute,
but that each have different values.  When this happens, we say the attribute
is a \emph{multi-valued attribute}, which is often shortened to be
\emph{multi-attribute}.

An object is defined by its augmentations and
\emph{not} by its identifier. An identifier is simply a label or pointer to the object. On subsequent runs of the same Soar program,
there may be an object with exactly the same augmentations, but a different
identifier, and the program will still reason about the object
appropriately. Identifiers are internal markers for Soar; they can appear
in working memory, but they never appear in a production.

There is no predefined relationship between objects in working memory and
``real objects'' in the outside world.  Objects in working memory may refer to
real objects, such as \soar{block A}; features of an object, such as the
color \soar{red} or shape \soar{cube}; a relation between objects, such as \soar{ontop}; classes of
objects, such as \soar{blocks}; etc. The actual names of attributes and
values have no meaning to the Soar architecture (aside from a few WME's
created by the architecture itself). For example, Soar doesn't care whether
the things in the blocks world are called ``blocks'' or ``cubes'' or
``chandeliers''. It is up to the Soar programmer to pick suitable labels and to
use them consistently.

The elements in working memory arise from one of four sources:\vspace{-12pt}
\begin{enumerate}
\item The actions of productions create most working memory elements. 
\vspace{-8pt}
\item The decision procedure automatically creates some special 
   state augmentations (type, superstate, impasse, ...)
	whenever a state is created.  States are created during 
	initialization (the first state) or because of an impasse (a 
substate).  \vspace{-8pt}
\item The decision procedure creates the operator augmentation of the state 
based on
preferences.  This records the selection of the current operator.\vspace{-8pt}
\item External I/O systems create working memory elements on the input-link
for sensory data.
\end{enumerate}

The elements in working memory are removed in six different ways:\vspace{-12pt}
\begin{enumerate}
\item The decision procedure automatically removes all state
augmentations it creates when the impasse that led to their creation is 
resolved.\vspace{-8pt}
\item The decision procedure removes the operator augmentation of the
state when that operator is no longer selected as the current operator.\vspace{-
8pt}
\item Production actions that use \soar{reject} preferences remove
      working memory elements that were created by other productions.\vspace{-8pt}
\item The architecture automatically removes i-supported WMEs when the productions that created them no longer match.\vspace{-8pt}
\item The I/O system removes sensory data from the input-link when it
is no longer valid. \vspace{-8pt}
\item The architecture automatically removes WME's that are no longer linked to 
a state (because some other WME has been removed).
\end{enumerate}


\index{state}
For the most part, the user is free to use any attributes and values
that are appropriate for the task. However, states have special
augmentations that cannot be directly created, removed, or modified by
rules.  These include the augmentations created when a state is created,
and the state's operator augmentation that signifies the current
operator (and is created based on preferences).  The specific
attributes that the Soar architecture automatically creates are listed in Section
\ref{SYNTAX-impasses}. Productions may create any other attributes for
states.

Preferences are held in a separate \emph{preference memory} where they cannot be
tested by productions; however, \soar{acceptable} preferences 
are held in \emph{both} preference memory and in working memory. By making the
\soar{acceptable} preferences available in working memory, 
the acceptable preferences can be tested for in productions allowing the
candidates operators to be compared before they are selected.

% ----------------------------------------------------------------------------
\section{Production Memory: Long-term Knowledge} 
\label{ARCH-pm}
\index{production memory|textbf}
\index{production}

\begin{figure}
\insertfigure{ab-prodmem}{3.5in}
\insertcaption{An abstract view of production memory. The productions are not
	related to one another.}
\label{fig:ab-prodmem}
\end{figure}

\index{production}
\index{production!firing}
Soar represents long-term knowledge as \emph{productions} that are stored in
\emph{production memory}, illustrated in Figure \ref{fig:ab-prodmem}. Each
production has a set of conditions and a set of actions.  If the
conditions of a production match working memory, the production
\emph{fires}, and the actions are performed.

\subsection{The structure of a production}
\label{ARCH-pm-structure}

In the simplest form of a production, conditions and actions refer directly to
the presence (or absence) of objects in working memory. For example, a
production might say:
\begin{verbatim}
  CONDITIONS: block A is clear 
              block B is clear 
  ACTIONS:    suggest an operator to move block A ontop of block B
\end{verbatim}
This is not the literal syntax of productions, but a simplification.
The actual syntax is presented in Chapter \ref{SYNTAX}.

The conditions of a production may also specify the \emph{absence} of patterns
in working memory. For example, the conditions could also specify that ``block
A is not red'' or ``there are no red blocks on the table''. But since these are
not needed for our example production, there are no examples of negated
conditions for now.

The order of the conditions of a production do not matter to Soar except
that the first condition must directly test the state. Internally, Soar
will reorder the conditions so that the matching process can be more
efficient. This is a mechanical detail that need not concern most
users. However, you may print your productions to the screen or save
them in a file; if they are not in the order that you expected them to
be, it is likely that the conditions have been reordered by Soar.

\subsubsection{Variables in productions and multiple instantiations}

In the example production above, the names of the blocks are ``hardcoded'',
that is, they are named specifically. In Soar productions, variables are used
so that a production can apply to a wider range of situations.
\index{production!instantiation}

The variables are bound to specific symbols in working memory elements
by Soar's matching process.  A production along with a specific and
consistent set of variable bindings is called an \emph{instantiation}. A
production instantiation is consistent only if every occurrence of a
variable is bound to the same value.  Since the same production may
match multiple times, each with different variable bindings, several
instantiations of the same production may match at the same time and,
therefore, fire at the same time.  If blocks \soar{A} and \soar{B} are
clear, the first production (without variables) will suggest one
operator. However, if a production was created that used variables to
test the names, this second production will be instantiated twice and
therefore suggest \textit{two} operators: one operator to move block
\soar{A} ontop of block \soar{B} and a second operator to move block
\soar{B} ontop of block \soar{A}.
\index{instantiation|see{production!instantiation or operator!instantiation}}

Because the identifiers of objects are determined at runtime, literal
identifiers cannot appear in productions. Since identifiers occur in
every working memory element, variables must be used to test for
identifiers, and using the same variables across multiple occurrences is what links conditions together.

Just as the elements of working memory must be linked to a state 
in working memory, so must the objects referred to in a production's
conditions. That is, one condition must test a state object 
\emph{and} all other conditions must test that same state or objects that
are linked to that state.

\subsection{Architectural roles of productions}
\label{ARCH-pm-roles}
\index{production!roles}

Soar productions can fulfill four different roles: the three
knowledge-retrieval problem-solving functions, and the state elaboration function, all described on page \pageref{LIST:5functions}:\vspace{-10pt}
\begin{enumerate}
\item Operator proposal\vspace{-10pt}
\item Operator comparison\vspace{-10pt}
\item \textit{(Operator selection is not an act of knowledge 
retrieval)}\vspace{-10pt}
\item Operator application\vspace{-10pt}
\item State elaboration
\end{enumerate}

A single production should not fulfill more than one of these roles
(except for proposing an operator and creating an absolute preference
for it).  Although productions are not declared to be of one type or the
other, Soar examines the structure of each production and classifies the
rules automatically based on whether they propose and compare operators,
apply operators, or elaborate the state. 

\subsection{Production Actions and Persistence}
\index{I-support}
\index{production actions}
\label{ARCH-prefmem-persistence}
\index{persistence}
\index{O-support}
\index{operator!support}
\label{PAGE:O-support}

Generally, actions of a production either create preferences for
operator selection, or create/remove working memory elements.  For
operator proposal and comparison, a production creates preferences for
operator selection.  These preferences should persist only as long as
the production instantiation that created them continues to match.  When
the production instantiation no longer matches, the situation has
changed, making the preference no longer relevant.  Soar automatically
removes the preferences in such cases.  These preferences are said to
have \emph{I-support} (for ``instantiation support'').  Similarly, state
elaborations are simple inferences that are valid only so long as the
production matches.  Working memory elements created as state
elaborations also have I-support and remain in working memory only as
long as the production instantiation that created them continues to
match working memory.  For example, the set of relevant operators changes
as the state changes, thus the proposal of operators is done with
I-supported preferences. This way, the operator proposals will be
retracted when they no longer apply to the current situation.

However, the actions of productions that \emph{apply} an operator, either by
adding or removing elements from working memory, need to persist even
after the operator is no longer selected and operator application
production instantiation no longer matches.  For example, in placing a
block on another block, a condition is that the second block be
clear. However, the action of placing the first block removes the fact
that the second block is clear, so the condition will no longer be
satisfied.

Thus, operator application productions do not retract their actions, even
if they no longer match working memory.  This is called \emph{O-support}
(for ``operator support''). Working memory elements that participate in
the application of operators are maintained throughout the existence of
the state in which the operator is applied, unless explicitly removed (or
if they become unlinked).  Working memory elements are removed by a
\emph{reject} action of a operator-application rule.  
\index{O-support!reject}

Whether a working memory element receives O-support or I-support is
determined by the structure of the production instantiation that creates
the working memory element.  O-support is given only to working memory
elements created by operator-application productions.

An operator-application production tests the current operator of a state
and modifies the state. Thus, a working memory element receives
O-support if it is for an augmentation of the current state or
substructure of the state, and the conditions of the instantiation that
created it test augmentations of the current operator.  

When productions are matched, all productions that have their conditions
met fire creating or removing working memory elements.  Also, working
memory elements and preferences that lose I-support are removed from
working memory. Thus, several new working memory elements and
preferences may be created, and several existing working memory elements
and preferences may be removed at the same time.  (Of
course, all this doesn't happen literally at the same time, but the
order of firings and retractions is unimportant, and happens in parallel
from a functional perspective.)

% ----------------------------------------------------------------------------
\section{Preference memory: Selection Knowledge} 
\label{ARCH-prefmem}
%\label{ch-abst-symb-pref}
\index{preference}
\index{preference memory}

\nocomment{need to find the right word there. None of selection, evaluation, or
	comparison seems quite right.}

The selection of the current operator is determined by the \emph{preferences} in
\emph{preference memory}. Preferences are suggestions or imperatives about the
current operator, or information about how suggested operators compare
to other operators.  Preferences refer to operators by using the
identifier of a working memory element that stands for the operator.
After preferences have been created for a state, the decision procedures
evaluates them to select the current operator for that state.

For an operator to be selected, there will be at least one preference
for it, specifically, a preference to say that the value is a candidate
for the operator attribute of a state (this is done with either an
``\soar{acceptable}'' or ``\soar{require}'' preference). There may also
be others, for example to say that the value is ``best''.

\index{persistence}
\index{preference!persistence|see{persistence}}
The different preferences available and the semantics of preferences are
explained in Section \ref{ARCH-prefmem-semantics}. Preferences remain in
preference memory until removed for one of the reasons previously discussed in
Section \ref{ARCH-prefmem-persistence}.

% ----------------------------------------------------------------------------
\subsection{Preference semantics}
\label{ARCH-prefmem-semantics}

\index{preference!semantics}
This section describes the semantics of each type of preference.  More
details on the preference resolution process are provided in
Appendix
\ref{PREFERENCES}.

\nocomment{preference resolution or preference evaluation? resolution.}

\index{decision!procedure}
Only a single value can be selected as the current operator, that is,
all values are mutually exclusive.  In addition, there is no implicit
transitivity in the semantics of preferences.  If A is indifferent to B,
and B is indifferent to C, A and C will not be indifferent to one
another unless there is a preference that A is indifferent to C (or C
and A are both indifferent to all competing values).

\begin{description}
\index{preference!acceptable}
\index{"+|see{preference!acceptable}}
\index{acceptable preference|see{preference!acceptable}}
\item [Acceptable (+)] An \soar{acceptable} preference states that a value is a
        candidate for selection. All values, except those with \soar{require}
        preferences, must have an \soar{acceptable} preference in order to be
        selected. If there is only one value with an \soar{acceptable} 
preference
        (and none with a \soar{require} preference), that value will be selected 
as
        long as it does not also have a \soar{reject} or a \soar{prohibit}
        preference.\vspace{-8pt}

\index{preference!reject}
\index{"-|see{preference!reject}}
\index{reject preference|see{preference!reject}}
\item [Reject ($-$)] A \soar{reject} preference states that the value is not a
        candidate for selection.\vspace{-8pt}

\index{preference!better}
\index{">|see{preference!better}}
\index{better preference|see{preference!better}}
\index{preference!worse}
\index{"<|see{preference!worse}}
\index{worse preference|see{preference!worse}}
\item [Better ($>$), Worse ($<$)] A \soar{better} or \soar{worse} preference
        states, for the two values involved, that one value should not be
        selected if the other value is a candidate. \soar{Better} and
        \soar{worse} allow for the creation of a partial ordering between
        candidate values. \soar{Better} and \soar{worse} are simple inverses
        of each other, so that \soar{A} better than \soar{B} is equivalent to
        \soar{B} worse than \soar{A}.\vspace{-8pt}

\index{preference!best}
\index{">|see{best preference}}
\index{best preference|see{best preference}}
\item [Best ($>$)] A \soar{best} preference states that the value may be
        better than any competing value (unless there are other competing
        values that are also ``best''). If a value is \soar{best} (and not
        \soar{reject}ed, \soar{prohibit}ed, or \soar{worse} than another), it
        will be selected over any other value that is not also \soar{best} (or
        \soar{require}d). If two such values are \soar{best}, then any
        remaining preferences for those candidates (\soar{worst},
        \soar{indifferent}) will be examined to determine the selection. Note that if a
        value (that is not \soar{reject}ed or \soar{prohibit}ed) is
        \soar{better} than a \soar{best} value, the \soar{better} value will
        be selected.  (This result is counter-intuitive, but allows explicit
        knowledge about the relative worth of two values to dominate knowledge
        of only a single value. A \soar{require} preference should be used
        when a value \emph{must} be selected for the goal to be 
achieved.)\vspace{-8pt}

\index{preference!worst}
\index{"<|see{preference!worst}}
\index{worst preference|see{preference!worst}}
\item [Worst ($<$)] A \soar{worst} preference states that the value should be 
selected
        only if there are no alternatives.  It allows for a simple type of
        default specification. The semantics of the \soar{worst} preference
        are similar to those for the \soar{best} preference.\vspace{-8pt}

\index{preference!indifferent}
\index{"=|see{preference!indifferent}}
\index{indifferent preference|see{preference!indifferent}}
\index{indifferent-selection}
\item [Indifferent (=)] An \soar{indifferent} preference states that there is
        positive knowledge that it does not matter which value is selected.
        This may be a binary preference, to say that two values are mutually
        indifferent, or a unary preference, to say that a single value is as
        good or as bad a choice as other expected alternatives.
        
	When \soar{indifferent} preferences are used to signal that it does
	not matter which operator is selected, by default, Soar
	chooses randomly from among the alternatives. (The
	\soar{indifferent-selection} function can be used to change this
	behavior as described on
	page \pageref{indifferent-selection} in Chapter
	\ref{INTERFACE}.)\vspace{-8pt}

\index{preference!numeric-indifferent}
\index{numeric-indifferent preference|see{preference!numeric-indifferent}}
\item [Numeric-Indifferent (= \emph{number})] A \soar{numeric-indifferent} preference
	is used to bias the random selection from mutually indifferent values. This
	preference includes a unary indifferent preference, so an operator with a
	\soar{numeric-indifferent} preference will not force a tie impasse. Additionally,
	the preference weights the operator's probability of being selected according to
	the number given. For instance, given the preferences
	
	\begin{verbatim}
	(<s> ^operator <o1> = 40)
	(<s> ^operator <o2> = 10)
	\end{verbatim}
	
	the operator bound to \verb!<o1>! would be more likely to be selected, whereas 
	
	\begin{verbatim}
	(<s> ^operator <o1> =)
	(<s> ^operator <o2> =)
	\end{verbatim}
	
	would give equal probability to the two choices. There are two schemes for
	combining multiple numeric-indifferent preferences and performing the probabilistic
	selection; details are given in the description of the \soar{numeric-indifferent-mode}
	command on page \pageref{numeric-indifferent-mode}.

\index{preference!require}
\index{"!|see{preference!require}}
\index{require preference|see{preference!require}}
\item [Require (!)] A \soar{require} preference states that the value must be
        selected if the goal is to be achieved.\vspace{-8pt}

\index{preference!prohibit}
\index{"~|see{preference!prohibit}}
\index{prohibit preference|see{preference!prohibit}}
\item [Prohibit ($\tild$)] A \soar{prohibit} preference states that the value 
cannot
        be selected if the goal is to be achieved.  If a value has a
        \soar{prohibit} preference, it will not be selected for a value of an
        augmentation, independent of the other preferences.\vspace{-8pt}
\end{description}


\index{preference!acceptable}
\index{preference!require}
If there is an
\soar{acceptable}\/ preference for a value of an operator, and there are no
other competing values, that operator will be selected.  If there are
multiple \soar{acceptable}\/ preferences for the same state but with
different values, the preferences must be evaluated to determine which
candidate is selected.

If the preferences can be evaluated without conflict, the appropriate
operator augmentation of the state will be added to working
memory. This can happen when they all suggest the same operator or when
one operator is preferable to the others that have been suggested. When
the preferences conflict, Soar reaches an impasse, as described in
Section \ref{ARCH-impasses}.

Preferences can be confusing; for example, there can be two suggested
values that are both ``best'' (which again will lead to an impasse
unless additional preferences resolve this conflict); or there may be
one preference to say that value \soar{A} is better than value \soar{B}
and a second preference to say that value \soar{B} is better than value
\soar{A}.



% ----------------------------------------------------------------------------
% ----------------------------------------------------------------------------
\section{Soar's Execution Cycle: Without Substates}
\label{ARCH-decision}

\index{decision cycle}
\index{quiescence}
The execution of a Soar program proceeds through a number of
\emph{cycles}. Each cycle has five phases:
\begin{enumerate} 
\item Input: New sensory data comes into working memory.
\item Proposal: Productions fire (and retract) to interpret new data (state 
elaboration), propose operators for the current situation (operator
proposal), and compare proposed operators (operator comparison).  All of
the actions of these productions are I-supported.  All matched
productions fire in parallel (and all retractions occur in parallel),
and matching and firing continues until there are no more additional
complete matches or retractions of productions (\emph{quiescence}).
\item Decision: A new operator is selected, or an impasse is detected
and a new state is created.
\item Application: Productions fire to apply the operator (operator
application).  The actions of these productions will be O-supported.
Because of changes from operator application productions, other
productions with I-supported actions may also match or retract. Just as
during proposal, productions fire and retract in parallel until quiescence.
\item Output: Output commands are sent to the external environment.
\end{enumerate}

The cycles continue until the halt action is issued from
the Soar program (as the action of a production) or until Soar is interrupted 
by the user.

During the processing of these phases, it is possible that the
preferences that resulted in the selection of the current operator could
change.  Whenever operator preferences change, the preferences are
re-evaluated and if a different operator selection would be made, then
the current operator augmentation of the state is immediately removed.
However, a new operator is not selected until the next decision phase, when
all knowledge has had a chance to be retrieved.

\begin{figure}
\insertfigure{decisioncycle}{6.75in}
\insertcaption{A detailed illustration of Soar's decision cycle: out of date}
\label{fig:decisioncycle}
\end{figure}

\begin{figure}
\index{decision cycle!pseudo code}
\begin{verbatim}
Soar
  while (HALT not true) Cycle;
  
Cycle
  InputPhase;
  ProposalPhase;
  DecisionPhase;
  ApplicationPhase;
  OutputPhase;


ProposalPhase
  while (some I-supported productions are waiting to fire or retract)
    FireNewlyMatchedProductions;
    RetractNewlyUnmatchedProductions;

DecisionPhase
  for (each state in the stack, 
       starting with the top-level state)
  until (a new decision is reached)
    EvaluateOperatorPreferences; /* for the state being considered */
    if (one operator preferred after preference evaluation)
      SelectNewOperator;
    else                  /* could be no operator available or */
      CreateNewSubstate;  /* unable to decide between more than one */

ApplicationPhase
  while (some productions are waiting to fire or retract)
    FireNewlyMatchedProductions;
    RetractNewlyUnmatchedProductions;
\end{verbatim}

\insertcaption{A simplified version of the Soar algorithm.}
\label{fig:pseudocode}
\end{figure}

% ----------------------------------------------------------------------------
\section{Impasses and Substates}
\label{ARCH-impasses}
\index{decision procedure}
\index{impasse}
\index{goal!subgoal}
\index{result}

When the decision procedure is applied to evaluate preferences and determine
the operator augmentation of the state, it is possible that the preferences are 
either
incomplete or inconsistent.
The preferences can be incomplete in that no \soar{acceptable} operators are 
suggested, or that there are insufficient preferences to distinguish among
\soar{acceptable} operators. The preferences can be inconsistent if, for instance, 
operator
\soar{A} is preferred to operator \soar{B}, and operator \soar{B} is preferred 
to operator \soar{A}. Since preferences are generated independently,
from different production instantiations, there is no guarantee that
they will be consistent.

% ----------------------------------------------------------------------------
\subsection{Impasse Types}
\label{ARCH-impasses-types}

\index{impasse!tie}
\index{impasse!conflict}
\index{impasse!constraint-failure}
\index{impasse!no-change}
\index{tie impasse|see{impasse!tie}}
\index{conflict impasse|see{impasse!conflict}}
\index{constraint-failure impasse|see{impasse!constraint-failure}}
\index{no-change impasse|see{impasse!no-change}}

There are four types of impasses that can arise from the preference scheme.
\vspace{-12pt}

\begin{description}
\item[Tie impasse ---] 
	A \emph{tie} impasse arises if the preferences do not
        distinguish between two or more operators with \soar{acceptable}
        preferences. If two operators both have \soar{best} or \soar{worst}
        preferences, they will tie unless additional preferences distinguish
        between them.\vspace{-8pt}
	\index{impasse!tie}
	\index{tie impasse}
\item[Conflict impasse ---]
	A \emph{conflict} impasse arises if at least two values have conflicting
        better or worse preferences (such as \soar{A} is better than \soar{B}
        and \soar{B} is better than \soar{A}) for an operator, and neither
        one is rejected, prohibited, or \soar{require}d.\vspace{-8pt}
	\index{impasse!conflict}
	\index{conflict impasse}
\item[Constraint-failure impasse ---]
	A \emph{constraint-failure} impasse arises if there is more than one
        \soar{require}d value for an operator, or if a value has both a
        \soar{require} and a \soar{prohibit} preference. These preferences
        represent constraints on the legal selections that can be made for a
        decision and if they conflict, no progress can be made from the
        current situation and the impasse cannot be resolved by additional
        preferences.\vspace{-8pt}
	\index{constraint-failure impasse}
	\index{impasse!constraint-failure}
\item[No-change impasse ---]
	A \emph{no-change} impasse arises if a new operator is not selected
        during the decision procedure. There are two types of no-change
        impasses: state no-change and operator no-change:\vspace{-8pt} 
        \begin{description}
        \item[State no-change impasse ---] 
		A state no-change impasse occurs when there are no
		\soar{acceptable} (or \soar{require}) preferences to suggest
		operators for the current state (or all the \soar{acceptable}
		values have also been \soar{reject}ed). The decision procedure
		cannot select a new operator.\vspace{-8pt}
        \item[Operator no-change impasse ---] 
		An operator no-change impasse occurs when either a new operator 
                is selected for the current state but no additional productions
                match during the application phase, or a new
                operator is not selected during the next decision phase.
        \end{description}
	\index{state no-change impasse}
	\index{operator no-change impasse}
	\index{no-change impasse}
	\index{impasse!state no-change}
	\index{impasse!operator no-change}
	\index{impasse!no-change}
\end{description}

There can be only one type of impasse at a given level of subgoaling at
a time. Given the semantics of the preferences, it is possible to have a
tie or conflict impasse and a constraint-failure impasse at the same
time.  In these cases, Soar detects only the constraint-failure impasse.

The impasse is detected \textit{during} the selection of the operator,
but happens \textit{because} one of the other four problem-solving
functions was incomplete.

% ----------------------------------------------------------------------------
\subsection{Creating New States}
%\label{elim-impa}
\index{goal!subgoal}
\index{subgoal|see{goal}}
Soar handles these inconsistencies by creating a new state in which the
goal of the problem solving is to resolve the impasse.  Thus, in the
substate, operators will be selected and applied in an attempt either to
discover which of the tied operators should be selected, or to apply the
selected operator piece by piece.  The substate is often called a
\emph{subgoal} because it exists to resolve the impasse, but is
sometimes called a substate because the representation of the subgoal in
Soar is as a state.

The initial state in the subgoal contains a complete description of the
cause of the impasse, such as the operators that could not be decided
among (or that there were no operators proposed) and the state that the
impasse arose in. From the perspective of the new state, the latter is
called the \emph{superstate}. Thus, the superstate is part of the
substructure of each state, represented by the Soar architecture using
the \soar{superstate} attribute. (The initial state, created in the 0th
decision cycle, contains a \soar{superstate} attribute with the value of
\soar{nil} --- the top-level state has no superstate.)

The knowledge to resolve the impasse may be retrieved by any type of
problem solving, from searching to discover the implications of different
decisions, to asking an outside agent for advice. There is no \emph{a priori}
restriction on the processing, except that it involves applying operators to
states.
\index{subgoal}

\begin{figure}
\insertfigure{stack1}{7.75in}
\insertcaption{A simplified illustration of a subgoal stack.}
\label{fig:stack1}
\end{figure}

\index{goal!stack}
\index{stack|see{goal}}
In the substate, operators can be selected and applied as Soar attempts to
solve the subgoal. (The operators proposed for solving the subgoal may be
similar to the operators in the superstate, or they may be entirely
different.) While problem solving in the subgoal, additional impasses may be
encountered, leading to new subgoals.  Thus, it is possible for Soar to have a
\emph{stack} of subgoals, represented as states: Each state has 
a single superstate (except the initial state) and each state may have at most 
one substate. Newly created
subgoals are considered to be added to the bottom of the stack; the first
state is therefore called the \emph{top-level state}.\footnote{The
original state is the ``top'' of the stack because as Soar
runs, this state (created first), will be at the top of the computer screen,
and substates will appear on the screen below the top-level state.}  See
Figure \ref{fig:stack1} for a simplified illustrations of a subgoal stack.

Soar continually attempts to retrieve knowledge relevant to all goals in the
subgoal stack, although problem-solving activity will tend to focus on the
most recently created state. However, problem solving is active at
all levels, and productions that match at any level will fire.

% ----------------------------------------------------------------------------
\subsection{Results}
\label{ARCH-impasses-results}
\index{goal!result|see{result}}
\index{result}

In order to resolve impasses, subgoals must generate results that allow
the problem solving at higher levels to proceed.  The {\em results} of a
subgoal are the working memory elements and preferences that were
created in the substate, and that are also linked directly or indirectly
to a superstate (\emph{any} superstate in the stack). A preference or
working memory element is said to be created in a state if the
production that created it tested that state and this is the most recent
state that the production tested. Thus, if a production tests multiple
states, the preferences and working memory elements in its actions are
considered to be created in the most recent of those states (and is not
considered to have been created in the other states). The architecture
automatically detects if a preference or working memory elmenet created
in a substate is also linked to a superstate.

These working memory elements and preferences will not be removed when
the impasse is resolved because they are still linked to a superstate,
and therefore, they are called the \textit{results of the subgoal}.  A
result has either I-support or O-support; the determination of support is
described below.

%A production that creates a result is illustrated in Figure \ref{fig:result}.
%The figure illustrates the result as a working memory element: 
%``\soar{new-attribute X1}''.

%\begin{figure}
%\insertfigure{result}{3.7in}
%\insertcaption{An abstract illustration of a production that creates a
%	result. In the figure, S2 is the lowest state in the subgoal stack
%	that is tested by the production, and the working memory element
%	is said to have been created in state S2.  }
%\label{fig:result}
%\end{figure}

%\begin{figure}
%\insertfigure{result-indirect}{3in}
%\insertcaption{An abstract illustration of a production that creates a
%	working memory element that indirectly becomes a result. S2 is the
%	lowest state in the subgoal stack that is tested by the production,
%	and the working memory element is said to be created in state S2. Some 
%other
%	production instantiation creates the working memory element that links X2 
%to the
%	superstate.
%	}
%\label{fig:result-indirect}
%\end{figure}

A working memory element or preference will be a result if
its identifier is already linked to a superstate.
%(as illustrated inFigure \ref{fig:result})
A working memory element or preference can also become a result
indirectly if, after it is created and it is still in working memory or
preference memory, its identifier becomes linked to a superstate through
the creation of another result. For example, if the problem solving in a
state constructs an operator for a superstate, it may wait until
the operator structure is complete before creating an
\soar{acceptable} preference for the operator in the superstate. The
\soar{acceptable} preference is a result because it was created in the
state and is linked to the superstate (and, through the superstate, is
linked to the top-level state). The substructures of the operator then
become results because the operator's identifier is now linked to the
superstate. 
% An indirect result is illustrated in Figure \ref{fig:result-indirect}). 

\subsubsection*{Justifications: Determination of support for results}

\index{I-support!of result}
\index{O-support!of result}
Some results receive I-support, while others receive O-support.  The
type of support received by a result is determined by the function it
plays in the superstate, and not the function it played in the state in
which it was created. For example, a result might be created through
operator application in the state that created it; however, it might
only be a state elaboration in the superstate. The first function would
lead to O-support, but the second would lead to I-support.

\index{justification}
\index{justification!creation}
In order for the architecture to determine whether a result receives I-support
or O-support, Soar must first determine the function that the working
memory element or preference plays
(that is, whether the result should be considered an operator application or
not). To do this, Soar creates a temporary production, called a
\textit{justification}. The justification summarizes the processing in the
substate that led to the result:\vspace{-10pt}
\begin{description}
\item[The conditions] of a justification are those working memory
elements that exist in the superstate (and above) that were necessary
for producing the result.  This is determined by collecting all of the
working memory elements tested by the productions that fired in the
subgoal that led to the creation of the result, and then removing those
conditions that test working memory elements created in the subgoal.
\vspace{-6pt}
\item[The action] of the justification is the result of the subgoal.
\end{description} 

Soar determines I-support or O-support for the justification just as it
would for any other production, as described in Section
\ref{ARCH-prefmem-persistence}.  If the justification is an operator
application, the result will receive O-support.  Otherwise, the result
gets I-support from the justification. If such a result loses
I-support from the justification, it will be retracted if there is no
other support.  Justification are not added to production memory, but
are otherwise treated as an instantiated productions that have already
fired.

Justifications include any negated conditions that were in the original
productions that participated in producing the results, and that test for
the absence of superstate working memory elements. Negated conditions that
test for the absence of working memory elements that are local to the
substate are not included, which can lead to overgeneralization
in the justification (see Section \ref{CHUNKING-problems} on page
\pageref{CHUNKING-problems} for details). 
\index{learning!overgeneral}
\index{chunk!overgeneral}
\index{justification!overgeneral}

% ----------------------------------------------------------------------------
\subsection{Removal of Substates: Impasse Resolution}
%\label{elim-impa}
\label{ARCH-impasses-elimination}
\index{impasse!resolution}
\index{impasse!elimination}
\index{goal!termination}

Problem solving in substates is an important part of what Soar
\textit{does}, and an operator impasse does not necessarily indicate a
problem in the Soar program.  They are a way to decompose a complex
problem into smaller parts and they provide a context for a program to
deliberate about which operator to select.  Operator impasses are necessary, for
example, for Soar to do any learning about problem solving (as will be
discussed in Chapter \ref{CHUNKING}). This section describes how
impasses may be resolved during the execution of a Soar program, how
they may be eliminated during execution without being resolved, and some
tips on how to modify a Soar program to prevent a specific impasse from
occurring in the first place.  

\subsubsection*{Resolving Impasses}

An impasse is \textit{resolved} when processing in a subgoal creates
results that lead to the selection of a new operator for the state
where the impasse arose. When an operator impasse is resolved, Soar has
an opportunity to learn, and the substate (and all its substructure) is
removed from working memory.

Here are possible approaches for resolving specific types
of impasses are listed below:\vspace{-12pt}
\begin{description}
\item[Tie impasse ---]
	A tie impasse can be resolved by productions that create preferences
	that prefer one option (\soar{better}, \soar{best}, \soar{require}),
	eliminate alternatives (\soar{worse}, \soar{worst}, \soar{reject},
	\soar{prohibit}), or make all of the objects indifferent
	(\soar{indifferent}).\vspace{-8pt}
\item[Conflict impasse ---]
	A conflict impasse can be resolved by productions that create
	preferences to \soar{require} one option (\soar{require}), or
	eliminate the alternatives (reject, prohibit).\vspace{-8pt}
\item[Constraint-failure impasse ---]
	A constraint-failure impasse cannot be resolved by additional
	preferences, but may be prevented by changing productions so that they
	create fewer \soar{require} or \soar{prohibit} preferences.\vspace{-8pt}
\item[State no-change impasse ---]
	A state no-change impasse can be resolved by productions that create 
	\soar{acceptable} or \soar{require} preferences for operators.\vspace{-
8pt}
\item[Operator no-change impasse ---]
	An operator no-change impasse can be resolved by productions that
	apply the operator, changing the state so the operator proposal
	no longer matches or other operators are proposed and preferred.
\end{description}

\subsubsection*{Eliminating Impasses}

An impasse is resolved when results are created that allow progress to
be made in the state where the impasse arose.  In Soar, an impasse can be
\textit{eliminated} (but not resolved) when a higher level impasse is
resolved, eliminated, or regenerated.  In these cases, the impasse
becomes irrelevant because higher-level processing can proceed.  An
impasse can also become irrelevant if input from the outside world
changes working memory which in turn causes productions to fire that
make it possible to select an operator.  In all these cases, the impasse
is eliminated, but not ``resolved'', and Soar does not learn in this
situation.

\subsubsection*{Regenerating Impasses}

An impasse is \textit{regenerated} when the problem solving in the
subgoal becomes {\em inconsistent} with the current situation.  During
problem solving in a subgoal, Soar monitors which aspect of the
surrounding situation (the working memory elements that exist in
superstates) the problem solving in the subgoal has depended upon.  If
those aspects of the surronding situation change, either because of
changes in input or because of results, the problem solving in the
subgoal is inconsistent, and the state created in response to the
original impasse is removed and a new state is created. Problem solving
will now continue from this new state.  The impasse is not ``resolved'',
and Soar does not learn in this situation.

The reason for regeneration is to guarantee that the working memory
elements and preferences created in a substate are consistent with
higher level states.  As stated above, inconsistency can arise when a
higher level state changes either as a result of changes in what is
sensed in the external environment, or from results produced in the
subgoal.  The problem with inconsistency is that once inconsistency
arises, the problem being solved in the subgoal may no longer be the
problem that actually needs to be solved.  Luckily, not all changes to a
superstate lead to inconsistency.

In order to detect inconsistencies, Soar maintains a {\em dependency set}
for every subgoal/substate.  The dependency set consists of all working
memory elements that were tested in the conditions of productions that
created O-supported working memory elements that are directly or
indirectly linked to the substate.  Thus, whenever such an O-supported
working memory element is created, Soar records which working memory
elements that exist in a superstate were tested, directly or indirectly
in creating that working memory element. \index {dependency-set} Whenever
any of the working memory elements in the dependency set of a substate
change, the substate is regenerated.

Note that the creation of I-supported structures in a subgoal does not
increase the dependency set, nor do O-supported results.  Thus, only
subgoals that involve the creation of internal O-support working memory
elements risk regeneration, and then only when the basis for the
creation of those elements changes.

\subsubsection*{Substate Removal}

Whenever a substate is removed, all working memory elements and
preferences that were created in the substate that are not
results are removed from working memory. In Figure \ref{fig:stack1},
state \soar{S3} will be removed from working memory when the impasse
that created it is resolved, that is, when sufficient preferences have
been generated so that one of the operators for state \soar{S2} can be
selected. When state \soar{S3} is removed, operator \soar{O9} will also be removed,
as will the acceptable
preferences for \soar{O7}, \soar{O8}, and \soar{O9}, and the
\soar{impasse}, \soar{attribute}, and \soar{choices} augmentations of state
\soar{S3}. These working memory elements are removed because they are no
longer linked to the subgoal stack. The acceptable preferences for
operators \soar{O4}, \soar{O5}, and \soar{O6} remain in working memory. They
were linked to state \soar{S3}, but since they are also linked to state
\soar{S2}, they will stay in working memory until \soar{S2} is removed (or
until they are retracted or rejected).

\subsection{Soar's Cycle: With Substates}
\label{ARCH-decision-substates}

When there are multiple substates, Soar's cycle remains basically the
same but has a few minor changes.  


The first change is that during the decision procedure, Soar will detect
impasses and create new substates.  For example, following the proposal
phase, the decision phase will detect if a decision cannot be made given
the current preferences.  If an impasse arises, a new substate is
created and added to working memory.  

%The decision procedure will detect an operator no-change impasse as soon
%as an operator is selected and added to working memory by checking to
%see whether or not productions will create O-supported actions during
%the next application phae.  If no O-suppored actions will be created,
%the decision procedure will immediately create an operator no-change
%impasse, and then proceed to output, input, and so on.  In these cases,
%the operator no-change is made in the same decision as the operator
%selection.  There will be cases where the operator no-change happens on
%the following decisions, such as when there are O-supported productions
%that will fire, but do not lead to a change in the selected operator.

The second change when there are multiple substates is that at each
phase, Soar goes through the substates, from oldest (highest) to newest
(lowest), completing any necessary processing at that level for that
phase before doing any processing in the next substate.  When firing
productions for the proposal or application phases, Soar processes the
firing (and retraction) of rules, starting from those matching the
oldest substate to the newest.  Whenever a production fires or retracts,
changes are made to working memory and preference memory, possibly
changing which productions will match at the lower levels (productions
firing within a given level are fired in parallel -- simulated).
Productions firings at higher levels can resolve impasses and thus
eliminate lower states before the productions at the lower level ever
fire.  Thus, whenever a level in the state stack is reached, all
production activity is guaranteed to be consistent with any processing
that has occurred at higher levels.


% ----------------------------------------------------------------------------
\section{Learning}
\label{ARCH-learning} 
\index{learning}
\index{chunk}
\index{goal!subgoal}

\index{learning}
\index{chunking|see{learning}}
When an operator impasse is resolved, it means that Soar has, through problem 
solving,
gained access to knowledge that was not readily available before. Therefore,
when an impasse is resolved, Soar has an opportunity to learn, by summarizing
and generalizing the processing in the substate.

\index{chunk}
Soar's learning mechanism is called \textit{chunking}; it attempts to create a
new production, called a chunk. The conditions of the chunk are the elements
of the state that (through some chain of production firings) allowed the
impasse to be resolved; the action of the production is the working
memory element or preference that
resolved the impasse (the result of the impasse). The conditions and action
are variablized so that this new production may match in a similar situation
in the future and prevent an impasse from arising. 

Chunks are very similar to justifications in that they are both
formed via the backtracing process and both create a result in their
actions. However, there are some important distinctions:\vspace{-12pt}
\begin{enumerate}
\item Chunks are productions and are added to production memory.
	Justifications do not appear in production memory.\vspace{-8pt}
\item Justifications disappear as soon as the working memory element or
         preference they provide support
	for is removed. \vspace{-8pt}
\item Chunks contain variables so that they may match working memory in other
	situations; justifications are similar to an instantiated chunk.
\end{enumerate}




% ----------------------------------------------------------------------------
\section{Input and Output}
\label{ARCH-io}	%\label{ch-abst-symb-inpu}
\index{I/O}

Many Soar users will want their programs to interact with a real or simulated
environment. For example, Soar programs may control a robot, receiving sensory
inputs and sending command outputs. Soar programs may also interact with
simulated environments, such as a flight simulator. Input is viewed as
Soar's perception and output is viewed as Soar's motor abilities.

When Soar interacts with an external environment, it must make use of
mechanisms that allow it to receive input from that environment and to effect
changes in that environment; the mechanisms provided in Soar are called
\textit{input functions} and \textit{output functions}.

\begin{description}
\item[Input functions] add and delete elements from working memory in response
	to changes in the external environment.
\item[Output functions] attempt to effect changes in the external
	environment. 
\end{description}

Input is processed at the beginning of each execution cycle and output
occurs at the end of each execution cycle.

For instructions on how to use input and output functions with Soar, refer to the
\textit{SML Quick Start Guide}.




