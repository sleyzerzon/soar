\subsection{\soarb{numeric-indifferent-mode}}
\label{numeric-indifferent-mode}
\index{numeric-indifferent-mode}
Select method for combining numeric preferences. 
\subsubsection*{Synopsis}
\begin{verbatim}
numeric-indifferent-mode [-as]
\end{verbatim}
\subsubsection*{Options}
\begin{tabular}{|l|l|}
\hline
\soar{ -a, --avg, --average } & Use average mode (default).  \\
\hline
\soar{ -s, --sum } & Use sum mode.  \\
\hline
\end{tabular}
\subsubsection*{Description}
 The numeric-indifferent-mode command is used to select the method for combining numeric preferences. This command is only meaningful in indifferent-selection --random  mode. 
 The default procedure is \textbf{--avg}
 (average) which assigns a final value to an operator according to the rule: \begin{itemize}
\item  If the operator has at least one numeric preference, assign it the value that is the average of all of its numeric preferences. 
\item  If the operator has no numeric preferences (but has been included in the indifferent selection through some combination of non-numeric preferences), assign it the value 50. 
\end{itemize}
 The intended range of numeric-preference values for \textbf{--avg}
 mode is 0-100. 
 The other combination option \textbf{--sum}
 assigns a final value according to the rule: \begin{itemize}
\item  Add together any numeric preferences for the operator (defaulting to 0 if there are none). 
\item  Assign the operator the value e\^{}\{PreferenceSum / AgentTemperature\}, where AgentTemperature is a compile-time constant currently set at 25.0. 
\end{itemize}
 Any real-numbered preference may be used in \textbf{--sum}
 mode. 
 Once a value has been computed for each operator, the next operator is selected probabilistically, with each candidate operator's chance weighted by its computed value. 
