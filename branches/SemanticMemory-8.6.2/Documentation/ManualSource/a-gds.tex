% ----------------------------------------------------------------------------
\typeout{--------------- appendix: GDS ------------------}
\chapter[A Goal Dependency Set Primer]{
A Goal Dependency Set Primer\footnote{A preliminary draft by Robert Wray, contact at \texttt{wrayre@acm.org}.
}}

\label{GDS}
\index{GDS}


% a list of hyphenation points for re-occuring words in the document
\hyphenation{con-temp-or-an-e-ous}
\hyphenation{OP-ER-AND}
\hyphenation{Mich-i-gan}

%\pagestyle{myheadings}
%\markboth{GDS Primer}{DRAFT: Not for Quotation or Distribution}

%\input{macros}


      

% use optional labels to link authors explicitly to addresses:
% \author[label1,label2]{}
% \address[label1]{}
% \address[label2]{}
%\author{Robert Wray  \\  Soar Technology \\ 3600 Green Road Suite 600 \\ Ann Arbor, MI 48105 \\ (734)327-8000 \\ \texttt{wrayre@acm.org}  
%        }


%\maketitle                        %%%% To set Title and Author names.
%\thispagestyle{empty}

%%%% Replace with your Abstract.

%%%%%%%%%%%%%%%%%%%%%%%%%%%%%%%%%

This document briefly describes the Goal Dependency Set (GDS), which
was introduced with Soar~8.  There are three sections: a brief
discussion of the motivation for the GDS, a discussion of the
consequences of the GDS from a behavior developer/modeler's point of
view, and some details on the kernel implementation of the GDS, for
anyone working at the architecture level.  This document is by no
means complete, but introduces the GDS in Soar-specific terms.

\section*{Why the GDS was needed}

As a symbol system, Soar attempts to approximate the knowledge level
but will necessarily always fall short \cite{Newell90:UTC}.  We can
informally think of the way in which Soar falls short of the knowledge
level as its peculiar ``psychology.''  Those interested in using Soar
to model human psychology would like Soar's ``psychology'' to
approximate human psychology.  Those using Soar to create agent
systems would like to make Soar's processing approximate the knowledge
level as closely as possible.  However, Soar~7 had a number of
symbol-level ``quirks'' that appeared inconsistent with human
psychology and that made building large-scale, knowledge-based systems
in Soar more difficult than necessary.  Bob Wray's thesis (1998)
\nocite{Wray98:Ensuring} addressed many of these symbol-level problems
in Soar, among them logical inconsistency in symbol manipulations,
non-contemporaneous constraints in chunks \cite{Wray96:Compilation},
race conditions in rule firings and in the decision process, and
contention between original task knowledge and learned knowledge
\cite{Wray01:Resolving}.

The Goal Dependency Set implements a solution to logical
inconsistencies between persistent (o-supported) working memory
elements (WMEs) in a substate and its ``context''.  The context
consists of all the WMEs in any superstates above the local
goal/state\footnote{This report will use ``state,'' not ``goal.''  At
the kernel level, states are still called ``goals'' and ``goal'' is often
still used to refer to states.    As a result, a
confusion in terminology results, with ``\textbf{Goal} Dependency Set'' a 
specific example, even though ``goals'' have not been
an explicit, behavior-level Soar construct since Soar~6}.  In Soar, any
action (application) of an operator receives an o-support preference.
This preference makes the resulting WME persistent: it will remain in
memory until explicitly removed (or until its local state is removed),
regardless of whether it continues to be justified.

Persistent WMEs are pervasive in Soar, because operators are the main
unit of problem solving.  Persistence is necessary for taking any
non-monotonic step in a problem space.  However, persistent WMEs also
are dependent on WMEs in the superstate context.  The problem in
Soar~7, especially when trying to create large-scale systems like
TacAir-Soar \cite{Jones99:Automated}, is that the knowledge developer
must always think about which dependencies can be ``ignored'' and
which need to result in a reconsideration of the persistent WME.  For
example, imagine an exploration robot that makes a persistent decision
to travel to some distant destination based, in part, on its power
reserves.  Now suppose that the agent notices that its power reserves
have failed.  If this change is not communicated to the state where
the travel decision was made, the agent will continue to act as if its
full power reserves were still available.

Of course, for this specific example, the knowledge designer can
encode some knowledge to react to this inconsistency.  The fundamental
problem is that the knowledge designer has to consider \emph{all}
possible interactions between all o-supported WMEs and all contexts.
Soar systems often use the architecture's impasse mechanism to realize
a form of decomposition.  These potential interactions mean that the
knowledge developer cannot focus on individual problem spaces when
creating knowledge, which makes knowledge development more difficult.
Further, in all but the simplest systems, the knowledge designer will
miss some potential interactions.  The result is agents are that were
unnecessarily brittle, failing in difficult-to-understand,
difficult-to-duplicate ways.  

The GDS also solves the the problem of non-contemporaneous constraints
in chunks.  A non-contemporaneous constraint refers to two or more
conditions that never co-occur simultaneously.  An example might be a
driving robot that learned a rule that attempted to match ``red
light'' and ``green light'' simultaneously. Obviously, for functioning
traffic lights, this rule would never fire.  By ensuring that local
persistent elements are always consistent with the higher-level
context, non-contemporaneous constraints in chunks are
\emph{guaranteed} not to happen.


The GDS captures context dependencies during processing, meaning the
architecture will identify and respond to inconsistencies
automatically.  The knowledge designer then does not have to consider
potential inconsistencies between local, o-supported WMEs and the
context.  The following sections describe further how the GDS works
and how to use the GDS in behavior systems, as well as how the GDS is
implemented in the Soar kernel.


\section*{Behavior-level view of the Goal Dependency Set}

This section discusses what the GDS does, and how that impacts
production knowledge design and implementation.

\subsection*{Operation of the Goal Dependency Set}


\begin{figure}
\insertfigure{simple-ncc}{3in}
\caption{Simplified Representation of the context dependencies (above the line), local os-upported WMEs (below the line), and the generation of a result.  In Soar~7, this situation led to non-contemporaneous constraints in the chunk that generates {\bf 3}.}
\label{'ncc'}
\end{figure}

Whenever a feature is created (added to working memory) in the Soar~7
architecture, that feature will persist for some time.  The
persistence of features may differ with respect to how long the
features remain in memory, and more importantly, what circumstances
cause the feature to be removed.  The Soar~7 architecture utilizes
three primary types of persistence: i-support, o-support, and
c-support.

The weakest persistence is instantiation support.  An i-supported
feature exists in memory only as long as the production which lead to
the feature's creation remains instantiated.  Thus, the WME depends
upon this production instantiation (and, more specifically, the
features the instantiation tests).  When one of the conditions in the
production instantiation no longer matches, the instantiation is
retracted, resulting in the loss of the acceptable preference for the
WME.\footnote{Importantly, in a technical sense, the WME is only
retracted when it loses instantiation support, not when the creating
production is retracting.  For example, a WME could receive i-support
from several different instantiations and the retraction of one would
not lead to the retraction of the WME.  However, the the following
generally discusses direct dependency unmediated by preferences,
ignoring this complication for clarity.}  I-support is illustrated in
Figure~\ref{'ncc'}. A copy of {\bf A} in the subgoal, {\bf A$_s$}, is
retracted automatically when {\bf A} changes to {\bf A'}.  The
substate WME persists only as long as it remains justified by {\bf A}.
This justification is called ``instantiation support'' (I-support) in
Soar (and should not be confused with result \emph{justifications}.)

In the broadest sense, we can say that some feature $<$b$>$ is
``dependent'' upon another element $<$a$>$ if $<$a$>$ was used in the
creation of $<$b$>$, i.e., if $<$a$>$ was tested in the production
instantiation that created $<$b$>$.  Further, a dependent change with
respect to feature $<$b$>$ is a change to any of its instantiating
features.  In Figure~\ref{'ncc'}, the change from {\bf A} to {\bf A'}
is a dependent change for feature {\bf 1} because {\bf A} was used to
create {\bf 1}.

In Soar 7, some features are insensitive to dependent changes.  These
features are often referred to as ``persistent WMEs'' because, unlike
i-supported WMEs, they remain in memory until explicitly removed.
There are two different types of this stronger persistence: o-support
and c-support.  

Any feature created by the action of an operator
receives ``operator support.''  An o-supported feature remains in
memory until explicitly rejected (or until the superstructure to which
it is attached is removed).  Removal is architecturally
independent of the WME's instantiating conditions.

Context-support affects the persistence of an operator itself, rather
than its effects.  Once a unique operator has been chosen by the
decision procedure, the choice persists until explicitly re-decided
(via a reconsider preference).  C-support ensures that the WME for a
selected operator remains available even if the production that
proposed the operator is no longer instantiated.  Soar~8 eliminates
c-support, so that operators now persist only as long as they receive
instantiation support.  This change was integral to the overall
solution Soar~8 provides, but is distinct from the GDS.

The GDS provides a solution to the first problem.  When {\bf A}
changes, the persistent WME {\bf 1} may be no longer consistent with
its context (e.g., {\bf A'}).  The specific solution is inspired by
the chunking algorithm.  In Soar~8, whenever an o-supported WME is
created in the local state, the superstate dependencies of that new
feature are determined and added to the {\em goal dependency set}
(GDS) of that state. Conceptually speaking, whenever a working memory
change occurs, the dependency sets for every state in the context
hierarchy are compared to working memory changes.\footnote{The
implementation is slightly different, trading additional memory
overhead to avoid scanning all the goal dependency sets after each WM
change.  See the next section.  }  If a removed element is found in a
GDS, the state is removed from memory (along with all existing
substructure). The dependency set includes only dependencies for
o-supported features.  For example, in Figure~\ref{'gds'}, at time
$t_0$, because only i-supported features have been created in the
subgoal, the dependency set is empty.

\begin{figure}
\insertfigure{gomor-o-support}{3in}
\caption{The Dependency Set in Soar~8.}
\label{'gds'}
\end{figure}


Three types of features can be tested in the creation of an
o-supported feature.  Each requires a slightly different type of
update to the dependency set.
\begin{description}
\item [Elements in the superstate:] WMEs in the superstate are added
directly to the goal's dependency set.  In Figure~\ref{'gds'}, the
persistent subgoal item {\bf 3} is dependent upon {\bf A} and {\bf
D}. These superstate WMEs are added to the subgoal's dependency set when
{\bf 3} is added to working memory at time $t_1$.  It does not matter
that {\bf A} is i-supported and {\bf D} o-supported.\footnote{In addition,
superstate WMEs can also include context slot preferences, which 
are represented in the architecture as working memory elements.}
\item [Local I-Supported Features:] Local i-supported features are not
added to the goal dependency set.  Instead, the superstate WMEs that
led to the creation of the i-supported feature are determined and
added to the GDS.  In the example, when {\bf 4} is created, {\bf A},
{\bf B} and {\bf C} must be added to the dependency set because they
are the superstate features that led to {\bf 1}, which in turn led to
{\bf 2} and finally {\bf 4}.  However, because item {\bf A} was
previously added to the dependency set at $t_1$, it is unnecessary to
add it again.
\item [Local O-Supported Features:] The dependencies of a local
o-supported feature have already been added to the state's GDS.  Thus,
tests of local o-supported WMEs do not require additions to the
dependency set.  In Figure~\ref{'gds'}, the creation of element {\bf
5} does not change the dependency set because it is dependent only
upon persistent items {\bf 3} and {\bf 4}, whose features had been
previously added to the GDS.
\end{description}

In Soar~8, any change to the current dependency set will cause
the retraction of all subgoal structure.  Thus, any time after time
$t_1$, either the {\bf D} to {\bf D'} or {\bf A} to {\bf A'}
transition would cause the removal of the entire subgoal. The {\bf E}
to {\bf E'} transition causes no retraction because {\bf E} is not in
the goal's dependency set.

\subsection*{The role of the GDS in agent design}

The GDS places some design time constraints on operator implementation.
These constraints are:
\begin{itemize}
\item Operator actions that are used to remember a previous state/situation should be asserted in the top state
\item All operator elaborations should be i-supported
\item Any operator with local actions should be designed to be re-entrant
\end{itemize}
This section describes these issues.

Soar says any operator effect is o-supported, regardless of whether
that assertion is entailed by the current situation, or whether it
reflects an assumption about it.  The GDS adds additional (needed)
constraint.  Because any context dependencies for subgoal, o-supported
assertions will be added to the GDS, the developer must decide if an
o-supported element should be represented in a substate or the top
state.

This decision is straightforward if the functional role of the
persistent element is considered.  Four important capabilities that
require persistence are:
\begin{enumerate}

\item \textbf{Reasoning hypothetically:} ~ Some assertions may need to
reflect hypothetical states.  Such assertions are ``assumptions''
because a hypothetical inference cannot always be grounded in the
current context.  In other problem solvers with truth maintenance,
only assumptions are persistent.

\item \textbf{Reasoning non-monotonically:} ~
Sometimes the result of an inference changes one of the assertions on
which the inference is dependent.  As an example, consider the task of
counting.  Each newly counted item replaces the old value of the
count. 

\item \textbf{Remembering:} ~
Agents oftentimes need to remember an external situation or stimulus,
even when that perception is no longer available.  

\item \textbf{Avoiding Expensive Computations:} ~ In some situations,
an agent may have the information needed to assert some belief in a
new world state but the expense of performing the computation
necessary for the assertion, given what is already known, makes the
computation avoidable.  For example, in dynamic, complex domains,
determining when to make an expensive calculation is often formulated
as an explicit agent task \cite{Jones99:Automated}.
\end{enumerate}

When remembering or avoiding an expensive computation, the
agent/designer is making a commitment to retain something even though
it might not be supported in the current context.  In Soar~8, these
WMEs should be asserted in the top state.  \emph{For many Soar systems,
especially those focused on execution in a dynamic environment, 
most o-supported elements will need to be stored on the top state.} 

For any kind of local, non-monotonic reasoning about the context
(counting, projection planning), features should be stored locally.
When a dependent context change occurs, the GDS interrupts the
processing by removing the state.  While this may seem like a severe
over-reaction, formal and empirical analysis have suggested that this
solution is less computationally expensive than attempting to identify
the specific dependent assumption \cite{Wray03:Ensuring}.


\subsection*{Operator Elaborations}

Operator elaborations (i.e., placing some information on an operator
WME) should be i-supported when using Soar~8, since this information
is, by definition, temporary/not persistent (because it's located on
the non-persistent operator).  However, the kernel itself hasn't kept
up with this change.  Prior to Soar~8.5, Soar's o-support modes
computed operator elaborations as o-supported, resulting in the
context conditions being added to the GDS.  This often leads to
unwanted/unnecessary retractions.  If you are using a version prior to
Soar~8, you should declare any operator elaborations i-supported (i.e.,
using :i-support).




\section*{Kernel-level view of the Goal Dependency Set}


The actual implementation of the GDS in the Soar kernel is slightly
more complex than the conceptual description of the previous section
(but not significantly so).  

Elements are added the GDS via elaborate\_gds(), a procedure in
decide.c that mimics the chunking backtrace function.  The algorithm
is shown in Figure~\ref{tab:dhj:proc}.  When an o-supported preference
is asserted, elaborate\_gds() is called.  Conditions in a production
instantiation that are located in a higher context can be added
directly to the GDS (1).  For local conditions, elaborate\_gds() first
checks whether the tested WME is o-supported, or if it has been
previously been back traced through (2). If either of these are true,
the WME can be ignored because it's dependencies have been added to
the GDS previously.  If not, elaborate\_gds() is called recursively,
to find the context dependencies for the local, contributing WME,~$c$
(3).

\begin{figure}[h]
%\rule{\textwidth}{.5mm}
\framebox[\textwidth]{
\begin{minipage}{\textwidth}
\begin{tabbing}
xxx\=xxx\=xxx\=xxx\=xxx\=xxxxxxxxxxxxxxxxxxxx\= \kill

\textbf{PROC} $create\_new\_assertion(\ldots)$ \\
\> Whenever a new o-supported element is asserted, the GDS is updated \\ 
\> to include any new context dependencies.  \\
\> $\ldots$\\
\> $A_{inst} \leftarrow $ instantiation that asserted acceptable preference for A  \\
\> \textbf{IF} A is an o-supported WME\\
%\>\>$A_{goal_{GDS}}$ : = $append(A\rightarrow goal\rightarrow GDS$ \\
%\>\>$G \leftarrow A_{goal} \quad$ 
\>\>G is the goal/state in which A is asserted \\
\>\>$G_{GDS} \leftarrow append(G_{GDS}, elaborate\_GDS(A)) $ \\

\>$\ldots$\\
\textbf{END} \\

\\
\textbf{PROC} $elaborate\_GDS(assertion\, A)$ \\
\> $S \leftarrow \{ NIL \} $ \\
        \>\textbf{FOR} Each assertion  $c$ in $A_{inst}$, the instantiation supporting A \\
$\bigcirc \! \! \! \! 1$      
          \>\>     \textbf{IF} $\left\{ GoalLevel(c) \quad\mbox{closer to top state than}\quad GoalLevel(A) \right\}$ \\

           \>\>\>\>              $append(c,S)\quad$ (append context dependency to GDS) \\
\\
$\bigcirc \! \! \! \! 2$
             \>\>   \textbf{ELSEIF} \{ \>\>\> $GoalLevel(c) \quad$ same as $GoalLevel(A)  \quad\mbox{AND}\quad $ \\
              \>\>\> \> \>    $c$ is NOT an o-supported WME $\quad\mbox{AND}\quad $ \\
\>\>\> \> \>  $c$ has not previously been inspected \} \\
$\bigcirc \! \! \! \! 3$               \>\>\>\>          $S \leftarrow append(S,elaborate\_GDS(c))$ \\
\>\>\>\>\>(compute GDS dependencies for $c$ and add to goal's GDS) \\
$\bigcirc \! \! \! \! 4$               \>\>\>\>          $c_{inspected} \leftarrow true \quad $ \\
\>\>\>\>\>($c$'s context dependencies have been added to the GDS;  \\
\>\>\>\>\>~~ no need to consider it again for this GDS)


\\
\> return S, the list of new dependencies in the GDS \\ 
\textbf{END} \\


%         \>\>\textbf{END(IF)} \\
%\>\textbf{END(FOR)} \\
%\textbf{END(PROC)} \\

\\
\textbf{PROC} $GoalLevel(assertion \quad A)$ \\
\> Return the goal level associated with assertion A

\end{tabbing}
\end{minipage}
}
%\rule{\textwidth}{.5mm}
\caption{The algorithm for determining members of the GDS.}
\label{tab:dhj:proc}
\end{figure}


When WME changes occur, each goal/state must be checked to determine
if the WME appeared on that goal's GDS. Because WME changes occur in
nearly every Soar elaboration cycle, we chose to extend the WME data
structure to avoid this scanning.  Figure~\ref{wme} illustrates the
relationship.  Each GDS consists of a pointer to its goal and a
pointer to a WME DLL list.  The gds\_next and gds\_prev pointers on
WME define the GDS WMEs for a particular GDS and the GDS pointer
provides a link back from each GDS WME to the GDS data structure.

When a WME is removed, the GDS pointer can be checked to determine
immediately if the goal should be removed.  No scanning is necessary.


\begin{figure}
INSERT DIAGRAM HERE
%\insertfigure{NEED DIAGRAM}{3in}
\caption{The GDS and WME data structures}.
\label{wme}
\end{figure}

\subsection*{Other implementation issues}

\begin{itemize}

\item Allocating memory for the GDS \\ The GDS memory is created for
each goal when the goal is created.  The GDS is deallocated when the
goal is removed.  A NIL WME pointer for the GDS indicates a goal has
no WMEs in its GDS.
\item Updating a WME GDS pointer \\ A WME should appear in only the
GDS of the highest goal for which it is dependent.  If a WME is
determined to already be in a GDS lower than the current goal, its GDS
pointer is updated to the higher goal, it is removed from the gds\_WME
DLL of the lower goal, and added to the higher one.  If there are no
other WMEs on the gds\_WME DLL of the lower goal, its WME pointer is
set to NIL (the GDS itself is retained, because we don't want to have
to reallocate memory for the GDS if we need to add to it later.)

\end{itemize}



%\bibliography{general,personal,soar}
%\bibliographystyle{acm}
 



