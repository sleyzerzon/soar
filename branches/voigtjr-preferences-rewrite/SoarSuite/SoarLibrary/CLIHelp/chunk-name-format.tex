\subsection{\soarb{chunk-name-format}}
\label{chunk-name-format}
\index{chunk-name-format}
Specify format of the name to use for new chunks. 
\subsubsection*{Synopsis}
\begin{verbatim}
chunk-name-format [-sl] -p [<prefix>]
chunk-name-format [-sl] -c [<count>]
\end{verbatim}
\subsubsection*{Options}
\begin{tabular}{|l|l|}
\hline 
 -s, --short  & Use the short format for naming chunks  \\
 \hline 
 -l, --long  & Use the long format for naming chunks (default)  \\
 \hline 
 -p, --prefix [$<$prefix$>$]  & If $<$prefix$>$ is given, use $<$prefix$>$ as the prefix for naming chunks. Otherwise, return the current \emph{prefix}
. (defaults to ``\textbf{chunk}
``)  \\
 \hline 
 -c, --count [$<$count$>$]  & If $<$count$>$ is given, set the chunk counter for naming chunks to $<$count$>$. Otherwise, return the current value of the chunk counter.  \\
 \hline 
\end{tabular}
\subsubsection*{Description}
 The short format for naming newly-created chunks is: 
 \emph{prefixChunknum}
 The long (default) format for naming chunks is: 
 \emph{prefix-Chunknum}
*d\emph{dc}
*\emph{impassetype}
*\emph{dcChunknum}
 where: 
 \emph{prefix}
 is a user-definable prefix string; \emph{prefix}
 defaults to ``\textbf{chunk}
`` when unspecified by the user. It many not contain the character *, 
 \emph{Chunknum}
 is $<$count$>$ for the first chunk created, $<$count$>$+1 for the second chunk created, etc. 
 \emph{dc}
 is the number of the decision cycle in which the chunk was formed, 
 \emph{impassetype}
 is one of \textbf{[tie | conflict | cfailure | snochange | opnochange]}
, 
 \emph{dcChunknum}
 is the number of the chunk within that specific decision cycle. 
