\subsection{\soarb{default-wme-depth}}
\label{default-wme-depth}
\index{default-wme-depth}
Set the level of detail used to print WME\~A�\^a�$\neg$\^a��s. 
\subsubsection*{Synopsis}
\begin{verbatim}
default-wme-depth [depth]
\end{verbatim}
\subsubsection*{Options}
\begin{tabular}{|l|l|}
\hline
\soar{ depth } & A non-negative integer.  \\
\hline
\end{tabular}
\subsubsection*{Description}
 The \textbf{default-wme-depth}
 command reflects the default depth used when working memory elements are printed (using the \textbf{print}
 command or \textbf{wmes}
 alias). The default value is 1. When the command is issued with no arguments, \textbf{default-wme-depth}
 returns the current value of the default depth. When followed by an integer value, \textbf{default-wme-depth}
 sets the default depth to the specified value. This default depth can be overridden on any particular call to the \textbf{print}
 or \textbf{wmes}
 command by explicitly using the \textbf{--depth}
 flag, e.g.,\textbf{print --depth 10 \emph{args}
}
. 
 By default, the \textbf{print}
 command prints \emph{objects}
 in working memory, not just the individual working memory element. To limit the output to individual working memory elements, the \textbf{--internal}
 flag must also be specified in the \textbf{print}
 command. Thus when the print depth is \textbf{0}
, by default Soar prints the entire object, which is the same behavior as when the print depth is \textbf{1}
. But if \textbf{--internal}
 is also specified, then a depth of \textbf{0}
 prints just the individual WME, while a depth of \textbf{1}
 prints all WMEs which share that same identifier. This is true when printing timetags, identifiers or WME patterns. 
 When the depth is greater than \textbf{1}
, the identifier links from the specified WME's will be followed, so that additional substructure is printed. For example, a depth of \textbf{2}
 means that the object specified by the identifier, wme-pattern, or timetag will be printed, along with all other objects whose identifiers appear as values of the first object. This may result in multiple copies of the same object being printed out. If \textbf{--internal}
 is also specified, then individuals WMEs and their timetags will be printed instead of the full objects. 
\subsubsection*{Default Aliases}
\begin{tabular}{|l|l|}
\hline
\soar{ Alias } & Maps to  \\
\hline
\soar{ set-default-depth } & default-wme-depth  \\
\hline
\end{tabular}
\subsubsection*{See Also}
\hyperref[print]{print} 