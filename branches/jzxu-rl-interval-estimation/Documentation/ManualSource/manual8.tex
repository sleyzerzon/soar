%---------------------------------------------------------------------------
% Description       : LaTeX source for Soar 8 User's Manual
% Author(s)         : J. Laird, C. Congdon
% Organization      : University of Michigan
%---------------------------------------------------------------------------
%---------------------------------------------------------------------------
% converted to LaTeX2e by Clare Bates Congdon -- Tue Feb 21 14:45:22 1995
%  * also removing latexinfo stuff; new manual will be in latex and html only
% 
% THIS VERSION:
%  * updates manual to Soar 8
% 
%
% TO PRINT A `FINAL' VERSION OF THIS MANUAL, here's the drill:
%   (not necessarily `final', but to be handed out to folks)
%
% 1. comment out any `includeonly's in this file.
% 2. change 'include' to 'input' for appendices -- may fix TOC problem (see
% 	NOTES below).
% 3. run latex repeatedly and fix source files until you don't get any more
% 	errors. (a few font errors seem to be unavoidable).
%       watch for ``overfull'' messages, which mean that your lines are too
%       long. ``underfull'' can often be ignored.
% 3b. run makeindex
% 3c. edit manual.ind for special characters, such as & 
%	I think the best approach here is probably to put \verb+ + around the
%	special chars.
%	Note that if we put the '\verb+ +' in the index command, the chars
%       might be indexed under \verb...which would maybe be okay, now that I
%       think about it. (should still be alphabetically at the front of the
%       manual.) 
% 3d. run latex yet again
% 4. run dvips and use ghostview to have a peek.
% 5. fix errors.
%     (note that the easiest way to remove all the comments, if that's really
%     what you want to do, is to redefine `comment' and 'betacomment' in this
%     file to be the same as 'nocomment')
% 6. pay particular attention to the toc (table of contents) and the function
% 	summary and index.
%    * the appendices often get screwed up, requiring a hand-edit of the toc
%       file 
%    * the commands in the toc look better if you remove the args to
%       user-interface commands (again, by hand)
%    * if any of the commands in the user-interface has changed, chances are
%    	good that the function summary is no longer current. You can edit
%       manual.glo, as described functions.tex, or just make individual
%       changes.
%    NOTE: there is no point in editing the TOC unless you've got everything
%    else perfect. You have to edit it one run before the last one because
%    latex will overwrite it. (It's sometimes a good idea to save an edited
%    toc if you have to make minor changes that will not change page
%    numbering.)
% 7. If everything looks fine, and you've just re-edited the manual.toc file
%    for the last time, run latex and dvips for the last time.
%
% TIPS ON DEBUGGING LATEX (random problems I'm having as I try to wrap this up)
% * DO NOT edit and save a file (including this one) while you are running
%   latex on said file. Latex seems to have a pointer into the file, and
%   saving can cause huge hunks of text to get skipped or duplicated. And
%   you'll get bizarre errors.
% * If you get an ``undefined reference'' error for a label you know you've
%   defined, the most likely cause is that you've forgotten an \end{verbatim}
%   line in the previous section. The next possibility is that you opened a
ls
%   comment environment and closed it way later than you thought.
% * If the page count seems to suddenly shoot up, its a good bet that you
%   forgot an \end{verbatim} somewhere. If it suddenly drops, you probably
%   forgot to close a comment environment.
%
% OTHER CHANGES MADE IN THIS VERSION OF THE MANUAL:
%  * set odd and even margins differently for twosided printing and binding
%  * created \soar mode, which is just \texttt, but will allow us to change
%    this at whim by editing this file only, rather than search and replaces
%    all over creation.
%  * other playing around with typestyles, trying to get bold typewriter font.
%    I think I've finally found one; sansserif font is redefined to the new
%    typewriter font (probably available only at UofM?), which has italic and
%    bold styles.
%  * created \carat, which puts the ^ character into a non-verbose environment
%    (e.g. soar mode). Might as well note here that you can't easily use
%    verbose mode in the middle of text because it won't add line endings in
%    the appropriate places (you'll just get overfulls). That's why soar mode
%    is needed.
%  * also created \tild for adding tildes in
%  * the function summary and index is now tied to the user-interface chapter.
%    It doesn't have to be in the future, but while the command set is in
%    flux, this seems the best way to keep things consistent. I'm using
%    glossary commands (in the user interface chapter) to write to the
%    manual.glo file, which I then edit into the function summary. See the
%    function summary file for instructions.
%
% NOTES:
%  * to print manual without the .ps files for figures, edit this file only.
%    comment out the definition of \insertfigures and uncomment the other
%    version of \insertfigures, which will just leave whitespace
%  * likewise, all ``comments'' in the text can be removed by redefining the
%    \comment environment to be the same as the \nocomment environment below,
%    rather than searching through all files
%  * to print only some of the chapters, use the \includeonly command; see
%    below 
%  * for final version of manual, have to edit the TOC by hand:
%    1. word ``APPENDIX'' appears the line *after* the listing of the first
%       appendix 
%    2. looks a heck of a lot better if we remove the command arguments from
%       the TOC listing for chapter 6. (This can be done in emacs with a
%       replace-rexexp.) (Replace-regexp ``.\texttt.{.*}}'' with ``}'')
%---------------------------------------------------------------------------
\documentclass[12pt,twoside,named]{book}
%\documentclass[12pt,twoside,final]{book}
% final restricts line overruns, but might be default
\usepackage{named}		% for bibliography (named.sty is in this dir)
\usepackage{makeidx}		% for index
\usepackage{epsfig}		% for figures (was 'psfig')
\usepackage[colorlinks]{hyperref}     % for hyperlinks
\usepackage[figure,figure*]{hypcap}   % to correct anchor placement for figures
%\usepackage{amstext}
%\usepackage{verbatim}
%\usepackage{subfigure}
%\usepackage{rotating}
%\usepackage{epic}
%\usepackage{lscape}


%---------------------------------------------------------------------------
% setup the environment
%---------------------------------------------------------------------------
\setlength{\topmargin}{-0.5in}		
\setlength{\oddsidemargin}{.5in}	% for binding, odd and even have
\setlength{\evensidemargin}{0in}	% different margins
\setlength{\textwidth}{6in}             % Gives 1 in. side margins; 1.5in on inside edge 
\setlength{\textheight}{9.2in}		% Gives 1 inch on top and bottom.

\setlength{\parskip}{8pt}
\setlength{\parindent}{0pt}

% try these for fixing spacing in itemize
%  --> none of these seem to work
\setlength{\itemsep}{0pt}	% space between items
\setlength{\parsep}{0pt}	% space between paragraphs w/in an item
\setlength{\topsep}{0pt}	% between prec. text and first item
\setlength{\partopsep}{0pt}	% extra space, if list env is preceeded by blank line

%changing those last two had bad side effects; also changed spacing after
%verbatim environments (but didn't always fix list environments; maybe because
%it is a rubber space)

%\setlength{\topskip}{-20pt}	% have no idea what this will change -->
				%didn't work

% write an .idx file; this may need to be edited to be incorporated
\makeindex
\renewcommand{\seealso}[1]{\textit{see also} #1}

% write an .glo file; this will need to be edited to be incorporated
% --> used for the function summary, NOT for the glossary
\makeglossary

% to play with the numbering depth (defaults for both are 2 in book.cls)
\setcounter{secnumdepth}{3}
\setcounter{tocdepth}{2}

%---------------------------------------------------------------------------
% specify an alternate directory for the ps figures
%---------------------------------------------------------------------------
%\psfigsearchpath{Figures}   
\graphicspath{{Figures/}}   % at Bowdoin

%---------------------------------------------------------------------------
% define new commands
%---------------------------------------------------------------------------

% to add functions to .glo file:
\newcommand{\funsum}[2]{\glossary{#1 & #2 & \textit{#1}}}

% to change spacing 
\newcommand{\dou}{\renewcommand{\baselinestretch}{1.6}\small\normalsize}
\newcommand{\singl}{\renewcommand{\baselinestretch}{1.0}\small\normalsize}
\newcommand{\halfs}{\renewcommand{\baselinestretch}{0.5}\small\normalsize}


% to insert figures (first option allows blank space in lieu of actual .ps file)
% insertcaption inserts a rule after the caption, which is handy in separating
% caption from text

%\newcommand{\insertfigure}[2]{\vspace{2in}}
\newcommand{\insertfigure}[2]{		%\protect\rule[0.0in]{6in}{.01in}
			   %   \capstart   % works with hypcap package, figures must have captions
			      \begin{center}
                              \ \epsfig{file=#1,height=#2}
                              \end{center}}
\newcommand{\inserttwofigs}[3]{\begin{center}
                              \ \psfig{figure=#1.ps,height=#3} \psfig{figure=#2.ps,height=#3}
                              \end{center}}
\newcommand{\insertthreefigs}[4]{\begin{center}
                              \ \psfig{figure=#1.ps,height=#4} \psfig{figure=#2.ps,height=#4}  \psfig{figure=#3.ps,height=#4}
                              \end{center}}

\newcommand{\insertcaption}[1]{\singl\caption{\small#1}
                                        \protect\rule[0.0in]{6in}{.01in}   }

% to include in-text comments (for drafts)
%\newcommand{\comment}[1]{\begin{quote}{\small {\em comment:} \ #1} \end{quote}}
\newcommand{\nocomment}[1]{}
\newcommand{\betacomment}[1]{\begin{quote}{\small {\em Version 7.0.3 comment:} \ #1} \end{quote}}

% To take all comments out, just redefine `comment' and `betacomment' like 'nocomment':
 \newcommand{\comment}[1]{}
% \newcommand{\betacomment}[1]{}

%\newcommand{\nocomment}[1]{\begin{quote}{\small {\em Old comment:} \ #1} \end{quote}}
%\newcommand{\cbc}[1]{\begin{quote}{\small {\em CBC comment:} \ #1} \end{quote}}
%\newcommand{\umcomment}[1]{\begin{quote}{\small {\em UMich comment: \\} \ #1} \end{quote}}

\renewcommand{\cite}[1]{}
%---------------------------------------------------------------------------
% aesthetics: typefaces and special characters
%---------------------------------------------------------------------------

% Change sanserif font to a typewriter font (this is different from the
% default typewriter font, which is a bit narrower, but doesn't have a bold
% weight -- with this command, \textsf will now be a typewriter font.
\renewcommand{\sfdefault}{pcr}    % new typewriter font
%\renewcommand{\sfdefault}{phv}   % helvetica
%\renewcommand{\sfdefault}{ppl}   % palatino, which I rather like

%\renewcommand{\sfdefault}{pcr}  % new typewriter font
%\renewcommand{\ttdefault}{pcr}  % new typewriter font

% to write soar code into text -- soar bold is now a bold tt font.
\newcommand{\soar}[1]{\texttt{#1}}
%\newcommand{\soarb}[1]{\texttt{\textbf{#1}}}   % can't get bold typewriter?
\newcommand{\soarb}[1]{\textsf{\textbf{#1}}}
\newcommand{\soarit}[1]{\textsf{\textit{#1}}}	% italic typewriter
\newcommand{\soarbit}[1]{\textsf{\textit{\textbf{#1}}}}	% bold italic typewriter

%try to get just bold to be the new typewriter font, while leaving regular tt alone
%....not sure how to do this, since it involves two commands.

\def\btt{\fontfamily{cmttss}\fontseries{b}\fontshape{n}\fontsize{12}{13.6}\selectfont}

%carat and tilde symbols
\newcommand{\carat}{\ensuremath{^\wedge}}  %This is how to get a carat symbol
					   %  (^) in the text
\newcommand{\tild}{\ensuremath{\sim}}      %This is how to get a tilde (~) in
					   %  the text 
\newcommand{\cmark}{\ensuremath{\surd}}    %This is how to get a checkmark


%---------------------------------------------------------------------------
% aesthetics: fill the page better with illustrations
% I believe the top \def's are redundant with the \renewcommands
%---------------------------------------------------------------------------
% to fill pages better when there's figures or tables (aka 'floats')
%\def\topfraction{1.0}        %maximum fraction of floats at the top of the page
%\def\bottomfraction{1.0}     %ditto, for the bottom of the page
%\def\textfraction{0}         %minimum fraction of text (--> 100% floats is okay)
%\def\floatpagefraction{0.8}  % if a page is full of floats, it'd better be FULL


% I don't like it when a figure floats on a page with no text, so fiddle with
% these parameters to change this. A trial and error sort of thing; the first
% command didn't seem to do anything. The percentages may be excessive; I
% don't know what the defaults are
\renewcommand{\topfraction}{1.0}          %up to 1.0 of a page can be a figure
\renewcommand{\bottomfraction}{1.0}       %up to 1.0 of a page can be a figure
\renewcommand{\textfraction}{0.0}         %up to 1.0 of a page can be text
\renewcommand{\floatpagefraction}{0.9}    %minimum of .9 of a page for floats only


\setcounter{topnumber}{4}		% up to 4 floats on a page
\setcounter{bottomnumber}{4}		% up to 4 floats on a page


%the distance between a figure and the text on the page (probably in addition
%  to parsep)
\renewcommand{\textfloatsep}{10pt}


%---------------------------------------------------------------------------
% aesthetics: don't use white space to fill to bottom of page
%---------------------------------------------------------------------------
\raggedbottom

%---------------------------------------------------------------------------
% aesthetics: set second level of itemize to be something other than dashes
%  (just playing around for now)
%---------------------------------------------------------------------------
%\renewcommand{\labelitemii}{$\triangleright$}
%\renewcommand{\labelitemii}{$\diamond$}
\renewcommand{\labelitemii}{$\star$}

%---------------------------------------------------------------------------
% I may need to use this to solve the headers problem
%   e.g. preface has header ``list of figures'' and function summary has
%   header ``bibliography''. I suspect that what needs to be redefined in
%   those situations is 'leftmark' and 'rightmark' and not the head itself.
%   Try to change these with renewcommands...
%     \def\@evenhead{\thepage\hfil\slshape\leftmark}%
%     \def\@oddhead{{\slshape\rightmark}\hfil\thepage}%
% Neither of the following (three) approaches work -- first two led to
% 'preface' being the header throughout the manual; third did nothing
%     \renewcommand\leftmark{\textit{PREFACE}}
%     \renewcommand\rightmark{\textit{PREFACE}}
%     \def\leftmark{\textit{PREFACE}}
%     \def\rightmark{\textit{PREFACE}}
%     \def\@evenhead{\thepage\hfil\slshape{PREFACE}}
%     \def\@oddhead{{\slshape{PREFACE}}\hfil\thepage}
%	[renewcommand didn't work with evenhead and oddhead]
%---------------------------------------------------------------------------


%---------------------------------------------------------------------------
% includeonly's, for drafts
%---------------------------------------------------------------------------

%\includeonly{m-preface8,m-intro8,m-architecture8}

%\includeonly{m-multiple,m-advanced}
%\includeonly{m-intro8}

%\includeonly{interface-8.6}
%\includeonly{m-preface8}

%\includeonly{m-preface8,m-architecture8}
%\includeonly{m-functions}

%\includeonly{a-grammars}

%---------------------------------------------------------------------------
% uncomment the next line if you manually edit the toc and index files.
%\nofiles          % don't overwrite the toc and ind files
%---------------------------------------------------------------------------
% BEGIN
%---------------------------------------------------------------------------
\begin{document}

\bibliographystyle{named}
%\pagestyle{empty}			% looks to be overwritten below

%----------------------------------------------------------------------------
% Title page
%----------------------------------------------------------------------------


\begin{titlepage}
\vspace{1.5in}

\begin{center}
\begin{huge}
	The Soar User's Manual \vspace{10pt} \\
	Version 8.6.3 \vspace{20pt} \\
\end{huge}
\begin{large}
	\nocomment{Edition 1}
\end{large} \vspace{36pt}


\begin{large}
	John E. Laird and Clare Bates Congdon  \\
        User interface sections by Karen J. Coulter \\
	Electrical Engineering and Computer Science Department\\
	University of Michigan \vspace{.3in} \\

\end{large}
	Draft of:
	\today

\end{center}


\vspace*{0pt plus 1filll}
	{\em
	Errors may be reported to John E. Laird (laird@umich.edu)\\
        \\
	}
	Copyright \copyright\ 1998 - 2006, The Regents of the University of Michigan
\vspace{.1in}

Development of earlier versions of this manual were supported under
contract N00014-92-K-2015 from the Advanced Systems Technology Office of
the Advanced Research Projects Agency and the Naval Research Laboratory,
and contract N66001-95-C-6013 from the Advanced Systems Technology
Office of the Advanced Research Projects Agency and the Naval Command
and Ocean Surveillance Center, RDT\&E division.

	%\vspace*{0pt plus 1filll}
\comment{
\newpage
\setlength{\parskip}{3pt}

\nocomment
	{\em This is a draft version of the manual; there are lots of little
	changes and a few big changes still to be completed....Some of the
	undone things:
	\begin{enumerate}
	\item all indexing needs to be redone
	\item compare index against glossary to see if I left anything out.
	\item change format for user-interface syntax -- use italics to
		specify args that are variable.
	\item have to regenerate a lot of the examples for the user-interface
		chapter. 
	\item the commands in the user-interface chapter may or may not
		correspond to the final release
	\item appendices are still on the sketchy side... getting better \\
		grammars and o-support definitely still need attention
	\item Needs a few more figures
	\item remember to check for consistency, such as upper/lowercase in
		section headings. (and i-support vs. I-support.) Looking over
		TOC and List of Figures would be a good idea.
	\item impasses are resolved; preferences are evaluated -- check for
		consistency 
	\item watch the difference between productions/operators making
		changes to the state and suggesting changes
	\item it would be nice to add a brief troubleshooting guide \\
		for example, to note what happens when someone is running
		``soar'' (rather than ``soartk''), but tries to load Tk code.
	\end{enumerate}


	Fri Jun 20 16:22:27 1997, Major undone things:

	\begin{enumerate}
	\item Haven't updated learning chapter
	\item Not sure how much ``advanced'' chapter has changed in the source
		since the last printing (leaving this chapter for next pass)

	\item Haven't incorporated comments from Frank and Aladin
	\item should, perhaps, include them in acknowledgements?

	\item indexing

	\item appendices

	\end{enumerate}

\nocomment{
	Things that I think are done:
	\begin{enumerate}
	\item Too much of the Tcl stuff got removed (moved into the ``advanced
		manual''). Added ``advanced'' chapter, but haven't written Tcl
		section yet.
	\item blocks-world task has become a big mess -- too many versions
		have been used for different things: \begin{enumerate}
		\item create THE version of the blocks-world to be used with
			this manual (or possibly two versions, one with
			subgoaling?) 
		\item redo figures
		\item include as an appendix
		\item add to distribution
		\end{enumerate}
	\item subgoal stack illustration is from old blocks-world task and
		should be updated
	\item need a brief description of SDE and pointer to docs
	\item also need pointers to tutorial and advanced manual

	\item check double and single quotes -- not always in latex format.
	\item ``evaluate preferences'', not ``resolve'' (impasses are
		resolved; preferences are evaluated). Exception: I left
		``resolved'' for preferences in just a couple of places, to
		mean that the preferences were evaluated and were not
		contradictory. Similarly, they cannot be resolved if they are
		contradictory.
 	\item change 'object' attribute to 'block' in all examples and
		illustrations -- no, change it to `thing'
	\item make sure that command names shown in examples in early chapters
		are consistent with actual command names -- go/run,
		load/source, etc.
	\item operators are not distinguished objects -- the operator
		augmentation of the state is distinguished (not entirely true
		-- operators ARE distinguished, for example, because their
		preferences are not evaluated until the decision phase)
	\item ``everything retracts'', is the impression we give early on and
		then later, we amend this. Not a great idea. From the
		beginning, make it clear that some retract and some don't
	\item use ``substate'' more often, as in ``soar creates a new
		substate''
	\item don't call them databases; call them memories
	\item we've changed the language for talking about preferences, and
	this has not been consistently incorporated. For example, we're trying
	to say that a preference has i-support or o-support, not that a
	production has i-support or o-support. Also a preference may be an
	operator proposal, operator application, operator termination, or
	elaboration, but productions aren't those four types because
	productions can create multiple preferences that fulfill different
	roles.
	
	\end{enumerate}
	}	 
	}
        }
\normalsize
\setlength{\parskip}{8pt}

\end{titlepage}

% ----------------------------------------------------------------------------
% Table of contents and preface
% ----------------------------------------------------------------------------
\cleardoublepage
\pagestyle{headings}
\pagenumbering{roman}
\setlength{\parskip}{0pt}	% try this for condensing TOC
\tableofcontents

\cleardoublepage

\addcontentsline{toc}{chapter}{Contents}

\listoffigures            	
\cleardoublepage
\setlength{\parskip}{8pt}

%\cleardoublepage
%\addcontentsline{toc}{chapter}{Preface}

%\markboth{PREFACE}{PREFACE}
%\chaptermark{WORKING?}
%\include{m-preface8}

% ----------------------------------------------------------------------------
% Body of document
%
% This version of the manual is primarily the intro and chapters 4, 5, and 8
%   of the old manual (SLCM syntax, chunking, encoding a task)
% leave out appendices for now
% ----------------------------------------------------------------------------
\cleardoublepage
\pagenumbering{arabic}

% ----------------------------------------------------------------------------
\typeout{--------------- INTROduction ---------------------------------------}
\chapter{Introduction}
\label{INTRO}

Soar has been developed to be an architecture for constructing general
intelligent systems. It has been in use since 1983, and has evolved through
many different versions. This manual documents the most current of these:
Soar, version 8.6.

Our goals for Soar include that it is to be an architecture that can: \vspace{-12pt}

\begin{itemize} 
\item be used to build systems that work on the full range of tasks expected
	of an \linebreak intelligent agent, from highly routine to extremely difficult,
	open-ended problems;\vspace{-6pt}
\item represent and use appropriate forms of knowledge, such as procedural,
	declarative, episodic, and possibly iconic;\vspace{-6pt}
\item employ the full range of problem solving methods;\vspace{-6pt}
\item interact with the outside world; and\vspace{-6pt}
\item learn about all aspects of the tasks and its performance on those tasks.
\end{itemize} 

In other words, our intention is for Soar to support all the capabilities
required of a general intelligent agent. Below are the major principles that
are the cornerstones of Soar's design:  \vspace{-12pt}

\begin{enumerate} 
\item The number of distinct architectural mechanisms should be minimized.
        In Soar there is a single representation of permanent knowledge
        (productions), a single representation of temporary knowledge (objects
        with attributes and values), a single mechanism for generating goals
        (automatic subgoaling), and a single learning mechanism (chunking).\vspace{-6pt}

\item All decisions are made through the combination of relevant knowledge at
        run-time.  In Soar, every decision is based on the current
        interpretation of sensory data and any relevant knowledge retrieved
        from permanent memory.  Decisions are never precompiled into
        uninterruptible sequences.
\end{enumerate}


% ----------------------------------------------------------------------------
% ----------------------------------------------------------------------------
\section{Using this Manual}

\nocomment{check that this describes the final form of the manual}

We expect that novice Soar users will read the manual in the order it is
presented: 

\begin{description}
\item[Chapter \ref{ARCH} and Chapter \ref{SYNTAX}] describe Soar from
different perspectives: \textbf{Chapter \ref{ARCH}} describes the Soar
architecture, but avoids issues of syntax, while \textbf{Chapter \ref{SYNTAX}}
describes the syntax of Soar, including the specific conditions and actions
allowed in Soar productions.

\item[Chapter \ref{CHUNKING}] describes chunking, Soar's learning
mechanism.  Not all users will make use of chunking, but it is important to
know that this capability exists.

\item[Chapter \ref{INTERFACE}] describes the Soar user interface --- how the
user interacts with Soar. The chapter is a catalog of user-interface commands,
grouped by functionality.  The most accurate and up-to-date information on the syntax of the Soar User Interface is found online, on the Soar Wiki, at


\hspace{2em}\soar{http://winter.eecs.umich.edu/soar}.

\end{description}

Advanced users will refer most often to Chapter \ref{INTERFACE}, flipping back
to Chapters \ref{ARCH} and \ref{SYNTAX} to answer specific questions.

There are several appendices included with this manual: 
\begin{description}

%\item[Appendix \ref{GLOSSARY}] is a glossary of terminology used in this manual.

\item[Appendix \ref{BLOCKSCODE}] contains an example Soar program for a simple
version of the blocks world. This blocks-world program is used as an example
throughout the manual.

%\item[Appendix \ref{USING}] is an overview of example programs currently available
%(provided with the Soar distribution) with explanations of how to run them,
%and pointers to other help sources available for novices.

%\item[Appendix \ref{DEFAULT}] describes Soar's default knowledge, which can be used
%(or not) with any Soar task.

\item[Appendix \ref{GRAMMARS}] provides a grammar for Soar productions.

\item[Appendix \ref{SUPPORT}] describes the determination of o-support.

\item[Appendix \ref{PREFERENCES}] provides a detailed explanation of the preference
resolution process.

%\item[Appendix \ref{Tcl-I/O}] gives an example of Soar I/O functions, written in Tcl.

\item[Appendix \ref{GDS}] provides an explanation of the Goal Dependency Set. 
\end{description}

\subsubsection*{Additional Back Matter}

The appendices are followed by an index; the last
pages of this manual contain a summary and index of the user-interface
functions for quick reference.


\subsubsection*{Not Described in This Manual}

Some of the more advanced features of Soar are not described in this
manual, such as how to interface with a simulator, or how to create Soar
applications using multiple interacting agents.  A discussion of
these topics is provided in a separate document, the \textit{SML Quick Start Guide}.

For novice Soar users, try \textit{The Soar 8 Tutorial}, which guides the reader 
through several example tasks and exercises.

See Section \ref{CONTACT} for information about obtaining Soar documentation.

% ----------------------------------------------------------------------------
%\section{Other Soar Documentation}
%\label{DOCUMENTATION}
%
%In addition to this manual, there are three other documents that you may want
%to obtain for more information about different aspects of Soar:
%
%\begin{description}
%\item[The Soar 8 Tutorial] is written for novice Soar users, and guides the
%	reader through several example tasks and exercises.
%\item[The Soar Advanced Applications Manual] is written for advanced Soar
%	users. This guide describes how to add input and output routines to
%	Soar programs, how to run multiple Soar ``agents'' from a single Soar
%	image, and how to extend Soar by adding your own user-interface
%	functions, simulators, or graphical user interfaces.
%\item[Soar Design Dogma] gives advice and examples about good Soar programming style. 
%        It may be helpful to both the novice and mid-level Soar user. 
%\end{description}
% ----------------------------------------------------------------------------
\section{Contacting the Soar Group}
\label{CONTACT}

\subsection*{Resources on the Internet}

The primary website for Soar is:

\hspace{2em}\soar{http://sitemaker.umich.edu/soar}.

Look here for the latest downloads, documentation, and Soar-related announcements, as well
as links to information about specific Soar research projects and researchers and a FAQ
(list of frequently asked questions) about Soar.

For questions about Soar, you may write to the Soar e-mail list at:

\hspace{2em}\soar{soar-group@lists.sourceforge.net}.

If you would like to be on this list yourself, visit:

\hspace{2em}\soar{http://lists.sourceforge.net/lists/listinfo/soar-group}.

To report Soar bugs, to check whether a bug has been reported, or to check the status
of a previously reported bug, visit:

\hspace{2em}\soar{https://winter.eecs.umich.edu/soar-bugzilla/}.


 

%The online FAQ will usually contain the most current information on Soar. It
%is available at: 

%\soar{http://acs.ist.psu.edu/soar-faq/soar-faq.html}


\subsection*{For Those Without Internet Access}

If you cannot reach us on the internet, please write to us at the following 
address:

\begin{flushleft}
\hspace{2em}The Soar Group \\
\hspace{2em}Artificial Intelligence Laboratory \\
\hspace{2em}University of Michigan\\
\hspace{2em}1101 Beal Ave.\\
\hspace{2em}Ann Arbor, MI 48109-2110 \\
\hspace{2em}USA 
\end{flushleft}

% ----------------------------------------------------------------------------
% ----------------------------------------------------------------------------
\section{A Note on Different Platforms and Operating Systems}
\label{INTRO-platforms}
\index{Unix}
\index{Linux}
\index{Macintosh}
\index{Personal Computer}
\index{Windows}
\index{Operating System}

Soar runs on a wide variety of computers, including Unix (and Linux) machines,
Macintoshes running OSX, and PCs running the Windows XP (and probably 2000 and NT) operating system.

This manual documents Soar generally, although all references to files
and directories use Unix format conventions rather than Windows-style folders.


%\include{m-overview}
% ----------------------------------------------------------------------------
\typeout{--------------- The Soar ARCHitecture ------------------------------}
\chapter{The Soar Architecture}
\label{ARCH}

This chapter describes the Soar architecture.  It covers all aspects of Soar
except for the specific syntax of Soar's memories and descriptions of the
Soar user-interface commands.

This chapter gives an abstract description of Soar.  It starts by giving
an overview of Soar and then goes into more detail for each of Soar's
main memories (working memory, production memory, and preference memory)
and processes (the decision procedure, learning, and input and output).

% ----------------------------------------------------------------------------
\section{An Overview of Soar}
\label{ARCH-overview}

The design of Soar is based on the hypothesis that all deliberate
\textit{goal}-oriented behavior can be cast as the selection and application
of \textit{operators} to a \textit{state}. A state is a representation of the
current problem-solving situation; an operator transforms a state (makes
changes to the representation); and a goal is a desired outcome of the
problem-solving activity.

As Soar runs, it is continually trying to apply the current operator and
select the next operator (a state can have only one operator at a time),
until the goal has been achieved. The selection and application of
operators is illustrated in Figure \ref{fig:select-apply}. 

\begin{figure}
\insertfigure{select-apply}{1.5in}
\insertcaption{Soar is continually trying to select and apply operators.}
\label{fig:select-apply}
\end{figure}

Soar has separate memories (and different representations) for
descriptions of its current situation and its long-term knowledge.  In
Soar, the current situation, including data from sensors, results of
intermediate inferences, active goals, and active operators is held in
\emph{working memory}.  Working memory is organized as
\emph{objects}. Objects are described in terms of their
\emph{attributes}; the values of the attributes may correspond to
sub-objects, so the description of the state can have a hierarchical
organization. (This need not be a strict hierarchy; for example, there's
nothing to prevent two objects from being ``substructure'' of each
other.)

The long-term knowledge, which specifies how to respond to different
situations in working memory, can be thought of as the program for Soar.
The Soar architecture cannot solve any problems without the addition of
long-term knowledge.  (Note the distinction between the ``Soar
architecture'' and the ``Soar program'': The former refers to the system
described in this manual, common to all users, and the latter refers to
knowledge added to the architecture.)

A Soar program contains the knowledge to be used for solving a specific
task (or set of tasks), including information about how to select and
apply operators to transform the states of the problem, and a means of
recognizing that the goal has been achieved.  

\subsection{Problem-Solving Functions in Soar}
\label{ARCH-functions}
\index{problem solving!functions}

\label{LIST:5functions}
All of Soar's long-term knowledge is organized around the functions of operator 
selection and operator application, which are organized into  four distinct types of
knowledge:\vspace{-10pt} 
\begin{description}
\item \textbf{Knowledge to select an operator} \\ \vspace{-20pt}
  \begin{enumerate}
  \item \textit{Operator Proposal:} Knowledge that an operator is appropriate for the current situation. 
  \item \textit{Operator Comparison:} Knowledge to compare candidate operators.
  \item 	\textit{Operator Selection:} Knowledge to select a single operator, based on the comparisons.
  \end{enumerate}
\item \textbf{Knowledge to apply an operator} \\  \vspace{-20pt}
  \begin{enumerate} \setcounter{enumi}{3}
  \item \textit{Operator Application:} Knowledge of how a specific operator modifies the state.
\end{enumerate}
\end{description}
In addition, there is a fifth type of knowledge in Soar that is
indirectly connected to both operator selection and operator application:

\begin{description}
% a hack to make the next bullet line up with the previous ones
\makeatletter
\@newlistfalse
\makeatother
% /hack
\begin{enumerate} \setcounter{enumi}{4}
\item Knowledge of monotonic inferences that can be made about the state
(\textit{state elaboration}). 
\end{enumerate}
\end{description}
State elaborations indirectly affect operator selection and application
by creating new descriptions of the current situation that can cue
the selection and application of operators.

\index{production}
\index{production!match}
\index{match|see{production!match}}
\index{fire|see{production!firing}}
\index{retract|see{production!retraction}}

These problem-solving functions are the primitives for generating behavior
in Soar.  Four of the functions require retrieving long-term knowledge
that is relevant to the current situation: elaborating the state,
proposing candidate operators, comparing the candidates, and applying
the operator by modifying the state. These functions are driven by the
knowledge encoded in a Soar program.  Soar represents that knowledge as
\textit{production} rules.  Production rules are similar to ``if-then''
statements in conventional programming languages. (For example, a
production might say something like ``if there are two blocks on the
table, then suggest an operator to move one block ontop of the other
block'').  The ``if'' part of the production is called its
\textit{conditions} and the ``then'' part of the production is called
its \textit{actions}. When the conditions are met in the current
situation as defined by working memory, the production is \emph{matched}
and it will \emph{fire}, which means that its actions are executed,
making changes to working memory. 

The other function, selecting the current operator, involves making a
decision once sufficient knowledge has been retrieved.  This is
performed by Soar's \emph{decision procedure}, which is a fixed
procedure that interprets \emph{preferences} that have been created by
the retrieval functions. The knowledge-retrieval and decision-making
functions combine to form Soar's \emph{decision cycle}.

\index{decision procedure}
\index{decision cycle}
\index{impasse}

When the knowledge to perform the problem-solving functions is not
directly available in productions, Soar is unable to make progress and
reaches an \textit{impasse}.  There are three types of possible impasses
in Soar:
\begin{enumerate}
\item An operator cannot be selected because none are proposed.\vspace{-4pt}
\item An operator cannot be selected because multiple operators are
        proposed and the comparisons are insufficient to determine which
        one should be selected.\vspace{-4pt}
\item An operator has been selected, but there is insufficient knowledge
        to apply it.\vspace{-4pt}
\end{enumerate}
In response to an impasse, the Soar architecture creates a
\textit{substate} in which operators can be selected and applied to
generate or deliberately retrieve the knowledge that was not directly
available; the goal in the substate is to resolve the impasse. For
example, in a substate, a Soar program may do a lookahead search to
compare candidate operators if comparison knowledge is not directly
available.  Impasses and substates are described in more detail in
Section \ref{ARCH-impasses}.


% ----------------------------------------------------------------------------
\subsection{An Example Task: The Blocks-World}

We will use a task called the blocks-world as an example throughout this
manual. In the blocks-world task, the initial state has three blocks named
\soar{A}, \soar{B}, and \soar{C} on a table; the operators move one block at a
time to another location (on top of another block or onto the table); and the
goal is to build a tower with \soar{A} on top, \soar{B} in the middle, and
\soar{C} on the bottom. The initial state and the goal are illustrated in
Figure \ref{fig:blocks}.

The Soar code for this task is included in Appendix \ref{BLOCKSCODE}. You do not
need to look at the code at this point.

\begin{figure}
\insertfigure{blocks}{2in}
\insertcaption{The initial state and goal of the ``blocks-world'' task.}
\label{fig:blocks}
\end{figure}

The operators in this task move a single block from its current location to a
new location; each operator is represented with the following information: 
\vspace{-12pt}
\begin{itemize}
\item the name of the block being moved \vspace{-9pt}
\item the current location of the block (the ``thing'' it is on top of) \vspace{-9pt}
\item the destination of the block (the ``thing'' it will be on top of) 
\vspace{-9pt}
\end{itemize}

The goal in this task is to stack the blocks so that \soar{C} is on the
table, with block \soar{B} on block \soar{C}, and block \soar{A} on
top of block \soar{B}.

% ----------------------------------------------------------------------------
\subsection{Representation of States, Operators, and Goals}
\label{OVERVIEW-ps-representation}
\index{state!representation}
\index{operator!representation}
\index{goal!representation}
\index{problem space!representation}
\index{attribute}

The initial state in our blocks-world task --- before any operators have been
proposed or selected --- is illustrated in Figure \ref{fig:ab-wmem}.

\begin{figure}
\insertfigure{ab-wmem}{3.5in}
\insertcaption{An abstract illustration of the initial state of the blocks
	world as working memory objects. At this stage of problem solving, no
	operators have been proposed or selected.}
\label{fig:ab-wmem}
\end{figure}

A state can have only one operator at a time, and the operator is represented
as substructure of the state. A state may also have as substructure a number
of \emph{potential} operators that are in consideration; however, these
suggested operators should not be confused with the current operator. 

Figure \ref{fig:ab-wmem2} illustrates working memory after the first operator
has been selected. There are six operators proposed, and only one of
these is actually selected.

\begin{figure}
\insertfigure{ab-wmem2}{4.25in}
\insertcaption{An abstract illustration of working memory in the blocks world
	after the first operator has been selected.}
\label{fig:ab-wmem2}
\end{figure}

Goals are either represented explicitly as substructure of the state
with general rules that recognize when the goal is achieved, or are
implicitly represented in the Soar program by goal-specific rules that
test the state for specific features and recognize when the goal is
achieved.  The point is that sometimes a description of the goal will be
available in the state for focusing the problem solving, whereas other
times it may not.  Although representing a goal explicitly has many advantages,
some goals are difficult to explicitly represent on the state.

The goal in our blocks-world task is represented implicitly in the Soar
program. A single production rule monitors the state for completion of the goal and
halts Soar when the goal is achieved.

% ----------------------------------------------------------------------------
\subsection{Proposing candidate operators}
\index{operator!proposal}

As a first step in selecting an operator, one or more candidate
operators are proposed.  Operators are proposed by rules that test
features of the current state.  When the blocks-world task is run, the
Soar program will propose six distinct (but similar) operators for the
initial state as illustrated in Figure \ref{fig:proposal}. These
operators correspond to the six different actions that are possible
given the initial state.

\begin{figure}
\insertfigure{blocks-proposal}{2.5in}
\insertcaption{The six operators proposed for the initial state of the blocks
	world each move one block to a new location.}
\label{fig:proposal}
\end{figure}


% ----------------------------------------------------------------------------
\subsection{Comparing candidate operators: Preferences}
\index{operator!comparison}

The second step Soar takes in selecting an operator is to evaluate or
compare the candidate operators. In Soar, this is done via rules that
test the proposed operators and the current state, and then create
\emph{preferences}.  Preferences assert the relative or absolute merits of the
candidate operators. For example, a preference may say that operator A
is a ``better'' choice than operator B at this particular time, or a
preference may say that operator A is the ``best'' thing to do at this
particular time.

% ----------------------------------------------------------------------------
\subsection{Selecting a single operator}
\index{operator!selection}

Soar attempts to select a single operator based on the preferences available
for the candidate operators. There are four different situations that may
arise:\vspace{-14pt}
\begin{enumerate}
\item The available preferences unambiguously prefer a single operator.\vspace{-
6pt}
\item The available preferences suggest multiple operators, and 
       prefer a subset that can be selected from randomly.\vspace{-6pt}
\item The available preferences suggest multiple operators,but neither case
       1 or 2 above hold.\vspace{-6pt}
\item The available preferences do not suggest any operators.
\end{enumerate}

In the first case, the preferred operator is selected.  In the second
case, one of the subset is selected randomly. In the third and fourth
cases, Soar has reached an ``impasse'' in problem solving, and a new
substate is created.  Impasses are discussed in Section
\ref{ARCH-impasses}.

In our blocks-world example, the second case holds, and Soar can select one of
the operators randomly.

% ----------------------------------------------------------------------------
\subsection{Applying the operator}
\index{operator!application}

An operator applies by making changes to the state; the specific changes
that are appropriate depend on the operator and the current state.

\index{I/O}
\index{problem solving!internal}
\index{problem solving!external}
There are two primary approaches to modifying the state: indirect and direct.
\emph{Indirect} changes are used in Soar programs that interact with
an external environment: The Soar program sends motor commands to the
external environment and monitors the external environment for
changes. The changes are reflected in an updated state description,
garnered from sensors. Soar may also make \emph{direct} changes to the
state; these correspond to Soar doing problem solving ``in its
head''. Soar programs that do not interact with an external environment
can make only direct changes to the state.

Internal and external problem solving should not be viewed as mutually
exclusive activities in Soar. Soar programs that interact with an
external environment will generally have operators that make direct and
indirect changes to the state: The motor command is represented as
substructure of the state \emph{and} it is a command to the environment. Also, a Soar program may maintain an internal
model of how it expects an external operator will modify the world; if
so, the operator must update the internal model (which is substructure
of the state).

When Soar is doing internal problem solving, it must know how to modify
the state descriptions appropriately when an operator is being
applied. If it is solving the problem in an external environment, it
must know what possible motor commands it can issue in order to affect
its environment.

The example blocks-world task described here does not interact with an external
environment. Therefore, the Soar program directly makes changes to the state
when operators are applied. There are four changes that may need to be made
when a block is moved in our task: \vspace{-14pt}
\label{LIST:blocks-app}
\begin{enumerate}
\item The block that is being moved is no longer where it was (it is no longer
   	``on top'' of the same thing).\vspace{-6pt}
\item The block that is being moved is now in a new location (it is ``on top''
	of a new thing).\vspace{-6pt}
\item The place that the block used to be is now clear.\vspace{-6pt}
\item The place that the block is moving to is no longer clear --- unless it
	is the table, which is always considered ``clear''\footnote{In this
	blocks-world task, the table always has room for another block, so it
	is represented as always being ``clear''.}.
\end{enumerate}

The blocks-world task could also be implemented using an external simulator. In this case,
the Soar program does not update all the ``on top'' and ``clear'' relations;
the updated state description comes from the simulator.

% ----------------------------------------------------------------------------
\subsection{Making inferences about the state}

Making monotonic inferences about the state is the other role that Soar
long-term knowledge may fulfill. Such elaboration knowledge can simplify
the encoding of operators because entailments of a set of core features
of a state do not have to be explicitly included in application of the
operator.  In Soar, these inferences will be automatically retracted
when the situation changes such that the inference no longer holds.

For instance, our example blocks-world task uses an elaboration to keep track
of whether or not a block is ``clear''. The elaboration tests for the absence
of a block that is ``on top'' of a particular block; if there is no such ``on top'',
the block is ``clear''. When an operator application creates a new ``on top'', the
corresponding elaboration retracts, and the block is no longer ``clear''.


% ----------------------------------------------------------------------------
\subsection{Problem Spaces}
\label{ARCH-functions-ps}

\index{problem space}
If we were to construct a Soar system that worked on a large number of
different types of problems, we would need to include large numbers of
operators in our Soar program. For a specific problem and a
particular stage in problem solving, only a subset of all possible operators
are actually relevant. For example, if our goal is to \textit{count} the
blocks on the table, operators having to do with moving blocks are probably
not important, although they may still be ``legal''. The operators that are
relevant to current problem-solving activity define the space of possible
states that might be considered in solving a problem, that is, they define the
\emph{problem space}.

Soar programs are implicitly organized in terms of problem spaces
because the conditions for proposing operators will restrict an operator
to be considered only when it is relevant.  The complete problem space
for the blocks world is show in Figure \ref{fig:blocks-ps}.  Typically,
when Soar solves a problem in this problem space, it does not explicitly
generate all of the states, examine them, and then create a path.
Instead, Soar is \emph{in} a specific state at a given time (represented
in working memory), attempting to select an operator that will move it
to a new state.  It uses whatever knowledge it has about selecting
operators given the current situation, and if its knowledge is
sufficient, it will move toward its goal.
\begin{figure}
\insertfigure{blocks-ps}{4.9in}
\insertcaption{The problem space in the blocks-world includes all operators
	that move blocks from one location to another and all possible
	configurations of the three blocks.}
\label{fig:blocks-ps}
\end{figure}
The same problem could be recast in Soar as a planning problem, where
the goal is to develop a plan to solve the problem, instead of just
solving the problem.  In that case, a state in Soar would consist of a
plan, which in turn would have representations of Blocks World states
and operators from the original space.  The operators would perform
editing operations on the plan, such as adding new Blocks World
operators, simulating those operators, etc.  In both formulations of the
problem, Soar is still applying operators to generate new states, it is
just that the states and operators have different content.

The remaining sections in this chapter describe the memories and processes of Soar:
working memory, production memory, preference memory, Soar's execution cycle (the decision
procedure), learning, and how  input and output fit in.
% ----------------------------------------------------------------------------
\section{Working memory: The Current Situation} 
\label{ARCH-wm}
\index{working memory|textbf}

\index{working memory}
Soar represents the current problem-solving situation in its \emph{working
memory}. Thus, working memory holds the current state and operator and is Soar's
``short-term'' knowledge, reflecting the current knowledge of the world and
the status in problem solving.

\index{working memory element}
\index{WME|see{working memory element}}
\index{identifier}
\index{attribute}
\index{value}
Working memory contains elements called working memory elements, or WME's for
short. Each WME contains a very specific piece of information; for example, a WME
might say that ``B1 is a block''. 
Several WME's collectively may provide more information about the same
\textit{object}, for example, ``B1 is a block'', ``B1 is named A'', ``B1 is on
the table'', etc. These WME's are related because they are all contributing to
the description of something that is internally known to Soar as ``B1''. B1 is
called an \textit{identifier}; the group of WME's that share this identifier
are referred to as an \textit{object} in working memory. 
Each WME describes a different \textit{attribute} of the object, for example,
its name or type or location; each \textit{attribute} has a \textit{value} associated
with it, for example, the name is A, the type is block, and the position is on
the table. Therefore, each WME is an identifier-attribute-value triple, and
all WME's with the same identifier are part of the same object.

\index{working memory!object}
\index{identifier}
\index{attribute}
\index{value}
\index{link}
Objects in working memory are \emph{linked} to other objects: The value of one
WME may be an identifier of another object. For example, a WME might say that
``B1 is ontop of T1'', and another collection of WME's might describe the
object T1: ``T1 is a table'', ``T1 is brown'', and ``T1 is ontop of F1''. And
still another collection of WME's might describe the object F1: ``F1 is a
floor'', etc. All objects in working memory must be linked to a state, either
directly or indirectly (through other objects). Objects that are not linked to
a state will be automatically removed from working memory by the Soar
architecture. 

\index{augmentation|see{working memory element}}
WME's are also often called \textit{augmentations} because they
``augment'' the object, providing more detail about it. While these two
terms are somewhat redundant, WME is a term that is used more often to
refer to the contents of working memory, while augmentation is a term
that is used more often to refer to the description of an object.
Working memory is illustrated at an abstract level in Figure
\ref{fig:ab-wmem} on page \pageref{fig:ab-wmem}. 


The attribute of an augmentation is usually a constant, such as \soar{name} or
\soar{type}, because in a sense, the attribute is just a label used to
distinguish one link in working memory from another.\footnote{In order to
allow these links to have some substructure, the attribute name may be an
identifier, which means that the attribute may itself have attributes and
values, as specified by additional working memory elements.}

The value of an augmentation may be either a constant, such as \soar{red}, or
an identifier, such as \soar{06}. When the value is an identifier, it refers
to an object in working memory that may have additional substructure. In
semantic net terms, if a value is a constant, then it is a terminal node with
no links; if it is an identifier it is a nonterminal node.

\index{multi-valued attribute}
\index{multi-attribute|see{multi-valued attribute}}
One key concept of Soar is that working memory is a set, which means that there can never be two elements in
working memory at the same time that have the same identifier-attribute-value
triple (this is prevented by the architecture). However, it is possible to have
multiple working memory elements that have the same identifier and attribute,
but that each have different values.  When this happens, we say the attribute
is a \emph{multi-valued attribute}, which is often shortened to be
\emph{multi-attribute}.

An object is defined by its augmentations and
\emph{not} by its identifier. An identifier is simply a label or pointer to the object. On subsequent runs of the same Soar program,
there may be an object with exactly the same augmentations, but a different
identifier, and the program will still reason about the object
appropriately. Identifiers are internal markers for Soar; they can appear
in working memory, but they never appear in a production.

There is no predefined relationship between objects in working memory and
``real objects'' in the outside world.  Objects in working memory may refer to
real objects, such as \soar{block A}; features of an object, such as the
color \soar{red} or shape \soar{cube}; a relation between objects, such as \soar{ontop}; classes of
objects, such as \soar{blocks}; etc. The actual names of attributes and
values have no meaning to the Soar architecture (aside from a few WME's
created by the architecture itself). For example, Soar doesn't care whether
the things in the blocks world are called ``blocks'' or ``cubes'' or
``chandeliers''. It is up to the Soar programmer to pick suitable labels and to
use them consistently.

The elements in working memory arise from one of four sources:\vspace{-12pt}
\begin{enumerate}
\item The actions of productions create most working memory elements. 
\vspace{-8pt}
\item The decision procedure automatically creates some special 
   state augmentations (type, superstate, impasse, ...)
	whenever a state is created.  States are created during 
	initialization (the first state) or because of an impasse (a 
substate).  \vspace{-8pt}
\item The decision procedure creates the operator augmentation of the state 
based on
preferences.  This records the selection of the current operator.\vspace{-8pt}
\item External I/O systems create working memory elements on the input-link
for sensory data.
\end{enumerate}

The elements in working memory are removed in six different ways:\vspace{-12pt}
\begin{enumerate}
\item The decision procedure automatically removes all state
augmentations it creates when the impasse that led to their creation is 
resolved.\vspace{-8pt}
\item The decision procedure removes the operator augmentation of the
state when that operator is no longer selected as the current operator.\vspace{-
8pt}
\item Production actions that use \soar{reject} preferences remove
      working memory elements that were created by other productions.\vspace{-8pt}
\item The architecture automatically removes i-supported WMEs when the productions that created them no longer match.\vspace{-8pt}
\item The I/O system removes sensory data from the input-link when it
is no longer valid. \vspace{-8pt}
\item The architecture automatically removes WME's that are no longer linked to 
a state (because some other WME has been removed).
\end{enumerate}


\index{state}
For the most part, the user is free to use any attributes and values
that are appropriate for the task. However, states have special
augmentations that cannot be directly created, removed, or modified by
rules.  These include the augmentations created when a state is created,
and the state's operator augmentation that signifies the current
operator (and is created based on preferences).  The specific
attributes that the Soar architecture automatically creates are listed in Section
\ref{SYNTAX-impasses}. Productions may create any other attributes for
states.

Preferences are held in a separate \emph{preference memory} where they cannot be
tested by productions; however, \soar{acceptable} preferences 
are held in \emph{both} preference memory and in working memory. By making the
\soar{acceptable} preferences available in working memory, 
the acceptable preferences can be tested for in productions allowing the
candidates operators to be compared before they are selected.

% ----------------------------------------------------------------------------
\section{Production Memory: Long-term Knowledge} 
\label{ARCH-pm}
\index{production memory|textbf}
\index{production}

\begin{figure}
\insertfigure{ab-prodmem}{3.5in}
\insertcaption{An abstract view of production memory. The productions are not
	related to one another.}
\label{fig:ab-prodmem}
\end{figure}

\index{production}
\index{production!firing}
Soar represents long-term knowledge as \emph{productions} that are stored in
\emph{production memory}, illustrated in Figure \ref{fig:ab-prodmem}. Each
production has a set of conditions and a set of actions.  If the
conditions of a production match working memory, the production
\emph{fires}, and the actions are performed.

\subsection{The structure of a production}
\label{ARCH-pm-structure}

In the simplest form of a production, conditions and actions refer directly to
the presence (or absence) of objects in working memory. For example, a
production might say:
\begin{verbatim}
  CONDITIONS: block A is clear 
              block B is clear 
  ACTIONS:    suggest an operator to move block A ontop of block B
\end{verbatim}
This is not the literal syntax of productions, but a simplification.
The actual syntax is presented in Chapter \ref{SYNTAX}.

The conditions of a production may also specify the \emph{absence} of patterns
in working memory. For example, the conditions could also specify that ``block
A is not red'' or ``there are no red blocks on the table''. But since these are
not needed for our example production, there are no examples of negated
conditions for now.

The order of the conditions of a production do not matter to Soar except
that the first condition must directly test the state. Internally, Soar
will reorder the conditions so that the matching process can be more
efficient. This is a mechanical detail that need not concern most
users. However, you may print your productions to the screen or save
them in a file; if they are not in the order that you expected them to
be, it is likely that the conditions have been reordered by Soar.

\subsubsection{Variables in productions and multiple instantiations}

In the example production above, the names of the blocks are ``hardcoded'',
that is, they are named specifically. In Soar productions, variables are used
so that a production can apply to a wider range of situations.
\index{production!instantiation}

The variables are bound to specific symbols in working memory elements
by Soar's matching process.  A production along with a specific and
consistent set of variable bindings is called an \emph{instantiation}. A
production instantiation is consistent only if every occurrence of a
variable is bound to the same value.  Since the same production may
match multiple times, each with different variable bindings, several
instantiations of the same production may match at the same time and,
therefore, fire at the same time.  If blocks \soar{A} and \soar{B} are
clear, the first production (without variables) will suggest one
operator. However, if a production was created that used variables to
test the names, this second production will be instantiated twice and
therefore suggest \textit{two} operators: one operator to move block
\soar{A} ontop of block \soar{B} and a second operator to move block
\soar{B} ontop of block \soar{A}.
\index{instantiation|see{production!instantiation or operator!instantiation}}

Because the identifiers of objects are determined at runtime, literal
identifiers cannot appear in productions. Since identifiers occur in
every working memory element, variables must be used to test for
identifiers, and using the same variables across multiple occurrences is what links conditions together.

Just as the elements of working memory must be linked to a state 
in working memory, so must the objects referred to in a production's
conditions. That is, one condition must test a state object 
\emph{and} all other conditions must test that same state or objects that
are linked to that state.

\subsection{Architectural roles of productions}
\label{ARCH-pm-roles}
\index{production!roles}

Soar productions can fulfill four different roles: the three
knowledge-retrieval problem-solving functions, and the state elaboration function, all described on page \pageref{LIST:5functions}:\vspace{-10pt}
\begin{enumerate}
\item Operator proposal\vspace{-10pt}
\item Operator comparison\vspace{-10pt}
\item \textit{(Operator selection is not an act of knowledge 
retrieval)}\vspace{-10pt}
\item Operator application\vspace{-10pt}
\item State elaboration
\end{enumerate}

A single production should not fulfill more than one of these roles
(except for proposing an operator and creating an absolute preference
for it).  Although productions are not declared to be of one type or the
other, Soar examines the structure of each production and classifies the
rules automatically based on whether they propose and compare operators,
apply operators, or elaborate the state. 

\subsection{Production Actions and Persistence}
\index{I-support}
\index{production actions}
\label{ARCH-prefmem-persistence}
\index{persistence}
\index{O-support}
\index{operator!support}
\label{PAGE:O-support}

Generally, actions of a production either create preferences for
operator selection, or create/remove working memory elements.  For
operator proposal and comparison, a production creates preferences for
operator selection.  These preferences should persist only as long as
the production instantiation that created them continues to match.  When
the production instantiation no longer matches, the situation has
changed, making the preference no longer relevant.  Soar automatically
removes the preferences in such cases.  These preferences are said to
have \emph{I-support} (for ``instantiation support'').  Similarly, state
elaborations are simple inferences that are valid only so long as the
production matches.  Working memory elements created as state
elaborations also have I-support and remain in working memory only as
long as the production instantiation that created them continues to
match working memory.  For example, the set of relevant operators changes
as the state changes, thus the proposal of operators is done with
I-supported preferences. This way, the operator proposals will be
retracted when they no longer apply to the current situation.

However, the actions of productions that \emph{apply} an operator, either by
adding or removing elements from working memory, need to persist even
after the operator is no longer selected and operator application
production instantiation no longer matches.  For example, in placing a
block on another block, a condition is that the second block be
clear. However, the action of placing the first block removes the fact
that the second block is clear, so the condition will no longer be
satisfied.

Thus, operator application productions do not retract their actions, even
if they no longer match working memory.  This is called \emph{O-support}
(for ``operator support''). Working memory elements that participate in
the application of operators are maintained throughout the existence of
the state in which the operator is applied, unless explicitly removed (or
if they become unlinked).  Working memory elements are removed by a
\emph{reject} action of a operator-application rule.  
\index{O-support!reject}

Whether a working memory element receives O-support or I-support is
determined by the structure of the production instantiation that creates
the working memory element.  O-support is given only to working memory
elements created by operator-application productions.

An operator-application production tests the current operator of a state
and modifies the state. Thus, a working memory element receives
O-support if it is for an augmentation of the current state or
substructure of the state, and the conditions of the instantiation that
created it test augmentations of the current operator.  

When productions are matched, all productions that have their conditions
met fire creating or removing working memory elements.  Also, working
memory elements and preferences that lose I-support are removed from
working memory. Thus, several new working memory elements and
preferences may be created, and several existing working memory elements
and preferences may be removed at the same time.  (Of
course, all this doesn't happen literally at the same time, but the
order of firings and retractions is unimportant, and happens in parallel
from a functional perspective.)

% ----------------------------------------------------------------------------
\section{Preference memory: Selection Knowledge} 
\label{ARCH-prefmem}
%\label{ch-abst-symb-pref}
\index{preference}
\index{preference memory}

\nocomment{need to find the right word there. None of selection, evaluation, or
	comparison seems quite right.}

The selection of the current operator is determined by the \emph{preferences} in
\emph{preference memory}. Preferences are suggestions or imperatives about the
current operator, or information about how suggested operators compare
to other operators.  Preferences refer to operators by using the
identifier of a working memory element that stands for the operator.
After preferences have been created for a state, the decision procedures
evaluates them to select the current operator for that state.

For an operator to be selected, there will be at least one preference
for it, specifically, a preference to say that the value is a candidate
for the operator attribute of a state (this is done with either an
``\soar{acceptable}'' or ``\soar{require}'' preference). There may also
be others, for example to say that the value is ``best''.

\index{persistence}
\index{preference!persistence|see{persistence}}
The different preferences available and the semantics of preferences are
explained in Section \ref{ARCH-prefmem-semantics}. Preferences remain in
preference memory until removed for one of the reasons previously discussed in
Section \ref{ARCH-prefmem-persistence}.

% ----------------------------------------------------------------------------
\subsection{Preference semantics}
\label{ARCH-prefmem-semantics}

\index{preference!semantics}
This section describes the semantics of each type of preference.  More
details on the preference resolution process are provided in
Appendix
\ref{PREFERENCES}.

\nocomment{preference resolution or preference evaluation? resolution.}

\index{decision!procedure}
Only a single value can be selected as the current operator, that is,
all values are mutually exclusive.  In addition, there is no implicit
transitivity in the semantics of preferences.  If A is indifferent to B,
and B is indifferent to C, A and C will not be indifferent to one
another unless there is a preference that A is indifferent to C (or C
and A are both indifferent to all competing values).

\begin{description}
\index{preference!acceptable}
\index{"+|see{preference!acceptable}}
\index{acceptable preference|see{preference!acceptable}}
\item [Acceptable (+)] An \soar{acceptable} preference states that a value is a
        candidate for selection. All values, except those with \soar{require}
        preferences, must have an \soar{acceptable} preference in order to be
        selected. If there is only one value with an \soar{acceptable} 
preference
        (and none with a \soar{require} preference), that value will be selected 
as
        long as it does not also have a \soar{reject} or a \soar{prohibit}
        preference.\vspace{-8pt}

\index{preference!reject}
\index{"-|see{preference!reject}}
\index{reject preference|see{preference!reject}}
\item [Reject ($-$)] A \soar{reject} preference states that the value is not a
        candidate for selection.\vspace{-8pt}

\index{preference!better}
\index{">|see{preference!better}}
\index{better preference|see{preference!better}}
\index{preference!worse}
\index{"<|see{preference!worse}}
\index{worse preference|see{preference!worse}}
\item [Better ($>$), Worse ($<$)] A \soar{better} or \soar{worse} preference
        states, for the two values involved, that one value should not be
        selected if the other value is a candidate. \soar{Better} and
        \soar{worse} allow for the creation of a partial ordering between
        candidate values. \soar{Better} and \soar{worse} are simple inverses
        of each other, so that \soar{A} better than \soar{B} is equivalent to
        \soar{B} worse than \soar{A}.\vspace{-8pt}

\index{preference!best}
\index{">|see{best preference}}
\index{best preference|see{best preference}}
\item [Best ($>$)] A \soar{best} preference states that the value may be
        better than any competing value (unless there are other competing
        values that are also ``best''). If a value is \soar{best} (and not
        \soar{reject}ed, \soar{prohibit}ed, or \soar{worse} than another), it
        will be selected over any other value that is not also \soar{best} (or
        \soar{require}d). If two such values are \soar{best}, then any
        remaining preferences for those candidates (\soar{worst},
        \soar{indifferent}) will be examined to determine the selection. Note that if a
        value (that is not \soar{reject}ed or \soar{prohibit}ed) is
        \soar{better} than a \soar{best} value, the \soar{better} value will
        be selected.  (This result is counter-intuitive, but allows explicit
        knowledge about the relative worth of two values to dominate knowledge
        of only a single value. A \soar{require} preference should be used
        when a value \emph{must} be selected for the goal to be 
achieved.)\vspace{-8pt}

\index{preference!worst}
\index{"<|see{preference!worst}}
\index{worst preference|see{preference!worst}}
\item [Worst ($<$)] A \soar{worst} preference states that the value should be 
selected
        only if there are no alternatives.  It allows for a simple type of
        default specification. The semantics of the \soar{worst} preference
        are similar to those for the \soar{best} preference.\vspace{-8pt}

\index{preference!indifferent}
\index{"=|see{preference!indifferent}}
\index{indifferent preference|see{preference!indifferent}}
\index{indifferent-selection}
\item [Indifferent (=)] An \soar{indifferent} preference states that there is
        positive knowledge that it does not matter which value is selected.
        This may be a binary preference, to say that two values are mutually
        indifferent, or a unary preference, to say that a single value is as
        good or as bad a choice as other expected alternatives.
        
	When \soar{indifferent} preferences are used to signal that it does
	not matter which operator is selected, by default, Soar
	chooses randomly from among the alternatives. (The
	\soar{indifferent-selection} function can be used to change this
	behavior as described on
	page \pageref{indifferent-selection} in Chapter
	\ref{INTERFACE}.)\vspace{-8pt}

\index{preference!numeric-indifferent}
\index{numeric-indifferent preference|see{preference!numeric-indifferent}}
\item [Numeric-Indifferent (= \emph{number})]
	A \soar{numeric-indifferent} preference is used to bias the random selection from mutually indifferent values.
	This preference includes a unary indifferent preference, so an operator with a \soar{numeric-indifferent} preference will not force a tie impasse.
	When a set of operators are determined to be indifferent based on all other asserted preference types and at least one operator has a numeric-indifferent preference, the decision mechanism will choose an operator based on their numeric-indifferent values and the exploration policy.
	The available exploration policies and how they calculate selection probability are detailed in the documentation for the \soar{indifferent-selection} command on page \pageref{indifferent-selection}.
	When a single operator is given multiple numeric-indifferent preferences, they are either averaged or summed into a single value based on the setting of the \soar{numeric-indifferent-mode} command (see page \pageref{numeric-indifferent-mode}).
	
	Numeric-indifferent preferences that are created by RL rules can be adjusted by the reinforcement learning mechanism.
	In this way, it's possible for an agent to begin a task with only arbitrarily initialized numeric indifferent preferences and with experience learn to make the optimal decisions.
	See chapter \ref{RL} for more information.
	
\index{preference!require}
\index{"!|see{preference!require}}
\index{require preference|see{preference!require}}
\item [Require (!)] A \soar{require} preference states that the value must be
        selected if the goal is to be achieved.\vspace{-8pt}

\index{preference!prohibit}
\index{"~|see{preference!prohibit}}
\index{prohibit preference|see{preference!prohibit}}
\item [Prohibit ($\tild$)] A \soar{prohibit} preference states that the value 
cannot
        be selected if the goal is to be achieved.  If a value has a
        \soar{prohibit} preference, it will not be selected for a value of an
        augmentation, independent of the other preferences.\vspace{-8pt}
\end{description}


\index{preference!acceptable}
\index{preference!require}
If there is an
\soar{acceptable}\/ preference for a value of an operator, and there are no
other competing values, that operator will be selected.  If there are
multiple \soar{acceptable}\/ preferences for the same state but with
different values, the preferences must be evaluated to determine which
candidate is selected.

If the preferences can be evaluated without conflict, the appropriate
operator augmentation of the state will be added to working
memory. This can happen when they all suggest the same operator or when
one operator is preferable to the others that have been suggested. When
the preferences conflict, Soar reaches an impasse, as described in
Section \ref{ARCH-impasses}.

Preferences can be confusing; for example, there can be two suggested
values that are both ``best'' (which again will lead to an impasse
unless additional preferences resolve this conflict); or there may be
one preference to say that value \soar{A} is better than value \soar{B}
and a second preference to say that value \soar{B} is better than value
\soar{A}.

% ----------------------------------------------------------------------------
% ----------------------------------------------------------------------------
\section{Soar's Execution Cycle: Without Substates}
\label{ARCH-decision}

\index{decision cycle}
\index{quiescence}
The execution of a Soar program proceeds through a number of
\emph{cycles}. Each cycle has five phases:
\begin{enumerate} 
\item Input: New sensory data comes into working memory.
\item Proposal: Productions fire (and retract) to interpret new data (state 
elaboration), propose operators for the current situation (operator
proposal), and compare proposed operators (operator comparison).  All of
the actions of these productions are I-supported.  All matched
productions fire in parallel (and all retractions occur in parallel),
and matching and firing continues until there are no more additional
complete matches or retractions of productions (\emph{quiescence}).
\item Decision: A new operator is selected, or an impasse is detected
and a new state is created.
\item Application: Productions fire to apply the operator (operator
application).  The actions of these productions will be O-supported.
Because of changes from operator application productions, other
productions with I-supported actions may also match or retract. Just as
during proposal, productions fire and retract in parallel until quiescence.
\item Output: Output commands are sent to the external environment.
\end{enumerate}

The cycles continue until the halt action is issued from
the Soar program (as the action of a production) or until Soar is interrupted 
by the user.

During the processing of these phases, it is possible that the
preferences that resulted in the selection of the current operator could
change.  Whenever operator preferences change, the preferences are
re-evaluated and if a different operator selection would be made, then
the current operator augmentation of the state is immediately removed.
However, a new operator is not selected until the next decision phase, when
all knowledge has had a chance to be retrieved.

\begin{figure}
\insertfigure{decisioncycle}{6.75in}
\insertcaption{A detailed illustration of Soar's decision cycle: out of date}
\label{fig:decisioncycle}
\end{figure}

\begin{figure}
\index{decision cycle!pseudo code}
\begin{verbatim}
Soar
  while (HALT not true) Cycle;
  
Cycle
  InputPhase;
  ProposalPhase;
  DecisionPhase;
  ApplicationPhase;
  OutputPhase;


ProposalPhase
  while (some I-supported productions are waiting to fire or retract)
    FireNewlyMatchedProductions;
    RetractNewlyUnmatchedProductions;

DecisionPhase
  for (each state in the stack, 
       starting with the top-level state)
  until (a new decision is reached)
    EvaluateOperatorPreferences; /* for the state being considered */
    if (one operator preferred after preference evaluation)
      SelectNewOperator;
    else                  /* could be no operator available or */
      CreateNewSubstate;  /* unable to decide between more than one */

ApplicationPhase
  while (some productions are waiting to fire or retract)
    FireNewlyMatchedProductions;
    RetractNewlyUnmatchedProductions;
\end{verbatim}

\insertcaption{A simplified version of the Soar algorithm.}
\label{fig:pseudocode}
\end{figure}

% ----------------------------------------------------------------------------
\section{Impasses and Substates}
\label{ARCH-impasses}
\index{decision procedure}
\index{impasse}
\index{goal!subgoal}
\index{result}

When the decision procedure is applied to evaluate preferences and determine
the operator augmentation of the state, it is possible that the preferences are 
either
incomplete or inconsistent.
The preferences can be incomplete in that no \soar{acceptable} operators are 
suggested, or that there are insufficient preferences to distinguish among
\soar{acceptable} operators. The preferences can be inconsistent if, for instance, 
operator
\soar{A} is preferred to operator \soar{B}, and operator \soar{B} is preferred 
to operator \soar{A}. Since preferences are generated independently,
from different production instantiations, there is no guarantee that
they will be consistent.

% ----------------------------------------------------------------------------
\subsection{Impasse Types}
\label{ARCH-impasses-types}

\index{impasse!tie}
\index{impasse!conflict}
\index{impasse!constraint-failure}
\index{impasse!no-change}
\index{tie impasse|see{impasse!tie}}
\index{conflict impasse|see{impasse!conflict}}
\index{constraint-failure impasse|see{impasse!constraint-failure}}
\index{no-change impasse|see{impasse!no-change}}

There are four types of impasses that can arise from the preference scheme.
\vspace{-12pt}

\begin{description}
\item[Tie impasse ---] 
	A \emph{tie} impasse arises if the preferences do not
        distinguish between two or more operators with \soar{acceptable}
        preferences. If two operators both have \soar{best} or \soar{worst}
        preferences, they will tie unless additional preferences distinguish
        between them.\vspace{-8pt}
	\index{impasse!tie}
	\index{tie impasse}
\item[Conflict impasse ---]
	A \emph{conflict} impasse arises if at least two values have conflicting
        better or worse preferences (such as \soar{A} is better than \soar{B}
        and \soar{B} is better than \soar{A}) for an operator, and neither
        one is rejected, prohibited, or \soar{require}d.\vspace{-8pt}
	\index{impasse!conflict}
	\index{conflict impasse}
\item[Constraint-failure impasse ---]
	A \emph{constraint-failure} impasse arises if there is more than one
        \soar{require}d value for an operator, or if a value has both a
        \soar{require} and a \soar{prohibit} preference. These preferences
        represent constraints on the legal selections that can be made for a
        decision and if they conflict, no progress can be made from the
        current situation and the impasse cannot be resolved by additional
        preferences.\vspace{-8pt}
	\index{constraint-failure impasse}
	\index{impasse!constraint-failure}
\item[No-change impasse ---]
	A \emph{no-change} impasse arises if a new operator is not selected
        during the decision procedure. There are two types of no-change
        impasses: state no-change and operator no-change:\vspace{-8pt} 
        \begin{description}
        \item[State no-change impasse ---] 
		A state no-change impasse occurs when there are no
		\soar{acceptable} (or \soar{require}) preferences to suggest
		operators for the current state (or all the \soar{acceptable}
		values have also been \soar{reject}ed). The decision procedure
		cannot select a new operator.\vspace{-8pt}
        \item[Operator no-change impasse ---] 
		An operator no-change impasse occurs when either a new operator 
                is selected for the current state but no additional productions
                match during the application phase, or a new
                operator is not selected during the next decision phase.
        \end{description}
	\index{state no-change impasse}
	\index{operator no-change impasse}
	\index{no-change impasse}
	\index{impasse!state no-change}
	\index{impasse!operator no-change}
	\index{impasse!no-change}
\end{description}

There can be only one type of impasse at a given level of subgoaling at
a time. Given the semantics of the preferences, it is possible to have a
tie or conflict impasse and a constraint-failure impasse at the same
time.  In these cases, Soar detects only the constraint-failure impasse.

The impasse is detected \textit{during} the selection of the operator,
but happens \textit{because} one of the other four problem-solving
functions was incomplete.

% ----------------------------------------------------------------------------
\subsection{Creating New States}
%\label{elim-impa}
\index{goal!subgoal}
\index{subgoal|see{goal}}
Soar handles these inconsistencies by creating a new state in which the
goal of the problem solving is to resolve the impasse.  Thus, in the
substate, operators will be selected and applied in an attempt either to
discover which of the tied operators should be selected, or to apply the
selected operator piece by piece.  The substate is often called a
\emph{subgoal} because it exists to resolve the impasse, but is
sometimes called a substate because the representation of the subgoal in
Soar is as a state.

The initial state in the subgoal contains a complete description of the
cause of the impasse, such as the operators that could not be decided
among (or that there were no operators proposed) and the state that the
impasse arose in. From the perspective of the new state, the latter is
called the \emph{superstate}. Thus, the superstate is part of the
substructure of each state, represented by the Soar architecture using
the \soar{superstate} attribute. (The initial state, created in the 0th
decision cycle, contains a \soar{superstate} attribute with the value of
\soar{nil} --- the top-level state has no superstate.)

The knowledge to resolve the impasse may be retrieved by any type of
problem solving, from searching to discover the implications of different
decisions, to asking an outside agent for advice. There is no \emph{a priori}
restriction on the processing, except that it involves applying operators to
states.
\index{subgoal}

\begin{figure}
\insertfigure{stack1}{7.75in}
\insertcaption{A simplified illustration of a subgoal stack.}
\label{fig:stack1}
\end{figure}

\index{goal!stack}
\index{stack|see{goal}}
In the substate, operators can be selected and applied as Soar attempts to
solve the subgoal. (The operators proposed for solving the subgoal may be
similar to the operators in the superstate, or they may be entirely
different.) While problem solving in the subgoal, additional impasses may be
encountered, leading to new subgoals.  Thus, it is possible for Soar to have a
\emph{stack} of subgoals, represented as states: Each state has 
a single superstate (except the initial state) and each state may have at most 
one substate. Newly created
subgoals are considered to be added to the bottom of the stack; the first
state is therefore called the \emph{top-level state}.\footnote{The
original state is the ``top'' of the stack because as Soar
runs, this state (created first), will be at the top of the computer screen,
and substates will appear on the screen below the top-level state.}  See
Figure \ref{fig:stack1} for a simplified illustrations of a subgoal stack.

Soar continually attempts to retrieve knowledge relevant to all goals in the
subgoal stack, although problem-solving activity will tend to focus on the
most recently created state. However, problem solving is active at
all levels, and productions that match at any level will fire.

% ----------------------------------------------------------------------------
\subsection{Results}
\label{ARCH-impasses-results}
\index{goal!result|see{result}}
\index{result}

In order to resolve impasses, subgoals must generate results that allow
the problem solving at higher levels to proceed.  The {\em results} of a
subgoal are the working memory elements and preferences that were
created in the substate, and that are also linked directly or indirectly
to a superstate (\emph{any} superstate in the stack). A preference or
working memory element is said to be created in a state if the
production that created it tested that state and this is the most recent
state that the production tested. Thus, if a production tests multiple
states, the preferences and working memory elements in its actions are
considered to be created in the most recent of those states (and is not
considered to have been created in the other states). The architecture
automatically detects if a preference or working memory elmenet created
in a substate is also linked to a superstate.

These working memory elements and preferences will not be removed when
the impasse is resolved because they are still linked to a superstate,
and therefore, they are called the \textit{results of the subgoal}.  A
result has either I-support or O-support; the determination of support is
described below.

%A production that creates a result is illustrated in Figure \ref{fig:result}.
%The figure illustrates the result as a working memory element: 
%``\soar{new-attribute X1}''.

%\begin{figure}
%\insertfigure{result}{3.7in}
%\insertcaption{An abstract illustration of a production that creates a
%	result. In the figure, S2 is the lowest state in the subgoal stack
%	that is tested by the production, and the working memory element
%	is said to have been created in state S2.  }
%\label{fig:result}
%\end{figure}

%\begin{figure}
%\insertfigure{result-indirect}{3in}
%\insertcaption{An abstract illustration of a production that creates a
%	working memory element that indirectly becomes a result. S2 is the
%	lowest state in the subgoal stack that is tested by the production,
%	and the working memory element is said to be created in state S2. Some 
%other
%	production instantiation creates the working memory element that links X2 
%to the
%	superstate.
%	}
%\label{fig:result-indirect}
%\end{figure}

A working memory element or preference will be a result if
its identifier is already linked to a superstate.
%(as illustrated inFigure \ref{fig:result})
A working memory element or preference can also become a result
indirectly if, after it is created and it is still in working memory or
preference memory, its identifier becomes linked to a superstate through
the creation of another result. For example, if the problem solving in a
state constructs an operator for a superstate, it may wait until
the operator structure is complete before creating an
\soar{acceptable} preference for the operator in the superstate. The
\soar{acceptable} preference is a result because it was created in the
state and is linked to the superstate (and, through the superstate, is
linked to the top-level state). The substructures of the operator then
become results because the operator's identifier is now linked to the
superstate. 
% An indirect result is illustrated in Figure \ref{fig:result-indirect}). 

\subsubsection*{Justifications: Determination of support for results}

\index{I-support!of result}
\index{O-support!of result}
Some results receive I-support, while others receive O-support.  The
type of support received by a result is determined by the function it
plays in the superstate, and not the function it played in the state in
which it was created. For example, a result might be created through
operator application in the state that created it; however, it might
only be a state elaboration in the superstate. The first function would
lead to O-support, but the second would lead to I-support.

\index{justification}
\index{justification!creation}
In order for the architecture to determine whether a result receives I-support
or O-support, Soar must first determine the function that the working
memory element or preference plays
(that is, whether the result should be considered an operator application or
not). To do this, Soar creates a temporary production, called a
\textit{justification}. The justification summarizes the processing in the
substate that led to the result:\vspace{-10pt}
\begin{description}
\item[The conditions] of a justification are those working memory
elements that exist in the superstate (and above) that were necessary
for producing the result.  This is determined by collecting all of the
working memory elements tested by the productions that fired in the
subgoal that led to the creation of the result, and then removing those
conditions that test working memory elements created in the subgoal.
\vspace{-6pt}
\item[The action] of the justification is the result of the subgoal.
\end{description} 

Soar determines I-support or O-support for the justification just as it
would for any other production, as described in Section
\ref{ARCH-prefmem-persistence}.  If the justification is an operator
application, the result will receive O-support.  Otherwise, the result
gets I-support from the justification. If such a result loses
I-support from the justification, it will be retracted if there is no
other support.  Justification are not added to production memory, but
are otherwise treated as an instantiated productions that have already
fired.

Justifications include any negated conditions that were in the original
productions that participated in producing the results, and that test for
the absence of superstate working memory elements. Negated conditions that
test for the absence of working memory elements that are local to the
substate are not included, which can lead to overgeneralization
in the justification (see Section \ref{CHUNKING-problems} on page
\pageref{CHUNKING-problems} for details). 
\index{learning!overgeneral}
\index{chunk!overgeneral}
\index{justification!overgeneral}

% ----------------------------------------------------------------------------
\subsection{Removal of Substates: Impasse Resolution}
%\label{elim-impa}
\label{ARCH-impasses-elimination}
\index{impasse!resolution}
\index{impasse!elimination}
\index{goal!termination}

Problem solving in substates is an important part of what Soar
\textit{does}, and an operator impasse does not necessarily indicate a
problem in the Soar program.  They are a way to decompose a complex
problem into smaller parts and they provide a context for a program to
deliberate about which operator to select.  Operator impasses are necessary, for
example, for Soar to do any learning about problem solving (as will be
discussed in Chapter \ref{CHUNKING}). This section describes how
impasses may be resolved during the execution of a Soar program, how
they may be eliminated during execution without being resolved, and some
tips on how to modify a Soar program to prevent a specific impasse from
occurring in the first place.  

\subsubsection*{Resolving Impasses}

An impasse is \textit{resolved} when processing in a subgoal creates
results that lead to the selection of a new operator for the state
where the impasse arose. When an operator impasse is resolved, Soar has
an opportunity to learn, and the substate (and all its substructure) is
removed from working memory.

Here are possible approaches for resolving specific types
of impasses are listed below:\vspace{-12pt}
\begin{description}
\item[Tie impasse ---]
	A tie impasse can be resolved by productions that create preferences
	that prefer one option (\soar{better}, \soar{best}, \soar{require}),
	eliminate alternatives (\soar{worse}, \soar{worst}, \soar{reject},
	\soar{prohibit}), or make all of the objects indifferent
	(\soar{indifferent}).\vspace{-8pt}
\item[Conflict impasse ---]
	A conflict impasse can be resolved by productions that create
	preferences to \soar{require} one option (\soar{require}), or
	eliminate the alternatives (reject, prohibit).\vspace{-8pt}
\item[Constraint-failure impasse ---]
	A constraint-failure impasse cannot be resolved by additional
	preferences, but may be prevented by changing productions so that they
	create fewer \soar{require} or \soar{prohibit} preferences.\vspace{-8pt}
\item[State no-change impasse ---]
	A state no-change impasse can be resolved by productions that create 
	\soar{acceptable} or \soar{require} preferences for operators.\vspace{-
8pt}
\item[Operator no-change impasse ---]
	An operator no-change impasse can be resolved by productions that
	apply the operator, changing the state so the operator proposal
	no longer matches or other operators are proposed and preferred.
\end{description}

\subsubsection*{Eliminating Impasses}

An impasse is resolved when results are created that allow progress to
be made in the state where the impasse arose.  In Soar, an impasse can be
\textit{eliminated} (but not resolved) when a higher level impasse is
resolved, eliminated, or regenerated.  In these cases, the impasse
becomes irrelevant because higher-level processing can proceed.  An
impasse can also become irrelevant if input from the outside world
changes working memory which in turn causes productions to fire that
make it possible to select an operator.  In all these cases, the impasse
is eliminated, but not ``resolved'', and Soar does not learn in this
situation.

\subsubsection*{Regenerating Impasses}

An impasse is \textit{regenerated} when the problem solving in the
subgoal becomes {\em inconsistent} with the current situation.  During
problem solving in a subgoal, Soar monitors which aspect of the
surrounding situation (the working memory elements that exist in
superstates) the problem solving in the subgoal has depended upon.  If
those aspects of the surronding situation change, either because of
changes in input or because of results, the problem solving in the
subgoal is inconsistent, and the state created in response to the
original impasse is removed and a new state is created. Problem solving
will now continue from this new state.  The impasse is not ``resolved'',
and Soar does not learn in this situation.

The reason for regeneration is to guarantee that the working memory
elements and preferences created in a substate are consistent with
higher level states.  As stated above, inconsistency can arise when a
higher level state changes either as a result of changes in what is
sensed in the external environment, or from results produced in the
subgoal.  The problem with inconsistency is that once inconsistency
arises, the problem being solved in the subgoal may no longer be the
problem that actually needs to be solved.  Luckily, not all changes to a
superstate lead to inconsistency.

In order to detect inconsistencies, Soar maintains a {\em dependency set}
for every subgoal/substate.  The dependency set consists of all working
memory elements that were tested in the conditions of productions that
created O-supported working memory elements that are directly or
indirectly linked to the substate.  Thus, whenever such an O-supported
working memory element is created, Soar records which working memory
elements that exist in a superstate were tested, directly or indirectly
in creating that working memory element. \index {dependency-set} Whenever
any of the working memory elements in the dependency set of a substate
change, the substate is regenerated.

Note that the creation of I-supported structures in a subgoal does not
increase the dependency set, nor do O-supported results.  Thus, only
subgoals that involve the creation of internal O-support working memory
elements risk regeneration, and then only when the basis for the
creation of those elements changes.

\subsubsection*{Substate Removal}

Whenever a substate is removed, all working memory elements and
preferences that were created in the substate that are not
results are removed from working memory. In Figure \ref{fig:stack1},
state \soar{S3} will be removed from working memory when the impasse
that created it is resolved, that is, when sufficient preferences have
been generated so that one of the operators for state \soar{S2} can be
selected. When state \soar{S3} is removed, operator \soar{O9} will also be removed,
as will the acceptable
preferences for \soar{O7}, \soar{O8}, and \soar{O9}, and the
\soar{impasse}, \soar{attribute}, and \soar{choices} augmentations of state
\soar{S3}. These working memory elements are removed because they are no
longer linked to the subgoal stack. The acceptable preferences for
operators \soar{O4}, \soar{O5}, and \soar{O6} remain in working memory. They
were linked to state \soar{S3}, but since they are also linked to state
\soar{S2}, they will stay in working memory until \soar{S2} is removed (or
until they are retracted or rejected).

\subsection{Soar's Cycle: With Substates}
\label{ARCH-decision-substates}

When there are multiple substates, Soar's cycle remains basically the
same but has a few minor changes.  


The first change is that during the decision procedure, Soar will detect
impasses and create new substates.  For example, following the proposal
phase, the decision phase will detect if a decision cannot be made given
the current preferences.  If an impasse arises, a new substate is
created and added to working memory.  

%The decision procedure will detect an operator no-change impasse as soon
%as an operator is selected and added to working memory by checking to
%see whether or not productions will create O-supported actions during
%the next application phae.  If no O-suppored actions will be created,
%the decision procedure will immediately create an operator no-change
%impasse, and then proceed to output, input, and so on.  In these cases,
%the operator no-change is made in the same decision as the operator
%selection.  There will be cases where the operator no-change happens on
%the following decisions, such as when there are O-supported productions
%that will fire, but do not lead to a change in the selected operator.

The second change when there are multiple substates is that at each
phase, Soar goes through the substates, from oldest (highest) to newest
(lowest), completing any necessary processing at that level for that
phase before doing any processing in the next substate.  When firing
productions for the proposal or application phases, Soar processes the
firing (and retraction) of rules, starting from those matching the
oldest substate to the newest.  Whenever a production fires or retracts,
changes are made to working memory and preference memory, possibly
changing which productions will match at the lower levels (productions
firing within a given level are fired in parallel -- simulated).
Productions firings at higher levels can resolve impasses and thus
eliminate lower states before the productions at the lower level ever
fire.  Thus, whenever a level in the state stack is reached, all
production activity is guaranteed to be consistent with any processing
that has occurred at higher levels.


% ----------------------------------------------------------------------------
\section{Learning}
\label{ARCH-learning} 
\index{learning}
\index{chunk}
\index{goal!subgoal}

\index{learning}
\index{chunking|see{learning}}
When an operator impasse is resolved, it means that Soar has, through problem 
solving,
gained access to knowledge that was not readily available before. Therefore,
when an impasse is resolved, Soar has an opportunity to learn, by summarizing
and generalizing the processing in the substate.

\index{chunk}
One of Soar's learning mechanisms is called \textit{chunking}; it attempts to create a
new production, called a chunk. The conditions of the chunk are the elements
of the state that (through some chain of production firings) allowed the
impasse to be resolved; the action of the production is the working
memory element or preference that
resolved the impasse (the result of the impasse). The conditions and action
are variablized so that this new production may match in a similar situation
in the future and prevent an impasse from arising. 

Chunks are very similar to justifications in that they are both
formed via the backtracing process and both create a result in their
actions. However, there are some important distinctions:\vspace{-12pt}
\begin{enumerate}
\item Chunks are productions and are added to production memory.
	Justifications do not appear in production memory.\vspace{-8pt}
\item Justifications disappear as soon as the working memory element or
         preference they provide support
	for is removed. \vspace{-8pt}
\item Chunks contain variables so that they may match working memory in other
	situations; justifications are similar to an instantiated chunk.
\end{enumerate}




% ----------------------------------------------------------------------------
\section{Input and Output}
\label{ARCH-io}	%\label{ch-abst-symb-inpu}
\index{I/O}

Many Soar users will want their programs to interact with a real or simulated
environment. For example, Soar programs may control a robot, receiving sensory
inputs and sending command outputs. Soar programs may also interact with
simulated environments, such as a flight simulator. Input is viewed as
Soar's perception and output is viewed as Soar's motor abilities.

When Soar interacts with an external environment, it must make use of
mechanisms that allow it to receive input from that environment and to effect
changes in that environment; the mechanisms provided in Soar are called
\textit{input functions} and \textit{output functions}.

\begin{description}
\item[Input functions] add and delete elements from working memory in response
	to changes in the external environment.
\item[Output functions] attempt to effect changes in the external
	environment. 
\end{description}

Input is processed at the beginning of each execution cycle and output
occurs at the end of each execution cycle.

For instructions on how to use input and output functions with Soar, refer to the
\textit{SML Quick Start Guide}.





% ----------------------------------------------------------------------------
\typeout{--------------- The SYNTAX of soar programs ------------------------}
\chapter{The Syntax of Soar Programs}
%\label{performance}
\label{SYNTAX}
\index{syntax!productions|see{production!syntax}}
\index{syntax!working memory elements|see{working memory element!syntax}}
\index{syntax!preferences|see{preference!syntax}}

This chapter describes in detail the syntax of elements in working
memory, preference memory, and production memory, and how impasses and
I/O are represented in working memory and in productions. Working memory
elements and preferences are created as Soar runs, while productions are
created by the user or through chunking. The bulk of this chapter
explains the syntax for writing productions.

The first section of this chapter describes the structure of working
memory elements in Soar; the second section describes the structure of
preferences; and the third section describes the structure of
productions. The fourth section describes the structure of impasses.
An overview of how input and output appear in working memory is
presented in the fifth section; the full discussion of Soar I/O can be
found in the \textit{SML Quick Start Guide}.

This chapter assumes that you understand the operating principles of
Soar, as presented in Chapter \ref{ARCH}.

% ----------------------------------------------------------------------------
\section{Working Memory}
\label{SYNTAX-wm}
\index{working memory!syntax}
\index{working memory element!syntax}

Working memory contains \emph{working memory elements} (WME's). As
described in Section \ref{ARCH-wm}, WME's can be created by the actions of 
productions, the evaluation of preferences, the Soar
architecture, and via the input/output system.

\index{identifier}
\index{attribute}
\index{value}
\index{^ (carat symbol)}
A WME is a list consisting of three symbols: an {\em identifier}, an
\emph{attribute}, and a \emph{value}, where the entire WME is enclosed in
parentheses and the attribute is preceded by an up-arrow (\carat ).
A template for a working memory element is:
\begin{verbatim}
(identifier ^attribute value)
\end{verbatim}

The identifier is an internal symbol, generated by the Soar architecture as
it runs. The attribute and value can be either identifiers or constants; if
they are identifiers, there are other working memory elements that have 
that identifier in their first position.  As the previous sentences
demonstrate, identifier is used to refer both to the first position of
a working memory element, as well as to the symbols that occupy that position.


% ----------------------------------------------------------------------------
\subsection{Symbols}
\label{SYNTAX-wm-symbols}

Soar distinguishes between two types of working memory symbols:
\emph{identifiers} and {\em constants}. 
\index{symbol}

\index{identifier}
\textbf{Identifiers: } An identifier is a unique symbol, created at runtime when 
a new object is added to working memory. The names of 
identifiers are
created by Soar, and consist of a single uppercase letter followed by a string
of digits, such as \soar{G37} or \soar{O22}.

(The Soar user interface will also allow users to specify identifiers using
lowercase letters, for example, when using the \texttt{print} command.
But internally, they are actually uppercase letters.)

\index{constant}
\textbf{Constants: } There are three types of constants: integers,
floating-point, and symbolic constants:\vspace{-10pt}
\index{constant}
\begin{itemize} 

\index{integer}
\item Integer constants (numbers).  The range of values depends on the
        machine and implementation you're using, but it is at least $[$-2
        billion..2 billion$]$.\vspace{-8pt}

\index{floating-point constants}
\item Floating-point constants (numbers).  The range depends on
        the machine and implementation you're using.\vspace{-8pt}
        
\item Symbolic constants.  These are symbols with arbitrary names. A constant
        can use any combination of letters, digits, or \verb.$%&*+-/:<=>?_.
        Other characters (such as blank spaces) can be included by surrounding
        the complete constant name with vertical bars: \soar{|This is a
        constant|}.  (The vertical bars aren't part of the name; they're just
        notation.)  A vertical bar can be included by prefacing it with a
        backslash inside surrounding vertical bars:
        \verb.|Odd-symbol\|name|.\vspace{-8pt}
\end{itemize} 
\index{attribute}
\index{value}
\index{constant}
\index{symbolic constant}

Identifiers should not be confused with constants, although they may ``look
the same''; identifiers are generated (by the Soar architecture) at runtime
and will not necessarily be the same for repeated runs of the same program.
Constants are specified in the Soar program and will be the same for repeated
runs.

Even when a constant ``looks like'' an identifier, it will not act like
an identifier in terms of matching. A constant is printed surrounded by
vertical bars whenever there is a possibility of confusing it with an
identifier: \soar{|G37|} is a constant while \soar{G37} is an
identifier. To avoid possible confusion, you should not use
letter-number combinations as constants or for production names.

\subsection{Objects}

Recall from Section \ref{ARCH-wm} that all WME's that share an
identifier are collectively called an \textit{object} in working memory.
The individual working memory elements that make up an object are often
called \emph{augmentations}, because they augment the object.  A
template for an object in working memory is:
\begin{verbatim}
(identifier ^attribute-1 value-1 ^attribute-2 value-2 
            ^attribute-3 value-3... ^attribute-n value-n)
\end{verbatim}

For example, if you run Soar with the example blocks-world program described
in Appendix \ref{BLOCKSCODE}, after one elaboration cycle, you can look at the
top-level state by using the \soar{print} command:
\label{example:prints1}
\begin{verbatim}
soar> print s1
(S1 ^io I1 ^ontop O2 ^ontop O3 ^ontop O1 ^problem-space blocks 
    ^superstate nil ^thing B3 ^thing T1 ^thing B1 ^thing B2 
    ^type state)
\end{verbatim} \vspace{12pt}
The attributes of an object are printed in alphabetical order to make it easier 
to find a specific attribute.

\index{attribute!multi-valued attribute}
\index{multi-attributes|see{attribute!multi-valued attribute}}
Working memory is a set, so that at any time, there are never duplicate
versions of working memory elements.  However, it is possible for
several working memory elements to share the same identifier and
attribute but have different values.  Such attributes are called
multi-valued attributes or \emph{multi-attributes}.  For example, state
\soar{S1}, above, has two attributes that are multi-valued: \soar{thing} and 
\soar{ontop}. 


% ----------------------------------------------------------------------------
\subsection{Timetags}
\index{timetag}
\index{working memory element!timetag|see{timetag}}

When a working memory element is created, Soar assigns it a unique
integer \textit{timetag}. The timetag is a part of the working memory
element, and therefore, WME's are actually quadruples, rather than
triples. However, the timetags are not represented in working memory and
cannot be matched by productions. The timetags are used to distinguish
between multiple occurrences of the same WME. As preferences change and
elements are added and deleted from working memory, it is possible for
a WME to be created, removed, and created again. The second creation of
the WME --- which bears the same identifier, attribute, and value as the
first WME --- is \textit{different}, and therefore is assigned a
different timetag. This is important because a production will fire only
once for a given instantiation, and the instantiation is determined by
the timetags that match the production and not by the
identifier-attribute-value triples.

To look at the timetags of WMEs, the \soar{wmes} command can be used:
\begin{verbatim}
soar> wmes s1
(3: S1 ^io I1)
(10: S1 ^ontop O2)
(9: S1 ^ontop O3)
(11: S1 ^ontop O1)
(4: S1 ^problem-space blocks)
(2: S1 ^superstate nil)
(6: S1 ^thing B3)
(5: S1 ^thing T1)
(8: S1 ^thing B1)
(7: S1 ^thing B2)
(1: S1 ^type state)
\end{verbatim} \vspace{12pt}
This shows all the individual augmentations of \soar{S1}, each is preceded by
an integer \textit{timetag}.

% ----------------------------------------------------------------------------
\subsection{Acceptable preferences in working memory}
\label{SYNTAX-wm-preferences}
\index{working memory!acceptable preference}
\index{preference!acceptable}

The acceptable preferences for the operator augmentations of states
appear in working memory as identifier-attribute-value-preference
quadruples. No other preferences appear in working memory. A template
for an acceptable preference in working memory is:
\begin{verbatim}
(identifier ^operator value +)
\end{verbatim} \vspace{12pt}

For example, if you run Soar with the example blocks-world program described
in Appendix \ref{BLOCKSCODE}, after the first operator
has been selected, you can again look at the top-level state using the
\soar{wmes} command:

\begin{verbatim}
soar> wmes s1
(3: S1 ^io I1)
(9: S1 ^ontop O3)
(10: S1 ^ontop O2)
(11: S1 ^ontop O1)
(48: S1 ^operator O4 +)
(49: S1 ^operator O5 +)
(50: S1 ^operator O6 +)
(51: S1 ^operator O7 +)
(54: S1 ^operator O7)
(52: S1 ^operator O8 +)
(53: S1 ^operator O9 +)
(4: S1 ^problem-space blocks)
(2: S1 ^superstate nil)
(5: S1 ^thing T1)
(8: S1 ^thing B1)
(6: S1 ^thing B3)
(7: S1 ^thing B2)
(1: S1 ^type state)
\end{verbatim} \vspace{12pt}

The state \soar{S1} has six augmentations of acceptable preferences for
different operators (\soar{O4} through \soar{O9}). These have plus signs
following the value to denote that they are acceptable preferences. The state
has exactly one operator, \soar{O7}. This state corresponds to the
illustration of working memory in Figure \ref{fig:ab-wmem2}.
\index{preference}
\index{object}

% ----------------------------------------------------------------------------
\subsection{Working Memory as a Graph}
\index{link}
\index{identifier}
\index{object}

Not only is working memory a set, it is also a graph structure where the
identifiers are nodes, attributes are links, and constants are terminal
nodes.  Working memory is not an arbitrary graph, but a graph rooted in
the states.  Therefore, all WMEs are \emph{linked} either directly or
indirectly to a state.  The impact of this constraint is that all WMEs
created by actions are linked to WMEs tested in the conditions.  The
link is one-way, from the identifier to the value. Less commonly, the
attribute of a WME may be an identifier.

\begin{figure}
\insertfigure{o43net}{4in}
\insertcaption{A semantic net illustration of four objects in working memory.}
\label{fig:o43net}
\end{figure}

Figure \ref{fig:o43net} illustrates four objects in working memory; the
object with identifier \soar{X44} has been linked to the object with
identifier \soar{O43}, using the attribute as the link, rather than the
value. The objects in working memory illustrated by this figure are:
\begin{verbatim}
(O43 ^isa apple ^color red ^inside O53 ^size small ^X44 200) 
(O87 ^isa ball ^color red ^inside O53 ^size big)
(O53 ^isa box ^size large ^color orange ^contains O43 O87)
(X44 ^unit grams ^property mass)
\end{verbatim} \vspace{12pt}

In this example, object \soar{O43} and object \soar{O87} are both linked to
object \soar{O53} through \soar{(O53 \carat contains O43)} and \soar{(O53
\carat contains O87)}, respectively (the \soar{contains} attribute
is a multi-valued attribute). Likewise, object \soar{O53} is linked to object
\soar{O43} through \soar{(O43 \carat inside O53)} and linked to object
\soar{O87} through \soar{(O87 \carat inside O53)}. Object \soar{X44} is linked
to object \soar{O43} through \soar{(O43 \carat X44 200)}.

Links are transitive so that \soar{X44} is linked to \soar{O53} (because
\soar{O43} is linked to \soar{O53} and \soar{X44} is linked to
\soar{O43}). However, since links are not symmetric, \soar{O53} is not
linked to \soar{X44}.


% ----------------------------------------------------------------------------
% ----------------------------------------------------------------------------
\section{Preference Memory}
\label{SYNTAX-prefmem}
\index{preference memory!syntax}
\index{preference!syntax}

Preferences are created by production firings and express the
relative or absolute merits for selecting an operator for a state.  When
preferences express an absolute rating, they are
identifier-attribute-value-preference quadruples; when preferences
express relative ratings, they are
identifier-attribute-value-preference-value quintuples

For example, 
\begin{verbatim}
(S1 ^operator O3 +)
\end{verbatim}
is a preference that asserts that operator O3 is an acceptable operator for
state S1, while
\begin{verbatim}
(S1 ^operator O3 > O4)
\end{verbatim}
is a preference that asserts that operator O3 is a better choice for the
operator of state S1 than operator O4.

The semantics of preferences and how they are processed were described in
Section \ref{ARCH-prefmem}, which also described each of the eleven different
types of preferences.  Multiple production instantiations may create identical 
preferences. Unlike working memory, preference memory is not a set: Duplicate 
preferences are allowed in preference memory.
% ----------------------------------------------------------------------------
% ----------------------------------------------------------------------------
\section{Production Memory}
\label{SYNTAX-pm}
\index{production!syntax}
\index{production memory!syntax}

\nocomment{XXXX start here with indexing}

Production memory contains productions, which can be loaded in by a user
(typed in while Soar is running or \soar{source}d from a file) or
generated by chunking while Soar is running. Productions (both
user-defined productions and chunks) may be examined using the
\soar{print} command, described in Section \ref{print} on page
\pageref{print}.

Each production has three required components: a name, a set of conditions
(also called the left-hand side, or LHS), and a set of actions (also called the
right-hand side, or RHS).  There are also two optional components: a 
documentation string and a type.

Syntactically, each production consists of the symbol \soar{sp}, followed
by: an opening curly brace, \soar{\{}; the production's name; the
documentation string (optional); the production type (optional);
comments (optional); the production's conditions; the symbol \soar{-->}
(literally: dash-dash-greaterthan); the production's actions; and a
closing curly brace, \soar{\}}.  Each element of a production is
separated by white space. Indentation and linefeeds are used by
convention, but are not necessary.

\begin{verbatim}
sp {production-name
    �Documentation string�
    :type
    CONDITIONS
    -->
    ACTIONS
    }
\end{verbatim}  \vspace{12pt}

\begin{figure}
\begin{verbatim}
sp {blocks-world*propose*move-block
   (state <s> ^problem-space blocks
              ^thing <thing1> {<> <thing1> <thing2>}
              ^ontop <ontop>)
   (<thing1> ^type block ^clear yes)
   (<thing2> ^clear yes)
   (<ontop> ^top-block <thing1>
            ^bottom-block <> <thing2>)
   -->
   (<s> ^operator <o> +)
   (<o> ^name move-block 
        ^moving-block <thing1> 
        ^destination <thing2>)}
\end{verbatim}
\insertcaption{An example production from the example blocks-world task.}
\label{fig:ex-prod}
\end{figure}

An example production, named ``\soar{blocks-world*propose*move-block}'', is
shown in Figure \ref{fig:ex-prod}. This production proposes operators named 
\soar{move-block} that move blocks
from one location to another. The details of this production will be described
in the following sections.

\subsubsection*{Conventions for indenting productions}

Productions in this manual are formatted using conventions designed to
improve their readability. These conventions are not part of the
required syntax. First, the name of the production immediately follows
the first curly bracket after the \soar{sp}.  All conditions are aligned
with the first letter after the first curly brace, and attributes of an
object are all aligned The arrow is indented to align with the
conditions and actions and the closing curly brace follows the last
action.

% ----------------------------------------------------------------------------
\subsection{Production Names}

The name of the production is  an almost arbitrary constant. (See Section
\ref{SYNTAX-wm-symbols} for a description of constants.) By convention, the
name describes the role of the production, but functionally, the name is
just a label primarily for the use of the programmer.  

A production name should never be a single letter followed by numbers, 
which is the format of identifiers.

The convention for naming productions is to separate important elements
with asterisks; the important elements that tend to appear in the name
are:\vspace{-12pt}
\begin{enumerate}
\item The name of the task or goal (e.g., \texttt{blocks-world}).\vspace{-10pt}
\item The name of the architectural function (e.g., \texttt{propose}).\vspace{-
10pt}
\item The name of the operator (or other object) at issue. (e.g.,
        \texttt{move-block})\vspace{-10pt} 
\item Any other relevant details.
\end{enumerate}


This name convention enables one to have a good idea of the function of
a production just by examining its name. This can help, for example,
when you are watching Soar run and looking at the specific productions
that are firing and retracting.  Since Soar uses white space to delimit
components of a production, if whitespace inadvertently occurs in the
production name, Soar will complain that an open parenthesis was
expected to start the first condition.

\subsection{Documentation string (optional)}

A production may contain an optional documentation string. The syntax
for a documentation string is that it is enclosed in double quotes and
appears after the name of the production and before the first condition
(and may carry over to multiple lines). The documentation string allows
the inclusion of internal documentation about the production; it will be
printed out when the production is printed using the \soar{print}
command.

% ----------------------------------------------------------------------------
\subsection{Production type (optional)}

A production may also include an optional \emph{production type}, which
may specify that the production should be considered a default
production (\soar{:default}) or a chunk (\soar{:chunk}), or may specify
that a production should be given O- support (\soar{:o-support}) or
I-support (\soar{:i-support}).  Users are discouraged from using these
types.  These types are described in Section \ref{sp}, which begins on
Page \pageref{sp}.

There is one additional flag (\soar{:interrupt}) which can be placed at this location
in a production. However this flag does not specify a production type, but is
a signal that the production should be marked for special debugging capabilities. For more
information, see Section \ref{sp} on Page \pageref{sp}.

% ----------------------------------------------------------------------------
\subsection{Comments (optional)}
\index{comments}

Productions may contain comments, which are not stored in Soar when the
production is loaded, and are therefore not printed out by the
\soar{print} command. A comment is begun with a pound sign character
\soar{\#} and ends at the end of the line.  Thus, everything following
the \soar{\#} is not considered part of the production, and comments
that run across multiple lines must each begin with a \soar{\#}.

For example:
\begin{verbatim}
sp {blocks-world*propose*move-block
   (state <s> ^problem-space blocks
              ^thing <thing1> {<> <thing1> <thing2>}
              ^ontop <ontop>)
   (<thing1> ^type block ^clear yes)
   (<thing2> ^clear yes)
#   (<ontop> ^top-block <thing1>
#           ^bottom-block <> <thing2>)
   -->
   (<s> ^operator <o> +)
   (<o> ^name move-block         # you can also use in-line comments
        ^moving-block <thing1>
        ^destination <thing2>)}
\end{verbatim}

When commenting out conditions or actions, be sure that all parentheses
remain balanced outside the comment.

\subsubsection*{External comments}

Comments may also appear in a file with Soar productions, outside
 the curly braces of the \soar{sp} command.  Comments
must either start a new line with a \soar{\#} or start with \soar{;\#}.
In both cases, the comment runs to the end of the line.

\begin{verbatim}
# imagine that this is part of a "Soar program" that contains 
# Soar productions as well as some other code.

source blocks.soar      ;# this is also a comment
\end{verbatim}


% ----------------------------------------------------------------------------
% ----------------------------------------------------------------------------
% ----------------------------------------------------------------------------
\subsection{The condition side of productions (or LHS)}
\label{SYNTAX-pm-conditions}            %perf-cond
\index{condition side}
\index{LHS of production}
\index{production!LHS}
\index{production!condition}

The condition side of a production, also called the left-hand side (or
LHS) of the production, is a pattern for matching one or more WMEs. When
all of the conditions of a production match elements in working memory,
the production is said to be instantiated, and is ready to perform its
action.

The following subsections describe the condition side of a production,
including predicates, disjunctions, conjunctions, negations, acceptable
preferences for operators, and a few advanced topics. 
% A grammar for the
% condition side is given in Appendix \ref{GRAMMARS}.

% ----------------------------------------------------------------------------
\subsubsection{Conditions}
\label{Conditions}
\index{Conditions}

The condition side of a production consists of a set of conditions.
Each condition tests for the existence or absence (explained later in
Section \ref{SYNTAX-pm-negated}) of working memory elements. Each
condition consists of a open parenthesis, followed by a test for the
identifier, and the tests for augmentations of that identifier, in terms
of attributes and values.  The condition is terminated with a close
parenthesis.  Thus, a single condition might test properties of a single
working memory element, or properties of multiple working memory
elements that constitute an object.  
\begin{verbatim}
(identifier-test ^attribute1-test value1-test 
                 ^attribute2-test value2-test
                 ^attribute3-test value3-test
                 ...)
\end{verbatim}
The first condition in a production must match against a state in
working memory.  Thus, the first condition must begin with the
additional symbol ``state''.  All other conditions and actions must be
\textit{linked} directly or indirectly to this condition. This linkage
may be direct to the state, or it may be indirect, through objects
specified in the conditions.  If the identifiers of the actions are not
linked to the state, a warning is printed when the production is parsed,
and the production is not stored in production memory.  In the actions
of the example production shown in Figure \ref{fig:ex-prod}, the
operator preference is directly linked to the state and the remaining
actions are linked indirectly via the operator preference.

Although all of the attribute tests in the template above are followed
by value tests, it is possible to test for only the existence of an
attribute and not test any specific value by just including the
attribute and no value.  Another exception to the above template is
operator preferences, which have the following structure where a plus
sign follows the value test.
\begin{verbatim}
(state-identifier-test ^operator value1-test +
                 ...)
\end{verbatim}

In the remainder of this section, we describe the different tests that
can be used for identifiers, attributes, and values.  The simplest of
these is a constant, where the constant specified in the attribute or
value must match the same constant in a working memory element.

% ----------------------------------------------------------------------------
\subsubsection{Variables in productions}
\label{SYNTAX-pm-variables}
\index{variables}

Variables match against constants in working memory elements in the
identifier, attribute, or value positions.  Variables can be further
constrained by additional tests (described in later sections) or by
multiple occurrences in conditions.  If a variable occurs more than once
in the condition of a production, the production will match only if the
variables match the same identifier or constant.  However, there is no
restriction that prevents different variables from binding to the same
identifier or constant.

Because identifiers are generated by Soar at run time, it impossible to
include tests for specific identifiers in conditions.  Therefore,
variables are used in conditions whenever an identifier is to be
matched.

Variables also provide a mechanism for passing identifiers and constants
which match in conditions to the action side of a rule.

Syntactically, a variable is a symbol that begins with
a left angle-bracket (i.e., \soar{<}), ends with a right angle-bracket (i.e.,
\soar{>}), and contains at least one alphanumeric symbol in between.

In the example production in Figure \ref{fig:ex-prod}, there are seven
variables: \soar{<s>}, \soar{<clear1>}, \soar{<clear2>}, \soar{<ontop>},
\soar{<block1>}, \soar{<block2>}, and \soar{<o>}.

The following table gives examples of legal and illegal variable names.

\begin{tabular}{| l | l |} \hline
\bf{Legal variables} &  \bf{Illegal variables} \\ \hline
\soar{<s>} &  \soar{<>} \\
\soar{<1>} & \soar{<1} \\
\soar{<variable1>} & \soar{variable>} \\
\soar{<abc1>} & \soar{<a b>} \\ \hline 
\end{tabular} \vspace{10pt}

% ----------------------------------------------------------------------------
\subsubsection{Predicates for values}
\label{SYNTAX-pm-predicates}    %perf-pred}
\index{predicates}
\index{=}
\index{<>}
\index{<}
\index{<=}
\index{>=}
\index{>}
\index{<=>}

A test for an identifier, attribute, or value in a condition (whether
constant or variable) can be modified by a preceding predicate. There
are six predicates that can be used:
\soar{<>, <=>, <, <=, >=, >}.  

\begin{tabular}{| l | l |} \hline
\bf{Predicate} &  \bf{Semantics of Predicate} \\ \hline
\soar{<>}  & Not equal. Matches anything except the value immediately \\
           &  following it. \\
\soar{<=>} & Same type.  Matches any symbol that is the same type (identifier, 
\\
           &  integer, floating-point, non-numeric constant) as the value \\
           &  immediately following it. \\
\soar{<}   & Numerically less than the value immediately following it. \\
\soar{<=}  & Numerically less than or equal to the value immediately \\
           &  following it. \\
\soar{>=}  & Numerically greater than or equal to the value immediately \\ 
           &  following it. \\
\soar{>}   & Numerically greater than the value immediately following it. \\  
\hline 
\end{tabular} \vspace{10pt}
\index{numeric comparisons}
\index{type comparisons}
\index{not equal test}

The following table shows examples of legal and illegal predicates:

\begin{tabular}{| l | l |} \hline
\bf{Legal predicates} &  \bf{Illegal predicates} \\ \hline
\soar{> <valuex>} & \soar{> > <valuey>} \\
\soar{< 1}  & \soar{1 >} \\
\soar{<=> <y>} & \soar{= 10} \\  \hline
\end{tabular} \vspace{10pt}

\subsubsection*{Example Production}

\begin{verbatim}
sp {propose-operator*to-show-example-predicate
   (state <s> ^car <c>)
   (<c> ^style convertible ^color <> rust)
   -->
   (<s> ^operator <o> +)
   (<o> ^name drive-car ^car <c>) }
\end{verbatim}

In this production, there must be a ``color'' attribute for the working memory
object that matches \verb+<c>+, and the value of that attribute must not be
``rust''. 

% ----------------------------------------------------------------------------
\subsubsection{Disjunctions of values}
\label{SYNTAX-pm-disjuncts}      %perf-disj
\index{disjunction of constants}
\index{<< >>}

A test for an identifier, attribute, or value may also be for a
disjunction of constants. With a disjunction, there will be a match if any
one of the constants is found in a working memory element (and the other
parts of the working memory element matches). Variables and predicates
may not be used within disjunctive tests.

Syntactically, a disjunctive test is specified with double angle brackets
(i.e., \soar{ <<} and \soar{>>}). There must be spaces separating the brackets
from the constants. 

The following table provides examples of legal and illegal disjunctions:

\begin{tabular}{| l | l |} \hline
\bf{Legal disjunctions} &  \bf{Illegal disjunctions} \\ \hline
\soar{<< A B C 45 I17 >>} &  \soar{<< <A> A >>}  \\
\soar{<< 5 10 >>} &  \soar{<< < 5  > 10 >>}  \\
\soar{<< good-morning good-evening >>} & \soar{<<A B C >>} \\  \hline 
\end{tabular} \vspace{10pt}

\subsubsection*{Example Production}
For example, the third condition of the following
production contains a disjunction that restricts the color of the table to
\soar{red} or \soar{blue}:

\begin{verbatim}
sp {blocks*example-production-conditions
   (state ^operator <o> + ^table <t>)
   (<o> ^name move-block)
   (<t> ^type table ^color << red blue >> )
   -->
   ... }
\end{verbatim}

\subsubsection*{Note}
Disjunctions of complete conditions are not allowed in Soar.  Multiple
(similar) productions fulfill this role.


% ----------------------------------------------------------------------------
\subsubsection{Conjunctions of values}
\label{SYNTAX-pm-conjunctions}  %perf-conj}
\index{conjunctive!conditions}

A test for an identifier, attribute, or value in a condition may include
a conjunction of tests, all of which must hold for there to be a match.

Syntactically, conjuncts are contained within curly braces (i.e., \soar{\{}
and \soar{\}}). The following table shows some examples of legal and illegal
conjunctive tests:

\begin{tabular}{| l | l |} \hline
\bf{Legal conjunctions} &  \bf{Illegal conjunctions} \\  \hline
\soar{\{ <= <a> >= <b> \}} & \soar{\{ <x> < <a> + <b> \}} \\
\soar{\{ <x> > <y> \}}     & \soar{\{ > > <b> \}} \\
\soar{\{ <> <x> <y> \}}    & \\
\soar{\{ << A B C >> <x> \}} & \\
\soar{\{ <=> <x> > <y> << 1 2 3 4 >> <z> \}} & \\  \hline
\end{tabular} \vspace{10pt}

Because those examples are a bit difficult to interpret, let's go over the
legal examples one by one to understand what each is doing.

In the first example, the value must be less than or equal to the value bound
to variable \soar{<a>} and greater than or equal to the value bound to
variable \soar{<b>}.

In the second example, the value is bound to the variable \soar{<x>}, which
must also be greater than the value bound to variable \soar{<y>}. 

In the third example, the value must not be equal to the value bound to
variable \soar{<x>} and should be bound to variable \soar{<y>}.  Note the
importance of order when using conjunctions with predicates: in the second
example, the predicate modifies \soar{<y>}, but in the third
example, the predicate modifies \soar{<x>}.

In the fourth example, the value must be one of \soar{A}, \soar{B}, or
\soar{C}, and the second conjunctive test binds the value to variable
\soar{<x>}. 

In the fifth example, there are four conjunctive tests. First, the value must
be the same type as the value bound to variable \soar{<x>}. Second, the value
must be greater than the value bound to variable \soar{<y>}. Third, the value
must be equal to \soar{1}, \soar{2}, \soar{3}, or \soar{4}. Finally, the value
should be bound to variable \soar{<z>}.

In Figure \ref{fig:ex-prod}, a conjunctive test is used for the \soar{thing}
attribute in the first condition.

% ----------------------------------------------------------------------------
\subsubsection{Negated conditions}
\label{SYNTAX-pm-negated}       %perf-nega-cond
\index{negated!conditions}
\index{-}

In addition to the positive tests for elements in working memory, conditions
can also test for the absence of patterns.  A \emph{negated condition} will be
matched only if there does not exist a working memory element consistent with
its tests and variable bindings. Thus, it is a test for the \textit{absence}
of a working memory element.

Syntactically, a negated condition is specified by preceding a condition with a
dash (i.e., ``\soar{-}'').

For example, the following condition tests the absence of a working memory
element of the object bound to \soar{<p1> \carat type father}.

\begin{verbatim}
-(<p1> ^type father)
\end{verbatim} \vspace{12pt}

A negation can be used within an object with many attribute-value pairs by
having it precede a specific attribute:

\begin{verbatim}
(<p1> ^name john -^type father ^spouse <p2>)
\end{verbatim} \vspace{12pt}

In that example, the condition would match if there is a working memory
element that matches \soar{(<p1> \carat name john)} and another that matches 
\soar{(<p1> \carat spouse <p2>)}, but is no working memory element that matches 
\soar{(<p1> \carat type father)} (when \soar{p1} is bound to the same 
identifier).

On the other hand, the condition:
\begin{verbatim}
-(<p1> ^name john ^type father ^spouse <p2>)
\end{verbatim}

would match only if there is no object in working memory that matches all
three attribute-value tests.

\subsubsection*{Example Production}
\begin{verbatim}
sp {default*evaluate-object
   (state <ss> ^operator <so>)
   (<so> ^type evaluation 
         ^superproblem-space <p>)
  -(<p> ^default-state-copy no)
   -->
   (<so> ^default-state-copy yes) }
\end{verbatim}

\subsubsection*{Notes}

One use of negated conditions to avoid is testing for the absence of the
working memory element that a production creates with I-support; this
would lead to an ``infinite loop'' in your Soar program, as Soar would
repeatedly fire and retract the production.


% ----------------------------------------------------------------------------
\subsubsection{Negated conjunctions of conditions}
\label{SYNTAX-pm-negaconj}      %perf-nega-conj}
\index{negated!conjunctions}
\index{conjunctive!negation}

Conditions can be grouped into conjunctive sets by surrounding the set of
conditions with \soar{\{} and \soar{\}}. The production compiler groups the
test in these conditions together. This grouping allows for negated tests of
more than one working memory element at a time. In the example below, the
state is tested to ensure that it does not have an object on the table. 

\begin{verbatim}
sp {blocks*negated-conjunction-example
   (state <s> ^name top-state)
  -{(<s> ^ontop <on>)
    (<on> ^bottom-object <bo>)
    (<bo> ^type table)}
   -->
   (<s> ^nothing-ontop-table true) } 
\end{verbatim}

When using negated conjunctions of conditions, the production has
nested curly braces. One set of curly braces delimits the production, while
the other set delimits the conditions to be conjunctively negated.

If only the last condition, \soar{(<bo> \carat type table)} were negated, the
production would match only if the state \emph{had} an ontop relation, and the
ontop relation had a bottom-object, but the bottom object wasn't a table.
Using the negated conjunction, the production will also match when the state
has no ontop augmentation or when it has an ontop augmentation that doesn't
have a bottom-object augmentation.

The semantics of negated conjunctions can be thought of in terms of
mathematical logic, where the negation of $(A \wedge B \wedge C)$:

$\neg (A \wedge B \wedge C)$

can be rewritten as:

$(\neg A) \vee (\neg B) \vee (\neg C)$

That is, ``not (A and B and C)'' becomes ``(not A) or (not B) or (not C)''.



% ----------------------------------------------------------------------------
\subsubsection{Multi-valued attributes}
\label{SYNTAX-pm-multi}
\index{multi-valued attribute}

An object in working memory may have multiple augmentations that specify
the same attribute with different values; these are called multi-valued
attributes, or multi-attributes for short.  To shorten the specification
of a condition, tests for multi-valued attributes can be shortened so
that the value tests are together.

For example, the condition:
\begin{verbatim}
(<p1> ^type father ^child sally ^child sue)
\end{verbatim}

could also be written as:
\begin{verbatim}
(<p1> ^type father ^child sally sue)
\end{verbatim}


% ----------------------------------------------------------------------------
\subsubsection*{Multi-valued attributes and variables}

When variables are used with multi-valued attributes, remember that
variable bindings are not unique unless explicitly forced to be so. For
example, to test that an object has two values for attribute
\soar{child}, the variables in the following condition can match to the same
value.

\begin{verbatim}
(<p1> ^type father ^child <c1> <c2>)
\end{verbatim} \vspace{12pt}

To do tests for multi-valued attributes with variables correctly,
conjunctive tests must be used, as in:

\begin{verbatim}
(<p1> ^type father ^child <c1> {<> <c1> <c2>})
\end{verbatim} \vspace{12pt}

The conjunctive test \soar{ \{<> <c1> <c2>\} } ensures that \soar{<c2>} will
bind to a different value than \soar{<c1>} binds to.


% ----------------------------------------------------------------------------
\subsubsection*{Negated conditions and multi-valued attributes}

A negation can also precede an attribute with multiple values.  In this case
it tests for the absence of the conjunction of the values.  For example

\begin{verbatim}
(<p1> ^name john -^child oprah uma)
\end{verbatim}

is the same as 

\begin{verbatim}
(<p1> ^name john)
-{(<p1> ^child oprah)
  (<p1> ^child uma)}
\end{verbatim}

and the match is possible if either \soar{(<p1> \carat child oprah)} or
\soar{(<p1> \carat child uma)} cannot be found in working memory with the
binding for \soar{<p1>} (but not if both are present).

% ----------------------------------------------------------------------------
\subsubsection{Acceptable preferences for operators}
\label{SYNTAX-pm-acceptable}
\index{condition!acceptable preference }
\index{preference!acceptable as condition}
\index{acceptable preference}
\index{+}

The only preferences that can appear in working memory are acceptable
preferences for operators, and therefore, the only preferences that may appear
in the conditions of a production are acceptable preferences for operators.

Acceptable preferences for operators can be matched in a condition by testing
for a ``\soar{+}'' following the value.  This allows a production to test the
existence of a candidate operator and its properties, and possibly create a
preference for it, before it is selected.

In the example below, \soar{\carat operator <o> +} matches the acceptable
preference for the operator augmentation of the state. \emph{This does not
test that operator} \soar{<o>} \emph{has been selected as the current
operator}.

\begin{verbatim}
sp {blocks*example-production-conditions
   (state ^operator <o> + ^table <t>)
   (<o> ^name move-block)
   -->
   ... }
\end{verbatim}


In the example below, the production tests the state for acceptable
preferences for two different operators (and also tests that these operators
move different blocks):

\begin{verbatim}
sp {blocks*example-production-conditions
   (state ^operator <o1> + <o2> + ^table <t>)
   (<o1> ^name move-block ^moving-block <m1> ^destination <d1>)
   (<o2> ^name move-block ^moving-block {<m2> <> <m1>} 
         ^destination <d2>)
   -->
   ... }
\end{verbatim}

\subsubsection{Attribute tests}

The previous examples applied all of the different test to the values of
working memory elements. 
All of the tests that can be used for values can also be used for
attributes and identifiers (except those including constants).

% ----------------------------------------------------------------------------
\subsubsection*{Variables in attributes}

Variables may be used with attributes, as in:

\begin{verbatim}
sp {blocks*example-production-conditions
   (state <s> ^operator <o> + 
              ^thing <t> {<> <t> <t2>} )
   (operator <o> ^name group 
                 ^by-attribute <a>
                 ^moving-block <t>
                 ^destination <t2>)
   (<t> ^type block ^<a> <x>)
   (<t2> ^type block ^<a> <x>)
   -->
   (<s> ^operator <o> >) }
\end{verbatim}

This production tests that there is acceptable operator that is trying to
group blocks according to some attribute, \soar{<a>}, and that block
\soar{<t>} and \soar{<t2>} both have this attribute (whatever it is), and have
the same value for the attribute.


% ----------------------------------------------------------------------------
\subsubsection*{Predicates in attributes}

Predicates may be used with attributes, as in:

\begin{verbatim}
sp {blocks*example-production-conditions
   (state ^operator <o> + ^table <t>)
   (<t> ^<> type table)
   -->
   ... }
\end{verbatim}

which tests that the object with its identifier bound to \soar{<t>} must have
an attribute whose value is \soar{table}, but the name of this attribute is
not \soar{type}.

% ----------------------------------------------------------------------------
\subsubsection*{Disjunctions of attributes}
\index{disjunctions of attributes}
\index{<< >>}

Disjunctions may also be used with attributes, as in:

\begin{verbatim}
sp {blocks*example-production-conditions
   (state ^operator <o> + ^table <t>)
   (<t> ^<< type name>> table)
   -->
   ... }
\end{verbatim}

which tests that the object with its identifier bound to \soar{<t>} must have
either an attribute \soar{type} whose value is \soar{table} or an attribute
\soar{name} whose value is \soar{table}.

% ----------------------------------------------------------------------------
\subsubsection*{Conjunctive tests for attributes}

Section \ref{SYNTAX-pm-conjunctions} illustrated the use of conjunctions for
the values in conditions. Conjunctive tests may also be used with attributes,
as in:

\begin{verbatim}
sp {blocks*example-production-conditions
   (state ^operator <o> + ^table <t>)
   (<t> ^{<ta> <> name} table)
   -->
   ... }
\end{verbatim}

which tests that the object with its identifier bound to \soar{<t>} must have
an attribute whose value is \soar{table}, and the name of this attribute is
not \soar{name}, and the name of this attribute (whatever it is) is bound to
the variable \soar{<ta>}.

When attribute predicates or attribute disjunctions are used with
multi-valued attributes, the production is rewritten internally to use a
conjunctive test for the attribute; the conjunctive test includes a
variable used to bind to the attribute name. Thus,

\begin{verbatim}
(<p1> ^type father ^ <> name sue sally)
\end{verbatim}

is interpreted to mean:

\begin{verbatim}
(<p1> ^type father ^ {<> name <a*1>} sue ^ <a*1> sally)
\end{verbatim}


% ----------------------------------------------------------------------------
\subsubsection{Attribute-path notation}
\label{SYNTAX-pm-path}
\index{dot notation}
\index{path notation}
\index{.}

Often, variables appear in the conditions of productions only to link the value
of one attribute with the identifier of another attribute. Attribute-path
notation provides a shorthand so that these intermediate variables do not need
to be included.

Syntactically, path notation lists a sequence of attributes separated by dots
(.), after the \carat \ in a condition.

For example, using attribute path notation, the production:

\begin{verbatim}
sp {blocks-world*monitor*move-block
   (state <s> ^operator <o>)
   (<o> ^name move-block
        ^moving-block <block1>
        ^destination <block2>)
   (<block1> ^name <block1-name>)
   (<block2> ^name <block2-name>)   
   -->
   (write (crlf) |Moving Block: | <block1-name>
                 | to: | <block2-name> ) }
\end{verbatim}

could be written as:

\begin{verbatim}
sp {blocks-world*monitor*move-block
   (state <s> ^operator <o>)
   (<o> ^name move-block
        ^moving-block.name <block1-name>
        ^destination.name <block2-name>)   
   -->
   (write (crlf) |Moving Block: | <block1-name>
                 | to: | <block2-name> ) }
\end{verbatim}

Attribute-path notation yields shorter productions that are easier to
write, less prone to errors, and easier to understand.

When attribute-path notation is used, Soar internally expands the conditions
into the multiple Soar objects, creating its own variables as needed.
Therefore, when you print a production (using the \soar{print} command), the
production will not be represented using attribute-path notation.


%----------------------------------------------------------------------------
\subsubsection*{Negations and attribute path notation}

\nocomment{can't negations be used with structured values? there's no
        description of this (yes -- bobd)}

A negation may be used with attribute path notation, in which case it amounts
to a negated conjunction. For example, the production:

\begin{verbatim}
sp {blocks*negated-conjunction-example
   (state <s> ^name top-state)
  -{(<s> ^ontop <on>)
    (<on> ^bottom-object <bo>)
    (<bo> ^type table)}
   -->
   (<s> ^nothing-ontop-table true) } 
\end{verbatim}

could be rewritten as:

\begin{verbatim}
sp {blocks*negated-conjunction-example
   (state <s> ^name top-state -^ontop.bottom-object.type table)
   -->
   (<s> ^nothing-ontop-table true) }
\end{verbatim}


% ----------------------------------------------------------------------------
\subsubsection*{Multi-valued attributes and attribute path notation}

\nocomment{can't multi-attributes be used with structured values? there's no
        description of this (yes -- bobd)}

Attribute path notation may also be used with multi-valued attributes, such as:

\begin{verbatim}
sp {blocks-world*propose*move-block
   (state <s> ^problem-space blocks
              ^clear.block <block1> { <> <block1> <block2> }
              ^ontop <ontop>)
   (<block1> ^type block)
   (<ontop> ^top-block <block1>
            ^bottom-block <> <block2>)
   -->
   (<s> ^operator <o> +)
   (<o> ^name move-block +
        ^moving-block <block1> +
        ^destination <block2> +) }
\end{verbatim}


\subsubsection*{Multi-attributes and attribute-path notation}
\label{SYNTAX-pm-caveat}

\textbf{Note:} It would not be advisable to write the production in Figure
\ref{fig:ex-prod} using attribute-path notation as follows:

\begin{verbatim}
sp {blocks-world*propose*move-block*dont-do-this
   (state <s> ^problem-space blocks
              ^clear.block <block1>
              ^clear.block { <> <block1> <block2> }
              ^ontop.top-block <block1>
              ^ontop.bottom-block <> <block2>)
   (<block1> ^type block)
   -->
   ...
   }
\end{verbatim}

This is not advisable because it corresponds to a different set of conditions
than those in the original production (the \soar{top-block} and
\soar{bottom-block} need not correspond to the same \soar{ontop} relation).
To check this, we could print the original production at the Soar prompt:

\begin{verbatim}
soar> print blocks-world*propose*move-block*dont-do-this
sp {blocks-world*propose*move-block*dont-do-this
    (state <s> ^problem-space blocks ^thing <thing2>
          ^thing { <> <thing2> <thing1> } ^ontop <o*1> ^ontop <o*2>)
    (<thing2> ^clear yes)
    (<thing1> ^clear yes ^type block)
    (<o*1> ^top-block <thing1>)
    (<o*2> ^bottom-block { <> <thing2> <b*1> })
    -->
    (<s> ^operator <o> +)
    (<o> ^name move-block 
         ^moving-block <thing1> 
         ^destination <thing2>) }
\end{verbatim}

Soar has expanded the production into the longer form, and created two
distinctive variables, \soar{$<$o*1$>$} and \soar{$<$o*2$>$} to represent the
\soar{ontop} attribute. These two variables will not necessarily bind to the
same identifiers in working memory.

% ----------------------------------------------------------------------------
\subsubsection*{Negated multi-valued attributes and attribute-path notation}

Negations of multi-valued attributes can be combined with attribute-path
notation. However; it is very easy to make mistakes when using negated
multi-valued attributes with attribute-path notation. Although it is
possible to do it correctly, we strongly discourage its use.

For example, 

\begin{verbatim}
sp {blocks*negated-conjunction-example
   (state <s> ^name top-state -^ontop.bottom-object.name table A)
   -->
   (<s> ^nothing-ontop-A-or-table true) }
\end{verbatim}

gets expanded to:

\begin{verbatim}
sp {blocks*negated-conjunction-example
   (state <s> ^name top-state)
  -{(<s> ^ontop <o*1>)
    (<o*1> ^bottom-object <b*1>)
    (<b*1> ^name A)
    (<b*1> ^name table)}
   -->
   (<s> ^nothing-ontop-A-or-table true) }
\end{verbatim}

This example does not refer to two different blocks with different
names. It tests that there is not an \soar{ontop} relation with a
\soar{bottom-block} that is named \soar{A} and named \soar{table}. Thus, this
production probably should have been written as:

\begin{verbatim}
sp {blocks*negated-conjunction-example
   (state <s> ^name top-state 
              -^ontop.bottom-object.name table
              -^ontop.bottom-object.name A)
   -->
   (<s> ^nothing-ontop-A-or-table true) }
\end{verbatim}

which expands to: 
\begin{verbatim}
sp {blocks*negated-conjunction-example
   (state <s> ^name top-state)
  -{(<s> ^ontop <o*2>)
    (<o*2> ^bottom-object <b*2>)
    (<b*2> ^name a)}
  -{(<s> ^ontop <o*1>)
    (<o*1> ^bottom-object <b*1>)
    (<b*1> ^name table)}
   -->
   (<s> ^nothing-ontop-a-or-table true +) }
\end{verbatim}

\subsubsection*{Notes on attribute-path notation}\vspace{-12pt}
\begin{itemize}
\item Attributes specified in attribute-path notation may not start with a
        digit. For example, if you type \soar{\carat foo.3.bar}, Soar thinks
        the \soar{.3} is a floating-point number. (Attributes that don't
        appear in path notation can begin with a number.)

\item Attribute-path notation may be used to any depth.

\item Attribute-path notation may be combined with structured values,
        described in Section \ref{SYNTAX-pm-structured}.

\end{itemize}


% ----------------------------------------------------------------------------
\subsubsection{Structured-value notation}
\label{SYNTAX-pm-structured}    %pref-struc-cond}
\index{structured value notation}
\index{production!structured values}
\index{value!structured notation}

Another convenience that eliminates the use of intermediate variables is 
structured-value notation. 

Syntactically, the attributes and values of a condition may be written where a
variable would normally be written. The attribute-value structure is delimited
by parentheses.

Using structured-value notation, the production in Figure \ref{fig:ex-prod}
(on page \pageref{fig:ex-prod}) may also be written as:

\begin{verbatim}
sp {blocks-world*propose*move-block
   (state <s> ^problem-space blocks
              ^thing <thing1> {<> <thing1> <thing2>}
              ^ontop (^top-block <thing1>
                      ^bottom-block <> <thing2>))
   (<thing1> ^type block ^clear yes)
   (<thing2> ^clear yes)
-->
   (<s> ^operator <o> +)
   (<o> ^name move-block
        ^moving-block <thing1>
        ^destination <thing2>) }
\end{verbatim}

Thus, several conditions may be ``collapsed'' into a single condition.


\subsubsection*{Using variables within structured-value notation}

Variables are allowed within the parentheses of structured-value notation to
specify an identifier to be matched elsewhere in the production. For example,
the variable \soar{<ontop>} could be added to the conditions (although it are
not referenced again, so this is not helpful in this instance):

\begin{verbatim}
sp {blocks-world*propose*move-block
   (state <s> ^problem-space blocks
              ^thing <thing1> {<> <thing1> <thing2>}
              ^ontop (<ontop> 
                      ^top-block <thing1>
                      ^bottom-block <> <thing2>))
   (<thing1> ^type block ^clear yes)
   (<thing2> ^clear yes)
   -->
   (<s> ^operator <o> +)
   (<o> ^name move-block
        ^moving-block <thing1>
        ^destination <thing2>) }
\end{verbatim}

Structured values may be nested to any depth. Thus, it is possible to write
our example production using a single condition with multiple structured
values:

\begin{verbatim}
sp {blocks-world*propose*move-block
   (state <s> ^problem-space blocks
              ^thing <thing1> 
                     ({<> <thing1> <thing2>}
                      ^clear yes)
              ^ontop (^top-block 
                        (<thing1>
                         ^type block 
                         ^clear yes)
                      ^bottom-block <> <thing2>) )
   -->
   (<s> ^operator <o> +)
   (<o> ^name move-block
        ^moving-block <thing1>
        ^destination <thing2>) }
\end{verbatim}


\subsubsection*{Notes on structured-value notation}\vspace{-12pt}
\begin{itemize}
\item Attribute-path notation and structured-value notation are orthogonal and
        can be combined in any way. A structured value can contain an
        attribute path, or a structure can be given as the value for an
        attribute path. 

\item Structured-value notation may also be combined with negations and with
        multi-attributes. 

\item Structured-value notation may not be used in the actions of productions.

\end{itemize}


% ----------------------------------------------------------------------------
% ----------------------------------------------------------------------------
\subsection{The action side of productions (or RHS)}
\label{SYNTAX-pm-action}
\index{RHS of production}
\index{production!RHS}
\index{action side of production}


The action side of a production, also called the right-hand side (or RHS) of
the production, consists of individual actions that can:
\begin{itemize}
\item Add new elements to working memory.
\item Remove elements from working memory.
\item Create preferences.
\item Perform other actions
\end{itemize}

When the conditions of a production match working memory, the production is
said to be instantiated, and the production will fire during the next
elaboration cycle. Firing the production involves performing the actions
\emph{using the same variable bindings} that formed the instantiation.

\subsubsection{Variables in Actions}
\index{variable!action side}
Variables can be used in actions.  A variable that appeared in the
condition side will be replaced with the value that is was bound to in
the condition.  A variable that appears only in the action side will be
bound to a new identifier that begins with the first letter of that
variable (e.g., \soar{<o>} might be bound to \soar{o234}). This symbol is
guaranteed to be unique and it will be used for all occurrences of the
variable in the action side, appearing in all working memory elements
and preferences that are created by the production action.

\subsubsection{Creating Working Memory Elements}
An element is created in working memory by specifying it as an action.
Multiple augmentations of an object can be combined into a single
action, using the same syntax as in conditions, including path notation
and multi-valued attributes. 
\begin{verbatim}
   -->
   (<s> ^block.color red
        ^thing <t1> <t2>) }
\end{verbatim}
The action above is expanded to be:
\begin{verbatim}
   -->
   (<s> ^block <*b>)
   (<*b> ^color red)
   (<s> ^thing <t1>)
   (<s> ^thing <t2>) }
\end{verbatim}
This will add four elements to working memory with the variables replaced
with whatever values they were bound to on the condition side.

Since Soar is case sensitive, different combinations of upper- and
lowercase letters represent \emph{different} constants. For example,
``\soar{red}'', ``\soar{Red}'', and ``\soar{RED}'' are all distinct symbols in
Soar. In many cases, it is prudent to choose one of uppercase or lowercase and
write all constants in that case to avoid confusion (and bugs).

The constants that are used for attributes and values have a few
restrictions on them:\vspace{-12pt} 
\begin{enumerate}
\item There are a number of architecturally created augmentations for state
        and impasse objects; see Section \ref{SYNTAX-impasses} for a listing of 
        these special augmentations. User-defined productions can not create
        or remove augmentations of states that use these
        attribute names.\vspace{-8pt}
\item Attribute names should not begin with a number if these attributes will
        be used in attribute-path notation.
\end{enumerate}

\subsubsection{Removing Working Memory Elements}

A element is explicitly removed from working memory by following the
value with a dash: \soar{-}, also called a reject.  

\begin{verbatim}
   -->
   (<s> ^block <b> -)}
\end{verbatim}

If the removal of a working memory element removes the only link between
the state and working memory elements that had the value of the removed
element as an identifier, those working memory elements will be
removed. This is applied recursively, so that all item that become
unlinked are removed.

The reject should be used with an action that will be o-supported.
If reject is attempted with I-support, the working memory element will
reappear if the reject loses I-support and the element still has
support.  

% ----------------------------------------------------------------------------
\subsubsection{The syntax of preferences}
\index{preference}

Below are the eleven types of preferences as they can appear in the actions of a
production for the selection of operators:
\label{pref-list}

\begin{tabular}{| l | l |} \hline
\bf{RHS preferences}                        & \bf{Semantics} \\ \hline
\soar{(id \carat operator value)}          & acceptable  \\ 
\soar{(id \carat operator value +)}        & acceptable  \\ 
\soar{(id \carat operator value !)}        & require \\ 
\soar{(id \carat operator value \tild)}    & prohibit \\
\soar{(id \carat operator value -)}        & reject \\
\soar{(id \carat operator value > value2)} & better \\
\soar{(id \carat operator value < value2)} & worse \\
\soar{(id \carat operator value >)}        & best  \\
\soar{(id \carat operator value <)}        & worst \\
\soar{(id \carat operator value =)}        & unary indifferent  \\
\soar{(id \carat operator value = value2)} & binary indifferent  \\
\soar{(id \carat operator value = number)} & numeric indifferent \\
\hline
\end{tabular} \vspace{10pt}
\index{+}
\index{"!}
\index{~}
\index{-}
\index{>}
\index{<}
\index{=}
\index{&}
\index{"@}


The identifier and value will always be variables, such as
\soar{(<s1> \carat operator <o1> > <o2>)}.

The preference notation appears similar to the predicate tests that
appear on the left-hand side of productions, but has very different
meaning. Predicates cannot be used on the right-hand side of a
production and you cannot restrict the bindings of variables on the
right-hand side of a production. (Such restrictions can happen only in
the conditions.)

Also notice that the \soar{+} symbol is optional when specifying acceptable
preferences in the actions of a production, although using this symbol
will make the semantics of your productions clearer in many instances. The
\soar{+} symbol will always appear when you inspect preference memory (with
the \soar{preferences} command).

Productions are never needed to delete preferences because preferences
will be retracted when the production no longer matches.  Preferences
should never be created by operator application rules, and they should
always be created by rules that will give only I-support to their actions.

% ----------------------------------------------------------------------------
\subsubsection{Shorthand notations for preference creation}

There are a few shorthand notations allowed for the creation of operator
preferences on the right-hand side of productions.

Acceptable preferences do not need to be specified with a \soar{+}
symbol. \soar{(<s> \carat operator <op1>)} is assumed to mean \soar{(<s> \carat
operator <op1> +)}.

Ambiguity can easily arise when using a preference that can be
either binary or unary: \soar{> < =}. The default assumption is that if a
value follows the preference, then the preference is binary. It will be unary
if a carat (up-arrow), a closing parenthesis, another preference, or a comma follows it. 

Below are four examples of legal, although unrealistic, actions that have the
same effect.

\begin{verbatim}
(<s> ^operator <o1> <o2> + <o2> < <o1> <o3> =, <o4>)
(<s> ^operator <o1> + <o2> + 
            <o2> < <o1> <o3> =, <o4> +)
(<s> ^operator <o1> <o2> <o2> < <o1> <o4> <o3> =)
(<s> ^operator <o1> ^operator <o2>
           ^operator <o2> < <o1> ^operator <o4> <o3> =)
\end{verbatim}

Any one of those actions could be expanded to the following list of
preferences: 
\begin{verbatim}
(<s> ^operator <o1> +)
(<s> ^operator <o2> +)
(<s> ^operator <o2> < <o1>)
(<s> ^operator <o3> =)
(<s> ^operator <o4> +)
\end{verbatim}

Note that structured-value notation may not be used in the actions of 
productions.

% ----------------------------------------------------------------------------
\subsubsection{Righthand-side Functions}

The fourth type of action that can occur in productions is called a 
\emph{righthand-side function}.  Righthand-side functions allow productions
to create side effects other than changing working memory.  The RHS functions
are described below, organized by the type of side effect they have.

% ----------------------------------------------------------------------------
\subsubsection{Stopping and pausing Soar}
\label{RHS-stopping}

\begin{description}
\index{halt}
\item [\soarb{halt} ---] Terminates Soar's execution and returns to 
the user prompt.  A \soar{halt} action irreversibly terminates the
running of a Soar program.
It should not be used if Soar is to be restarted (see the
 \soar{interrupt} RHS action below.)
\begin{verbatim}
sp {
    ...
    -->
    (halt) }
\end{verbatim} 

\item [\soarb{interrupt} --- ]
\index{interrupt}
        Executing this function causes Soar to stop at the end of the
        current phase, and return to the user prompt. This is similar 
        to \soar{halt}, but does not terminate the run.
        The run may be continued by issuing a \soar{run} command from
	the user interface.  The \soar{interrupt} RHS function has the
	same effect as typing \soar{stop-soar} at the prompt, except
	that there is more control because it takes effect exactly
	at the end of the phase that fires the production.
\begin{verbatim}
sp {
    ...
    -->
    (interrupt) }
\end{verbatim}
	
	\label{interrupt-directive}
	Soar execution may also be stopped immediately before a production
	fires, using the \soar{:interrupt} directive. This functionality is
	called a matchtime interrupt and is very useful for debugging. See
	Section	\ref{sp} on Page \pageref{sp} for more information.
	
\begin{verbatim}
sp {production*name
    :interrupt
    ...
    -->
    ...
    }
\end{verbatim}
\end{description}

% ----------------------------------------------------------------------------
\subsubsection{Text input and output}

The function \soar{write} is provided as a production
action to do simple output of text in Soar. Soar applications that
do extensive input and output of text should use Soar Markup Language (SML). To learn
about SML, read the "SML Quick Start Guide" which should be located in the "Documentation" 
folder of your Soar install.

 
\begin{description}
\index{write}
\item [\soarb{write} --- ] This function writes its arguments to the standard
        output. It does not automatically insert blanks, linefeeds, or carriage
        returns.  For example, if \soar{<o>} is bound to 4, then
\begin{verbatim}
sp {
    ...
    -->
    (write  <o> <o> <o> | x| <o> | | <o>) }
\end{verbatim}
        prints

\begin{verbatim}
444 x4 4
\end{verbatim}

\index{carriage return, line feed}
\index{crlf}
\item [\soarb{crlf} --- ] Short for ``carriage return, line feed'', this
        function can be called only within \soar{write}. It forces a new line
        at its position in the \soar{write} action. 
\begin{verbatim}
sp {
    ...
    -->
    (write <x> (crlf) <y>) }
\end{verbatim}


%\index{accept}
%\item [\soarb{accept} --- ] Suspends Soar's execution and waits for the user
%        to type a constant, followed by a carriage return. The result of
%        \soar{accept} is the constant. The accept function does not read 
%	in strings.  It accepts a
%        single constant (which may look like a string).
%        Soar applications that make extensive use of text input should be
%        implemented using Tcl and Tk functionality, described in the
%        \emph{Soar Advanced Applications Manual}.

%The \soarb{accept} function does not work properly under the TSI 
%(Tcl-Soar Interface), or any other Soar program that has a separate 
%``Agent Window'' instead of a Tcl or Wish Console.  In this instance, 
%users should employ the \soar{tcl} RHS function 
%(described on page \pageref{SYNTAX-pm-otheractions-tcl}) to get user
%input through a text widget.
%\begin{verbatim} 
%sp {
%    ...
%    -->
%    (<s> ^input (accept)) }
%\end{verbatim}

        \nocomment{Does this imply that a CR is not needed? I.e., will the
                constant be 'accepted' after a space is hit?
                }

\end{description}

% ----------------------------------------------------------------------------
\subsubsection{Mathematical functions}

The expressions described in this section can be nested to any depth. For all
of the functions in this section, missing or non-numeric arguments result 
in an error.


\begin{description}
\index{compute}
\index{arithmetic operations}
\index{floating-point number}
\item [\soarb{+, -, *, /} --- ]
        These symbols provide prefix notation mathematical functions.
        These symbols work similarly to C functions.  They will take either 
        integer or real-number arguments. The first three functions return 
        an integer when all arguments are integers
        and otherwise return a real number, and the last two functions
        always return a real number.  The \soar{-} symbol is also a
	unary function which, given a single argument, returns the
	product of the argument and \soar{-1}.

\begin{verbatim}
sp {
    ...
    -->
    (<s> ^sum (+ <x> <y>)
         ^product-sum (* (+ <v> <w>) (+ <x> <y>))
         ^big-sum (+ <x> <y> <z> 402)
         ^negative-x (- <x>))
}
\end{verbatim}

\item [\soarb{div, mod} --- ]
        These symbols provide prefix notation binary mathematical functions
        (they each take two arguments). These symbols work similarly to C
        functions: They will take only integer arguments (using reals results
        in an error) and return an integer: \soar{div} takes two integers and
        returns their integer quotient; \soar{mod} returns their remainder.

\begin{verbatim}
sp {
    ...
    -->
    (<s> ^quotient (div <x> <y>)
         ^remainder (mod <x> <y>)) }
\end{verbatim}

\item [\soarb{abs, atan2, sqrt, sin, cos} --- ]   
        These symbols provide prefix notation unary mathematical functions
        (they each take one argument). These symbols work similarly to C
        functions: They will take either integer or real-number arguments. The
        first function (\soar{abs}) returns an integer when its argument is an
        integer and otherwise returns a real number, and the last four
        functions always return a real number.  \soar{atan2} returns as
	a float in radians, the arctangent of (first\_arg / second\_arg).
	\soar{sin} and \soar{cos} take as arguments the angle in radians.

\begin{verbatim}
sp {
    ...
    -->
    (<s> ^abs-value (abs <x>)
         ^sqrt (sqrt <x>)) }
\end{verbatim}


% ----------------------------------------------------------------------------
\index{int}
\item [\soarb{int} --- ] Converts a single symbol to an integer constant. This
        function expects either an integer constant, symbolic constant, or
        floating point constant. The symbolic constant must be a string which
        can be interpreted as a single integer. The floating point constant is
        truncated to only the integer portion. This function essentially
        operates as a type casting function.

        For example, the expression \soar{2 + sqrt(6)} could be printed
        as an integer using the following:

\begin{verbatim}
sp {
    ...
    -->
    (write (+ 2 (int sqrt(6))) ) }
\end{verbatim}

% ----------------------------------------------------------------------------
\index{float}
\item [\soarb{float} --- ] Converts a single symbol to a floating point 
constant.
        This function expects either an integer constant, symbolic constant,
        or floating point constant. The symbolic constant must be a string
        which can be interpreted as a single floating point number. This
        function essentially operates as a type casting function. 

        For example, if you wanted to print out an integer expression as a
        floating-point number, you could do the following:

\begin{verbatim}
sp {
    ...
    -->
    (write (float (+ 2 3))) }
\end{verbatim}
\end{description}

        \nocomment{is there a reason you'd ever need to cast a int to a float? I
                can't think of a simple example}


% ----------------------------------------------------------------------------
\subsubsection{Generating and manipulating symbols}

A new symbol (an identifier) is generated on the right-hand side of a
production whenever a previously unbound variable is used. This section
describes other ways of generating and manipulating symbols on the right-hand
side. 

\begin{description}
\index{timestamp}
\item [\soarb{timestamp} --- ] This function returns a symbol whose print name 
is a
        representation of the current date and time. 

        For example:

\begin{verbatim}
sp {
    ...
    -->
    (write (timestamp)) }
\end{verbatim}

        When this production fires, it will print out a representation of the
        current date and time, such as:
\begin{verbatim}
soar> run 1 e
8/1/96-15:22:49
\end{verbatim}    


\index{make-constant-symbol}
\item [\soarb{make-constant-symbol} --- ] This function returns a new constant 
symbol
        guaranteed to be different from all symbols currently present in the
        system.  With no arguments, it returns a symbol whose name starts with
        ``\soar{constant}''.  With one or more arguments, it takes those
        argument symbols, concatenates them, and uses that as the
        prefix for the new symbol. (It may also append a number to the 
	resulting symbol, 
	if a symbol with that prefix as its name already exists.)

\begin{verbatim}
sp {
    ...
    -->
    (<s> ^new-symbol (make-constant-symbol)) }
\end{verbatim}

        When this production fires, it will create an augmentation in working
        memory such as:

\begin{verbatim}
(S1 ^new-symbol constant5)
\end{verbatim} \vspace{12pt}

        The production:

\begin{verbatim}
sp {
    ...
    -->
    (<s> ^new-symbol (make-constant-symbol <s> )) }
\end{verbatim}
        
        will create an augmentation in working memory such as:
\begin{verbatim}
(S1 ^new-symbol |S14|)
\end{verbatim}

        when it fires. The vertical bars denote that the symbol is a
        constant, rather than an identifier; in this example, the number 4 has
        been appended to the symbol S1.

        This can be particularly useful when used in conjunction with the
        \soar{timestamp} function; by using \soar{timestamp} as an argument to
        \soar{make-constant-symbol}, you can get a new symbol that is
        guaranteed to be unique. For example:

\begin{verbatim}
sp {
    ...
    -->
    (<s> ^new-symbol (make-constant-symbol (timestamp))) }
\end{verbatim}

        When this production fires, it will create an augmentation in working
        memory such as:

\begin{verbatim}
(S1 ^new-symbol 8/1/96-15:22:49)
\end{verbatim}    


\index{capitalize-symbol}
\item [\soarb{capitalize-symbol} --- ] Given a symbol, this function returns a 
new
        symbol with the first character capitalized. This function is provided
        primarily for text output, for example, to allow the first word in a
        sentence to be capitalized.

        \nocomment{This command is possibly obsolete, since Soar7 is case 
sensitive?}

\begin{verbatim}
(capitalize-symbol foo)
\end{verbatim}    

\end{description}

% ----------------------------------------------------------------------------
\subsubsection{User-defined functions and interface commands as RHS actions}
%\label{SYNTAX-pm-otheractions-tcl}

Any function which has a certain function signature may be registered with the
Kernel and called as a RHS function.  The function must have the following signature:

\begin{verbatim}
std::string MyFunction(smlRhsEventId id, void* pUserData, Agent* pAgent,
                  char const* pFunctionName, char const* pArgument);
\end{verbatim}

The Tcl and Java interfaces have similar function signatures. Any arguments passed
to the function on the RHS of a production are concatenated and passed to the function
in the pArgument argument.

Such a function can be registered with the kernel via the client interface by calling:

\begin{verbatim}
Kernel::AddRhsFunction(char const* pRhsFunctionName, RhsEventHandler 
                   handler, void* pUserData);
\end{verbatim}

The \soar{exec} and \soar{cmd} functions are used to call user-defined functions and interface
commands on the RHS of a production.

\begin{description}
\index{exec}
\item [\soarb{exec} --- ] Used to call user-defined registered functions. Any arguments are concatenated
without spaces. For example, if \soar{<o>} is bound to \soar{x}, then

\begin{verbatim}
sp {
   ...
   -->
   (exec MakeANote <o> 1) }
\end{verbatim}
   
will call the user-defined \soar{MakeANote} function with the argument "\soar{x1}".

The return value of the function, if any, may be placed in working memory or passed
to another RHS function. For example, the log of a number \soar{<x>} could be printed this way:

\begin{verbatim}
sp {
   ...
   -->
   (write |The log of | <x> | is: | (exec log(<x>))|) }
\end{verbatim}

where "\soar{log}" is a registered user-defined function.

\index{cmd}
\item[\soarb{cmd} --- ] Used to call built-in Soar commands. Spaces are inserted between concatenated 
arguments. For example, the production

\begin{verbatim}
sp {
   ...
   -->
   (write (cmd print --depth 2 <s>)) }
\end{verbatim}

will have the effect of printing the object bound to \soar{<s>} to depth 2.
\end{description}

%There are no safety nets with this function, and users are warned that they
%can get themselves into trouble if not careful.  Users should
%\emph{never} use the \soar{tcl} RHS function to invoke \soar{add-wme},
%\soar{remove-wme} or \soar{sp}.

% ----------------------------------------------------------------------------
\subsubsection{Controlling learning}
\label{SYNTAX-pm-actions-learning}

\nocomment{These RHS actions have not been implemented as of this writing. The
        functionality is achieved using the user-interface functions
        ``chunky-problem-spaces'' and ``chunk-free-problem-spaces''; see
        online help or the web pages for details on these functions.}


Soar's learning mechanism, called Chunking, is described in Chapter 4.

The following two functions are provided as RHS actions to assist in
development of Soar programs; they are not intended to correspond to any
theory of learning in Soar. This functionality is provided as a development 
tool, so that learning may be turned off in specific problem spaces,
preventing otherwise buggy behavior.

The \soar{dont-learn} and \soar{force-learn} RHS actions are to be used with
specific settings for the \soar{learn} command (see page \pageref{learn}.)
Using the \soar{learn} command, learning may be set to one of \soar{on},
\soar{off}, \soar{except}, or \soar{only}; learning must be set to
\soar{except} for the \soar{dont-learn} RHS action to have any effect and
learning must be set to \soar{only} for the \soar{force-learn} RHS action to
have any effect.

\begin{description}
\index{dont-learn}
\item [\soarb{dont-learn} --- ] When learning is set to \soar{except},
        by default chunks can be formed in all states; the \soar{dont-learn}
        RHS action will cause learning to be turned off for the specified
        state.

\begin{verbatim}
sp {turn-learning-off
    (state <s> ^feature 1 ^feature 2 -^feature 3)
     -->
    (dont-learn <s>) }
\end{verbatim}

        The \soar{dont-learn} RHS action applies when \soar{learn} is 
	set to \soar{-except}, and has no effect when other settings for
        \soar{learn} are used.


\index{force-learn}
\item [\soarb{force-learn} --- ] When learning is set to \soar{only},
        by default chunks are not formed in any state; the \soar{force-learn}
        RHS action will cause learning to be turned on for the specified
        state.

\begin{verbatim}
sp {turn-learning-on
    (state <s> ^feature 1 ^feature 2 -^feature 3)
     -->
    (force-learn <s>) }
\end{verbatim}

        The \soar{force-learn} RHS action applies when \soar{learn}
	is set to \soar{-only}, and has no effect when other settings for
        \soar{learn} are used.

\end{description}

% ----------------------------------------------------------------------------
%\subsection{Writing Productions that Create O-supported Preferences}

\nocomment{there's no discussion of o-support in this chapter, and probably
        there should be. maybe a quick separate section on the syntax of
        o-supported productions?

        [things to mention in this section: you can't always tell whether a
        preference will have o-support just by looking at the production
        (o-support is determined at runtime), and rules for determining
        o-support.]  
        }


% ----------------------------------------------------------------------------
% ----------------------------------------------------------------------------
\section{Impasses in Working Memory and in Productions}
\label{SYNTAX-impasses}
\index{subgoal}
\index{impasse}

When the preferences in preference memory cannot be resolved unambiguously,
Soar reaches an impasse, as described in Section \ref{ARCH-impasses}:\vspace{-
12pt}
\begin{itemize}
\item When Soar is unable to select a new operator (in the decision cycle), it
        is said to reach an operator impasse.\vspace{-8pt}
\end{itemize}

All impasses appear as states in working memory, where they can be
tested by productions.  This section describes the structure of state
objects in working memory.

% ----------------------------------------------------------------------------
\subsection{Impasses in working memory}
\label{SYNTAX-impasseaug}       %perf-goal-impa}

There are four types of impasses. 

\nocomment{rewrite this section to show templates of what the objects look like
        in working memory for different types of impasses}

\index{decision!procedure}
\index{impasse}
Below is a short description of the four types of impasses. (This was
described in more detail in Section \ref{ARCH-impasses} on page
\pageref{ARCH-impasses}.)\vspace{-12pt}
\begin{enumerate}
\item \emph{tie}: when there is a collection of equally eligible operators
        competing for the value of a particular attribute;\vspace{-8pt}
\item \emph{conflict}: when two or more objects are better than each other,
        and they are not dominated by a third operator;\vspace{-8pt}
\item \emph{constraint-failure}: when there are conflicting necessity
        preferences; \vspace{-8pt}
\item \emph{no-change}: when the proposal phase runs to quiescence without 
        suggesting a new operator.
\end{enumerate}
\index{impasse!types}
\index{tie impasse}
\index{conflict impasse}
\index{constraint-failure impasse}
\index{no-change impasse}
\index{elaboration!phase}
\index{impasse!resolution}
\index{goal!termination}
\index{subgoal!termination}

The list
below gives the seven augmentations that the architecture creates on the
substate generated when an impasse is reached, and the
values that each augmentation can contain:\vspace{-12pt}
\begin{description} 
\item [\soar{\carat type state}] \vspace{-8pt} 

\item [\soar{\carat impasse}] Contains the impasse type: \soar{tie}, 
    \soar{conflict}, \soar{constraint-failure}, or \soar{no-change}.\vspace{-
8pt} 

\item [\soar{\carat choices}]Either \soar{multiple} (for tie and conflict
        impasses), \soar{constraint-failure} (for constraint-failure
        impasses), or \soar{none} (for no-change impasses).\vspace{-8pt} 

\item [\soar{\carat superstate}] Contains the identifier of the state in which 
        the impasse arose.\vspace{-8pt}
        \index{superstate}

\item [\soar{\carat attribute}] For multi-choice and constraint-failure 
impasses,
        this contains \soar{operator}. For
        no-change impasses, this contains the attribute of the last 
        decision with a value (\soar{state} or \soar{operator}).\vspace{-8pt}
\index{subgoal!augmentations}

\item [\soar{\carat item}] For multi-choice and constraint-failure impasses, 
this 
        contains all values involved in the tie, conflict, or
        constraint-failure. If the set of items that tie or conflict changes
        during the impasse, the architecture removes or adds the appropriate
        item augmentations without terminating the existing impasse.\vspace{-
8pt}
        \index{item (attribute)}

\item [\soar{\carat quiescence}] States are the only objects with 
\soar{quiescence
        t}, which is an explicit statement that quiescence (exhaustion of the
        elaboration cycle) was reached in the superstate.  If problem solving
        in the subgoal is contingent on quiescence having been reached, the
        substate should test this flag.  The side-effect is that no chunk will
        be built if it depended on that test. See Section
        \ref{CHUNKING-creation} on page \pageref{CHUNKING-creation} for
        details. This attribute can be ignored when learning is turned off.
        \index{quiescence t (augmentation)}
        \index{exhaustion}
\end{description} 

Knowing the names of these architecturally defined attributes and their
possible values will help you to write productions that test for the presence
of specific types of impasses so that you can attempt to resolve the impasse
in a manner appropriate to your program. Many of the default
productions in the \soar{demos/defaults} directory of the Soar distribution
 provide means for resolving
certain types of impasses. You may wish to make use of some of all of these
productions or merely use them as guides for writing your own set of
productions to respond to impasses.

\subsubsection*{Examples}

The following is an example of a substate that is created for a tie among
three operators:
\index{goal!examples}
\index{impasse!examples}
\begin{verbatim}
(S12 ^type state ^impasse tie ^choices multiple ^attribute operator 
     ^superstate S3 ^item O9 O10 O11 ^quiescence t)
\end{verbatim} \vspace{12pt}

The following is an example of a substate that is created for a no-change
impasse to apply an operator:
\begin{verbatim}
(S12 ^type state ^impasse no-change ^choices none ^attribute operator 
     ^superstate S3 ^quiescence t)
(S3 ^operator O2)
\end{verbatim} \vspace{12pt}

% ----------------------------------------------------------------------------
\subsection{Testing for impasses in productions}

Since states appear in working memory, they may also be
tested for in the conditions of productions.

% There are numerous examples of this in the set of default productions (see
% Section \ref{default} or Appendix \ref{DEFAULT} for more information).

For example, the following production tests for a constraint-failure impasse
on the top-level state.

\begin{verbatim}
sp {default*top-goal*halt*operator*failure
    "Halt if no operator can be selected for the top goal."
    :default
    (state <ss> ^impasse constraint-failure ^superstate <s>)
    (<s> ^superstate nil)
-->
    (write (crlf) |No operator can be selected for top goal.| )
    (write (crlf) |Soar must halt.| )
    (halt)
}
\end{verbatim}

% ----------------------------------------------------------------------------
\section{Soar I/O: Input and Output in Soar}
\label{SYNTAX-io}
\index{I/O}
\index{motor commands|see{I/O}}

Many Soar users will want their programs to interact with a real or simulated
environment. For example, Soar programs could control a robot, receiving sensory
\emph{inputs} and sending command \textit{outputs}. Soar programs might 
also interact with
simulated environments, such as a flight simulator. The mechanisms by which
Soar receives inputs and sends outputs to an external process is called
\emph{Soar I/O}.

This section describes how input and output are represented in working memory
and in productions.  The details of creating and registering the input and 
output functions for Soar are beyond the scope of this manual, but they are
described in the \textit{SML Quick Start Guide}.
This section is provided for the sake of Soar users who will be making
use of a program that has already been implemented, or for those who would
simply like to understand how I/O is implemented in Soar.
% A simple example
% of Soar I/O using Tcl is provided in Section (Appendix?) \ref{Interface-Tcl_I/O}.


% ----------------------------------------------------------------------------
\subsection{Overview of Soar I/O}

When Soar interacts with an external environment, it must make use of
mechanisms that allow it to receive input from that environment and to effect
changes in that environment. An external environment may be the real world or
a simulation; input is usually viewed as Soar's perception and output is
viewed as Soar's motor abilities.


\index{I/O!input functions}
\index{I/O!output functions}
\index{input functions|see{I/O!input functions}}
\index{output functions|see{I/O!output functions}}
Soar I/O is accomplished via \emph{input functions} and
\emph{output functions}. Input functions are called at the 
\emph{start}
of every execution cycle, and add elements directly to specific input
structures in working memory.  These changes to working memory
may change the set of productions that will fire or retract. 
Output functions are called
at the \emph{end} of every execution cycle and are processed in response to
changes to specific output structures in working memory.  An output function
is called only if changes have been made to the output-link structures in
working memory.

\index{I/O!io attribute}
\index{io attribute|see{I/O!io attribute}}
\index{I/O!input links}
\index{I/O!output links}
\index{input links|see{I/O!input links}}
\index{output links|see{I/O!output links}}
The structures for manipulating input and output in Soar are linked
to a predefined attribute of the
top-level state, called the \soar{io} attribute.  The \soar{io} attribute has
substructure to represent sensor inputs from the environment called
\emph{input links}; because these are represented in working memory, Soar
productions can match against input links to respond to an external
situation. Likewise, the \soar{io} attribute has substructure to
represent motor commands, called \emph{output links}. Functions that 
execute motor commands in the environment use the values on the output links 
to determine when and how they should execute an action.  Generally,
input functions create and remove elements on the input link to update
Soar's perception of the environment.  Output functions respond to values
of working memory elements that appear on Soar's output link strucure.



% ----------------------------------------------------------------------------
\subsection{Input and output in working memory}
\label{ADVANCED-io-wm}

All input and output is represented in working memory as substructure of the
\soar{io} attribute of the top-level state.  By default, the architecture
creates an \soar{input-link} attribute of the \soar{io} object and
an \soar{output-link} attribute of the io object. 
The values of the \soar{input-link} and \soar{output-link} attributes
are identifiers whose augmentations are the complete set of input and
output working memory elements, respectively.  Some Soar systems may 
benefit from having multiple input and output links, or that use names
which are more
descriptive of the input or output function, such as \soar{vision-input-link},
\soar{text-input-link}, or \soar{motor-output-link}.  In addition to
providing  the default \soar{io} substructure, the architecture allows
users to create multiple input and output links via productions
and I/O functions.  Any identifiers for \soar{io} substructure created
by the user will be assigned at run time and are not guaranteed to be
the same from run to run.  Therefore users should always employ
variables when referring to input and output links in productions.

Suppose a blocks-world task is implemented using a robot to move
actual blocks around, with a camera creating input to Soar and a robotic arm
executing command outputs. 
\begin{figure}
\insertfigure{blocks-inputlink}{3.5in}
\insertcaption{An example portion of the input link for the blocks-world task.}
\label{fig:blocks-inputlink}
\end{figure}
The camera image might be analyzed by a separate vision program; this program
could have as its output the locations of blocks on an xy plane.  
The Soar input function could take the
output from the vision program and create the following working memory
elements on the input link (all identifiers are assigned at runtime; 
this is just an example of possible bindings):

\begin{verbatim}
(S1 ^io I1)          [A]
(I1 ^input-link I2)  [A]
(I2 ^block B1)
(I2 ^block B2)
(I2 ^block B3)
(B1 ^x-location 1)
(B1 ^y-location 0)
(B1 ^color red)
(B2 ^x-location 2)
(B2 ^y-location 0)
(B2 ^color blue)
(B3 ^x-location 3)
(B3 ^y-location 0)
(B3 ^color yellow)
\end{verbatim} \vspace{12pt}

The '[A]' notation in the example is used to indicate the working memory
elements that are created by the architecture and not by the input function.
This configuration of blocks corresponds to all blocks on the table, as
illustrated in the initial state in Figure \ref{fig:blocks}.

\begin{figure}
\insertfigure{blocks-outputlink}{3.5in}
\insertcaption{An example portion of the output link for the blocks-world task.}
\label{fig:blocks-outputlink}
\end{figure}

Then, during the Apply Phase of the execution cycle, Soar productions could 
respond to an operator, such as ``move the red block
ontop of the blue block'' by creating a structure on the output link, such as:

\begin{verbatim}
(S1 ^io I1)           [A]
(I1 ^output-link I3)  [A]
(I3 ^name move-block)
(I3 ^moving-block B1)
(I3 ^x-destination 2)
(I3 ^y-destination 1)
(B1 ^x-location 1)
(B1 ^y-location 0)
(B1 ^color red)
\end{verbatim}  \vspace{12pt}

The '[A]' notation is used to indicate the working memory elements 
that are created by the architecture and not by productions.
An output function would look for specific structure in this output link and
translate this into the format required by the external program that controls
the robotic arm. Movement by the robotic arm would lead to changes in the 
vision system, which would later be reported on the input-link.

Input and output are viewed from Soar's perspective. An \emph{input
function} adds or deletes augmentations of the \soar{input-link} 
providing Soar with information about some occurrence external to Soar. An
\emph{output function} responds to substructure of the \soar{output-link}
produced by production firings, and causes some occurrence external to
Soar. Input and output occur through the \soar{io} attribute of the top-level
state exclusively.
\index{top-state!for I/O}

Structures placed on the input-link by an input function remain there until removed
by an input function. During this time, the structure continues to provide support for
any production that has matched against it. The structure does \emph{not} cause the production
to rematch and fire again on each cycle as long as it remains in working memory;
to get the production to refire, the structure must be removed and added again.



%The substructure of the input-link will remain in working memory until 
%the input function that
%created it removes it.  Thus working memory elements produced by an
%input function provide support for condition-matching
%in productions as long as the input persists in working memory, i.e.
%until the input function specifically removes the elements of the
%substructure.  However,
%a production that tests only a single element on the input structure will 
%result in instantiations that fire only once for each input element that
%matches.  The instantiation will not continue to fire for each matched
%input element, unless the element is removed and then added again.


% ----------------------------------------------------------------------------
\subsection{Input and output in production memory}
\label{ADVANCED-io-pm}

Productions involved in \emph{input} will test for specific attributes and
values on the input-link, while productions involved in \emph{output} will
create preferences for specific attributes and values on the output link.
For example, a simplified production that responds to the vision input 
for the blocks task might look like this:

\begin{verbatim}
sp {blocks-world*elaborate*input
    (state <s> ^io.input-link <in>)
    (<in> ^block <ib1>)
    (<ib1> ^x-location <x1> ^y-location <y1>)
    (<in> ^block {<ib2> <> <ib1>})
    (<ib2> ^x-location <x1> ^y-location {<y2> > <y1>})
    -->
    (<s> ^block <b1>)
    (<s> ^block <b2>)
    (<b1> ^x-location <x1>  ^y-location <y1> ^clear no)
    (<b2> ^x-location <x1>  ^y-location <y2> ^above <b1>)
}
\end{verbatim}  \vspace{12pt}

This production ``copies'' two blocks and their locations directly to 
the top-level state. 
%This is a generally a good idea when using input, since the input
%function may change the information on the link before the Soar program has
%finished using it.
It also adds information about the
relationship between the two blocks.  The variables used
for the blocks on the RHS of the production are deliberately different from the
variable name used for the block on the input-link in the LHS of the
production. If the variable were the same, the production would create 
a link into the structure of the input-link, rather than copy the information.
The attributes \soar{x-location} and
\soar{y-location} are assumed to be values and not identifiers, so the same
variable names may be used to do the copying.


A production that creates wmes on the output-link for the blocks task 
might look like this:

\begin{verbatim}
sp {blocks-world*apply*move-block*send-output-command
    (state <s> ^operator <o> ^io.output-link <out>)
    (<o> ^name move-block ^moving-block <b1> ^destination <b2>)
    (<b1> ^x-location <x1> ^y-location <y1>)
    (<b2> ^x-location <x2> ^y-location <y2>)
    -->
    (<out> ^move-block <b1>
           ^x-destination <x2> ^y-destination (+ <y2> 1))
}
\end{verbatim} \vspace{12pt}

This production would create substructure on the output-link that 
the output function could interpret as being a command to 
move the block to a new location.



% ----------------------------------------------------------------------------
\typeout{--------------- learning (CHUNKING) --------------------------------}
\chapter{Learning}
\label{CHUNKING}
\index{chunking}
\index{learning}
\index{result}
\index{subgoal}

\nocomment{

	This chapter really needs a general explanation of learning in
	Soar before it dives into WHEN chunks are and are not formed,
	and HOW ... cf. Chpater 2-level discussions

	I'd like to add two figures to this chapter: \\
	1. an illustration of the backtracing process \\
	2. an illustration of how learning fits into the decision cycle.

	Also, I'm pretty sure this chapter hasn't changed much since
	justifications were added to Soar. It would be nice to connect
	justifications to chunks.
	}


\nocomment{
	\begin{figure}
	%\insertfigure{learn}{7.75in}
	\insertcaption{Will somehow be an illustration of learning in
		the decision cycle.} 
	\label{fig:learn}
	\end{figure}
	}

Chunking is Soar's learning mechanism, the sole learning mechanism in Soar.
Chunking creates productions, called \emph{chunks}, that summarize the
processing required to produce the results of subgoals. When a chunk is built,
it is added to production memory, where it will be matched in similar
situations, avoiding the need for the subgoal. Chunks are created only when
results are formed in subgoals; since most Soar programs are continuously
subgoaling and returning results to higher-level states, chunks are typically
created continuously as Soar runs.

This chapter begins with a discussion of when chunks are built (Section
\ref{CHUNKING-creation} below), followed by a detailed discussion of
how Soar determines a chunk's conditions and actions (Section
\ref{CHUNKING-determining}). Sections \ref{CHUNKING-variablizing} through
\ref{CHUNKING-ordering} examine the construction of chunks in further
detail. Section \ref{CHUNKING-inhibition} explains how and why chunks are
prevented from matching with the WME's that led to their creation. Section
\ref{CHUNKING-problems} reviews the problem of overgeneral chunks.


% ----------------------------------------------------------------------------
\section{Chunk Creation}
\label{CHUNKING-creation}
\index{chunking!creation}

Several factors govern when chunks are built. Soar chunks the results of every
subgoal, \emph{unless} one of the following conditions is true:
\index{chunking!when active}

\index{learn}
\begin{enumerate}
\item Learning is \soar{off}. (See Section \ref{learn} on page \pageref{learn}
	for details of \soar{learn} used to turn learning off.) 

	Learning can be set to \soar{on} or \soar{off}.
	When \soar{learn} is \soar{on} chunks are built.  
	When \soar{learn} is \soar{off}, chunks are not built. 

\item Learning is set to \soar{bottom-up} and a chunk has already 
	been built for a subgoal of the state that generated the results. 
	(See Section \ref{learn} on page \pageref{learn} for details of 
	\soar{learn} used to set learning to bottom-up.) 

	With bottom-up learning, chunks are learned only in states in which no
	subgoal has yet generated a chunk. In this mode, chunks are learned
	only for the ``bottom'' of the subgoal hierarchy and not the
	intermediate levels. With experience, the subgoals at the bottom will
	be replaced by the chunks, allowing higher level subgoals to be
	chunked.\footnote{For some tasks, bottom-up chunking facilitates
	modelling power-law speedups, although its long-term theoretical
	status is problematic.}
	\index{bottom-up chunking}
	\index{chunking!bottom-up}
	
	\nocomment{this would be the appropriate place in the manual to discuss
		the rationale behind the existence of bottom-up chunking. I
		don't believe it's explained anywhere. 
		}

\item The chunk duplicates a production or chunk already in production memory.
	In some rare cases, a duplicate production will not be detected because the
	order of the conditions or actions is not the same as an existing production.  
	\index{chunking!duplicate chunks}

\item The augmentation, \soar{\carat quiescence t}, of the substate that
	produced the result is backtraced through.

	This mechanism is motivated by the \emph{chunking from exhaustion}
	problem, where the results of a subgoal are dependent on the
	exhaustion of alternatives (see Section \ref{CHUNKING-problems} on page
	\pageref{CHUNKING-problems}). If this substate augmentation is
	encountered when determining the conditions of a chunk, then no chunk
	will be built for the currently considered action. This is recursive, 
	so that if an un-chunked result is relevant to a second result, no 
	chunk will be built for the second result. This does not prevent the
	creation of a chunk that would include \soar{\carat quiescence t} as a
	condition.  
	\index{quiescence t (augmentation)} \index{exhaustion}
      
\item Learning has been temporarily turned off via a call to the
	\soar{dont-learn} production action (described on page
	\pageref{SYNTAX-pm-actions-learning} in Section 
	\ref{SYNTAX-pm-actions-learning}).

	This capability is provided for debugging and system development, and
	it is not part of the theory of Soar.
\end{enumerate}

If a result is to be chunked, Soar builds the chunk \emph{as soon as the
result is created}, rather than waiting until subgoal termination.
\index{result}
\index{subgoal!result}

	\nocomment{CBC: As soon as it's identified as a result, I assume.
		E.g., for the case where a ``result'' is created
		first, and not linked to the superstate until later.
	
		BobD: it doesn't become a result until it's linked.}

% ----------------------------------------------------------------------------
\section{Determining Conditions and Actions}
\label{CHUNKING-determining}

Chunking is an experience-based learning mechanism that summarizes  as 
productions the problem solving that occurs within a state. In order to 
maintain a
history of the processing to be used for chunking, Soar builds a 
\emph{trace} of the productions that fire in the subgoals. This section
describes how the relevant actions are determined, how information is 
stored in a trace, and finally, how the trace and the actions together 
determine the conditions for the chunk.

In order for the chunk to apply at the appropriate time, its conditions must
test exactly those working memory elements that were necessary to produce the
results of the subgoal. 
Soar computes a chunk's conditions based on the
productions that fire in the subgoal, beginning with the results of
the subgoal,
and then \emph{backtracing} through the productions that created 
each result.  It recursively backtraces through the working memory
elements that matched the conditions of the productions, finding the
actions that led to the WME's creation, etc., until conditions are
found that test elements that are linked to a superstate.
\index{backtracing}

\index{working memory!trace}
\index{trace!memory}

\nocomment{  This is what is used to say...
Soar computes a chunk's conditions based on the productions that 
fire in the subgoal. Chunking begins with the results of the subgoal,
and then \emph{backtraces} through the productions that created the preference
for each result. It then recursively backtraces through the working memory
elements that matched the conditions of the productions, finding the
acceptable preferences that led to their creation, etc., until conditions are
found that test elements that are linked to a superstate.}

% ----------------------------------------------------------------------------
\subsection{Determining a chunk's actions}
\index{result}
\index{subgoal result}
\index{chunking!determining actions}

A chunk's actions are built from the results of a subgoal.  A \emph{result} is
any working memory element created in the substate that is linked to a 
superstate.  A working memory element
is linked if its identifier is either the value of a superstate
WME, or the value of an augmentation  for an object that is linked to a
superstate.

\index{linked!chunk action}
\index{chunking!actions}

The results produced by a single production firing are the basis for creating
the actions of a chunk. A new result can lead to other results by linking a
superstate to a WME in the substate. This WME may in turn link
other WMEs in the substate to the superstate, making them results.
Therefore, the creation of a single WME that is linked to a superstate
can lead to the creation of a large number of results. All of the newly
created results become the basis of the chunk's actions.

% ----------------------------------------------------------------------------
\subsection{Tracing the creation and reference of working memory elements} 

Soar automatically maintains information on the creation of each 
working memory element in every state.  When a production fires, a
trace of the production is saved with the appropriate state. A \emph{trace} is
a list of the working memory elements matched by the production's conditions,
together with the actions created by the production.  The appropriate state
is the most recently created state (i.e., the state \emph{lowest} in the
subgoal hierarchy) that occurs in the production's matched working memory
elements.
\index{trace!memory}

Recall that when a subgoal is created, the \carat item augmentation lists all
values that lead to the impasse.
Chunking is complicated by the fact that the \soar{\carat item} augmentation
of the substate is created by the architecture and not by productions.
Backtracing cannot determine the cause of these substate augmentations in the
same way as other working memory elements. To overcome this, Soar maps these
augmentations onto the acceptable preferences for the operators in the 
\soar{\carat item} augmentations.


\subsubsection*{Negated conditions}
\index{negated conditions}
\index{chunking!negated conditions}

Negated conditions are included in a trace in the following way: when a
production fires, its negated conditions are fully instantiated with its
variables' appropriate values. This instantiation is based on the working
memory elements that matched the production's positive conditions. If the
variable is not used in any positive conditions, such as in a conjunctive
negation, a dummy variable is used that will later become a variable in a
chunk.

If the identifier used to instantiate a negated condition's identifier field
is linked to the superstate, then the instantiated negated condition is
added to the trace as a negated condition. In all other cases, the negated
condition is ignored because the system cannot determine why a working memory
element \emph{was not} produced in the subgoal and thus allowed the production
to fire. Ignoring these negations of conditions internal to the subgoal may
lead to overgeneralization in chunking (see Section \ref{CHUNKING-problems} on
page \pageref{CHUNKING-problems}). 
\index{overgeneral chunk}
     
% ----------------------------------------------------------------------------
\subsection{Determining a chunk's conditions}
\index{chunking!determining conditions}

The conditions of a chunk are determined by a dependency analysis of
production traces --- a process called \emph{backtracing}.  For each
instantiated production that creates a subgoal result, backtracing examines
the production trace to determine which working memory elements were matched.
If a matched working memory element is linked to a superstate, it is included
in the chunk's conditions. If it is not linked to a superstate, then
backtracing recursively examines the trace of the production that created the
working memory element. Thus, backtracing begins with a subgoal result, traces
backwards through all working memory elements that were used to produce that
result, and collects all of the working memory elements that are linked to a
superstate. This method ignores when the working memory elements were created,
thus allowing the conditions of one chunk to test the results of a chunk
learned earlier in the subgoal. The user can observe the backtracing process
by setting setting backtracing on, using the watch command: \soar{watch
backtracing -on} (see Section \ref{watch} on page \pageref{watch}). 
This prints out a trace of the conditions as they are collected.
\index{backtracing}
\index{chunking!conditions}

\index{desirability preference} 
Certain productions do not participate in backtracing. If a production creates
only a \soar{reject} preference or a desirability preference (\soar{better},
\soar{worse}, \soar{indifferent}, or \soar{parallel}), then neither the
preference nor the objects that led to its creation will be included in the
chunk. (The exception to this is that if the desirability or \soar{reject}
preference is a {\em result} of a subgoal, it will be in the chunk's actions.)
Desirability and reject preferences should be used only as search control for
choosing between legal alternatives and should not be used to guarantee the
correctness of the problem solving. The argument is that such preferences
should affect only the \emph{efficiency} and not the \emph{correctness} of
problem solving, and therefore are not necessary to produce the results.
Necessity preferences (\soar{require} or \soar{prohibit}) should be used to
enforce the correctness of problem solving; the productions that create these
preferences will be included in backtracing.
\index{preference!require}
\index{preference!prohibit}
\index{require preference}
\index{prohibit preference}

Given that results can be created at any point during a subgoal, it is
possible for one result to be relevant to another result. Whether or not the 
first result is included in the chunk for the second result depends on the
links that were used to match the first result in the subgoal. If the elements
are linked to the superstate, they are included as conditions. If the
elements are not linked to the superstate, then the result is traced through.
In some cases, there may be more than one set of links, so it is possible for
a result to be both backtraced through, and included as a condition.


% ----------------------------------------------------------------------------
\section{Variablizing Identifiers}
\label{CHUNKING-variablizing}
\index{identifier!variablization of}
\index{chunking!variablization}
\index{variablization}

Chunks are constructed by examining the traces, which include working memory
elements and operator preferences. To achieve any useful generality in chunks,
identifiers of actual objects must be replaced by variables when the chunk is
created; otherwise chunks will only ever fire when the exact same objects
are matched.  However, a constant value is never variablized; the actual 
value always appears directly in the chunk.

When a chunk is built, all occurrences of the same identifier are replaced
with the same variable. This can lead to an overspecific chunk, when two
variables are forced to be the same in the chunk, even though distinct
variables in the original productions just happened to match the same
identifier.

A chunk's conditions are also constrained by any not-equal (\soar{<>}) tests
for pairs of indentifiers used in the conditions of productions that are
included in the chunk. These tests are saved in the production traces and then
added in to the chunk.
\index{chunking!conditions}

% ----------------------------------------------------------------------------
\section{Ordering Conditions}
\label{CHUNKING-ordering}
\index{matcher}
\index{ordering chunk conditions}

	\nocomment{I think we need an actual section (earlier in the
		manual), describing the rete matcher and the
		reordering of conditions (which also happens 
		internally for user-defined productions). Then this
		section would mention that it's the same reordering process.}

Since the efficiency of the Rete matcher  \cite{Forg81} depends
heavily upon the order of a production's conditions, the chunking mechanism
attempts to write the chunk's conditions in the most favorable order. At each
stage, the condition-ordering algorithm tries to determine which eligible
condition, if placed next, will lead to the fewest number of partial
instantiations when the chunk is matched. A condition that matches an object
with a multi-valued attribute will lead to multiple partial instantiations, so
it is generally more efficient to place these conditions later in the
ordering.
\index{chunking!ordering conditions}
\index{multi-valued attribute}

This is the same process that internally reorders the conditions in
user-defined productions, as mentioned briefly in Section \ref{ARCH-pm-structure}. 


% ----------------------------------------------------------------------------
\section{Inhibition of Chunks}
\label{CHUNKING-inhibition}
\index{chunking!refractory inhibition}
\index{refractory inhibition of chunks}

When a chunk is built, it may be able to match immediately with the same
working memory elements that participated in its creation. If the production's
actions include preferences for new operators, the production would immediately
fire and create a preference for a new operator, which duplicates the 
operator preference
that was the original result of the subgoal. To prevent this,
\emph{inhibition} is used. This means that each production that is built 
during chunking is considered to have already fired with the instantiation of
the exact set of working memory elements used to create it. This does not
prevent a newly learned chunk from matching other working memory elements
that are present and firing with those values.

	\nocomment{any insights to why its called ``refractory''?}

% ----------------------------------------------------------------------------
\section{Problems that May Arise with Chunking}
\label{CHUNKING-problems}
\index{chunking!overgeneral}
\index{chunking!incorrect chunks}
\index{incorrect chunks}
\index{overgeneral chunk}

\nocomment{Moved from chapter 2: If there are no variables in justifications, I
	don't quite understand how overgeneralization can occur. \\
        RD: can still be overgeneral due to lack of conditions, e.g.,
	chunking from exhaustion (from testing a negative condition in the
	subgoal) \\
	JEL: it's from losing the local negations}

\nocomment{BobD: there are more problems with chunking than this. See last two
	slides from ``guts of chunking'', 11th soar workshop (and talk to Bob
	and/or John for an explanation
	}

One of the weaknesses of Soar is that chunking can create overgeneral productions
that apply in inappropriate situations, or overspecific productions that will
never fire. These problems arise when chunking cannot accurately summarize the
processing that led to the creation of a result. Below is a description of
three known problems in chunking.

\subsection{Using search control to determine correctness}
\index{desirability preference}

Overgeneral chunks can be created if a result of problem solving in a subgoal
is dependent on search-control knowledge. Recall that desirability
preferences, such as \soar{better}, \soar{best}, and \soar{worst}, are not
included in the traces of problem solving used in chunking (Section
\ref{CHUNKING-determining} on page \pageref{CHUNKING-determining}). In theory,
these preferences do not affect the validity of search. In practice, however,
a Soar program can be written so that search control \emph{does} affect the
correctness of search. Here are two examples:\vspace{-12pt}

\begin{enumerate} 
\item Some of the tests for correctness of a result are included in
	productions that prefer operators that will produce correct results.
  	The system will work correctly only when those productions are loaded.\vspace{-8pt}
\item An operator is given a worst preference, indicating that it
  	should be used only when all other options have been exhausted.
  	Because of the semantics of worst, this operator will be selected
  	after all other operators; however, if this operator then produces a
  	result that is dependent on the operator occurring after all others,
  	this fact will not be captured in the conditions of the chunk.
\end{enumerate}
\index{necessity preference} 
\index{preference!require} \index{preference!prohibit}
\index{require preference} \index{prohibit preference}

In both of these cases, part of the test for producing a result is {\em
implicit} in search control productions. This move allows the explicit state
test to be simpler because any state to which the test is applied is
guaranteed to satisfy some of the requirements for success. However, chunks
created in such a problem space will be overgeneral because the implicit parts
of the state test do not appear as conditions. 

\textbf{Solution:} To avoid this problem, necessity preferences
(\soar{require} and \soar{prohibit}) should be used whenever a control
decision is being made that also incorporates goal-attainment knowledge.  The
necessity preferences are included in the backtrace by chunking, thereby
avoiding overgenerality.

\subsection{Testing for local negated conditions}

Overgeneral chunks can be created when negated conditions test for the absence
of a working memory element that, if it existed, would be local to the
substate.  Chunking has no mechanism for determining \textit{why} a given
working memory element does not exist, and thus a condition that occurred in a
production in the subgoal is not included in the chunk. For example, if a
production tests for the absence of a local flag, and that flag is copied down
to the substate from a superstate, then the chunk should include a test that
the flag in the superstate does not exist. 
Unfortunately, it is computationally expensive to determine why a given
working memory element does not exist. Chunking only includes negated tests if
they test for the absence of superstate working memory elements. 

\textbf{Solution:} To avoid using negated conditions for local data, the local
data can be made a result by attaching it to the superstate. This increases
the number of chunks learned, but a negated condition for the superstate can
be used that leads to correct chunks.
\index{negated!conditions}
\index{chunking!negated conditions}

\subsection{Testing for the substate}

Overgeneral chunks can be created if a result of a subgoal is dependent on the
creation of an impasse within the substate. For example, processing in a
subgoal may consist of exhaustively applying all the operators in the problem
space. If so, then a convenient way to recognize that all operators have
applied and processing is complete is to wait for a state no-change impasse to
occur. When the impasse occurs, a production can test for the resulting
substate and create a result for the original subgoal. This form of state test
builds overgeneral chunks because no pre-existing structure is relevant to the
result that terminates the subgoal. The result is dependent only on the
existence of the substate within a substate.
\index{quiescence t (augmentation)}
\index{exhaustion}

\textbf{Solution:} The current solution to this problem is to allow the
problem solving to signal the architecture that the test for a substate is
being made.  The signal used by Soar is a test for the \soar{\carat quiescence
t} augmentation of the subgoal.  The chunking mechanism recognizes this test
and does not build a chunk when it is found in a backtrace of a subgoal.  The
history of this test is maintained, so that if the result of the substate is
then used to produce further results for a superstate, no higher chunks will
be built.  However, if the result is used as search control (it is a
desirability preference), then it does not prevent the creation of chunks
because the original result is not included in the backtrace.  If the
\soar{\carat quiescence t} being tested is connected to a superstate, it will
not inhibit chunking and it will be included in the conditions of the chunk.



\nocomment{
	\subsection{Overuse of predicates}

	Moved from chapter 2:

	All of the predicate tests are lost in the chunk, and only the
	exact value is included. If the predicate is explicitly
	represented as a relation between two objects in working
	memory, chunking will capture that abstract relationship 
	and create a much more general chunk.

	(also needs clarification about the use of predicates in the
	blocks world task, where we have to say  that the block that's
	being moved is not the same as the block that's 
        being moved to)

	}

% ----------------------------------------------------------------------------
\typeout{--------------- The Soar User INTERFACE -----------------------------}
\chapter{The Soar User Interface}
\label{INTERFACE}
\index{interface}
%\index{user interface}
%\index{function definitions}

\nocomment{for each command, use the 'funsum' command with a brief
	description. This writes to the manual.glo file which can be edited
	into the funtion summary and index (see that file for more
	instructions). This is a bit tedious, but the reason I've set it up
	this way is that the command set is in flux right now -- this lessens
	the chance that a command will be inadvertently omitted from the
	function summary (or that a defunct command will be inadvertently
	included). 
	}

\nocomment{\begin{figure}[h]
\psfig{figure=dilbert-living.ps,height=2.2in} \vspace{12pt}
\end{figure}
}
% ----------------------------------------------------------------------------


This chapter describes the set of user interface commands for Soar. All commands and examples are presented as 
if they are being entered at the Soar command prompt.

This chapter is organized into 7 sections:
\begin{enumerate}
\item Basic Commands for Running Soar
\item Examining Memory
\item Configuring Trace Information and Debugging
\item Configuring Soar's Run-Time Parameters
\item File System I/O Commands
\item Soar I/O commands
\item Miscellaneous Commands
\end{enumerate}

Each section begins with a summary description of the commands covered
in that section, including the role of the command and its importance
to the user.  Commands are then described fully, in alphabetical order.
The most accurate and up-to-date information on the syntax of the Soar 
User Interface is found online, on the Soar Wiki, at

\begin{verbatim}
   http://winter.eecs.umich.edu/soarwiki/Soar_Command_Line_Interface}.
\end{verbatim}

Throughout this chapter, each function description includes a specification of
its syntax and an example of its use. 

For a concise overview of the Soar interface functions, see the Function
Summary and Index on page \pageref{func-sum}. This index is intended to be a
quick reference into the commands described in this chapter.

\subsubsection*{Notation}

\nocomment{check for all commands that I've got the notation current}

The notation used to denote the syntax for each user-interface command follows
some general conventions:\vspace{-12pt}
\begin{itemize}
\item The command name itself is given in a \soarb{bold} font.\vspace{-8pt}
\item Optional command arguments are enclosed within square brackets,
	\soar{[} and \soar{]}.\vspace{-8pt}
\item A vertical bar, \soar{|}, separates alternatives.\vspace{-8pt}
\item Curly braces, \soar{\{\}}, are used to group arguments when at least
one argument from the set is required.
\item The commandline prompt that is printed by Soar, is normally
the agent name, followed by '\soar{>}'.  In the examples in this manual, 
we use ``\soar{soar>}''.
\item Comments in the examples are preceded by
a '\soar{\#}', and in-line comments are preceded by '\soar{;\#}'.
\end{itemize}

For many commands, there is some flexibility in the order in which the
arguments may be given. (See the online help for each command for more
information.)  We have not incorporated this flexible ordering into the syntax
specified for each command because doing so complicates the specification of
the command.  When the order of arguments will affect the output
produced by a command, the reader will be alerted.

% ----------------------------------------------------------------------------
\section{Basic Commands for Running Soar}
\label{BASIC}

This section describes the commands used to start, run and stop a Soar 
program; to invoke on-line help information; and to create and 
delete Soar productions.  The specific commands described in this
section are:

\paragraph{Summary}
\begin{quote}
\begin{description}
%\item[d] - Run the Soar program for one decision cycle.
%\item[e] - Run the Soar program for one elaboration cycle.
\item[excise] - Delete Soar productions from production memory.
%\item[exit] - Terminate Soar and return to the operating system.
\item[gp] - Define a pattern used to generate and source a set of Soar productions.
\item[help] - Provide formatted, on-line information about Soar commands.
\item[init-soar] - Reinitialize Soar so a program can be rerun from scratch.
\item[quit] - Close log file, terminate Soar, and return user to the operating system.
\item[run] - Begin Soar's execution cycle.
\item[sp] - Create a production and add it to production memory.
\item[stop-soar] - Interrupt a running Soar program.
\end{description}
\end{quote}
These commands are all frequently used anytime Soar is run.

\subsection{\soarb{excise}}
\label{excise}
\index{excise}
Delete Soar productions from production memory. 
\subsubsection*{Synopsis}
\begin{verbatim}
excise production_name [production_name ...]
excise -[acdtu]
\end{verbatim}
\subsubsection*{Options}
\begin{tabular}{|l|l|}
\hline 
 -a, --all  & Remove all productions from memory and perform an init-soar command  \\
 \hline 
 -c, --chunks  & Remove all chunks (learned productions) and justifications from memory  \\
 \hline 
 -d, --default  & Remove all default productions (:default) from memory  \\
 \hline 
 -t, --task  & Remove chunks, justifications, and user productions from memory  \\
 \hline 
 -u, --user  & Remove all user productions (but not chunks or default rules) from memory  \\
 \hline 
production\_name & Remove the specific production with this name.  \\
 \hline 
\end{tabular}
\subsubsection*{Description}
 This command removes productions from Soar's memory. The command must be called with either a specific production name or with a flag that indicates a particular group of productions to be removed. Using the flag \textbf{-a}
 or \textbf{--all}
 also causes an init-soar. 
\subsubsection*{Examples}
 This command removes the production my*first*production and all chunks: \begin{verbatim}
excise my*first*production --chunks
\end{verbatim}
 This removes all productions and does an init-soar: \begin{verbatim}
excise --all
\end{verbatim}
\subsubsection*{Default Aliases}
\begin{tabular}{|l|l|}
\hline 
 Alias  & Maps to  \\
 \hline 
 ex  & excise  \\
 \hline 
\end{tabular}
\subsubsection*{See Also}
\hyperref[init-soar]{init-soar} 
\input{wikicmd/tex/gp}
\subsection{\soarb{help}}
\label{help}
\index{help}
Provide formatted usage information about Soar commands. 
 Status: Incomplete\\ 
Currently uses working directory to find command-names and help/ subdir.--Jonathan 14:02, 25 Mar 2005 (EST) 
\subsubsection*{Synopsis}
\begin{verbatim}
help [command_name]
\end{verbatim}
\subsubsection*{Options}
\begin{tabular}{|l|l|}
\hline 
 command\_name  & Print usage syntax for the command.  \\
 \hline 
\end{tabular}
\subsubsection*{Description}
 This command prints formatted help for the given command name. 
\subsubsection*{Examples}
 To see the syntax for the \emph{excise}
 command: \begin{verbatim}
help excise
\end{verbatim}
 To see what commands help is available for: \begin{verbatim}
help
\end{verbatim}
\subsubsection*{Default Aliases}
\begin{tabular}{|l|l|}
\hline 
 Alias  & Maps to  \\
 \hline 
�?  & help  \\
 \hline 
 man  & help  \\
 \hline 
\end{tabular}
\subsubsection*{See Also}
\hyperref[helpex]{helpex} 
\subsection{\soarb{init-soar}}
\label{init-soar}
\index{init-soar}
empties working memory and resets run-time statistics. 
 Status: Complete
\subsubsection*{Synopsis}
\begin{verbatim}
init-soar
\end{verbatim}
\subsubsection*{Options}
 No options. 
\subsubsection*{Description}
 The init-soar command initializes Soar. It removes all elements from working memory, wiping out the goal stack, and resets all runtime statistics. The firing counts for all productions is reset to zero. The init-soar command allows a Soar program that has been halted to be reset and start its execution from the beginning. 
 init-soar does not remove any productions from production memory; to do this, use the excise command. Note however, that all justifications will be removed because they will no longer be supported. 
\subsubsection*{Examples}
\subsubsection*{See Also}
 excise
\subsubsection*{Structured Output:}
\paragraph*{On Success}
\begin{verbatim}
<result output="raw">true</result>
\end{verbatim}
\subsubsection*{Error Values:}
\paragraph*{During Parsing}
 No errors, all arguments ignored. 
\paragraph*{During Execution}
 kAgentRequired

\documentclass[10pt]{article}
\usepackage{fullpage, graphicx, url}
\title{Quit - Soar Wiki}
\begin{document}
\section*{Quit}
\subsubsection*{From Soar Wiki}


 This is part of the Soar Command Line Interface. 
\section*{ Name }


 \textbf{quit}
 - Close log file, terminate Soar, and return user to the operating system. 


 Status: Complete
\section*{ Synopsis }
\begin{verbatim}
quit

\end{verbatim}
\section*{ Options }


 No options. 
\section*{ Description }


 This command stops the run, quits the log and closes Soar. 
\section*{ Structured Output }
\subsection*{ On Success }
\begin{verbatim}
<result>
  <arg name="message" type="string">Goodbye.</arg>
</result>

\end{verbatim}
\section*{ Error Values }
\subsection*{ During Parsing }


 No errors. 
\subsection*{ During Execution }


 No errors. 

\end{document}

\subsection{\soarb{run}}
\label{run}
\index{run}
Begin Soar\^a��s execution cycle. 
 Complete Complete, except --output may work incorrectly due to gSKI--Jonathan 14:07, 18 Feb 2005 (EST) 
\subsubsection*{Synopsis}
\begin{verbatim}
run [count]
run -[d|e|p|o][fs] [count]
\end{verbatim}
\subsubsection*{Options}
\begin{tabular}{|l|l|}
\hline 
 -d, --decision  & Run Soar for count decision cycles.  \\
 \hline 
 -e, --elaboration  & Run Soar for count elaboration cycles.  \\
 \hline 
 -f, --forever  & Run until halted by problem-solving completion or until stopped by an interrupt.  \\
 \hline 
 -o, --output  & Run Soar until the nth time output is generated by the agent. Limited by the value of max-nil-output-cycles.  \\
 \hline 
 -p, --phase  & Run Soar by phases. A phase is either an input phase, proposal phase, decision phase, apply phase, or output phase.  \\
 \hline 
 -s, --self  & If other agents exist within the kernel, do not run them at this time.  \\
 \hline 
 count  & A single integer which specifies the number of cycles to run Soar.  \\
 \hline 
\end{tabular}
\paragraph*{Deprecated Options}
 These may be reimplemented in the future. 
\begin{tabular}{|l|l|}
\hline 
 --operator  & Run Soar until the nth time an operator is selected.  \\
 \hline 
 --state  & Run Soar until the nth time a state is selected.  \\
 \hline 
\end{tabular}
\subsubsection*{Description}
 The \textbf{run}
 command starts the Soar execution cycle or continues any execution that was temporarily stopped. The default behavior of \textbf{run}
, with no arguments, is to cause Soar to execute until it is halted or interrupted by an action of a production, or until an external interrupt is issued by the user. The \textbf{run}
 command can also specify that Soar should run only for a specific number of Soar cycles or phases (which may also be prematurely stopped by a production action or a control-C). This is helpful for debugging sessions, where users may want to pay careful attention to the specific productions that are firing and retracting. 
 The \textbf{run}
 command takes two optional arguments: an integer, \emph{count}
, which specifies how many units to run; and a \emph{units}
 flag indicating what steps or increments to use. If \emph{count}
 is specified, but no \emph{units}
 are specified, then Soar is run by decision cycles. If \emph{units}
 are specified, but \emph{count}
 is unpecified, then \emph{count}
 defaults to '1'. 
 If there are multiple Soar agents that exist in the same Soar process, then issuing a \textbf{run}
 command in any agent will cause all agents to run with the same set of parameters, unless the flag \textbf{--self}
 is specified, in which case only that agent will execute. 
\paragraph*{Note}
 If Soar has been stopped due to a \textbf{halt}
 action, an \textbf{init-soar}
 command must be issued before Soar can be restarted with the \textbf{run}
 command. 
\subsubsection*{Default Aliases}
\begin{tabular}{|l|l|}
\hline 
 Alias  & Maps to  \\
 \hline 
 d  & run -d 1  \\
 \hline 
 e  & run -e 1  \\
 \hline 
 step  & run 1  \\
 \hline 
\end{tabular}

\subsection{\soarb{sp}}
\label{sp}
\index{sp}
Define a Soar production. 
 Status: Complete
\subsubsection*{Synopsis}
\begin{verbatim}
sp {production_body}
\end{verbatim}
\subsubsection*{Options}
\begin{tabular}{|l|l|}
\hline 
 production\_body  & A Soar production.  \\
 \hline 
\end{tabular}
\subsubsection*{Description}
 This command defines a new Soar production. rule is a single argument parsed by the Soar kernel, so it should be enclosed in curly braces to avoid being parsed by other scripting languages that might be in the same proces. The overall syntax of a rule is as follows: \begin{verbatim}
  name 
      ["documentation-string"] 
      [FLAG*]
      LHS
      -->
      RHS
\end{verbatim}
 The first element of a rule is its name. Conventions for names are given in the Soar Users Manual. If given, the documentation-string must be enclosed in double quotes. Optional flags define the type of rule and the form of support its right-hand side assertions will receive. The specific flags are listed in a separate section below. The LHS defines the left-hand side of the production and specifies the conditions under which the rule can be fired. Its syntax is given in detail in a subsequent section. The --$>$ symbol serves to separate the LHS and RHS portions. The RHS defines the right-hand side of the production and specifies the assertions to be made and the actions to be performed when the rule fires. The syntax of the allowable right-hand side actions are given in a later section. The Soar Users Manual gives an elaborate discussion of the design and coding of productions. Please see that reference for tutorial information about productions. 
  More complex productions can be formed by surrounding the rule with double quotes instead of curly braces. This enables variable and command result substitutions in productions. If another production with the same name already exists, it is excised, and the new production is loaded. 
 \textbf{RULE FLAGS}
\\ 
 The optional FLAGs are given below. Note that these switches are preceeded by a colon instead of a dash -- this is a Soar parser convention. \begin{verbatim}
:o-support      specifies that all the RHS actions are to be given
                o-support when the production fires 
\end{verbatim}
 \begin{verbatim}
:no-support     specifies that all the RHS actions are only to be given
                i-support when the production fires 
\end{verbatim}
 \begin{verbatim}
:default        specifies that this production is a default production 
                (this matters for excise -task and watch task) 
\end{verbatim}
 \begin{verbatim}
:chunk          specifies that this production is a chunk 
                (this matters for learn trace)
\end{verbatim}
\subsubsection*{Examples}
 There are many examples in the Soar Users Manual and the demos subdirectory. Here is a simple production to create a problem space. It comes from the critter-world demo (see the file critter.tcl): \begin{verbatim}
sp {critter*create*space*critter
   "Formulate the initial problem space"
   (state <state> ^superstate nil)
   -->
   (<state> ^name move-around ^problem-space <p1>)
   (<p1> ^name critter)}
\end{verbatim}
 The production above has the name critter*create*space*critter. It has a documentation string that is surrounded by double quotes. The LHS is (state $<$state$>$ \^{}superstate nil) and indicates that this rule will match whenever there is a state object that has the attribute-value pair \^{}superstate nil. The --$>$ arrow separates the left and right-hand sides. The RHS consists of two lines. The first asserts that the state object is to be augmented with the name move-around and a problem space should be created. The second line of the RHS indicates that this problem space should be named critter. 
  New for Soar 8, is right-hand-side dot notation. So this production could also be written: \begin{verbatim}
sp {critter*create*space*critter
   "Formulate the initial problem space"
   (state <state> ^superstate nil)
   -->
   (<state> ^name move-around ^problem-space.name critter)}
\end{verbatim}
 Here is a variant of the above example using double quotes instead of curly braces. Double quotes are needed in order to imbed the value of the Tcl variable soar\_agent\_name in the production. The value of this variable is used to name the problem-space created. \begin{verbatim}
sp "critter*create*space*critter
   (state <state> ^superstate nil)
  -->
  (<state> ^name move-around ^problem-space <p1>)
  (<p1> ^name $soar_agent_name)"
\end{verbatim}
 \textbf{ the rest of this may no longer apply, depending on parsing...}
\\ 
 The primary change in the rule is the last clause of the RHS. In that clause, the scripting (Tcl) variable soar\_agent\_name is expanded. If this rule is given in an interpreter which has the variable soar\_agent\_name set to fred, then the RHS would expand to the following before being sent to the Soar kernel to be parsed: \begin{verbatim}
 (<p1> ^name fred)
\end{verbatim}
 Please be aware that when using double quotes, both the dollar sign (variable expansion) and square brackets (command result substitution) could be interpreted by a scripting language such as Tcl, if loaded into the process that is running Soar. If these characters (\$, [, and ]) are to be passed to the Soar production parser, they must be escaped (using a backslash) to avoid interpretation by the scripting language. 
\subsubsection*{See Also}
 excise learn watch
\subsubsection*{Structured Output:}
\paragraph*{On Success}
\begin{verbatim}
<result output="raw">true</result>
\end{verbatim}
\subsubsection*{Error Values:}
\paragraph*{During Parsing}
 kTooFewArgs, kTooManyArgs, kInvalidProduction
\paragraph*{During Execution}
 kAgentRequired, kgSKIError

\subsection{\soarb{stop-soar}}
\label{stop-soar}
\index{stop-soar}
Pause Soar. 
 Status: Complete\\ 
Reason for stopping currently ignored, not sure what this is for/why this is here.--Jonathan 13:59, 4 Feb 2005 (EST) 
\subsubsection*{Synopsis}
\begin{verbatim}
stop-soar [-s] [reason string]
\end{verbatim}
\subsubsection*{Options}
\begin{tabular}{|l|l|}
\hline 
 -s, --self  & Stop only the soar agent where the command is issued. All other agents continue running as previously specified.  \\
 \hline 
 reason\_string  & An optional string which will be printed when Soar is stopped, to indicate why it was stopped. If left blank, no message will be printed when Soar is stopped.  \\
 \hline 
\end{tabular}
\subsubsection*{Description}
 The \textbf{stop-soar}
 command stops any running Soar agents. It sets a flag in the Soar kernel so that Soar will stop running at a ``safe'' point and return control to the user. This command is usually not issued at the command line prompt - a more common use of this command would be, for instance, as a side-effect of pressing a button on a Graphical User Interface (GUI). 
\subsubsection*{Default Aliases}
\begin{tabular}{|l|l|}
\hline 
 Alias  & Maps to  \\
 \hline 
 stop  & stop-soar  \\
 \hline 
 interrupt  & stop-soar  \\
 \hline 
\end{tabular}
\subsubsection*{See Also}
\hyperref[run]{run} \subsubsection*{Warnings}
 If the graphical interface doesn't periodically do an ``update'' of flush the pending I/O, then it may not be possible to interrupt a Soar agent from the command line. 


\section{Examining Memory}
\label{MEMORY}

This section describes the commands used to inspect production memory,
working memory, and preference memory; to see what productions will 
match and fire in the next Propose or Apply phase;  and to examine the 
goal dependency set.  These commands are particularly useful when
running or debugging Soar, as they let users see what Soar is ``thinking.''
The specific commands described in this section are:

\paragraph{Summary}
\begin{quote}
\begin{description}
\item[default-wme-depth] - Set the level of detail used to print WMEs.
\item[gds-print] - Print the WMEs in the goal dependency set for each goal.
\item[internal-symbols] - Print information about the Soar symbol table.
\item[matches] - Print information about the match set and partial matches.
\item[memories] - Print memory usage for production matches.
\item[preferences] - Examine items in preference memory.
\item[print] - Print items in working memory or production memory.
\item[production-find] - Find productions that contain a given pattern.
%\item[wmes] - An alias for the print command; prints items in working memory.
\end{description}
\end{quote}

Of these commands, \textbf{print} is the most often used (and the most
complex) followed by \textbf{matches} and \textbf{memories}.  \textbf{preferences}
is used to examine which candidate operators have been proposed.
\textbf{production-find} is especially useful when the number of
productions loaded is high.  \textbf{gds-print}
is useful for examining the goal dependecy set when subgoals seem to
be disappearing unexpectedly.  \textbf{default-wme-depth} is related to the \textbf{print} command.
\textbf{internal-symbols} is not often used but is helpful when debugging Soar extensions or
trying to locate memory leaks.

\subsection{\soarb{default-wme-depth}}
\label{default-wme-depth}
\index{default-wme-depth}
Set the level of detail used to print WME\^a��s. 
\subsubsection*{Synopsis}
\begin{verbatim}
default-wme-depth [depth]
\end{verbatim}
\subsubsection*{Options}
\begin{tabular}{|l|l|}
\hline
\soar{ depth } & A non-negative integer.  \\
\hline
\end{tabular}
\subsubsection*{Description}
 The \textbf{default-wme-depth}
 command reflects the default depth used when working memory elements are printed (using the \textbf{print}
 command or \textbf{wmes}
 alias). The default value is 1. When the command is issued with no arguments, \textbf{default-wme-depth}
 returns the current value of the default depth. When followed by an integer value, \textbf{default-wme-depth}
 sets the default depth to the specified value. This default depth can be overridden on any particular call to the \textbf{print}
 or \textbf{wmes}
 command by explicitly using the \textbf{--depth}
 flag, e.g.,\textbf{print --depth 10 \emph{args}
}
. 
 By default, the \textbf{print}
 command prints \emph{objects}
 in working memory, not just the individual working memory element. To limit the output to individual working memory elements, the \textbf{--internal}
 flag must also be specified in the \textbf{print}
 command. Thus when the print depth is \textbf{0}
, by default Soar prints the entire object, which is the same behavior as when the print depth is \textbf{1}
. But if \textbf{--internal}
 is also specified, then a depth of \textbf{0}
 prints just the individual WME, while a depth of \textbf{1}
 prints all WMEs which share that same identifier. This is true when printing timetags, identifiers or WME patterns. 
 When the depth is greater than \textbf{1}
, the identifier links from the specified WME's will be followed, so that additional substructure is printed. For example, a depth of \textbf{2}
 means that the object specified by the identifier, wme-pattern, or timetag will be printed, along with all other objects whose identifiers appear as values of the first object. This may result in multiple copies of the same object being printed out. If \textbf{--internal}
 is also specified, then individuals WMEs and their timetags will be printed instead of the full objects. 
\subsubsection*{Default Aliases}
\begin{tabular}{|l|l|}
\hline
\soar{ Alias } & Maps to  \\
\hline
\soar{ set-default-depth } & default-wme-depth  \\
\hline
\end{tabular}
\subsubsection*{See Also}
\hyperref[print]{print} 
\subsection{\soarb{gds-print}}
\label{gds-print}
\index{gds-print}
Print the WMEs in the goal dependency set for each goal. 
\subsubsection*{Synopsis}
\begin{verbatim}
gds-print
\end{verbatim}
\subsubsection*{Options}
 No options. 
\subsubsection*{Description}
 The Goal Dependency Set (GDS) is described in an appendix of the Soar manual. This command is a debugging command for examining the GDS for each goal in the stack. First it steps through all the working memory elements in the rete, looking for any that are included in \emph{any}
 goal dependency set, and prints each one. Then it also lists each goal in the stack and prints the wmes in the goal dependency set for that particular goal. This command is useful when trying to determine why subgoals are disappearing unexpectedly: often something has changed in the goal dependency set, causing a subgoal to be regenerated prior to producing a result. 
\subsubsection*{Warnings}
 gds-print is horribly inefficient and should not generally be used except when something is going wrong and you need to examine the Goal Dependency Set. 
\subsubsection*{Default Aliases}
\begin{tabular}{|l|l|}
\hline
\soar{ Alias } & Maps to  \\
\hline
\soar{ gds\_print } & gds-print  \\
\hline
\end{tabular}
 Categories: Command Line Interface

\subsection{\soarb{internal-symbols}}
\label{internal-symbols}
\index{internal-symbols}
Print information about the Soar symbol table. 
 Priority: 4; Status: Incomplete, EvilBackDoor\\ 
Result generated by kernel.--Jonathan 16:16, 23 Feb 2005 (EST) 
\subsubsection*{Synopsis}
\begin{verbatim}
internal-symbols
\end{verbatim}
\subsubsection*{Options}
 No options. 
\subsubsection*{Description}
\subsubsection*{Structured Output:}
\paragraph*{On Success}
 Returns the output from the kernel in a string message. 
\paragraph*{Notes}
\subsubsection*{Error Values:}
 No errors. 
\paragraph*{During Parsing}
\paragraph*{During Execution}

\subsection{\soarb{matches}}
\label{matches}
\index{matches}
Prints information about partial matches and the match set. 
 Priority: 1�; Status: Incomplete, EvilBackDoor\\ 
Result generated by kernel.--Jonathan 12:18, 7 Feb 2005 (EST) 
\subsubsection*{Synopsis}
\begin{verbatim}
matches [-nc0t1w2] production name
matches -[a|r] [-nc0t1w2]
\end{verbatim}
\subsubsection*{Options}
\begin{tabular}{|l|l|}
\hline 
production\_name & Print partial match information for the named production.  \\
 \hline 
 -0, -n, --names, -c, --count  & For the match set, print only the names of the productions that are about to fire or retract (the default). If printing partial matches for a production, just list the partial match counts.  \\
 \hline 
 -1, -t, --timetags  & Also print the timetags of the wmes at the first failing condition  \\
 \hline 
 -2, -w, --wmes  & Also print the full wmes, not just the timetags, at the first failing condition.  \\
 \hline 
 -a, --assertions  & List only productions about to fire.  \\
 \hline 
 -r, --retractions  & List only productions about to retract.  \\
 \hline 
\end{tabular}
\subsubsection*{Description}
 The matches command prints a list of productions that have instantiations in the match set, i.e., those productions that will retract or fire in the next Propose or Apply phase. It also will preint partial match information for a single, named production. 
\subsection*{Printing the match set}
 When printing the match set (i.e., no production name is specified), the default action prints only the names of the productions which are about to fire or retract. If there are multiple instantiations of a production, the total number of instantiations of that production is printed after the production name, unless \textbf{--timetags|1}
 or \textbf{--wmes|2}
 are specified, in which case each instantiation is printed on a separate line. 
 When printing the match set, the \textbf{--assertions}
 and \textbf{--retractions}
 arguments may be specified to restrict the output to print only the assertions or retractions. 
\subsection*{Printing partial matches for productions}
. The pointer \textbf{$>$$>$$>$$>$}
 before a condition indicates that this is the first condition that failed to match. 
 When printing partial matches, the default action is to print only the counts of the number of WME's that match, and is a handy tool for determining which condition failed to match for a production that you thought should have fired. At levels \textbf{1}
 and \textbf{2}
 (or \textbf{--timetags}
 and \textbf{--wmes}
\subsection*{Notes}
 In Soar 8, the execution cycle (decision cycle) is input, propose, decide, apply output; it no longer stops for user input after the decision phase when running by decision cycles (\textbf{run -d 1}
). If a user wishes to print the match set immediately after the decision phase and before the apply phase, then the user must run Soar by \emph{phases}
 (\textbf{run -p 1}
). 
\subsubsection*{Examples}
 (fix this) - output? This example prints the productions which are about to fire and the wmes that match the productions on their left-hand sides: \begin{verbatim}
matches --assertions --wmes
\end{verbatim}
 This example prints the wme timetags for a single production. \begin{verbatim}
matches -t my*first*production</code.
\end{verbatim}
\subsubsection*{See Also}
 monitor

\subsection{\soarb{memories}}
\label{memories}
\index{memories}
Print memory usage for partial matches. 
 Status: Complete
\subsubsection*{Synopsis}
\begin{verbatim}
memories [-cdju] [\emph{n}
]
memories production_name 
\end{verbatim}
\subsubsection*{Options}
\begin{tabular}{|l|l|}
\hline 
 -c, --chunks  & Print memory usage of chunks.  \\
 \hline 
 -d, --default  & Print memory usage of default productions.  \\
 \hline 
 -j, --justifications  & Print memory usage of justifications.  \\
 \hline 
 -u, --user  & Print memory usage of user-defined productions.  \\
 \hline 
production\_name & Print memory usage for a specific production.  \\
 \hline 
\emph{n}
 & Number of productions to print, sorted by those that use the most memory.  \\
 \hline 
\end{tabular}
\subsubsection*{Description}
 is given, only \emph{n}
 productions will be printed: the \emph{n}
 productions that use the most memory. Output may be restricted to print memory usage for particular types of productions using the command options. 
 Memory usage is recorded according to the tokens that are allocated in the rete network for the given production(s). This number is a function of the number of elements in working memory that match each production. Therefore, this command will not provide useful information at the beginning of a Soar run (when working memory is empty) and should be called in the middle (or at the end) of a Soar run. 
 As a rule of thumb, numbers less than 100 mean that the production is using a small amount of memory, numbers above 1000 mean that the production is using a large amount of memory, and numbers above 10,000 mean that the production is using a \emph{very}
 large amount of memory. 
\subsubsection*{Examples}
 To show how to use the command in context, do this: \begin{verbatim}
command --option arg
\end{verbatim}
 and possibly explain the results. 
\subsubsection*{See Also}
 matches

\documentclass[10pt]{article}
\usepackage{fullpage, graphicx, url}
\setlength{\parskip}{1ex}
\setlength{\parindent}{0ex}
\title{Preferences - Soar Wiki}
\begin{document}
\section*{Preferences}
\subsubsection*{From Soar Wiki}


 This is part of the Soar Command Line Interface. 
\section*{ Name }


 \textbf{preferences}
 - Examine details about the preferences that support the specified \emph{id}
 and \emph{attribute}
. 


 Priority: 2; Status: Incomplete, EvilBackDoor\\ 
Result generated by kernel.--Jonathan 15:45, 18 Feb 2005 (EST) 
\section*{ Synopsis }
\begin{verbatim}
preferences [-0123nNtw] [id] [[^]attribute]

\end{verbatim}
\section*{ Options }


\begin{tabular}{|c|c|}
\hline 
 -0, -n, --none  & Print just the preferences themselves  \\
 \hline 
 -1, -N, --names  & Print the preferences and the names of the productions that generated them  \\
 \hline 
 -2, -t, --timetags  & Print the information for the --names option above plus the timetags of the wmes matched by the indicated productions  \\
 \hline 
 -3, -w, --wmes  & Print the information for the --timetags option above plus the entire wme.  \\
 \hline 
id & Must be an existing Soar object identifier.  \\
 \hline 
attribute & Must be an existing \emph{\^{}attribute}
 of the specified identifier.  \\
 \hline 

\end{tabular}



 \\ 

\section*{ Description }


 This command prints all the preferences for the given object id and attribute. If \emph{id}
 and \emph{attribute}
 are not specified, they default to the current state and the current operator. The '\^{}' is optional when specifying the attribute. The optional arguments indicates the level of detail to print about each preference. 
\section*{ Examples }


 This example prints the preferences on the (S1 \^{}operator) and the production names which created the preferences: \begin{verbatim}
preferences S1 operator --names

\end{verbatim}



 if the current state is S1, then the above syntax is equivalent to: \begin{verbatim}
 preferences -n

\end{verbatim}

\section*{ See Also }
\section*{ Structured Output }


 preferences returns formatted output in a string, this needs to be re-done.--Jonathan 15:44, 18 Feb 2005 (EST) 
\subsection*{ On Success }
\begin{verbatim}
<result>
  <arg param="message" type="string">output_string</arg>
</result>

\end{verbatim}
\subsection*{ Notes }
\section*{ Error Values }
\subsection*{ During Parsing }


 kUnrecognizedOption, kGetOptError, kTooManyArgs
\subsection*{ During Execution }


 kAgentRequired, kKernelRequired, kgSKIError Retrieved from ``\url{http://winter.eecs.umich.edu/soarwiki/Preferences}``

\end{document}

\subsection{\soarb{print}}
\label{print}
\index{print}
Print items in working memory or production memory. 
 1 Incomplete EvilBackDoor ResultByKernel
\subsubsection*{Synopsis}
\begin{verbatim}
print [-fFin] production_name
print -[a|c|D|j|u][fFin]
print [-i] [-d <depth>] \emph{identifier}
|\emph{timetag}
|\emph{pattern}
print -s[oS]
\end{verbatim}
\subsubsection*{Options}
\subsection*{Printing items in production memory}
\begin{tabular}{|l|l|}
\hline 
 -a, --all  & print the names of all productions currently loaded  \\
 \hline 
 -c, --chunks  & print the names of all chunks currently loaded  \\
 \hline 
 -D, --defaults  & print the names of all default productions currently loaded  \\
 \hline 
 -f, --full  & When printing productions, print the whole production. This is the default when printing a named production.  \\
 \hline 
 -F, --filename  & also prints the name of the file that contains the production.  \\
 \hline 
 -i, --internal  & items should be printed in their internal form. For productions, this means leaving conditions in their reordered (rete net) form.  \\
 \hline 
 -j, --justifications  & print the names of all justifications currently loaded.  \\
 \hline 
 -n, --name  & When printing productions, print only the name and not the whole production. This is the default when printing any category of productions, as opposed to a named production.  \\
 \hline 
 -u, --user  & print the names of all user productions currently loaded  \\
 \hline 
production\_name & print the production named production-name \\
 \hline 
\end{tabular}
\subsection*{Printing items in working memory}
\begin{tabular}{|l|l|}
\hline 
 -d, --depth \emph{n}
 & This option overrides the default printing depth (see the default-wme-depth command for more detail).  \\
 \hline 
 -i, --internal  & items should be printed in their internal form. For working memory, this means printing the individual elements with their timetags, rather than the objects.  \\
 \hline 
\emph{identifier}
 & print the object \emph{identifier}
. \emph{identifier}
 must be a valid Soar symbol such as \textbf{S1 }
 \hline 
\emph{pattern}
 & print the object whose working memory elements matching the given pattern. See Description for more information on printing objects matching a specific pattern.  \\
 \hline 
\emph{timetag}
 & print the object in working memory with the given \emph{timetag}
 \hline 
\end{tabular}
\subsection*{Printing the current subgoal stack}
\begin{tabular}{|l|l|}
\hline 
 -s, --stack  & Specifies that the Soar goal stack should be printed. By default this includes both states and operators.  \\
 \hline 
 -o, --operators  & When printing the stack, print only \textbf{operators}
.  \\
 \hline 
 -S, --states  & When printing the stack, print only \textbf{states}
.  \\
 \hline 
\end{tabular}
\subsubsection*{Description}
 The \textbf{print}
 command is used to print items from production memory or working memory. It can take several kinds of arguments. When printing items from working memory, the Soar objects are printed unless the --internal flag is used, in which case the wmes themselves are printed. \begin{verbatim}
(\emph{identifier}
 ^\emph{attribute value}
 [+])
\end{verbatim}
 The pattern is surrounded by parentheses. The \emph{identifier}
, \emph{attribute}
, and \emph{value}
 must be valid Soar symbols or the wildcard symbol * which matches all occurences. The optional \textbf{+}
 symbol restricts pattern matches to acceptable preferences. 
\subsubsection*{Examples}
 Print the working memory elements (and their timetags) which have the identifier s1 as object and v2 as value: \begin{verbatim}
print --internal (s1 ^* v2)
\end{verbatim}
 Print the Soar stack which includes states and operators: \begin{verbatim}
print --stack
\end{verbatim}
 Print the named production in its RETE form: \begin{verbatim}
print -if prodname
\end{verbatim}
 Print the names of all user productions currently loaded: \begin{verbatim}
print -u
\end{verbatim}
\subsubsection*{Default Aliases}
\begin{tabular}{|l|l|}
\hline 
 Alias  & Maps to  \\
 \hline 
 p  & print  \\
 \hline 
 wmes  & print -i  \\
 \hline 
\end{tabular}
\subsubsection*{See Also}
\hyperref[default-wme-depth]{default-wme-depth} \hyperref[predefined-aliases]{predefined-aliases} 
\subsection{\soarb{production-find}}
\label{production-find}
\index{production-find}
\subsubsection*{Synopsis}
production-find [-lrs[n|c]] \emph{pattern}
\end{verbatim}
\subsubsection*{Options}
\hline
\soar{\soar{ -c, --chunks }} & Look \emph{only}
 for chunks that match the pattern.  \\
\hline
\soar{\soar{ -l, --lhs }} & Match pattern only against the conditions (left-hand side) of productions (default).  \\
\hline
\soar{\soar{ -n, --nochunks }} &\emph{Disregard}
 chunks when looking for the pattern.  \\
\hline
\soar{\soar{ -r, --rhs }} & Match pattern against the actions (right-hand side) of productions.  \\
\hline
\soar{\soar{ -s, --show-bindings }} & Show the bindings associated with a wildcard pattern.  \\
\hline
\soar{\soar{ pattern }} & Any pattern that can appear in productions.  \\
\hline
\end{tabular}
\subsubsection*{Description}
 The production-find command is used to find productions in production memory that include conditions or actions that match a given \emph{pattern}
. The pattern given specifies one or more condition elements on the left hand side of productions (or negated conditions), or one or more actions on the right-hand side of productions. Any pattern that can appear in productions can be used in this command. In addition, the asterisk symbol, *, can be used as a wildcard for an attribute or value. It is important to note that the whole pattern, including the parenthesis, must be enclosed in curly braces for it to be parsed properly. 
 The variable names used in a call to production-find do not have to match the variable names used in the productions being retrieved. 
 The production-find command can also be restricted to apply to only certain types of productions, or to look only at the conditions or only at the actions of productions by using the flags. 
\subsubsection*{Examples}
 Find productions that test that some object \emph{gumby}
 has an attribute \emph{alive}
 with value \emph{t}
. In addition, limit the rules to only those that test an operator named \emph{foo}
production-find (<state> ^gumby <gv> ^operator.name foo)(<gv> ^alive t)
\end{verbatim}
 Note that in the above command, $<$state$>$ does not have to match the exact variable name used in the production. 
 Find productions that propose the operator \emph{foo}
production-find --rhs (<x> ^operator <op> +)(<op> ^name foo)
\end{verbatim}
production-find --chunks (<x> ^pokey *)
\end{verbatim}
source demos/water-jug/water-jug.soar
production-find (<s> ^name *)(<j> ^volume *)
production-find (<s> ^name *)(<j> ^volume 3)
production-find --rhs (<j> ^* <volume>)
\end{verbatim}
\subsubsection*{See Also}
\hyperref[sp]{sp} 

% ****************************************************************************
% ----------------------------------------------------------------------------
\section{Configuring Trace Information and Debugging}
\label{DEBUG}

This section describes the commands used primarily for debugging or
to configure the trace output printed by Soar as it runs.  Users may:
specify the content of the runtime trace output; ask that
they be alerted when specific productions fire and retract; 
or request details on Soar's performance.

The specific commands described in this section are:


\paragraph{Summary}
\begin{quote}
\begin{description}
\item[chunk-name-format] - Specify format of the name to use for new chunks.
\item[firing-counts] - Print the number of times productions have fired.
%\item[format-watch] - Change the trace output that's printed as Soar runs.
%\item[interrupt] - Add \& remove pre-firing interrupts on specific productions.
%\item[monitor] - Manage attachment of Tcl scripts to Soar events.
\item[pwatch] - Trace firings and retractions of specific productions.
\item[stats] - Print information on Soar's runtime statistics.
\item[verbose] -  Control detailed information printed as Soar runs.
\item[warnings] - Toggle whether or not warnings are printed.
\item[watch] - Control the information printed as Soar runs.
\item[watch-wmes] -  Print information about wmes that match a certain pattern as they are added and removed
\end{description}
\end{quote}

Of these commands, \soar{watch} is the most often used (and the most 
complex). \soar{pwatch} is related to \soar{watch}, but applies only 
to specific, named productions. \soar{firing-counts} and \soar{stats} 
are useful for understanding how much work Soar is doing. \soar{chunk-name-format} is less-frequently
used, but allows for detailed control of Soar's chunk naming.

\subsection{\soarb{chunk-name-format}}
\label{chunk-name-format}
\index{chunk-name-format}
Specify format of the name to use for new chunks. 
 Priority: 4; Status: Complete, EvilBackDoor
\subsubsection*{Synopsis}
\begin{verbatim}
chunk-name-format [-sl] -p [<prefix>]
chunk-name-format [-sl] -c [<count>]
\end{verbatim}
\subsubsection*{Options}
\begin{tabular}{|l|l|}
\hline 
 -s, --short  & Use the short format for naming chunks  \\
 \hline 
 -l, --long  & Use the long format for naming chunks (default)  \\
 \hline 
 -p, --prefix [$<$prefix$>$]  & If $<$prefix$>$ is given, use $<$prefix$>$ as the prefix for naming chunks. Otherwise, return the current \emph{prefix}
. (defaults to ``\textbf{chunk}
``)  \\
 \hline 
 -c, --count [$<$count$>$]  & If $<$count$>$ is given, set the chunk counter for naming chunks to $<$count$>$. Otherwise, return the current value of the chunk counter.  \\
 \hline 
\end{tabular}
\subsubsection*{Description}
 The short format for naming newly-created chunks is: 
 \emph{prefixChunknum}
 The long (default) format for naming chunks is: 
 \emph{prefix-Chunknum}
*d\emph{dc}
*\emph{impassetype}
*\emph{dcChunknum}
 where: 
 \emph{prefix}
 is a user-definable prefix string; \emph{prefix}
 defaults to ``\textbf{chunk}
`` when unspecified by the user. It many not contain the character *, 
 \emph{Chunknum}
 is $<$count$>$ for the first chunk created, $<$count$>$+1 for the second chunk created, etc. 
 \emph{dc}
 is the number of the decision cycle in which the chunk was formed, 
 \emph{impassetype}
 is one of \textbf{[tie | conflict | cfailure | snochange | opnochange]}
, 
 \emph{dcChunknum}
 is the number of the chunk within that specific decision cycle. 

\subsection{\soarb{firing-counts}}
\label{firing-counts}
\index{firing-counts}
Print the number of times each production has fired. 
 Status: Complete
\subsubsection*{Synopsis}
\begin{verbatim}
firing-counts [\emph{n}
]
firing-counts \emph{production_names}
\end{verbatim}
\subsubsection*{Options}
 If given, an option can take one of two forms -- an integer or a list of production names: 
\begin{tabular}{|l|l|}
\hline 
\emph{n}
 & List the top \emph{n}
 productions. If \emph{n}
 is 0, only the productions which haven't fired are listed  \\
 \hline 
 production\_name  & For each production in production\_names, print how many times the production has fired  \\
 \hline 
\end{tabular}
\subsubsection*{Description}
, is given, only the top \emph{n}
 productions are listed. If \textbf{n}
 is zero (0), only the productions that haven't fired at all are listed. If one or more production names are given as arguments, only firing counts for these productions are printed. 
 Note that firing counts are reset by a call to \textbf{init-soar}
. 
\subsubsection*{Examples}
 This example prints the 10 productions which have fired the most times along with their firing counts: \begin{verbatim}
firing-counts 10
\end{verbatim}
 This example prints the firing counts of productions my*first*production and my*second*production: \begin{verbatim}
firing-counts my*first*production my*second*production
\end{verbatim}
\subsubsection*{Warnings}
 Firing-counts are reset to zero after an init-soar. \\ 
 NB: This command is slow, because the sorting takes time O(n*log n) 
\subsubsection*{Default Aliases}
\begin{tabular}{|l|l|}
\hline 
 Alias  & Maps to  \\
 \hline 
 fc  & firing-counts  \\
 \hline 
\end{tabular}
\subsubsection*{See Also}
 init-soar

\subsection{\soarb{pwatch}}
\label{pwatch}
\index{pwatch}
Trace firings and retractions of specific productions. 
 Complete EvilBackDoor
\subsubsection*{Synopsis}
\begin{verbatim}
pwatch [-d|e] [production name]
\end{verbatim}
\subsubsection*{Options}
\begin{tabular}{|l|l|}
\hline 
 -d, --disable, --off  & Turn production watching off for the specified production. If no production is specified, turn production watching off for all productions.  \\
 \hline 
 -e, --enable, --on  & Turn production watching on for the specified production. The use of this flag is optional, so this is pwatch's default behavior. If no production is specified, all productions currently being watched are listed.  \\
 \hline 
production name & The name of the production to watch.  \\
 \hline 
\end{tabular}
\subsubsection*{Description}
 The \textbf{pwatch}
 command enables and disables the tracing of the firings and retractions of individual productions. This is a companion command to \textbf{watch}
, which cannot specify individual productions by name. 
 With no arguments, \textbf{pwatch}
 lists the productions currently being traced. With one production-name argument, \textbf{pwatch}
 enables tracing the production; \textbf{--enable}
 can be explicitly stated, but it is the default action. 
 If \textbf{--disable}
 is specified followed by a production-name, tracing is turned off for the production. When no production-name is specified, \textbf{pwatch --enable}
 lists all productions currently being traced, and \textbf{pwatch --disable}
 disables tracing of all productions. 
 Note that \textbf{pwatch}
 now only takes one production per command. Use multiple times to watch multiple functions. 
\subsubsection*{See Also}
\hyperref[watch]{watch} 
\documentclass[10pt]{article}
\usepackage{fullpage, graphicx, url}
\title{Stats - Soar Wiki}
\begin{document}
\section*{Stats}
\subsubsection*{From Soar Wiki}


 This is part of the Soar Command Line Interface. 
\section*{ Name }


 \textbf{stats}
 - Print information on Soar\^a��s runtime statistics. 


 Priority: 1; Status: Incomplete\\ 
Memory pool and rete stats not implemented with structured output.--Jonathan 16:04, 8 Mar 2005 (EST) \\ 
stats -r (raw output) looks like trash output.--Jonathan 16:01, 8 Mar 2005 (EST) 
\section*{ Synopsis }
\subsection*{ Structured Output }
\begin{verbatim}
stats

\end{verbatim}
\subsection*{ Raw Output }
\begin{verbatim}
stats [-s|-m|-r]

\end{verbatim}
\section*{ Options }


\begin{tabular}{|p{1in}|p{5in}|}
\hline 
 -m, --memory  & report usage for Soar's memory pools  \\
 \hline 
 -r, --rete  & report statistics about the rete structure  \\
 \hline 
 -s, --system  & report the system (agent) statistics. This is the default if no args are specified.  \\
 \hline 

\end{tabular}



 \\ 

\section*{ Description }


 This command prints Soar internal statistics. The module indicates the component of interest. If specified, module must be one of --system, --memory, or --rete. All statistics are listed for that module. 
\section*{ Examples }


 This prints all statistics in the --system module: \begin{verbatim}
stats --system

\end{verbatim}

\section*{ See Also }


 timers


 \\ 

\section*{ A Note on Timers }


 The current implementation of Soar uses a number of timers to provide time-based statistics for use in the stats command calculations. These timers are: \\ 
 total CPU time total kernel time phase kernel time (per phase) phase callbacks time (per phase) input function time output function time \\ 
 Total CPU time is calculated from the time a decision cycle (or number of decision cycles) is initiated until stopped. Kernel time is the time spent in core Soar functions. In this case, kernel time is defined as the all functions other than the execution of callbacks and the input and output functions. The total kernel timer is only stopped for these functions. The phase timers (for the kernel and callbacks) track the execution time for individual phases of the decision cycle (i.e., input phase, preference phase, working memory phase, output phase, and decision phase). Because there is overhead associated with turning these timers on and off, the actual kernel time will always be greater than the derived kernel time (i.e., the sum of all the phase kernel timers). Similarly, the total CPU time will always be greater than the derived total (the sum of the other timers) because the overhead of turning these timers on and off is included in the total CPU time. In general, the times reported by the single timers should always be greater than than the corresponding derived time. Additionally, as execution time increases, the difference between these two values will also increase. For those concerned about the performance cost of the timers, all the run time timing calculations can be compiled out of the code by defining NO\_TIMING\_STUFF (in soarkernel.h) before compilation. 


 \\ 

\section*{ Structured Output }


 The following arg parameters are returned: \begin{verbatim}
kParamStatsProductionCountDefault, kTypeInt
kParamStatsProductionCountUser, kTypeInt
kParamStatsProductionCountChunk, kTypeInt
kParamStatsProductionCountJustification, kTypeInt
kParamStatsCycleCountDecision, kTypeInt
kParamStatsCycleCountElaboration, kTypeInt
kParamStatsProductionFiringCount, kTypeInt
kParamStatsWmeCountAddition, kTypeInt
kParamStatsWmeCountRemoval, kTypeInt
kParamStatsWmeCount, kTypeInt
kParamStatsWmeCountAverage, kTypeDouble
kParamStatsWmeCountMax, kTypeInt
kParamStatsKernelTimeTotal, kTypeDouble
kParamStatsMatchTimeInputPhase, kTypeDouble
kParamStatsMatchTimeDetermineLevelPhase, kTypeDouble
kParamStatsMatchTimePreferencePhase, kTypeDouble
kParamStatsMatchTimeWorkingMemoryPhase, kTypeDouble
kParamStatsMatchTimeOutputPhase, kTypeDouble
kParamStatsMatchTimeDecisionPhase, kTypeDouble
kParamStatsOwnershipTimeInputPhase, kTypeDouble
kParamStatsOwnershipTimeDetermineLevelPhase, kTypeDouble
kParamStatsOwnershipTimePreferencePhase, kTypeDouble
kParamStatsOwnershipTimeWorkingMemoryPhase, kTypeDouble
kParamStatsOwnershipTimeOutputPhase, kTypeDouble
kParamStatsOwnershipTimeDecisionPhase, kTypeDouble
kParamStatsChunkingTimeInputPhase, kTypeDouble
kParamStatsChunkingTimeDetermineLevelPhase, kTypeDouble
kParamStatsChunkingTimePreferencePhase, kTypeDouble
kParamStatsChunkingTimeWorkingMemoryPhase, kTypeDouble
kParamStatsChunkingTimeOutputPhase, kTypeDouble
kParamStatsChunkingTimeDecisionPhase, kTypeDouble
kParamStatsMemoryUsageMiscellaneous, kTypeInt
kParamStatsMemoryUsageHash, kTypeInt
kParamStatsMemoryUsageString, kTypeInt
kParamStatsMemoryUsagePool, kTypeInt
kParamStatsMemoryUsageStatsOverhead, kTypeInt

\end{verbatim}

\section*{ Error Values }
\subsection*{ During Parsing }


 kTooManyArgs, kUnrecognizedOption, kGetOptError
\subsection*{ During Execution }


 kAgentRequired, kgSKIError

\end{document}

\subsection{\soarb{verbose}}
\label{verbose}
\index{verbose}
Control detailed information printed as Soar runs. 
\subsubsection*{Synopsis}
verbose [-ed]
\end{verbatim}
\subsubsection*{Options}
\hline
\soar{\soar{\soar{ -d, --disable, --off }}} & Turn verbosity off.  \\
\hline
\soar{\soar{\soar{ -e, --enable, --on }}} & Turn verbosity on.  \\
\hline
\end{tabular}
\subsubsection*{Description}
 Invoke with no arguments to query. (fix this) - More about what this command does? 

\subsection{\soarb{warnings}}
\label{warnings}
\index{warnings}
 Complete EvilBackDoor
\subsubsection*{Synopsis}
\begin{verbatim}
warnings -[e|d]
\end{verbatim}
\subsubsection*{Options}
\begin{tabular}{|l|l|}
\hline 
 -e, --enable, --on  & Default. Print all warning messages from the kernel.  \\
 \hline 
 -d, --disable, --off  & Disable all, except most critical, warning messages.  \\
 \hline 
\end{tabular}
\subsubsection*{Description}
 Enables and disables the printing of warning messages. If an argument is specified, then the warnings are set to that state. If no argument is given, then the current warnings status is printed. At startup, warnings are initially enabled. If warnings are disabled using this command, then some warnings may still be printed, since some are considered too important to ignore. 
 The warnings that are printed apply to the syntax of the productions, to notify the user when they are not in the correct syntax. When a lefthand side error is discovered (such as conditions that are not linked to a common state or impasse object), the production is generally loaded into production memory anyway, although this production may never match or may seriously slow down the matching process. In this case, a warning would be printed only if \textbf{warnings}
 were \textbf{--on}
. Righthand side errors, such as preferences that are not linked to the state, usually result in the production not being loaded, and a warning regardless of the \textbf{warnings}
 setting. 
\subsubsection*{Examples}
\subsubsection*{See Also}

\documentclass[10pt]{article}
\usepackage{fullpage, graphicx, url}
\setlength{\parskip}{1ex}
\setlength{\parindent}{0ex}
\title{Watch - Soar Wiki}
\begin{document}
\section*{Watch}
\subsubsection*{From Soar Wiki}


 This is part of the Soar Command Line Interface. 
\section*{ Name }


 \textbf{watch}
 - Control the run-time tracing of Soar. 


 Status: Complete, EvilBackDoor
\section*{ Synopsis }
\begin{verbatim}
watch
watch [--level] [0|1|2|3|4|5]
watch -N
watch -[dpPwrDujcbi] [<remove>] -[n|t|f]
watch --learning [<print|noprint|fullprint>]

\end{verbatim}
\section*{ Options }


\begin{tabular}{|c|c|c|}
\hline 
\emph{Option Flag}
 &\emph{Argument to Option}
 &\emph{Description}
 \\
 \hline 
 -l, --level  & 0 to 5 (see \textbf{Watch Levels}
 below)  & This flag is optional but recommended. Set a specific watch level using an integer 0 to 5, this is an inclusive operation  \\
 \hline 
 -N, --none  & No argument  & Turns off all printing about Soar's internals, equivalent to --level 0  \\
 \hline 
 -d, --decisions  & remove (optional, see \textbf{Remove}
 below)  & Controls whether state and operator decisions are printed as they are made  \\
 \hline 
 -p, --phases  & remove (optional, see \textbf{Remove}
 below)  & Controls whether decisions cycle phase names are printed as Soar executes  \\
 \hline 
 -P, --productions  & remove (optional, see \textbf{Remove}
 below)  & Controls whether the names of productions are printed as they fire and retract, equivalent to -Dujc  \\
 \hline 
 -w, --wmes  & remove (optional, see \textbf{Remove}
 below)  & Controls the printing of working memory elements that are added and deleted as productions are fired and retracted  \\
 \hline 
 -r, --preferences  & remove (optional, see \textbf{Remove}
 below)  & Controls whether the preferences generated by the traced productions are printed when those productions fire or retract  \\
 \hline 
 -D, --default  & remove (optional, see \textbf{Remove}
 below)  & Control only default-productions as they fire and retract  \\
 \hline 
 -u, --user  & remove (optional, see \textbf{Remove}
 below)  & Control only user-productions as they fire and retract  \\
 \hline 
 -c, --chunks  & remove (optional, see \textbf{Remove}
 below)  & Control only chunks as they fire and retract  \\
 \hline 
 -j, --justifications  & remove (optional, see \textbf{Remove}
 below)  & Control only justifications as they fire and retract  \\
 \hline 
 -n, --nowmes  & No argument  & When watching productions, do not print any information about wmes as they are added or retracted  \\
 \hline 
 -t, --timetags  & No argument  & When watching productions, print only the timetags for wmes as they are added or retracted  \\
 \hline 
 -f, --fullwmes  & No argument  & When watching productions, print the full wmes as they are added or retracted  \\
 \hline 
 -b, --backtracing  & remove (optional, see \textbf{Remove}
 below)  & Controls the printing of backtracing information when a chunk or justification is created  \\
 \hline 
 -i, --indifferent-selection  & remove (optional, see \textbf{Remove}
 below)  & Controls the printing of the scores for tied operators in random indifferent selection mode  \\
 \hline 
 -L, --learning  & noprint, print, or fullprint (see \textbf{Learning}
 below)  & Controls the printing of chunks/justifications as they are created  \\
 \hline 

\end{tabular}

\subsection*{ Watch Levels }


 Use of the --level (-l) flag is optional but recommended. 

\begin{tabular}{|c|c|}
\hline 
 0  & watch nothing; equivalent to \^a��N  \\
 \hline 
 1  & watch decisions; equivalent to -d  \\
 \hline 
 2  & watch phases and decisions; equivalent to -dp  \\
 \hline 
 3  & watch productions, phases, and decisions; equivalent to -dpP  \\
 \hline 
 4  & watch wmes, productions, phases, and decisions; equivalent to -dpPw  \\
 \hline 
 5  & watch preferences, wmes, productions, phases, and decisions; equivalent to -dpPwr  \\
 \hline 

\end{tabular}




 \\ 
 It is important to note that watch level 0 turns off ALL watch options, including backtracing, indifferent selection and learning. However, the other watch levels do not change these settings. That is, if any of these settings is changed from its default, it will retain its new setting until it is either explicitly changed again or the watch level is set to 0. 
\subsection*{ Remove }


 The remove argument has a numeric alias; you can use 0 for remove. A mix of formats is acceptable, even in the same command line. 

\begin{tabular}{|c|c|c|}
\hline 
 remove  & 0  & Turn watching off only for the specified option  \\
 \hline 

\end{tabular}


\subsection*{ Learning }


 The learning options have numeric aliases; you can use 0 for noprint, 1 for print, and 2 for fullprint. A mix of formats is acceptable, even in the same command line. 

\begin{tabular}{|c|c|c|}
\hline 
 noprint  & 0  & Print nothing about new chunks or justifications (default)  \\
 \hline 
 print  & 1  & Print the names of new chunks and justifications when created  \\
 \hline 
 fullprint  & 2  & Print entire chunks and justifications when created  \\
 \hline 

\end{tabular}




 \\ 

\section*{ Description }


 The watch command controls run-time tracing of Soar. With no arguments, this command prints out the current watch status. The various levels are used to modify the current watch settings. Each level can be indicated with either a number or a series of flags as follows: \begin{verbatim}
0 or --none
1 or --decisions
2 or --decisions --phases
3 or --decisions --phases --productions
4 or --decisions --phases --productions --wmes
5 or --decisions --phases --productions --wmes --preferences

\end{verbatim}



 The numerical arguments \emph{inclusively}
 turn on all levels up to the number specified. To use numerical arguments to turn off a level, specify a number which is less than the level to be turned off. For instance, to turn off watching of productions, specify ``watch --level 2'' (or 1 or 0). Numerical arguments are provided for shorthand convenience. For more detailed control over the watch settings, the named arguments should be used. 


 For the named arguments, including the named argument turns on only that setting. To turn off a specific setting, follow the named argument with \emph{remove}
 or \emph{0}
. 


 The named argument --productions is shorthand for the four arguments --default, --user, --justifications, and --chunks. 


 The pwatch command is used to watch individual productions specified by name rather than watch a type of productions, such as --user. 
\section*{ Examples }


 The most common uses of watch are by using the numeric arguments which indicate watch levels. To turn off all printing of Soar internals, do any one of the following (not all possibilities listed): \begin{verbatim}
watch --level 0
watch -l 0
watch -N

\end{verbatim}



 Although the --level flag is optional, its use is recommended: \begin{verbatim}
watch --level 5 \emph{... OK}

watch 5         \emph{... OK, but try to avoid}


\end{verbatim}



 Be careful of where the level is on the command line, for example, if you want level 2 and preferences: \begin{verbatim}
watch -r -l 2 \emph{... Incorrect: -r flag ignored, level 2 parsed after it and overrides the setting}

watch -r 2    \emph{... Syntax error: 0 or remove expected as optional argument to -r}

watch -r -l 2 \emph{... Incorrect: -r flag ignored, level 2 parsed after it and overrides the setting}

watch 2 -r    \emph{... OK, but try to avoid}

watch -l 2 -r \emph{... OK}


\end{verbatim}



 To turn on printing of decisions, phases and productions, do any one of the following (not all possibilities listed): \begin{verbatim}
watch --level 3
watch -l 3
watch --decisions --phases --productions
watch -d -p -P

\end{verbatim}



 Individual options can be changed as well. To turn on printing of decisions and wmes, but not phases and productions, do any one of the following (not all possibilities listed): \begin{verbatim}
watch --level 1 --wmes
watch -l 1 -w
watch --decisions --wmes
watch -d --wmes
watch -w --decisions
watch -w -d

\end{verbatim}



 To turn on printing of decisions, productions and wmes, and turns phases off, do any one of the following (not all possibilities listed): \begin{verbatim}
watch --level 4 --phases remove
watch -l 4 -p remove
watch -l 4 -p 0
watch -d -P -w -p remove

\end{verbatim}



 To watch the firing and retraction of decisions and \emph{only}
 user productions, do any one of the following (not all possibilities listed): \begin{verbatim}
watch -l 1 -u
watch -d -u

\end{verbatim}



 To watch decisions, phases and all productions \emph{except}
 user productions and justifications, and to see full wmes, do any one of the following (not all possibilities listed): \begin{verbatim}
watch --decisions --phases --productions --user remove --justifications remove --fullwmes
watch -d -p -P -f -u remove -j 0 
watch -f -l 3 -u 0 -j 0

\end{verbatim}

\section*{ See Also }


 pwatch print run watch-wmes
\section*{ Structured Output }
\subsection*{ On Query }


 The following arg parameters are returned: \begin{verbatim}
kParamWatchDecisions, kTypeBoolean
kParamWatchPhases, kTypeBoolean
kParamWatchProductionDefault, kTypeBoolean
kParamWatchProductionUser, kTypeBoolean
kParamWatchProductionChunks, kTypeBoolean
kParamWatchProductionJustifications, kTypeBoolean
kParamWatchWMEDetail, kTypeInt
kParamWatchWorkingMemoryChanges, kTypeBoolean
kParamWatchPreferences, kTypeBoolean
kParamWatchLearning, kTypeInt
kParamWatchBacktracing, kTypeBoolean
kParamWatchIndifferentSelection, kTypeBoolean

\end{verbatim}

\subsection*{ Otherwise }
\begin{verbatim}
<result output="raw">true</result>

\end{verbatim}


 \\ 

\section*{ Error Values }
\subsection*{ During Parsing }


 kMissingOptionArg, kUnrecognizedOption, kGetOptError, kTooManyArgs, kIntegerExpected, kIntegerMustBeNonNegative, kIntegerOutOfRange, kInvalidLearnSetting, kRemoveOrZeroExpected
\subsection*{ During Execution }


 kAgentRequired, kKernelRequired Retrieved from ``\url{http://winter.eecs.umich.edu/soarwiki/Watch}``

\end{document}

\subsection{\soarb{watch-wmes}}
\label{watch-wmes}
\index{watch-wmes}
\subsubsection*{Synopsis}
\begin{verbatim}
watch-wmes -[a|r]  -t <type>  pattern
watch-wmes -[l|R] [-t <type>]
\end{verbatim}
\subsubsection*{Options}
\begin{tabular}{|l|l|}
\hline
\soar{ -a, --add-filter } & Add a filter to print wmes that meet the type and pattern criteria.  \\
\hline
\soar{ -r, --remove-filter } & Delete filters for printing wmes that match the type and pattern criteria.  \\
\hline
\soar{ -l, --list-filter } & List the filters of this type currently in use. Does not use the pattern argument.  \\
\hline
\soar{ -R, --reset-filter } & Delete all filters of this type. Does not use pattern arg.  \\
\hline
\soar{ -t, --type } & Follow with a type of wme filter, see below.  \\
\hline
\end{tabular}
\paragraph*{Pattern}
 The pattern is an id-attribute-value triplet: \begin{verbatim}
\emph{id}
 \emph{attribute}
 \emph{value}
\end{verbatim}
 Note that \textbf{*}
 can be used in place of the id, attribute or value as a wildcard that maches any string. Note that braces are not used anymore. 
\paragraph*{Types}
 When using the -t flag, it must be followed by one of the following: 
\begin{tabular}{|l|l|}
\hline
\soar{ adds } & Print info when a wme is \emph{added}
.  \\
\hline
\soar{ removes } & Print info when a wme is \emph{retracted}
.  \\
\hline
\soar{ both } & Print info when a wme is added \emph{or}
 retracted.  \\
\hline
\end{tabular}
 When issuing a \textbf{-R}
 or \textbf{-l}
, the \textbf{-t}
 flag is optional. Its absence is equivalent to \textbf{-t both}
. 
\subsubsection*{Description}
 This commands allows users to improve state tracing by issuing filter-options that are applied when watching wmes. Users can selectively define which \emph{object-attribute-value}
 triplets are monitored and whether they are monitored for addition, removal or both, as they go in and out of working memory. 
 \textbf{Note:}
 The functionality of \textbf{watch-wmes}
 resided in the \textbf{watch}
 command prior to Soar 8.6. 
\subsubsection*{Examples}
 Users can \textbf{watch}
 an \emph{attribute}
 of a particular object (as long as that object already exists):  \begin{verbatim}
soar> watch-wmes --add-filter -t both D1 speed *
\end{verbatim}
 or print WMEs that retract in a specific state (provided the \textbf{state}
 already exists):  \begin{verbatim}
soar> watch-wmes --add-filter -t removes S3 * *
\end{verbatim}
  or watch any relationship between objects:  \begin{verbatim}
soar> watch-wmes --add-filter -t both * ontop *
\end{verbatim}


% ----------------------------------------------------------------------------
\section{Configuring Soar's Runtime Parameters}
\label{RUNTIME}

This section describes the commands that control Soar's Runtime Parameters.
Many of these commands provide options that simplify or restrict 
runtime behavior to enable easier and more localized debugging.
Others allow users to select alternative algorithms or methodologies.
Users can configure Soar's learning mechanism; examine the
backtracing information that supports chunks and justifications;
provide hints that could improve the efficiency of the Rete matcher;
limit runaway chunking and production firing;
choose an alternative algorithm for determining whether a working memory
element receives O-support;  and 
configure options for selecting between mutually indifferent operators.

The specific commands described in this section are:

\paragraph{Summary}
\begin{quote}
\begin{description}
\item[explain-backtraces] - Print information about chunk and justification backtraces.
\item[indifferent-selection] -  Controls indifferent preference arbitration.
\item[learn] - Set the parameters for chunking, Soar's learning mechanism.
\item[max-chunks] - Limit the number of chunks created during a decision cycle.
\item[max-elaborations] - Limit the maximum number of elaboration cycles in a given phase.
\item[max-memory-usage] - Set the number of bytes that when exceeded by an agent, will trigger the memory usage exceeded event. 
\item[max-nil-output-cycles] - Limit the maximum number of decision cycles executed without producing output. 
\item[multi-attributes] - Declare multi-attributes so as to increase Rete matching efficiency.
\item[numeric-indifferent-mode] - Select method for combining numeric preferences.
\item[o-support-mode] - Choose experimental variations of o-support.
\item[predict] - Predict the next selected operator 
\item[rl] - Get/Set Soar-RL parameters and statistics 
\item[save-backtraces] - Save trace information to explain chunks and justifications.
\item[select] - Force the next selected operator 
\item[set-stop-phase] -  Controls the phase where agents stop when running by decision.
\item[timers] - Toggle on or off the internal timers used to profile Soar.
\item[waitsnc] - Generate a wait state rather than a state-no-change impasse.
\end{description}
\end{quote}

% ----------------------------------------------------------------------------
\subsection{\soarb{explain-backtraces}}
\label{explain-backtraces}
\index{explain-backtraces}
Print information about chunk and justification backtraces. 
 Priority: 3; Status: Incomplete, EvilBackDoor\\ 
Result generated by kernel.--Jonathan 18:16, 25 Feb 2005 (EST) 
\subsubsection*{Synopsis}
\begin{verbatim}
explain-backtraces -f prod_name
explain-backtraces [-c <n>] prod_name
\end{verbatim}
\subsubsection*{Options}
\begin{tabular}{|l|l|}
\hline 
 prod\_name  & List all conditions and grounds for the chunk or justification.  \\
 \hline 
 -c, --condition  & Explain why condition number \emph{n}
 is in the chunk or justification.  \\
 \hline 
 -f, --full  &�?  \\
 \hline 
\end{tabular}
\subsubsection*{Description}
 This command provides some interpretation of backtraces generated during chunking. If no option is given, then a list of all chunks and justifications is printed. 
 The two most useful variants are: \begin{verbatim}
explain-backtraces prodname 
explain-backtraces name n
\end{verbatim}
 The first variant lists all of the conditions for the named chunk or justification, and the ground which resulted in inclusion in the chunk/justification. A ground is a working memory element (WME) which was tested in the supergoal. Just knowing which WME was tested may be enough to explain why the chunk/justification exists. If not, the conditions can be listed with an integer value. This value can be used in explain-backtraces name n to obtain a list of the productions which fired to obtain this condition in the chunk/justification (and the crucial WMEs tested along the way). Why use an integer value to specify the condition? To save a big parsing job. 
 save\_backtraces mode must be on when a chunk or justification is created or no explanation will be available. 
\subsubsection*{Structured Output:}
\paragraph*{On Success}
\paragraph*{Notes}
\subsubsection*{Error Values:}
\paragraph*{During Parsing}
 kNotImplemented
\paragraph*{During Execution}

\subsection{\soarb{indifferent-selection}}
\label{indifferent-selection}
\index{indifferent-selection}
Controls indifferent preference arbitration. 
\subsubsection*{Synopsis}
\begin{verbatim}
indifferent-selection [-aflr]
\end{verbatim}
\subsubsection*{Options}
\begin{tabular}{|l|l|}
\hline
\soar{ -a, --ask } & Ask the user to choose. Not implemented. \\
\hline
\soar{ -f, --first } & Select the first indifferent object from Soar's internal list.  \\
\hline
\soar{ -l, --last } & Select the last indifferent object from Soar's internal list.  \\
\hline
\soar{ -r, --random } & Select randomly (default).  \\
\hline
\end{tabular}
\subsubsection*{Description}
 The \textbf{indifferent-selection}
 command allows the user to set which option should be used to select between operator proposals that are mutally indifferent in preference memory. 
 The default option is \textbf{--random}
 which chooses an operator at random from the set of mutually indifferent proposals, with the selection biased by any existing numeric preferences. For repeatable results, the user may choose the \textbf{--first}
 or \textbf{--last}
 option. ``First'' refers to the list of operator augmentations internal to Soar; the ordering of the augmentations is arbitrary but deterministic, so that if you run Soar repeatedly, \textbf{--first}
 will always make the same decision. Similarly, \textbf{--last}
 chooses the last of the tied objects from the internal list. For complete control over the decision process, the \textbf{--ask}
 option prompts the user to select the next operator from a list of the tied operators. 
 If no argument is provided, \textbf{indifferent-selection}
 will display the current setting. 
\subsubsection*{Default Aliases}
\begin{tabular}{|l|l|}
\hline
\soar{ Alias } & Maps to  \\
\hline
\soar{ inds } & indifferent-selection  \\
\hline
\end{tabular}
\subsubsection*{See Also}
\hyperref[numeric-indifferent-mode]{numeric-indifferent-mode} 
\documentclass[10pt]{article}
\usepackage{fullpage, graphicx, url}
\setlength{\parskip}{1ex}
\setlength{\parindent}{0ex}
\title{Learn - Soar Wiki}
\begin{document}
\section*{Learn}
\subsubsection*{From Soar Wiki}


 This is part of the Soar Command Line Interface. 
\section*{ Name }


 \textbf{learn}
 - Set the parameters for chunking, Soar\^a��s learning mechanism. 


 Status: Complete, EvilBackDoor
\section*{ Synopsis }
\begin{verbatim}
learn [-l]
learn -[d|E|o]
learn -e [ab]

\end{verbatim}
\section*{ Options }


\begin{tabular}{|c|c|}
\hline 
 -a, --all-levels  & Build chunks whenever a subgoal returns a result. Learning must be --enabled.  \\
 \hline 
 -b, --bottom-up  & Build chunks only for subgoals that have not yet had any subgoals with chunks built. Learning must be --enabled.  \\
 \hline 
 -d, --disable, --off  & Turn all chunking off.  \\
 \hline 
 -e, --enable, --on  & Turn chunking on. Can be modified by -a or -b  \\
 \hline 
 -E, --except  & Learning is on, except as specified by RHS \emph{dont-learn}
 actions.  \\
 \hline 
 -l, --list  & Prints listings of dont-learn and force-learn states.  \\
 \hline 
 -o, --only  & Chunking is on only as specified by RHS \emph{force-learn}
 actions.  \\
 \hline 

\end{tabular}



 \\ 

\section*{ Description }


 The learn command controls the parameters for chunking (Soar's learning mechanism). With no arguments, this command prints out the current learning environment status. If arguments are provided, they will alter the learning environment as described in the options and arguments table. The watch command can be used to provide various levels of detail when productions are learned. Learning is \textbf{disabled}
 by default. 


 \\ 

\section*{ Examples }


 To enable learning only at the lowest subgoal level: \begin{verbatim}
learn -e b 

\end{verbatim}



 To see all the \emph{force-learn}
 and \emph{dont-learn}
 states registered by RHS actions \begin{verbatim}
learn -l

\end{verbatim}

\section*{ See Also }
\begin{description}
watch, explain-backtraces, save-backtraces

\end{description}


 \\ 

\section*{ Structured Output }
\subsection*{ On Query }


 If learning is on: \begin{verbatim}
<result>
  <arg param="learnsetting" type="boolean">true</arg>
  <arg param="learnonlysetting" type="boolean">setting</arg>
  <arg param="learnexceptsetting" type="boolean">setting</arg>
  <arg param="learnalllevelssetting" type="boolean">setting</arg>
</result>

\end{verbatim}



 If learning is off: \begin{verbatim}
<result>
  <arg param="learnsetting" type="boolean">false</arg>
</result>

\end{verbatim}

\subsection*{ On List }


 When the list flag is issued, the results of a query are returned plus the following two arg tags: \begin{verbatim}
<arg param="learnforcelearnstates" type="string">string</arg>
<arg param="learndontlearnstates" type="string">string</arg>

\end{verbatim}

\subsection*{ Otherwise }
\begin{verbatim}
<result output="raw">true</result>

\end{verbatim}
\subsection*{ Notes }
\begin{itemize}
\item  Setting is true (learning enabled) or false (disabled). 
\item  learnalllevelssetting true means all-levels enabled, false means bottom-up 

\end{itemize}
\section*{ Error Values }
\subsection*{ During Parsing }


 kUnrecognizedOption, kGetOptError, kTooManyArgs
\subsection*{ During Execution }


 kAgentRequired Retrieved from ``\url{http://winter.eecs.umich.edu/soarwiki/Learn}``

\end{document}

\subsection{\soarb{max-chunks}}
\label{max-chunks}
\index{max-chunks}
Limit the number of chunks created during a decision cycle. 
\subsubsection*{Synopsis}
\begin{verbatim}
max-chunks [n]
\end{verbatim}
\subsubsection*{Options}
\begin{tabular}{|l|l|}
\hline
\soar{ n } & Maximum number of chunks allowed during a decision cycle.  \\
\hline
\end{tabular}
\subsubsection*{Description}
 The \textbf{max-chunks}
 command is used to limit the maximum number of chunks that may be created during a decision cycle. The initial value of this variable is 50; allowable settings are any integer greater than 0. 
 The chunking process will end after \textbf{max-chunks}
 chunks have been created, \emph{even if there are more results that have not been backtraced through to create chunks}
, and Soar will proceed to the next phase. A warning message is printed to notify the user that the limit has been reached. 
 This limit is included in Soar to prevent getting stuck in an infinite loop during the chunking process. This could conceivably happen because newly-built chunks may match immediately and are fired immediately when this happens; this can in turn lead to additional chunks being formed, etc. If you see this warning, something is seriously wrong; Soar is unable to guarantee consistency of its internal structures. You should not continue execution of the Soar program in this situation; stop and determine whether your program needs to build more chunks or whether you've discovered a bug (in your program or in Soar itself). 

\subsection{\soarb{max-elaborations}}
\label{max-elaborations}
\index{max-elaborations}
Limit the maximum number of elaboration cycles in a given phase. Print a warning message if the limit is reached during a run. 
\subsubsection*{Synopsis}
\begin{verbatim}
max-elaborations [n]
\end{verbatim}
\subsubsection*{Options}
\begin{tabular}{|l|l|}
\hline 
\emph{n}
 & Maximum allowed elaboration cycles, must be a positive integer.  \\
 \hline 
\end{tabular}
\subsubsection*{Description}
 This command sets and prints the maximum number of elaboration cycles allowed. If \emph{n}
 is given, it must be a positive integer and is used to reset the number of allowed elaboration cycles. The default value is 100. \textbf{max-elaborations}
 with no arguments prints the current value. 
 \textbf{max-elaborations}
 controls the maximum number of elaborations allowed in a single decision cycle. The elaboration phase will end after \emph{max-elaboration}
 cycles have completed, even if there are more productions eligible to fire or retract; and Soar will proceed to the next phase after a warning message is printed to notify the user. This limits the total number of cycles of parallel production firing but does not limit the total number of productions that can fire during elaboration. 
 This limit is included in Soar to prevent getting stuck in infinite loops (such as a production that repeatedly fires in one elaboration cycle and retracts in the next); if you see the warning message, it may be a signal that you have a bug your code. However some Soar programs are designed to require a large number of elaboration cycles, so rather than a bug, you may need to increase the value of \emph{max-elaborations}
. 
 In Soar8, \emph{max-elaborations}
 is checked during both the Propose Phase and the Apply Phase. If Soar8 runs more than the max-elaborations limit in either of these phases, Soar8 proceeds to the next phase (either Decision or Output) even if quiescence has not been reached. 
\subsubsection*{Examples}
 The command issued with no arguments, returns the max elaborations allowed: \begin{verbatim}
max-elaborations 
\end{verbatim}
 to set the maximum number of elaborations in one phase to 50: \begin{verbatim}
max-elaborations 50
\end{verbatim}

\subsection{\soarb{max-memory-usage}}
\label{max-memory-usage}
\index{max-memory-usage}
Set the amount of bytes necessary to trigger the memory usage exceeded event. 
\subsubsection*{Synopsis}
max-memory-usage [n]
\end{verbatim}
\subsubsection*{Options}
\hline
\soar{\soar{\soar{ n }}} & Size of limit in bytes.  \\
\hline
\end{tabular}
\subsubsection*{Description}
 The \textbf{max-memory-usage}
 command is used to trigger the memory usage exceeded event. The initial value of this is 100MB (100,000,000); allowable settings are any integer greater than 0. 
 Using the command with no arguments displays the current limit. 

\subsection{\soarb{max-nil-output-cycles}}
\label{max-nil-output-cycles}
\index{max-nil-output-cycles}
Limit the maximum number of decision cycles that are executed without producing output when run is invoked with run-til-output args. 
\subsubsection*{Synopsis}
\begin{verbatim}
max-nil-output-cycles [n]
\end{verbatim}
\subsubsection*{Options}
\begin{tabular}{|l|l|}
\hline
\emph{n}
 & Maximum number of consecutive output cycles allowed without producing output. Must be a positive integer.  \\
\hline
\end{tabular}
\subsubsection*{Description}
 This command sets and prints the maximum number of nil output cycles (output cycles that put nothing on the output link) allowed when running using run-til-output (run --output). If \emph{n}
 is not given, this command prints the current number of nil-output-cycles allowed. If \emph{n}
 is given, it must be a positive integer and is used to reset the maximum number of allowed nil output cycles. 
 \textbf{max-nil-output-cycles}
 controls the maximum number of output cycles that generate no output allowed when a \textbf{run --out}
 command is issued. After this limit has been reached, Soar stops. The default initial setting of \emph{n}
 is 15. 
\subsubsection*{Examples}
 The command issued with no arguments, returns the max empty output cycles allowed: \begin{verbatim}
max-nil-output-cycles 
\end{verbatim}
 to set the maximum number of empty output cycles in one phase to 25: \begin{verbatim}
max-nil-output-cycles 25 
\end{verbatim}
\subsubsection*{See Also}
\hyperref[run]{run}  Categories: Command Line Interface

\subsection{\soarb{multi-attributes}}
\label{multi-attributes}
\index{multi-attributes}
Declare a symbol to be multi-attributed. 
 Complete
\subsubsection*{Synopsis}
\begin{verbatim}
multi-attributes [symbol [\emph{n}
]] 
\end{verbatim}
\subsubsection*{Options}
\begin{tabular}{|l|l|}
\hline 
symbol & Any Soar attribute.  \\
 \hline 
\emph{n}
 & Integer $>$ 1, estimate of degree of simultaneous values for attribute.  \\
 \hline 
\end{tabular}
\subsubsection*{Description}
 This command declares the given symbol to be an attribute which can take on multiple values. The optional \emph{n}
 is an integer ($>$1) indicating an upper limit on the number of expected values that will appear for an attribute. If \emph{n}
 is not specified, the value 10 is used for each declared multi-attribute. More informed values will tend to result in greater efficiency. This command is used only to provide hints to the production condition reorderer so it can produce better condition orderings. Better orderings enable the rete network to run faster. This command has no effect on the actual contents of working memory and most users needn't use this at all. 
 Note that multi-attributes declarations must be made before productions are loaded into soar or this command will have no effect. 
\subsubsection*{Examples}
 Declare the symbol ``thing'' to be an attribute likely to take more than 1 but no more than 4 values: \begin{verbatim}
 multi-attributes thing 4 
\end{verbatim}

\subsection{\soarb{numeric-indifferent-mode}}
\label{numeric-indifferent-mode}
\index{numeric-indifferent-mode}
Select method for combining numeric preferences. 
 Status: Complete
\subsubsection*{Synopsis}
\begin{verbatim}
numeric-indifferent-mode [-as]
\end{verbatim}
\subsubsection*{Options}
\begin{tabular}{|l|l|}
\hline 
 -a, --avg, --average  & Use average mode (default).  \\
 \hline 
 -s, --sum  & Use sum mode.  \\
 \hline 
\end{tabular}
\subsubsection*{Description}
 The numeric-indifferent-mode command is used to select the method for combining numeric preferences. This command is only meaningful in indifferent-selection --random  mode. 
 The default procedure is \textbf{-avg}
 (average) which assigns a final value to an operator according to the rule: \begin{itemize}
\item  If the operator has at least one numeric preference, assign it the value that is the average of all of its numeric preferences. 
\item  If the operator has no numeric preferences (but has been included in the indifferent selection through some combination of non-numeric preferences), assign it the value 50. 
\end{itemize}
 The intended range of numeric-preference values for \textbf{-avg}
 mode is 0-100. 
 The other combination option \textbf{-sum}
 assigns a final value according to the rule: \begin{itemize}
\item  Add together any numeric preferences for the operator (defaulting to 0 if there are none). 
\item  Assign the operator the value \textbf{Failed to parse (Missing texvc executable; please see math/README to configure.): e\^{}\{PreferenceSum / AgentTemperature\}}
\end{itemize}
 , where AgentTemperature is a compile-time constant currently set at 25.0. 
 Any real-numbered preference may be used in \textbf{-sum}
 mode. 
 Once a value has been computed for each operator, the next operator is selected probabilistically, with each candidate operator's chance weighted by its computed value. 

\subsection{\soarb{o-support-mode}}
\label{o-support-mode}
\index{o-support-mode}
Choose experimental variations of o-support. 
\subsubsection*{Synopsis}
\begin{verbatim}
o-support-mode [0|1|2|3|4]
\end{verbatim}
\subsubsection*{Options}
\begin{tabular}{|l|l|}
\hline 
 0  & Mode 0 is the base mode. O-support is calculated based on the structure of working memory that is tested and modified. Testing an operator or operator acceptable preference results in state or operator augmentations being o-supported. The support computation is very complex (see soar manual).  \\
 \hline 
 1  & Not available through gSKI.  \\
 \hline 
 2  & Mode 2 is the same as mode 0 except that all support is calculated the production structure, not from working memory structure. Augmentations of operators are still o-supported.  \\
 \hline 
 3  & Mode 3 is the same as mode 2 except that operator elaborations (adding attributes to operators) now get i-support even though you have to test the operator to elaborate an operator.  \\
 \hline 
 4  & Mode 4 is the default.  \\
 \hline 
\end{tabular}
\subsubsection*{Description}
 The \textbf{o-support-mode}
 command is used to control the way that o-support is determined for preferences. Only o-support modes 3 \& 4 can be considered current to Soar8, and o-support mode 4 should be considered an improved version of mode 3. The default o-support mode is mode 4. 
 In o-support modes 3 \& 4, support is given production by production; that is, all preferences generated by the RHS of a single instantiated production will have the same support. The difference between the two modes is in how they handle productions with both operator and non-operator augmentations on the RHS. For more information on o-support calculations, see the relevant appendix in the Soar manual. 
 Running o-support-mode with no arguments prints out the current o-support-mode. 

\input{wikicmd/tex/predict}
\chapter{Reinforcement Learning}
\label{RL}
\index{reinforcement learning}
\index{preference!numeric-indifferent}
\index{rl}

Soar has a reinforcement learning (RL) mechanism that tunes operator selection knowledge based on a given reward function.
This chapter describes the RL mechanism and how it is integrated with production memory, the decision cycle, and the state stack.
We assume that the reader is familiar with basic reinforcement learning concepts and notation. If not, we recommend first reading \emph{Reinforcement Learning: An Introduction} (1998) by Richard S. Sutton and Andrew G. Barto.
The detailed behavior of the RL mechanism is determined by numerous parameters that can be controlled and configured via the \soarb{rl} command.
Please refer to the documentation for that command in section \ref{rl} on page \pageref{rl}.

\section{RL Rules}
\label{RL-rules}

Soar's RL mechanism learns Q-values for state-operator\footnote{In this context, the term ``state'' refers to the state of the task or environment, not a state identifier.
For the rest of this chapter, bold capital letter names such as \soarb{S1} will refer to identifiers and italic lowercase names such as $s_1$ will refer to task states.} pairs.
Q-values are stored as numeric indifferent preferences asserted by specially formulated productions called \emph{RL rules}.
RL rules are identified by syntax.
A production is a RL rule if and only if its left hand side tests for a proposed operator, its right hand side asserts a single numeric indifferent preference, and it is not a template rule (see \ref{RL-templates}).
These constraints ease the technical requirements of identifying/updating RL rules and makes it easy for the agent programmer to add/maintain RL capabilities within an agent.

The following is an RL rule:

\begin{verbatim}
sp {rl*3*12*left
   (state <s> ^name task-name
              ^x 3
              ^y 12
	          ^operator <o> +)
   (<o> ^name move
	    ^direction left)
-->
   (<s> ^operator <o> = 1.5)
}
\end{verbatim}

Note that the LHS of the rule can test for anything as long as it contains a test for a proposed operator.
The RHS is constrained to exactly one action: asserting a numeric indifferent preference for the proposed operator.

The following are not RL rules:

\begin{verbatim}
sp {multiple*preferences
   (state <s> ^operator <o> +)
-->
   (<s> ^operator <o> = 5, >)
}
\end{verbatim}  \vspace{12pt}

\begin{verbatim}
sp {variable*binding
    (state <s> ^operator <o> +
               ^value <v>)
-->
    (<s> ^operator <o> = <v>)
}
\end{verbatim}

The first rule proposes multiple preferences for the proposed operator and thus does not comply with the rule format.
The second rule does not comply because it does not provide a \emph{constant} for the numeric indifferent preference value.

In the typical RL use case, the agent should learn to choose the optimal operator in each possible state of the environment.
The most straightforward way to achieve this is to give the agent a set of RL rules, each matching exactly one possible state-operator pair.
This approach is equivalent to a table-based RL algorithm, where the Q-value of each state-operator pair corresponds to the numeric indifferent preference asserted by exactly one RL rule.

In the more general case, multiple RL rules can match a single state-operator pair, and a single RL rule can match multiple state-operator pairs.
Assuming that the value of \soarb{numeric-indifferent-mode} is set to \soarb{sum} (see page \pageref{numeric-indifferent-mode}), all numeric indifferent preferences for an operator are summed when calculating the operator's Q-value.
In this context, RL rules can be interpreted more generally as binary features in a linear approximator of each state-operator pair's Q-value, and their numeric indifferent preference values their weights.
In other words,
$$Q(s, a) = w_1 \phi_2 (s, a) + w_2 \phi_2 (s, a) + \ldots + w_n \phi_n (s, a)$$
where all RL rules in production memory are numbered $1 \dots n$, $Q(s, a)$ is the Q-value of the state-operator pair $(s, a)$, $w_i$ is the numeric indifferent preference value of RL rule $i$, $\phi_i (s, a) = 0$ if RL rule $i$ does not match $(s, a)$, and $\phi_i (s, a) = 1$ if it does.
This interpretation allows RL rules to simulate a number of popular function approximation schemes used in RL such as tile coding and sparse coarse coding.

\section{Reward Representation}
\label{RL-reward}

RL updates are driven by reward signals.
In Soar, these reward signals are fed to the RL mechanism through a working memory link called the \soarb{reward-link}.
Each state in Soar's goal stack is automatically populated with a \soarb{reward-link} structure upon creation.
Soar will check this structure for a numeric reward signal for the last operator executed in the associated state at the beginning of every decision phase.
Reward is also counted when the agent is halted or a substate is retracted.
% What happens when an agent with multiple states is halted? Do the rewards in the substates get counted?

In order to be recognized, the reward signal must follow this pattern:

\begin{verbatim}
(<r1> ^reward <r2>)
(<r2> ^value [val])
\end{verbatim}

where \verb=<r1>= is the \soarb{reward-link} identifier, \verb=<r2>= is some intermediate identifier, and \verb=[val]= is any constant numeric value.
Any structure that does not match this pattern are ignored.
If there are multiple matching WMEs, their values are summed into a single reward signal.

As an example, consider the following state:

\begin{verbatim}
(S1 ^reward-link R1)
  (R1 ^reward R2)
    (R2 ^value 1.0)
    (R2 ^source environment)
  (R1 ^reward R3)
    (R3 ^value -0.2)
    (R3 ^source intrinsic)
\end{verbatim}  

In this state, there are two reward signals with values 1.0 and -0.2.
They will be summed together for a total reward of 0.8 and this will be the value given to the RL update algorithm.
The \verb=(R2 ^source environment)= and \verb=(R3 ^source intrinsic)= WMEs are not counted as rewards or special in any way, but were added by the agent to keep track of where the rewards came from.

Note that the \soarb{reward-link} is not part of the \soarb{io} structure and is not modified directly by the environment.
Reward information from the environment should be copied, via rules, from the \soarb{input-link} to the \soarb{reward-link}.
Also note that when counting rewards, Soar simply scans the \soarb{reward-link} and sums the values of all valid reward WMEs.
The WMEs are not modified and no bookkeeping is done to keep track of previously seen WMEs.
This means that reward WMEs that exist for multiple decision cycles such as o-supported WMEs will be counted multiple times.

\section{Updating RL Rule Values}
\label{RL-algo}

Soar's RL mechanism is integrated naturally with the decision cycle and performs online updates of RL rules.
Whenever an operator supported by RL rules is selected, the values of those RL rules are updated.
The update can be on-policy (Sarsa) or off-policy (Q-Learning), as controlled by the \soarb{learning-policy} parameter of the \soarb{rl} command.
For Sarsa, the update is
$$ \delta_t = \alpha \left[ r_{t+1} + \gamma Q(s_{t+1}, a_{t+1}) - Q(s_t, a_t) \right] $$
where 
\begin{itemize}
\item $Q(s_t, a_t)$ is the Q-value of the state and chosen operator in decision cycle $t$.
\item $Q(s_{t+1}, a_{t+1})$ is the Q-value of the state and chosen operator in the next decision cycle.
\item $r_{t+1}$ is the total reward counted in the next decision cycle.
\item $\alpha$ and $\gamma$ are the settings of the \soarb{learning-rate} and \soarb{discount-rate} parameters of the \soarb{rl} command, respectively.
\end{itemize}

For Q-Learning, the update is
$$ \delta_t = \alpha \left[ r_{t+1} + \gamma \underset{a \in A_{t+1}}{\max} Q(s_{t+1}, a) - Q(s_t, a_t) \right] $$
where $A_{t+1}$ is the set of operators proposed in the next decision cycle.

Finally, $\delta_t$ is divided by the number of RL rules comprising the Q-value for the operator and the numeric indifferent values for each RL rule is updated by that amount.

An example walkthrough of a Sarsa update with $\alpha = 0.3$ and $\gamma = 0.9$ follows.

\begin{enumerate}

\item In decision cycle $t$, an operator \soarb{O1} is proposed, and RL rules \soarb{rl-1} and \soarb{rl-2} assert the following numeric indifferent preferences for it:
\begin{verbatim}
   rl-1: (S1 ^operator O1 = 2.3)
   rl-2: (S1 ^operator O1 =  -1)
\end{verbatim}  
	The Q-value for \soarb{O1} is $Q(s_t, \soarb{O1}) = 2.3 - 1 = 1.3$.
	 
\item \soarb{O1} is selected and executed, so $Q(s_t, a_t) = Q(s_t, \soarb{O1}) = 1.3$.

\item In decision cycle $t+1$, a total reward of 1.0 is counted on the \soarb{reward-link}, an operator \soarb{O2} is proposed, and another RL rule \soarb{rl-3} asserts the following numeric indifferent preference for it:
\begin{verbatim}
	rl-3: (S1 ^operator O2 = 0.5)
\end{verbatim}
	So $Q(s_{t+1}, \soarb{O2}) = 0.5$.

\item \soarb{O2} is selected, so $Q(s_{t+1}, a_{t+1}) = Q(s_{t+1}, \soarb{O2}) = 0.5$
	Therefore, 
	$$\delta_t = \alpha \left[r_{t+1} + \gamma Q(s_{t+1}, a_{t+1}) - Q(s_t, a_t) \right] = 0.3 \times [ 1.0 + 0.9 \times 0.5 - 1.3 ] = 0.045$$
	Since \soarb{rl-1} and \soarb{rl-2} both contributed to the Q-value of \soarb{O1}, $\delta_t$ is evenly divided amongst them, resulting in updated values of
\begin{verbatim}
   rl-1: (<s> ^operator <o> = 2.3225)
   rl-2: (<s> ^operator <o> = -0.9775)
\end{verbatim}

\end{enumerate}

\subsection{Gaps in Rule Coverage}
\label{RL-gaps}

Call an operator with numeric indifferent preferences an RL operator.
The previous description had assumed that RL operators were selected in both decision cycles $t$ and $t+1$.
If the operator selected in $t+1$ is not an RL operator, then $Q(s_{t+1}, a_{t+1})$ would not be defined, and an update for the RL operator selected at time $t$ will be undefined.
% This is true for Sarsa, but what about Q-Learning?
We will call a sequence of one or more decision cycles in which RL operators are not selected between two decision cycles in which RL operators are selected a \emph{gap}.
Conceptually, it is desirable to use the temporal difference information from the RL operator after the gap to update the Q-value of the RL operator before the gap.
There are just no intermediate storage locations for these updates.
Requiring that RL rules support operators at every decision can be difficult for agent programmers, particularly when operators are required that do not represent steps in a task, but instead perform generic maintenance functions, such as cleaning processed output-link structures.

To address this issue, Soar's RL mechanism supports automatic propagation of updates over gaps.
For a gap of length $n$, the Sarsa update is
$$\delta_t = \alpha \left[ \sum_{i=t}^{t+n}{\gamma^{i-t} r_i} + \gamma^{n+1} Q(s_{t+n+1}, a_{t+n+1}) - Q(s_t, a_t) \right]$$
and the Q-Learning update is
$$\delta_t = \alpha \left[ \sum_{i=t}^{t+n}{\gamma^{i-t} r_i} + \gamma^{n+1} \underset{a \in A_{t+n+1}}{\max} Q(s_{t+n+1}, a) - Q(s_t, a_t) \right]$$

Note that rewards will still be counted during the gap, but they are discounted based on the number of decisions removed they are from the initial RL operator.

Gap propagation can be disabled by setting the \soarb{temporal-extension} parameter of the \soarb{rl} command to \soarb{off}.
When gap propagation is disabled, the RL rules supporting an operator that is followed by a gap are simply not updated.
The \soarb{rl} setting of the \soarb{watch} command (see Section \ref{watch} on page \pageref{watch}) is useful in identifying gaps.


\subsection{RL and Substates}
\label{RL-substates}

When an agent has multiple states in its state stack, the RL mechanism will treat each substate independently.
As mentioned previously, each state has its own \soarb{reward-link}.
When an RL operator is selected in a state \soarb{S}, the RL updates for that operator are only affected by the rewards counted on the \soarb{reward-link} for \soarb{S} and the Q-values of subsequent RL operators selected in \soarb{S}.

The only exception to this independence is when a selected RL operator forces an operator-no-change impasse.
When this occurs, the number of decision cycles the RL operator at the superstate remains selected is dependent upon the processing in the impasse state.
Consider the operator trace in Figure \ref{fig:rl-optrace}.

\begin{itemize}
\item At decision cycle 1, RL operator \soarb{O1} is selected in \soarb{S1} and causes an operator-no-change impass for three decision cycles.
\item In the substate \soarb{S2}, operators \soarb{O2}, \soarb{O3}, and \soarb{O4} are selected and applied sequentially.
\item Meanwhile in \soarb{S1}, reward values $r_2$, $r_3$, and $r_4$ are put on the \soarb{reward-link} sequentially.
\item Finally, the impasse is resolved by \soarb{O4}, the proposal for \soarb{O1} is retracted, and RL operator \soarb{O5} is selected in \soarb{S1}.
\end{itemize}

\begin{figure}
\insertfigure{Figures/rl-optrace}{1.5in}
\insertcaption{Example Soar subgoal operator trace.}
\label{fig:rl-optrace}
\end{figure}

In this scenario, only the RL update for $Q(s_1, \soarb{O1})$ will be different from the ordinary case.
Its value depends on the setting of the \soarb{hrl-discount} parameter of the \soarb{rl} command.
When this parameter is set to the default value \soarb{on}, the rewards on \soarb{S1} and the Q-value of \soarb{O5} are discounted by the number of decision cycles they are removed from the selection of \soarb{O1}.
In this case the update for $Q(s_1, \soarb{O1})$ is
$$\delta_1 = \alpha \left[ r_2 + \gamma r_3 + \gamma^2 r_4 + \gamma^3 Q(s_5, \soarb{O5}) \right]$$
which is equivalent to having a three decision gap separating \soarb{O1} and \soarb{O5}.

When \soarb{hrl-discount} is set to \soarb{off}, the number of cycles \soarb{O1} has been impassed will be ignored.
Thus the update would be
$$\delta_1 = \alpha \left[ r_2 + r_3 + r_4 + \gamma Q(s_5, \soarb{O5}) \right]$$

For impasses other than operator no-change, RL acts as if the impasse hadn't occurred.
If \soarb{O1} is the last RL operator selected before the impasse, $r_2$ the reward received in the decision cycle immediately following, and \soarb{On} the first operator selected after the impasse, then \soarb{O1} is updated with 
$$\delta_1 = \alpha \left[ r_2 + \gamma Q(s_n, \soarb{On}) \right]$$

Soar's automatic subgoaling and RL mechanisms can be combined to naturally implement hierarchical reinforcement learning algorithms such as MAXQ and options.

\subsection{Eligibility Traces}
\label{RL-et}
The RL mechanism supports eligibility traces, which can improve the speed of learning by updating RL rules across multiple sequential steps.
The \soarb{eligibility-trace-decay-rate} and \soarb{eligibility-trace-tolerance} parameters control this mechanism.
By setting \soarb{eligibility-trace-decay-rate} to \soarb{0} (default), eligibility traces are in effect disabled.
When eligibility traces are enabled, the particular algorithm used is dependent upon the learning policy.
For Sarsa, the eligibility trace implementation is \emph{Sarsa($\lambda$)}. 
For Q-Learning, the eligibility trace implementation is \emph{Watkin's Q($\lambda$)}.

\subsubsection{Exploration}

When operator selection is decided on the basis of numeric preferences, the decision mechanism should usually choose the operator with the highest numeric preferences, that is, to exploit the present operator selection knowledge.
However, for reinforcement learning to discover the optimal policy, it is necessary that the agent sometimes choose an action that does not have the maximum predicted value.
Such exploration is necessary because actions may be undervalued.
This situation can occur both during the initial learning of a task and as a result of change in the dynamics or reward structure of a task.

The exploration policy is selected and configured using the \soarb{indifferent-selection} command (see Section \ref{indifferent-selection} on page \pageref{indifferent-selection}).
In an effort to maintain backwards compatibility, the default exploration policy is \soarb{softmax}.
However, the first time that the reinforcement learning mechanism is enabled, the architecture changes this policy to \soarb{episilon-greedy} (a more suitable default for RL agents) and issues a message to the trace.

\section{Automatic Generation of RL Rules}

The number of RL rules required for an agent to accurately approximate operator Q-values is usually infeasibly large to write by hand, even for small domains.
Therefore, several methods exist to automate this.

\subsection{The gp Command}
The \soar{gp} command can be used to generate productions based on simple patterns.
This is useful if the states and operators of the environment can be distinguished by a fixed number of dimensions with finite domains.
An example is a grid world where the states are described by integer row/column coordinates, and the available operators are to move north, south, east, or west.
In this case, a single \soar{gp} command will generate all necessary RL rules:
	
\begin{verbatim}
gp {gen*rl*rules
   (state <s> ^name gridworld
              ^operator <o> +
              ^row [ 1 2 3 4 ]
              ^col [ 1 2 3 4 ])
   (<o> ^name move
        ^direction [ north south east west ])
-->
   (<s> ^operator <o> = 0.0)
}
\end{verbatim}
	
For more information see the documentation for this command on page \pageref{gp}.

\subsection{Rule Templates}
\label{RL-templates}

Rule templates allow Soar to dynamically generate new RL rules based on a predefined pattern as the agent encounters novel states.
This is useful when either the domains of environment dimensions are not known ahead of time, or when the enumerable state space of the environment is too large to capture in its entirety using \soar{gp}, but the agent will only encounter a small fraction of that space during its execution.
For example, consider the grid world example with 1000 rows and columns.
Attempting to generate RL rules for each grid cell and action a priori will result in $1000 \times 1000 \times 4 = 4 \times 10^6$ productions.
However, if most of those cells are unreachable due to walls, then the agent will never fire or update most of those productions.
Templates give the programmer the convenience of the \soar{gp} command without filling production memory with unnecessary rules.

Rule templates have variables that are filled in to generate RL rules as the agent encounters novel combinations of variable values.
A rule template is valid if and only if it is marked with the \soarb{:template} flag and, in all other respects, adheres to the format of an RL rule.
However, whereas an RL rule may only use constants as the numeric indifference preference value, a rule template may use a variable.
Consider the following rule template:

\begin{verbatim}
sp {sample*rule*template
    :template
    (state <s> ^operator <o> +
               ^value <v>)
-->
    (<s> ^operator <o> = <v>)
}
\end{verbatim}

During agent execution, this rule template will match working memory and fire like any other rule.
However, the rule firing will not create the numeric indifferent preference on the RHS.
Instead, a new production is created by substituting all variables in the rule template that matched against constant values with the values themselves.
Suppose that the LHS of the rule template matched against the state

\begin{verbatim}
(S1 ^value 3.2)
(S1 ^operator O1 +)
\end{verbatim}

Then the following production will be added to production memory:

\begin{verbatim}
sp {rl*sample*rule*template*1
    (state <s> ^operator <o> +
               ^value 3.2)
-->
    (<s> ^operator <o> = 3.2)
}
\end{verbatim}

The variable \soar{<v>} is replaced by \soar{3.2} on both the LHS and the RHS, but \soar{<s>} and \soar{<o>} are not replaced because they matches against identifiers (\soar{S1} and \soar{O1}).
As with other RL rules, the value of \soar{3.2} on the RHS of this rule may be updated later by reinforcement learning, whereas the value of \soar{3.2} on the LHS will remain unchanged.
If \soar{<v>} had matched against a non-numeric constant, it will be replaced by that constant on the LHS, but the RHS numeric indifference preference value will be set to zero to make the new rule valid.

The new production's name adheres to the following pattern:
\soarb{rl*template-name*id}, where \soarb{template-name} is the name of the originating rule template and \soarb{id} is the smallest positive integer such that the new production's name is unique.

If an identical production already exists in production memory, then the newly generate production is discarded.
It should be noted that the current process of identifying unique template match instances can become quite expensive in long agent runs.
Therefore, it is recommended to generate all necessary RL rules using the \soar{gp} command or via custom scripting when possible.

\subsection{Chunking}
Since RL rules are regular productions, they can be learned by chunking just like any other production.
This method is more general than using the \soar{gp} command or rule templates, and is useful if the environment state consists of arbitrarily complex relational structures that cannot be enumerated.

\subsection{\soarb{save-backtraces}}
\label{save-backtraces}
\index{save-backtraces}
Save trace information to explain chunks and justifications. 
 Priority: 3; Status: Complete, EvilBackDoor
\subsubsection*{Synopsis}
\begin{verbatim}
save-backtraces [-ed]
\end{verbatim}
\subsubsection*{Options}
\begin{tabular}{|l|l|}
\hline 
 -e, --enable, --on  & Turn explain sysparam on.  \\
 \hline 
 -d, --disable, --off  & Turn explain sysparam off.  \\
 \hline 
\end{tabular}
\subsubsection*{Description}
, backtracing information can be retrieved by using the explain-backtraces command. Saving backtracing information may slow down the execution of your Soar program, but it can be a very useful tool in understanding how chunks are formed. 
\subsubsection*{See Also}
\hyperref[explain-backtraces]{explain-backtraces} 
\input{wikicmd/tex/select}
\subsection{\soarb{set-stop-phase}}
\label{set-stop-phase}
\index{set-stop-phase}
Controls the phase where agents stop when running by decision. 
\subsubsection*{Synopsis}
set-stop-phase -[ABadiop] 
\end{verbatim}
\subsubsection*{Options}
 Options -A and -B are optional and mutually exclusive. If not specified, the default is -B. 
 Only one of -a, -d, -i, -o, -p must be selected. 
 With no options, reports the current stop phase. 
\hline
\soar{\soar{ -A, --after }} & Stop after specified phase.  \\
\hline
\soar{\soar{ -B, --before }} & Stop before specified phase (the default).  \\
\hline
\soar{\soar{ -a, --apply }} & Select the apply phase.  \\
\hline
\soar{\soar{ -d, --decision }} & Select the decision phase.  \\
\hline
\soar{\soar{ -i, --input }} & Select the input phase.  \\
\hline
\soar{\soar{ -o, --output }} & Select the output phase.  \\
\hline
\soar{\soar{ -p, --proposal }} & Select the proposal phase.  \\
\hline
\end{tabular}
\subsubsection*{Description}
 When running by decision cycle it can be helpful to have agents stop at a particular point in its execution cycle. This command allows the user to control which phase Soar stops in. The precise definition is that \emph{running for $<$n$>$ decisions and stopping before phase $<$ph$>$}
 means to run until the decision cycle counter has increased by $<$n$>$ and then stop when the next phase is $<$ph$>$. The phase sequence (as of this writing) is: input, proposal, decision, apply, output. Stopping after one phase is exactly equivalent to stopping before the next phase. 
 On initialization Soar defaults to stopping before the input phase (or after the output phase, however you like to think of it). 
 Setting the stop phase applies to all agents. 
\subsubsection*{Examples}
set-stop-phase -Bi                 // stop before input phase
set-stop-phase -Ad                 // stop after decision phase (before apply phase)
set-stop-phase -d                  // stop before decision phase
set-stop-phase --after --output    // stop after output phase
set-stop-phase                     // reports the current stop phase
\end{verbatim}
\subsubsection*{See Also}

\subsection{\soarb{timers}}
\label{timers}
\index{timers}
Toggle on or off the internal timers used to profile Soar. 
 Status: Complete, EvilBackDoor
\subsubsection*{Synopsis}
\begin{verbatim}
timers [-ed]
\end{verbatim}
\subsubsection*{Options}
\begin{tabular}{|l|l|}
\hline 
 -d, --disable, --off  & Disable all timers.  \\
 \hline 
 -e, --enable, --on  & Enable timers as compiled.  \\
 \hline 
\end{tabular}
\subsubsection*{Description}
 This command is used to control the timers that collect internal profiling information while Soar is running. With no arguments, this command prints out the current timer status. Timers are ENABLED by default. The default compilation flags for soar enable the basic timers and disable the detailed timers. The timers command can only enable or disable timers that have already been enabled with compiler directives. See the stats command for more info on the Soar timing system. 
\subsubsection*{Examples}
 To show how to use the command in context, do this: \begin{verbatim}
command --option arg
\end{verbatim}
 and possibly explain the results. 
\subsubsection*{See Also}
 stats
\subsubsection*{Structured Output:}
\paragraph*{On Query}
\begin{verbatim}
<result>
  <arg name="timers" type="boolean">setting</arg>
</result>
\end{verbatim}
\paragraph*{Otherwise}
\begin{verbatim}
<result output="raw">true</result>
\end{verbatim}
\subsubsection*{Error Values:}
\paragraph*{During Parsing}
 kUnrecognizedOption, kGetOptError, kTooManyArgs
\paragraph*{During Execution}
 kAgentRequired, kKernelRequired

\subsection{\soarb{waitsnc}}
\label{waitsnc}
\index{waitsnc}
\subsubsection*{Synopsis}
\begin{verbatim}
wait -[e|d]
\end{verbatim}
\subsubsection*{Options}
\begin{tabular}{|l|l|}
\hline
\soar{ -e, --enable, --on } & Turns a state-no-change into a \emph{wait}
 state.  \\
\hline
\soar{ -d, --disable, --off } & Default. A state-no-change generates an impasse.  \\
\hline
\end{tabular}
\subsubsection*{Description}
 In some systems, espcially those that model expert (fully chunked) knowledge, a state-no-change may represent a \emph{wait state}
 rather than an impasse. The waitsnc command allows the user to switch to a mode where a state-no-change that would normally generate an impasse (and subgoaling), instead generates a \emph{wait}
 state. At a \emph{wait}
 state, the decision cycle will repeat (and the decision cycle count is incremented) but no state-no-change impasse (and therefore no substate) will be generated. 
 When issued with no arguments, waitsnc returns its current setting. 
 Categories: Command Line Interface


% ----------------------------------------------------------------------------

\section{File System I/O Commands}
\label{FILE-IO}

This section describes commands which interact in one way or another
with operating system input and output, or file I/O.  Users can
save/retrieve information to/from files, redirect the information
printed by Soar as it runs, and save and load the binary representation
of productions.
The specific commands described in this section are:

\paragraph{Summary}
\begin{quote}
\begin{description}
%\item[command-to-file] - Evaluate a command and print its results to a file.
%\item[\emph{directory functions}] - \soar{cd, dirs, popd, pushd, pwd}
\item[cd] - Change directory.
\item[clog] - Record all user-interface input and output to a file. \emph{(was \soar{log})}
\item[command-to-file] - Dump the printed output and results of a command to a file. 
\item[dirs] - List the directory stack.
\item[echo] -  Print a string to the current output device.
\item[ls] - List the contents of the current working directory.
\item[popd] - Pop the current working directory off the stack and change to the next directory on the stack.
\item[pushd] - Push a directory onto the directory stack, changing to it.
\item[pwd] - Print the current working directory.
\item[rete-net] - Save the current Rete net, or restore a previous one.
\item[set-library-location] - Set the top level directory containing demos/help/etc.
%\item[output-strings-destination] - Redirect the Soar output stream.
\item[source] - Load and evaluate the contents of a file.
\end{description}
\end{quote}

The \textbf{source} command is used for nearly every Soar program.  The
directory functions are important to understand so that users can
navigate directories/folders to load/save the files of interest.  
Soar applications that include a graphical interface or other
simulation environment will often require the use of \textbf{echo}  .


\documentclass[10pt]{article}
\usepackage{fullpage, graphicx, url}
\setlength{\parskip}{1ex}
\setlength{\parindent}{0ex}
\title{Cd - Soar Wiki}
\begin{document}
\section*{Cd}
\subsubsection*{From Soar Wiki}


 This is part of the Soar Command Line Interface. 
\section*{ Name }


 \textbf{cd}
 - Change directory. 


 Status: Complete
\section*{ Synopsis }
\begin{verbatim}
cd [directory]

\end{verbatim}
\section*{ Options }


\begin{tabular}{|c|c|}
\hline 
 directory  & The directory to change to, can be relative or full path.  \\
 \hline 

\end{tabular}



 \\ 

\section*{ Description }


 Change the current working directory. If run with no arguments, returns to the directory that the command line interface was started in, often referred to as the \emph{home}
 directory. 
\section*{ Examples }


 To move to the relative directory named ../home/soar/agents \begin{verbatim}
cd ../home/soar/agents


\end{verbatim}

\section*{ See Also }
\begin{description}
dirs home ls pushd popd source topd

\end{description}
\section*{ Structured Output }
\subsection*{ On Success }
\begin{verbatim}
<result output="raw">true</result>

\end{verbatim}
\subsection*{ Notes }
\section*{ Error Values }
\subsection*{ During Parsing }


 kTooManyArgs
\subsection*{ During Execution }


 kchdirFail Retrieved from ``\url{http://winter.eecs.umich.edu/soarwiki/Cd}``

\end{document}

\subsection{\soarb{clog}}
\label{clog}
\index{clog}
Record all user-interface input and output to a file. 
\subsubsection*{Synopsis}
\begin{verbatim}
clog -[Ae] filename
clog –a string
clog [–cdoq]
\end{verbatim}
\subsubsection*{Options}
\begin{tabular}{|l|l|}
\hline
\soar{ filename } & Open filename and begin logging.  \\
\hline
\soar{ -c, --close, -o, --off, -d, --disable } & Stop logging, close the file.  \\
\hline
\soar{ -a, --add string } & Add the given string to the open log file.  \\
\hline
\soar{ -q, --query } & Returns \emph{open}
 if logging is active or \emph{closed}
 if logging is not active.  \\
\hline
\soar{ -A, --append, -e, --existing } & Opens existing log file named filename and logging is added at the end of the file.  \\
\hline
\end{tabular}
\subsubsection*{Description}
 The \textbf{clog}
 command allows users to save all user-interface input and output to a file. When Soar is logging to a file, everything typed by the user and everything printed by Soar is written to the file (in addition to the screen). 
 Invoke \textbf{clog}
 with no arguments (or with \textbf{-q}
) to query the current logging status. Pass a filename to start logging to that file (relative to the command line interface's home directory (see the home command)). Use the \textbf{close}
 option to stop logging. 
\subsubsection*{Examples}
 To initiate logging and place the record in foo.log: \begin{verbatim}
clog foo.log
\end{verbatim}
 To append log data to an existing foo.log file: \begin{verbatim}
clog -A foo.log
\end{verbatim}
 To terminate logging and close the open log file: \begin{verbatim}
clog -c
\end{verbatim}
\subsubsection*{Known Issues}
 Does not log everything when structured output is selected. 
\subsubsection*{See also}
\hyperref[command-to-file]{command-to-file} 
\subsection{\soarb{command-to-file}}
\label{command-to-file}
\index{command-to-file}
Dump the printed output and results of a command to a file. 
\subsubsection*{Synopsis}
\begin{verbatim}
command-to-file [-a] filename command [args]
\end{verbatim}
\subsubsection*{Options}
\begin{tabular}{|l|l|}
\hline
\soar{ -a, --append } & Append if file exists.  \\
\hline
\soar{ filename } & The file to log the results of the command to  \\
\hline
\soar{ command } & The command to log  \\
\hline
\soar{ args } & Arguments for command  \\
\hline
\end{tabular}
\subsubsection*{Description}
 This command logs a single command. It is almost equivalent to opening a log using clog, running the command, then closing the log, the only difference is that input isn't recorded. 
 Running this command while a log is open is an error. There is currently not support for multiple logs in the command line interface, and this would be an instance of multiple logs. 
 This command echos output both to the screen and to a file, just like clog. 
\subsubsection*{See also}
\hyperref[clog]{clog}  Categories: Command Line Interface

\subsection{\soarb{dirs}}
\label{dirs}
\index{dirs}
List the directory stack 
\subsubsection*{Synopsis}
\begin{verbatim}
dirs
\end{verbatim}
\subsubsection*{Options}
 No options. 
\subsubsection*{Description}
 This command lists the directory stack. Agents can move through a directory structure by pushing and popping directory names. The \textbf{dirs}
 command returns the stack. 
 The command \textbf{pushd}
 places a new ``agent current directory'' on top of the directory stack and cd's to it. The command \textbf{popd}
 removes the directory at the top of the directory stack and cd's to the previous directory which now appears at the top of the stack. 
\subsubsection*{See Also}
\hyperref[cd]{cd} \hyperref[home]{home} \hyperref[ls]{ls} \hyperref[pushd]{pushd} \hyperref[popd]{popd} \hyperref[source]{source} \hyperref[topd]{topd}  Categories: Command Line Interface

\documentclass[10pt]{article}
\usepackage{fullpage, graphicx, url}
\title{Echo - Soar Wiki}
\begin{document}
\section*{Echo}
\subsubsection*{From Soar Wiki}


 This is part of the Soar Command Line Interface. 
\section*{ Name }


 \textbf{echo}
 - Print a string to the current output device. 


 Status: Complete
\section*{ Synopsis }
\begin{verbatim}
echo string

\end{verbatim}
\section*{ Options }


\begin{tabular}{|p{1in}|p{5in}|}
\hline 
 string  & The string to print.  \\
 \hline 

\end{tabular}



 \\ 

\section*{ Description }


 This command echos the args to the current output stream. This is normally stdout but can be set to a variety of channels. If an arg is -nonewline then no newline is printed at the end of the printed strings. Otherwise a newline is printed after printing all the given args. Echo is the easiest way to add user comments or identification strings in a log file. 
\section*{ Examples }


 This example will add these comments to the screen and any open log file. \begin{verbatim}
echo This is the first run with disks = 12

\end{verbatim}

\section*{ See Also }
\begin{description}
log

\end{description}


 \\ 

\section*{ Structured Output }
\subsection*{ On Success }
\begin{verbatim}
<result>
  <arg param="message" type="string">message</arg>
</result>

\end{verbatim}
\subsection*{ Notes }
\section*{ Error Values }
\subsection*{ During Parsing }
\subsection*{ During Execution }


 No errors. 

\end{document}

\subsection{\soarb{ls}}
\label{ls}
\index{ls}
List the contents of the current working directory. 
\subsubsection*{Synopsis}
\begin{verbatim}
ls
\end{verbatim}
\subsubsection*{Options}
 No options. 
\subsubsection*{Description}
 List the contents of the working directory. 
\subsubsection*{Default Aliases}
\begin{tabular}{|l|l|}
\hline
\soar{ Alias } & Maps to  \\
\hline
\soar{ dir } & ls  \\
\hline
\end{tabular}
\subsubsection*{See Also}
\hyperref[cd]{cd} \hyperref[dirs]{dirs} \hyperref[home]{home} \hyperref[pushd]{pushd} \hyperref[popd]{popd} \hyperref[source]{source} \hyperref[topd]{topd}  Categories: Command Line Interface

\documentclass[10pt]{article}
\usepackage{fullpage, graphicx, url}
\title{Popd - Soar Wiki}
\begin{document}
\section*{Popd}
\subsubsection*{From Soar Wiki}


 This is part of the Soar Command Line Interface. 
\section*{ Name }


 \textbf{popd}
 - Pop the current working directory off the stack and change to the next directory on the stack. Can be relative pathname or fully specified path. 


 Status: Complete
\section*{ Synopsis }
\begin{verbatim}
popd

\end{verbatim}
\section*{ Options }


 No options. 
\section*{ Description }
\section*{ Soar8.3 Description }


 This command pops a directory off of the directory stack and cd's to it. See the dirs command for an explanation of the directory stack. 
\section*{ Examples }


 \\ 

\section*{ See Also }
\begin{description}
cd dirs home ls pushd \textbf{popd}
 source topd

\end{description}
\section*{ Structured Output }
\subsection*{ On Success }
\begin{verbatim}
<result output="raw">true</result>

\end{verbatim}
\section*{ Error Values }
\subsection*{ During Parsing }


 kTooManyArgs
\subsection*{ During Execution }


 kDirectoryStackEmpty

\end{document}

\subsection{\soarb{pushd}}
\label{pushd}
\index{pushd}
Push a directory onto the directory stack, changing to it. 
 Complete
\subsubsection*{Synopsis}
\begin{verbatim}
pushd directory
\end{verbatim}
\subsubsection*{Options}
\begin{tabular}{|l|l|}
\hline 
 directory  & Directory to change to, saving the current directory on to the stack.  \\
 \hline 
\end{tabular}
\subsubsection*{Description}
 Maintain a stack of working directories and push the directory on to the stack. Can be relative path name or fully specified. 
\subsubsection*{See Also}
\hyperref[cd]{cd} \hyperref[dirs]{dirs} \hyperref[home]{home} \hyperref[ls]{ls} \hyperref[popd]{popd} \hyperref[source]{source} \hyperref[topd]{topd} 
\subsection{\soarb{pwd}}
\label{pwd}
\index{pwd}
Print the current working directory. 
\subsubsection*{Synopsis}
pwd
\end{verbatim}
\subsubsection*{Options}
 No options. 
\subsubsection*{Description}
 Prints the current working directory of Soar. 
\subsubsection*{Default Aliases}
\hline
\soar{\soar{ Alias }} & Maps to  \\
\hline
\soar{\soar{ topd }} & pwd  \\
\hline
\end{tabular}

\subsection{\soarb{rete-net}}
\label{rete-net}
\index{rete-net}
Save the current Rete net, or restore a previous one. 
 Status: Complete
\subsubsection*{Synopsis}
\begin{verbatim}
rete-net -s|l filename
\end{verbatim}
\subsubsection*{Options}
\begin{tabular}{|l|l|}
\hline 
 -s, --save  & Save the Rete net in the named file. Cannot be saved when there are justifications present. Use excise -j \\
 \hline 
 -l, -r, --load, --restore  & Load the named file into the Rete network. working memory and production memory must both be empty. Use excise -a \\
 \hline 
filename & The name of the file to save or load.  \\
 \hline 
\end{tabular}
\subsubsection*{Description}
 The rete-net command saves the current Rete net to a file or restores a Rete net previously saved. The Rete net is Soar's internal representation of production and working memory; the conditions of productions are reordered and common substructures are shared across different productions. This command provides a fast method of saving and loading productions since a special format is used and no parsing is necessary. Rete-net files are portable across platforms that support Soar. 
 Normally users wish to save only production memory. Note that \emph{justifications}
 cannot be present when saving the Rete net. Issuing an init-soar before saving a Rete net will remove all justifications and working memory elements. \\ 
 If the filename contains a suffix of ``.Z'', then the file is compressed automatically when it is saved and uncompressed when it is loaded. Compressed files may not be portable to another platform if that platform does not support the same uncompress utility. 
\subsubsection*{See Also}
\hyperref[excise]{excise} \hyperref[init-soar]{init-soar} 
\documentclass[10pt]{article}
\usepackage{fullpage, graphicx, url}
\title{Set-library-location - Soar Wiki}
\begin{document}
\section*{Set-library-location}
\subsubsection*{From Soar Wiki}


 This is part of the Soar Command Line Interface. 
\section*{ Name }


 \textbf{set-library-location}
 - Set the top level directory containing demos/help/etc.\\ 
 Status: Complete
\section*{ Synopsis }
\begin{verbatim}
set-library-location [directory] 

\end{verbatim}
\section*{ Options }


\begin{tabular}{|p{1in}|p{5in}|}
\hline 
 directory  & The new desired library location.  \\
 \hline 

\end{tabular}



 \\ 

\section*{ Description }


 Invoke with no arguments to query what the current library location is. The library location should contain at least the help/ subdirectory and the command-names file for help to work. 
\section*{ Examples }
\begin{verbatim}
directory "c:\Documents and Settings\User\My Documents\Soar\SoarIO\bin"
directory /usr/local/share/soar/library

\end{verbatim}
\section*{ See Also }


 help
\section*{ Structured Output }
\subsection*{ On Success }


 Returns true on success, or kParamDirectory, kTypeString element with the directory on query. 
\section*{ Error Values }
\subsection*{ During Parsing }


 kTooManyArgs
\subsection*{ During Execution }


 No errors. 

\end{document}

\subsection{\soarb{source}}
\label{source}
\index{source}
Load and evaluate the contents of a file. 
 Status: Complete
\subsubsection*{Synopsis}
\begin{verbatim}
source filename
\end{verbatim}
\subsubsection*{Options}
\begin{tabular}{|l|l|}
\hline 
filename & The file of Soar productions and commands to load.  \\
 \hline 
\end{tabular}
\subsubsection*{Description}
 Load and evaluate the contents of a file. The \emph{filename}
\subsubsection*{See Also}


% ***************************************************************************
% ----------------------------------------------------------------------------
\section{Soar I/O Commands}
\label{SOAR-IO}

This section describes the commands used to manage Soar's Input/Output
(I/O) system, which provides a mechanism for allowing Soar to interact 
with external systems, such as a computer game environment or a robot.  

Soar I/O functions make calls to \soar{add-wme} and \soar{remove-wme}
to add and remove elements to the \textbf{io} structure of Soar's working
memory. 
 
The specific commands described in this section are:

\paragraph{Summary}
\begin{quote}
\begin{description}
\item[add-wme] - Manually add an element to working memory.
\item[remove-wme] - Manually remove an element from working memory.
\end{description}
\end{quote}

These commands are used mainly  when Soar needs to interact with an
external environment.  Users might take advantage of these commands when
debugging agents, but care should be used in adding and removing wmes this
way as they do not fall under Soar's truth maintenance system.

\subsection{\soarb{add-wme}}
\label{add-wme}
\index{add-wme}
Manually add an element to working memory. 
 Status: Complete, EvilBackDoor
\subsubsection*{Synopsis}
  \begin{verbatim}
add-wme id [^]attribute value [+]
\end{verbatim}
\subsubsection*{Options}
\begin{tabular}{|l|l|}
\hline 
 id  & Must be an existing identifier.  \\
 \hline 
 \^{}  & Leading \^{} on attribute is optional.  \\
 \hline 
 attribute  & Attribute can be any Soar symbol. Use * to have Soar create a new identifier.  \\
 \hline 
 value  & Value can be any soar symbol. Use * to have Soar create a new identifier.  \\
 \hline 
 +  & If the optional preference is specified, its value must be + (acceptable).  \\
 \hline 
\end{tabular}
\subsubsection*{Description}
\subsubsection*{Examples}
 This example adds the attribute/value pair ``message-status received'' to the identifier (symbol) S1: \begin{verbatim}
 add-wme S1 ^message-status received
\end{verbatim}
 This example adds an attribute/value pair with an acceptable preference to the identifier (symbol) Z2. The attribute is ``message'' and the value is a unique identifier generated by Soar. Note that since the \^{} is optional, it has been left off in this case. \begin{verbatim}
 add-wme Z2 message * + 
\end{verbatim}
\subsubsection*{Warnings}
\subsubsection*{See Also}
\hyperref[remove-wme]{remove-wme} 
\subsection{\soarb{remove-wme}}
\label{remove-wme}
\index{remove-wme}
Manually remove an element from working memory. 
 Complete
\subsubsection*{Synopsis}
\begin{verbatim}
remove-wme \emph{timetag}
\end{verbatim}
\subsubsection*{Options}
\begin{tabular}{|l|l|}
\hline 
 timetag  & A positive integer matching the timetag of an existing working memory element.  \\
 \hline 
\end{tabular}
\subsubsection*{Description}
 The remove-wme command removes the working memory element with the given timetag. This command is provided primarily for use in Soar input functions; although there is no programming enforcement, remove-wme should only be called from registered input functions to delete working memory elements on Soar's input link. 
 Beware of weird side effects, including system crashes. 
\subsubsection*{See Also}
\hyperref[add-wme]{add-wme} \subsubsection*{Warnings}
 remove-wme should never be called from the RHS: if you try to match a wme on the LHS of a production, and then remove the matched wme on the RHS, Soar will crash. 
 If used other than by input and output functions interfaced with Soar, this command may have weird side effects (possibly even including system crashes). Removing input wmes or context/impasse wmes may have unexpected side effects. You've been warned. 


% ***************************************************************************
% ----------------------------------------------------------------------------
\section{Miscellaneous}
\label{MISC}


\comment{this section still needs to be rewritten...}

\nocomment{This section describes the commands used to inspect production memory,
working memory, and preference memory; to see what productions will 
match and fire in the next Propose or Apply phase;  and to examine the 
goal dependency set.  These commands are particularly useful when
running or debugging Soar, as they let users see what Soar is ``thinking.''}
The specific commands described in this section are:


\paragraph{Summary}
\begin{quote}
\begin{description}
\item[alias] - Define a new alias, or command, using existing commands and arguments. 
\item[edit-production] - Fire event to Move focus in an open editor to this production.
%\item  Default Rules
%\item  Predefined Aliases 
%\item  The soar.tcl file
\item[soarnews] - Prints information about the current release.
\item[srand] -  Seed the random number generator.
%\item  Soar Variables
\item[time] - Uses a default system clock timer to record the wall time required while executing a command.
\item[unalias] - Remove an existing alias.
\item[version] - Returns version number of Soar kernel.
\end{description}
\end{quote}

\subsection{\soarb{alias}}
\label{alias}
\index{alias}
Define a new alias, or command, using existing commands and arguments. 
\subsubsection*{Synopsis}
\begin{verbatim}
alias name [cmd <args>]
alias -d name
alias
\end{verbatim}
\subsubsection*{Options}
\begin{tabular}{|l|l|}
\hline
\soar{ -d, --disable, --off } & Remove the named alias.  \\
\hline
\soar{ name } & The name of the alias, i.e. the new command.  \\
\hline
\soar{ cmd } & An existing command that will be invoked when the alias is entered on the commandline.  \\
\hline
\soar{ args } & Valid arguments to the cmd (optional \& optional number).  \\
\hline
\end{tabular}
\subsubsection*{Description}
 This command defines new aliases by creating Soar procedures with the given name. The new procedure can then take an arbitrary number of arguments which are post-pended to the given definition and then that entire string is executed as a command. The definition must be a single command, multiple commands are not allowed. The \textbf{alias}
 procedure checks to see if the name already exists, and does not destroy existing procedures or aliases by the same name. Existing aliases can be removed by using the \textbf{-d}
 flag. With no arguments, \textbf{alias}
 returns the list of defined aliases. With only the name given, \textbf{alias}
 returns the current definition. 
\subsubsection*{Examples}
 The alias \emph{wmes}
 is defined as: \begin{verbatim}
alias wmes print -i
\end{verbatim}
 If the user executes a command such as: \begin{verbatim}
wmes {(* ^superstate nil)}
\end{verbatim}
 it is as if the user had typed this command: \begin{verbatim}
print -i {(* ^superstate nil)}
\end{verbatim}
 To check what a specific alias is defined as, you would type \begin{verbatim}
alias wmes
\end{verbatim}
\subsubsection*{Default Aliases}
\begin{tabular}{|l|l|}
\hline
\soar{ Alias } & Maps to  \\
\hline
\soar{ a } & alias  \\
\hline
\soar{ un } & alias -d  \\
\hline
\soar{ unalias } & alias -d  \\
\hline
\end{tabular}
\subsubsection*{See Also}
\hyperref[unalias]{unalias} 
\subsection{\soarb{edit-production}}
\label{edit-production}
\index{edit-production}
Move focus in an editor to this production. 
\subsubsection*{Synopsis}
edit-production production_name
\end{verbatim}
\subsubsection*{Options}
 production\_name The name of the production to edit. 
\subsubsection*{Description}
 If an editor (currently limited to Visual Soar) is open and connected to Soar, this command causes the editor to open the file containing this production and move the cursor to the start of the production. If there is no editor connected to Soar, the command does nothing. In order to connect Visual Soar to Soar, launch Visual Soar and choose Connect from the Soar Runtime menu. Then open the Visual Soar project that you're working on. At that point, you're set up and edit-production will start to work. 
\subsubsection*{Examples}
edit-production my*production*name
\end{verbatim}
\subsubsection*{See Also}
\hyperref[sp]{sp} 
\subsection{\soarb{srand}}
\label{srand}
\index{srand}
Seed the random number generator. 
\subsubsection*{Synopsis}
srand [seed]
\end{verbatim}
\subsubsection*{Options}
\hline
\soar{\soar{\soar{ seed }}} & Random number generator seed.  \\
\hline
\end{tabular}
\subsubsection*{Description}
 Seeds the random number generator with the passed seed. Calling srand without providing a seed will seed the generator based on the contents of /dev/urandom (if available) or else based on time() and clock() values. 
\subsubsection*{Examples}
srand 0
\end{verbatim}
\subsubsection*{See Also}

\subsection{\soarb{soarnews}}
\label{soarnews}
\index{soarnews}
 Priority: 4; Status: Incomplete\\ 
Decide what to put in the news and where to read it from.--Jonathan 15:42, 23 Feb 2005 (EST) 
\subsubsection*{Synopsis}
\subsubsection*{Options}
\begin{tabular}{|l|l|}
\hline 
 & \\
 \hline 
 & \\
 \hline 
\end{tabular}
\subsubsection*{Description}

\subsection{\soarb{time}}
\label{time}
\index{time}
Use a default system clock timer to record the wall time required while executing a command. 
\subsubsection*{Synopsis}
time command [arguments]
\end{verbatim}
\subsubsection*{Options}
\hline
\soar{\soar{ command }} & The command to execute.  \\
\hline
\soar{\soar{ arguments }} & Optional command arguments.  \\
\hline
\end{tabular}
\subsubsection*{Description}

\subsection{\soarb{unalias}}
\label{unalias}
\index{unalias}
Undefine an existing alias 
\subsubsection*{Synopsis}
unalias name
\end{verbatim}
\subsubsection*{Options}
 No options. 
\subsubsection*{Description}
 This command undefines a previously created alias. This command takes exactly one argument: the name of the alias to remove. Use the alias command by itself to list all defined aliases. 
\subsubsection*{Examples}
unalias varprint
\end{verbatim}
\subsubsection*{Default Aliases}
\hline
\soar{\soar{ Alias }} & Maps to  \\
\hline
\soar{\soar{un}} &\textbf{unalias}
\hline
\end{tabular}
\subsubsection*{See Also}
\hyperref[alias]{alias} 
\subsection{\soarb{version}}
\label{version}
\index{version}
\subsubsection*{Synopsis}
\begin{verbatim}
 version
\end{verbatim}
\subsubsection*{Options}
 No options 
\subsubsection*{Description}
 This command gives version information about the current Soar kernel. It returns the version number and build date which can then be stored by the agent or the application. 
 Categories: Command Line Interface



% ****************************************************************************
% ----------------------------------------------------------------------------
% ****************************************************************************
% ****************************************************************************
% ****************************************************************************

\nocomment{

\subsection{Starting Soar}
\funsum{soar}{Starts Soar.}
\label{soar}
\index{soar}

To start Soar, you'll first have to find out where the Soar program is kept at
your site. Then \soar{cd} to the appropriate directory and type \soar{soar},
or specify the full pathname from your current directory.

\paragraph{Example}
\begin{verbatim}
unix% cd ~soar/soar-current
unix% soar
7.0.3. TCL TK

Bugs and questions should be sent to soar-bugs@cs.cmu.edu
The current bug-list may be obtained by sending mail to
soarhack@cs.cmu.edu with the Subject: line "bug list".

This software is in the public domain, and is made available AS IS.
Carnegie Mellon University, The University of Michigan, and
The University of Southern California/Information Sciences Institute
make no warranties about the software or its performance, implied
or otherwise.

Type "help" for information on various topics.
Type "quit" to exit.  Use ctrl-c to stop a Soar run.
Type "soarnews" for news.
Type "version" for complete version information.

soar> 
\end{verbatim} 

The Soar prompt (\soar{soar>}) indicates that you are running Soar and may
issue the commands documented in this chapter. 

\subsubsection*{Notes}

If you have problems starting Soar, it may be that your environment variables
are not set; see Section \ref{INTERFACE-tcl} for suggestions.

There are several optional arguments that can be given to the \soar{soar}
command, such as to start up multiple agents or to control windows. These are
considered advanced usage, and described in the \emph{Soar Advanced
Applications Manual}.

\comment{Karen: ``or set the path to look for startup files''

	K says to read help soartk -path option for better info}

% ----------------------------------------------------------------------------
\subsection{Files automatically loaded at startup}
\label{INTERFACE-files}

\comment{I have a note from Karen that 2 and 3 are switched for 7.0.3, but
  	that .soarrc will come before soar.soar for 7.1. Check}

\comment{smooth out a bit so that it's also appropriate for multiple agents} 

There are three files that may be loaded when Soar is first started up, if
they exist:\vspace{-12pt}
\begin{enumerate}
\item The \soar{\$soar\_library/soar.tcl} file, for your local Soar
	installation.\vspace{-6pt}
\item A \soar{.soarrc} file, in your home directory.\vspace{-6pt}
\item A \soar{soar.soar} file either in the current directory, or in your home
	directory. (Soar will check the current directory first, and load the
	first \soar{soar.soar} file it finds.)
\end{enumerate}

The files will be loaded in the order listed.\footnote{There are also some Tcl
and Tk files that are sourced at startup, but these are system files that
individual users cannot change.}

Note that the \soar{soar.soar} file is more generally
\soarit{name}\soar{.soar}, as described below.


\subsubsection*{The \soarb{\$soar\_library/soar.tcl} file}

The \soar{\$soar\_library/soar.tcl} file is loaded for all users at a local
site, and can be reconfigured by the local Soar administrator. This file is used
to load local aliases; it also contains the Tcl code that implements many of
the user-interface functions. It is also the appropriate place for
platform-dependent code.

Individual Soar users have no control over this file.


\subsubsection*{The \soarb{.soarrc} file}

	\betacomment{I'm not sure what \$HOME means for non-Unix users.}

Soar will check for a \soar{.soarrc} file in your home directory (as defined
in your \soar{\$HOME} variable), and load the file if it exists. This works
the same as if you had typed ``\soar{source .soarrc}'' at the prompt.

The \soarb{.soarrc} file is used to load personal aliases and Tcl code that
the user wants to use for all Soar applications. When multiple agents and
interpreters are used, they will all \soar{source} this file.

\comment{Note from K: currently loaded only once at startup. On the list to be
	fixed for 7.1

	I'm not sure what she means -- that it doesn't or didn't work for
	multiple agents in 7.0.4?}


\subsubsection*{The \soarb{soar.soar} file}

Soar will check for a \soar{soar.soar} file first in the current directory,
and then in your home directory (as defined in your \soar{\$HOME} variable),
and load the first \soar{soar.soar} file it finds. This is executed 
as if you had typed ``\soar{source soar.soar}'' at the Soar prompt.

	\comment{Once again, I'm not sure how \$HOME is resolved on non-Unix
	platforms.}

The \soar{soar.soar} file will typically contain other \soar{source} commands
(see Section \ref{source} below). For example, this file might be used to
automatically load in a set of individual alias definitions (see the
\soar{alias} command in Section \ref{alias}), to automatically load the
default rules (see Section \ref{default}), or even to automatically load in
the productions for a task.

When running multiple agents and interpreters, it is important to know that
the \soar{soar.soar} initialization file is generically
\soar{$<$agent-name$>$.soar}, which will be different for each agent. For
single-agent soar, the default agent is called \soar{soar}. The use of
multiple agents and interpreters is described in Chapter \ref{ADVANCED}, and
more thoroughly in the \textit{Soar Advanced Applications Manual}.


% ----------------------------------------------------------------------------
\subsection{Comments and Caveats about Tcl}
\label{INTERFACE-tcl}

The addition of Tcl to Soar provides additional functionality to the Soar
interface, but also imposes some restrictions on syntax. This section
describes some of the features and potential problems you may encounter.


\subsubsection*{Environment variables}

There are several environment variables that must be set in your operating
system for Soar to run. Many sites will have scripts installed so that the
Soar user need not set these variables explicitly. However, if you have
problems starting up Soar, you may wish to check these as a first course of
action. See the file \soar{SOAR\_INSTALLATION\_SETTINGS} (in the directory in
which Soar was built), which should contain a listing of the environment
variables and their correct settings.

\comment{Again, it's not clear what the equivalent is for non-Unix
	installations.} 

If you cannot locate this file, consult the person who installed Soar at your
site, or send mail to \soar{soar-bugs@cs.cmu.edu} to ask for help.

\subsubsection*{The Tcl interpreter}

All commands entered at the Soar prompt pass through the Tcl interpreter. This
has several implications for Soar users:\vspace{-12pt}
\begin{enumerate}
\item Commands entered at the Soar prompt may be Soar commands, Tcl commands,
	or operating system commands (the latter are specific to the machine
	you're running on). Soar attempts to execute a command first as a Soar
	command, second as a Tcl command, and third as an operating system
	command; in the rare case of commands that have the same name, the
	first system that can execute the command will do so.\vspace{-6pt}

\item Since Tcl is case sensitive, Soar is also case sensitive. This means, 
	for example, that ``red'', ``RED'', and ``Red'' are three different
	symbols to Soar. The one exception to case-sensitivity in Soar is
	identifiers; internally, these begin with an uppercase letter, but you
	may type identifiers with a lowercase letter also, and Soar will
	resolve the identifier correctly.\vspace{-6pt}

\item Tcl has a command-completion facility, which allows the user to type a
	partial command name in lieu of the full command name when the
	substring typed is long enough to distinguish it from other commands.

        This facility is often helpful, but may be confusing to some users: If
	the substring typed is not long enough to distinguish it from other
	commands, a listing of all the commands that may have been intended is
	printed. This listing will contain Tcl commands that may be unfamiliar
	to most Soar users.\vspace{-6pt}

\item Tcl allows variations in syntax for any command that uses
	curly braces to delimit arguments, and requires this variation for
	some specific uses. Most Soar users can proceed in ignorance of this
	variation, but it is mentioned here for completeness.

	The Soar commands affected by this variation are \soar{sp}, \soar{production-find},
	\soar{print}, \soar{wmes}, \soar{alias}, \soar{echo}, and
	\soar{command-to-file}; when these commands include Tcl variables,
	they must use double quotes as delimiters, rather than the curly
	braces.

	Using double quotes tells the Tcl interpreter to parse the string
	within the delimiters and resolve the variable reference immmediately.
	When curly braces are used, Tcl does not parse the enclosed string and
	the variable reference is not resolved; the string is passed on to the
	command for execution. Usually, the latter is not the desired
	behavior when Tcl variables are used, though in some cases, it may be
	desirable.

	The upshot of this is that in almost any command that uses curly
	braces as delimiters, double quotes may safely be substituted for the
	curly braces. However, it is not as safe to switch from double quotes
	to curly braces.
\end{enumerate}

\subsubsection*{Tcl error messages}

One unfortunate consequence of commands passing through the Tcl interpreter is
that Tcl does not print all of its error messages to the screen. In rare
instances, you'll be able to find an error message only by inspecting the Tcl
variable \soar{errorInfo}. To do this, you must use the Tcl command \soar{set}
(which both displays and sets the values of variables):

\begin{verbatim}
soar> help-me
invalid command name "help-me"
soar> set errorInfo
invalid command name "help-me"
    while executing
"help-me"
soar> 
\end{verbatim}

Should you need to check this variable, be sure to note that the letter
\soar{I} is capitalized in \soar{errorInfo}, and that it is the only
capitalized letter in the variable name. 

The \soar{errorInfo} variable saves only the most recent error message; there
is no way to recover previous error messages. Note that if you make a
typographical error when you enter ``\soar{set errorInfo}'', your new last
error will be the typo, and you won't know what the problem was. (It may make
sense to \soar{alias} ``set errorInfo'' to something easier to type
accurately.)


% ----------------------------------------------------------------------------
\subsection{Conventions for entering commands}

Input to Soar is usually typed in at the Soar prompt, e.g.:
\begin{verbatim}
soar> print s1
\end{verbatim}

Multiple commands can be given on the same line as long as they are separated
by semicolons, e.g.:

\begin{verbatim}
soar> run 3 d; print s1
\end{verbatim}

Multiple commands on the same line will be called in the order listed on the
line, with each being executed to completion before the next is executed. For
example, if one of the commands starts executing Soar productions, the
following commands will not be executed until Soar halts.


While Soar is executing productions (often called \emph{running}), it can be
interrupted by typing the break character, which is usually \soar{control-C}.
This will cause Soar to stop at the end of the current elaboration cycle and
return to a Soar prompt. (This only works while Soar is running, and not while
it's loading productions or waiting for you to type input to a RHS
\soar{accept} action or to the \soar{indifferent-selection} prompt.)

\nocomment{no textio anymore, so that last bit needs to be changed...does it
	still apply to user-select? (did it ever?) are there any other
	situations that it might apply to?}


Commands may also be placed in a file (see the \soar{source} command); if a
graphical user interface (GUI) is in use, commands may be executed via menus
or buttons. (Writing your own GUI for Soar is an advanced topic, covered in
the \emph{Soar Advanced Applications Manual}.)

Commands that refer to the attributes of WME's often use a carat symbol
(\soar{\carat}) to denote the attribute; this symbol is optional in most
commands, but is required \soar{sp}, \soar{print}, \soar{production-find}, and
the built-in alias \soar{wmes}.

	\nocomment{It seems that print/wmes is also affected, though I don't
	think it's supposed to be. And production-find.

	KJC: yes, soar does the parsing and expects ^ to identify the
	attribute, just like in sp.}

Several commands have flags that can be specified by using either numeric or
named arguments. This applies to commands in which the amount of information
printed depends on the arguments: The numeric arguments are provided for users
who prefer to think in terms of the amount of information printed, and the
named arguments are provided for users who prefer a mnemonic means of
remembering which flag to use. The commands that allow for either numeric or
named arguments are \soar{watch}, \soar{preferences}, and \soar{matches}.

% ----------------------------------------------------------------------------
\section{Beginners: Basic Commands}

The commands in this section will tell you how to load productions into Soar,
how to run a Soar program, how to quit, and how to get online help.

% ----------------------------------------------------------------------------
\section{Beginners: Additional Commands}

The commands in this section will tell you how to control the amount of output
produced during a run, how to turn learning on and off, how to define
productions, and how to control Soar's default behavior when there are
multiple objects with indifferent preferences.

% ----------------------------------------------------------------------------

% ----------------------------------------------------------------------------
\section{Beginners: Inspecting and Debugging}
\label{INTERFACE-beg-inspect}

The commands in this section will help you inspect production memory, working
memory, and preference memory to understand how Soar is working or to debug a
program that isn't working correctly.

% ----------------------------------------------------------------------------
% ----------------------------------------------------------------------------
% ----------------------------------------------------------------------------
\section{Beginners: Starting a New Task}

The commands in this section allow to you start a new task without quitting
and restarting Soar.

% ----------------------------------------------------------------------------
\section{Beginners: Interacting with the file system}

There are a number of commands available via Soar that allow to you change and
display the current directory for loading files into Soar. These are not
strictly ``Soar commands'' per se; some of them are available because Soar
will pass non-Soar commands to the operating system for execution (as
described in Section \ref{INTERFACE-tcl}).

The commands in this section are described only briefly because they are
really operating system commands. They are mentioned here primarily to let you
know that this functionality is available in Soar.

\paragraph{Notes}

These commands may not work or may work differently on non-Unix machines.

\nocomment{I'm uncertain about how any of these commands work on non-Unix
	platforms} 

% ----------------------------------------------------------------------------
\section{Beginners: Miscellaneous}

The commands in this section cover a few miscellaneous features not yet
described. 

% ----------------------------------------------------------------------------
\section{Intermediate: Running and Tracing}

The commands in this section (and following sections) are more advanced than
those previously described; not all users will need this information. Included
in this section are commands for running Soar and tracing what happens as Soar
runs; this includes more advanced descriptions of commands that have already
been presented (\soar{run}, \soar{watch}, and \soar{learn}).

% ----------------------------------------------------------------------------
\subsection{\soarb{run \soar{ [n] [unit]  }}}
\funsum{run}{More arguments for running Soar.}
\label{run2}
\index{run}


Without arguments, \soar{run} causes Soar to 

\paragraph{Examples}
\begin{verbatim}
run 5 o   --> run for 3 operator selections
run       --> run until halted by a control-C or a production action
\end{verbatim}

\paragraph{Notes}

The run command is more complex when multiple agents and interpreters are in
use. By default, the command applies to all agents and interpreters, a
\soar{-self} flag is used to denote the current agent. Non-agent interpreters
may use the \soar{d}, \soar{e}, and \soar{p} arguments, but not \soar{s} or
\soar{o}. The use of multiple agents and interpreters is documented in the
\textit{Soar Advanced Applications Manual}.

\comment{I suspect \soar{s} and \soar{o} have not been well defined yet for
	use with multiple agents and interpreters. I.e., should they run until
	the current agent reaches the next state (or operator), and keep all
	agents in step in terms of decision cycles? Or should they allow all
	agents to advance to the next state (or operator), and allow the
	decision cycle count to be different for different agents? If the
	latter, how far do non-agent interpreters get to run?

	I think the former is more likely to be in effect.
	}


% ----------------------------------------------------------------------------
\section{Intermediate: Inspecting and Debugging}

The commands in this section will help you inspect production memory, working
memory, and preference memory to understand how Soar is working or to debug a
program that isn't working correctly. The \soar{print} and \soar{matches}
commands were previously described in Section \ref{INTERFACE-beg-inspect}; in
this section, these commands are explained in more depth.

% ----------------------------------------------------------------------------
\subsection{\soarb{p}}
\funsum{p}{Alias for the print command.}
\label{p}
\index{p}

The \soar{p} alias is a shorthand for the \soar{print} command and works
exactly the same way as the \soar{print} command.

\paragraph{Example}
\begin{verbatim}
soar> p s1
(S1 ^io I1 ^ontop O3 ^ontop O2 ^ontop O1 ^operator O4 + ^operator O6
       ^operator O5 + ^operator O6 + ^operator O7 + ^operator O8 +
       ^operator O9 + ^problem-space blocks ^superstate nil ^thing T1
       ^thing B1 ^thing B3 ^thing B2 ^type state)
soar> p -stack
      : ==>S: S1 
      :    O: O6 (move-block)
\end{verbatim}

% ----------------------------------------------------------------------------
\subsection{\soarb{ps}}
\funsum{ps}{Alias for the print command; prints the current subgoal stack.}
\label{ps}
\index{ps}

The \soar{ps} alias is a shorthand for the \soar{print -stack} command. It
prints the current subgoal stack.

% ----------------------------------------------------------------------------
\subsection{\soarb{pf}}
\funsum{pf}{Alias for the production-find command; finds productions that
	match a given pattern.}
\label{pf}
\index{pf}

The \soar{pf} alias is a shorthand for the \soar{prodution-find} command. It
finds productions in production memory that match a specified pattern.

% ----------------------------------------------------------------------------

% ----------------------------------------------------------------------------
% ----------------------------------------------------------------------------
\section{Intermediate: Miscellaneous}

The commands in this section cover a few miscellaneous features not yet
described. 

% ----------------------------------------------------------------------------
\section{Advanced: Running and Tracing}

The commands in this section (and following sections) are more advanced than
those previously described; only advanced Soar users will need this
information. Included in this section are commands for explaining the
formation of specific chunks and justifications, controlling the output
from the watch command, and getting online help on Tcl commands.

% ----------------------------------------------------------------------------
\section{Advanced: Evaluating and increasing efficiency}

The commands in this section are provided to evaluate the efficiency of a Soar
program and of Soar itself and also to increase the efficiency of a Soar
program by providing more information about the program to Soar.

% ----------------------------------------------------------------------------
\section{Advanced: Debugging Soar}

% ----------------------------------------------------------------------------
\section{Advanced: Experimental variations in Soar}

There are two Soar variable that represent experimental changes to the Soar
architecture. These variables are provided so that users may try out these
proposed changes to Soar and evaluate how they help or hinder the development
of their program.

These variables are set and displayed using the Tcl \soar{set} command.

	\nocomment{note that both of these commands should be variables with
	the new scheme.

	True in 7.0.3, but not in earlier versions.}


% ----------------------------------------------------------------------------
% ----------------------------------------------------------------------------
% ----------------------------------------------------------------------------
\section{Advanced: Running Multiple Agents and Interpreters}
\label{INTERFACE-advanced-multiple}

When running Soar with multiple agents (as described in Chapter
\ref{ADVANCED}), there are some additional commands that may be used. Also, 
the full set of Soar user interface commands may be used, but care must be
taken by the user to distinguish commands that apply to the current agent as
opposed to commands that apply to all agents.

\comment{have to update all of these lists:}

The following commands are independent of agents:

\soar{help, soarnews, version, exit, quit} \vspace{12pt}


The following commands by default apply only to the currently selected agent:

\begin{verbatim}
add-wme, alias, default-wme-depth, echo, 
excise, explain, firing-counts, init-soar, learn, 
matches, max-chunks, max-elaborations, memories, monitor, ms, 
multi-attributes, o-support-mode, output-strings-destination, pf, pgs, 
preferences, print, print-alias, print-all-symbols, print-stats, pwatch, 
remove-wme, rete-net, sp, spr, stats, trace-format, unalias, user-select,
warnings, watch, wmes
\end{verbatim} \vspace{12pt}


The following commands apply by default to all agents, and their use with
multiple agents will be described in this section:

\soar{run, send, command-to-interpreters}  \vspace{12pt}

\comment{command-to-interpreters.... changed to send-to-interpreters?

	KJC: yes.}


The following commands are used only with multiple agents, and will be
described in this section:

\begin{verbatim}
create-interpreter, destroy-interpreter, eval-in-interpreters,
init-interpreter, list-interpreters, schedule-interpreter,
select-interpreter
\end{verbatim} \vspace{12pt}


Not sure what to do with these yet:

\soar{command-to-file, dirs, log, send, stop-soar, tksoar}  \vspace{12pt}


\comment{Most commands can't be issued to non-agent interpreters,
	e.g. preferences, wmes, etc., but I'm not sure what happens if you
	try

	KJC: tcl error: command not found (or something like that)} 


% ----------------------------------------------------------------------------
\subsection{\soarb{run} \soar{[-self]}}
\funsum{run}{Another run.}
\label{run3}
\index{run}

\comment{have to fix other descriptions of run, plus function summary, to
	mention this section.}

The basic \soar{run} command is described on page \pageref{run}; more advanced
usage is described on page \pageref{run2}.

When multiple agents are in use, this command works roughly the same way that
it does when a single agent is in use; the only difference is that by default
it applies to \textit{all} agents and interpreters. To run one agent or
interpreter at a time (for debugging), you may use the \soar{-self} argument,
which will restrict the \soar{run} command to apply to the current agent only.

% ----------------------------------------------------------------------------
\subsection{\soarb{send} \soar{[-self]}}
\funsum{send}{send....}
\label{send}
\index{send}

\comment{This command is still being hashed out; and I'm still not clear on
	the concept anyway.

	I think the idea is that by default, you might send a command to
	multiple agents, but that you could also specify a subset.

	KJC: see 7.0.3+ man page.

	KJC: implements the Tk 'send' command for non-tk interps --- sends
	command to one interp only.
	}

% ----------------------------------------------------------------------------
\section{Advanced: Additional Arguments When Starting Soar}
\funsum{soar}{Starts Soar.}
\label{INTERFACE-advanced-soar}
\label{soar2}
\index{soar}

When Soar is first started (e.g., by typing ``\soar{soar}'' at the Unix
prompt), there are a number of additional arguments that may be included.

These options enable the user to start Soar with a specific configuration, for
example, to have three agent interpreters and one Tk non-agent interpreter, or
to use a directory other than the current directory for scanning for startup
files for the agents.

The following options are available:

\begin{tabular}{| l | l | } \hline
argument  & effect  \\ \hline
\soar{-help}  & List the available options.\\ \hline

\soar{-agent name} & Create an agent interpreter with the specified name.\\ 
\soar{-tclsh name} & Create a non-agent Tcl interpreter with the specified name.\\ 
\soar{-wish name}  & Create a non-agent Tk intepreter with the specified name.\\ \hline

\soar{-noTk} & Turn off Tk for all interpreters specified after this argument. \\
\soar{-file filename}  & Read commands from the named file. \\ 
\soar{-path pathname}  & Scan the named path when defining intepreters. \\
\soar{-verbose} & Print detailed information about option processing. \\
\soar{-useIPC} & enable IPC for all intepreters. \\ \hline
\end{tabular} \vspace{10pt}

Arguments that apply only to wish-based (Tk) shells:

\begin{tabular}{| l | l | } \hline
argument  & effect  \\ \hline
\soar{-display displayname}  & Display all windows on the named display. \\
\soar{-sync} & Use synchronous mode for the display server. \\
\soar{-geometry geometry} & Initial geometry for window. \\ \hline
\end{tabular} \vspace{10pt}


\comment{I don't believe these are documented anywhere. Not online (more than
	something similar to the above table), and not in the advanced manual.
	}

\paragraph{Notes}

You may provide multiple names for interpreters and multiple paths to search.

You may not provide multiple file names.

Optional arguments, not predefined, are passed on to Tcl and can be used to
pass information to user-defined Tcl procedures.


	\nocomment{

		Can you source multiple files when you start up Soar? No. But
		couldn't you source a file that in turn sourced other files?

		FROM KARL: Yes, it sources the file.  No, this argument is not
		designed to initialize multiple interpreters -- that's what
		the *.soar files are for.  Following the normal Tcl usage,
		this argument is used to startup an *application* from a Tcl
		script (like the -f demos..gui.tcl does) and suppress the
		normal prompt.  This enables a GUI-controlled application to
		be started so that the command line is hidden from the user.
	}

\nocomment{Give args here for starting non-agent interpreters.

	WISH = windowing shell; tk + tcl; TCLSH = tcl shell

	arguments that apply only to wish-based shells: 
		display, geometry, sync 

	noTk is weird in that it applies to everything that follows it on the
		line, and not everything on the line. I'm also not sure why
		it's needed. The only thing I can think of is if you wanted to
		create some agents that had Tk and others that didn't (because
		we already have WISH and TCLSH to distinguish the non-agent
		interpreters that have Tk from those that don't)

	also note that optional arguments, not predefined, are passed on to
		Tcl and can be used to pass info to user-defined tcl procedures
	}

\nocomment{CLARE: I don't understand -notk option. Is this around so that some
	agents would have Tk and others wouldn't? (It would seem to apply only
	to agents, and not interpreters, since interpreters are started with
	or without Tk, depending on whether -WISH or -TCLSH is specified. But
	I want to make sure I understand.)

	KARL: This is present mainly to turn off Tk when desired (say when
	you're using Soar via a dialup line and no X).  While the argument
	processing checks to see if DISPLAY is set and a connection can be
	made, its possible that users have these vars set even when they don't
	want to use X.  This forces the matter.  Since it applies to all
	interpreters mentioned after it on the command line, it can also be
	used to turn off Tk in some of the interpreters.  -noTk applies to all
	types of interpreters, not just agent ones.  So a sequence like

	-noTk -wish foo -agent bar 

	results in a tclsh interpreter (which is a wish shell minus the Tk
	part) and an agent shell with no Tk.  This makes it easy to switch in
	and out of Tk and to test non-GUI portions of programs.  Its also
	possible that users might add other kinds of interpreters (say one
	with Tcl-DP in it) and the -noTk switch can apply to other
	interpreters as well.
	}


% ----------------------------------------------------------------------------
\section{Advanced: Tcl and Tk Functionality}
\label{INTERFACE-advanced-tcl}


% ----------------------------------------------------------------------------
\section{Advanced: Input and output concerns}
\label{INTERFACE-advanced-io}

The commands in this section describe commands that control input and output
functionality in Soar: where text is printed (from commands or the Soar trace
itself), how often input is accepted, and manually changing working memory
(bypassing the preference process). 


% ----------------------------------------------------------------------------

}


%\cleardoublepage
\appendix
\addcontentsline{toc}{chapter}{Appendices}

% change 'include's to 'input' for final version
%  (use 'include' instead if you're only printing part of the manual)
%\include{a-glossary}
% ----------------------------------------------------------------------------
\typeout{--------------- BLOCKSCODE ------------------------------------------}
\chapter{The Blocks-World Program}
\label{BLOCKSCODE}

\footnotesize
\begin{verbatim}
###############################################################################
###
### File              : blocks.soar
### Original author(s): John E. Laird <laird@eecs.umich.edu>
### Organization      : University of Michigan AI Lab
### Created on        : 15 Mar 1995, 13:53:46
### Last Modified By  : Clare Bates Congdon <congdon@eecs.umich.edu>
### Last Modified On  : 17 Jul 1996, 16:35:14
### Soar Version      : 7
###
### Description : A new, simpler implementation of the blocks world
###               with just three blocks being moved at random.
###
### Notes: 
###   CBC, 6/27: Converted to Tcl syntax
###   CBC, 6/27: Added extensive comments
###############################################################################

 
###############################################################################
# Create the initial state with blocks A, B, and C on the table.
#
# This is the first production that will fire; Soar creates the initial state
#   as an architectural function (in the 'zeroth' decision cycle), which will
#   match against this production.
# This production does a lot of work because it is creating (preferences for)
# all the structure for the initial state:
# 1. The state has a problem-space named 'blocks'. The problem-space limits
#    the operators that will be selected for a task. In this simple problem,
#    it isn't really necessary (there is only one operator), but it's a
#    programming convention that you should get used to.
# 2. The state has four 'things' -- three blocks and the table.
# 3. The state has three 'ontop' relations
# 4. Each of the things has substructure: their type and their names. Note that
#    the fourth thing is actually a 'table'.
# 5. Each of the ontop relations has substructure: the top thing and the
#    bottom thing.
# Finally, the production writes a message for the user.
#
# Note that this production will fire exactly once and will never retract.

sp {blocks-world*elaborate*initial-state
   (state <s> ^superstate nil)
-->
   (<s> ^problem-space blocks
        ^thing <block-A> <block-B> <block-C> <table>
        ^ontop <ontop-A> <ontop-B> <ontop-C>)
   (<block-A> ^type block ^name A)
   (<block-B> ^type block ^name B)
   (<block-C> ^type block ^name C)
   (<table> ^type table ^name TABLE)
   (<ontop-A> ^top-block <block-A> ^bottom-block <table>)
   (<ontop-B> ^top-block <block-B> ^bottom-block <table>)
   (<ontop-C> ^top-block <block-C> ^bottom-block <table>)
   (write (crlf) |Initial state has A, B, and C on the table.|)}


###############################################################################
# State elaborations - keep track of which objects are clear
# There are two productions - one for blocks and one for the table.
###############################################################################

###############################################################################
# Assert table always clear
#
# The conditions establish that:
#  1. The state has a problem-space named 'blocks'.
#  2. The state has a thing of type table.
# The action:
#  1. creates an acceptable preference for an attribute-value pair asserting
#     the table is clear.
#
# This production will also fire once and never retract.

sp {elaborate*table*clear
   (state <s> ^problem-space blocks
              ^thing <table>)
   (<table> ^type table)
-->
   (<table> ^clear yes)}

###############################################################################
# Calculate whether a block is clear
#
# The conditions establish that:
#  1. The state has a problem-space named 'blocks'.
#  2. The state has a thing of type block.
#  3. There is no 'ontop' relation having the block as its 'bottom-block'.
# The action:
#  1. create an acceptable preference for an attribute-value pair asserting
#     the block is clear.
#
# This production will retract whenever an 'ontop' relation for the given block
#  is created. Since the (<block> ^clear yes) wme only has i-support, it will
#  be removed from working memory automatically when the production retracts.

sp {elaborate*block*clear
   (state <s> ^problem-space blocks
              ^thing <block>)
   (<block> ^type block)
   -(<ontop> ^bottom-block <block>)
-->
   (<block> ^clear yes)}


###############################################################################
# Suggest MOVE-BLOCK operators
#
# This production proposes operators that move one block ontop of another block.  
# The conditions establish that:
#  1. The state has a problem-space named 'blocks'
#  2. The block moved and the block moved TO must be both be clear.
#  3. The block moved is different from the block moved to.
#  4. The block moved must be type block.
#  5. The block moved must not already be ontop the block being moved to.
# The actions:
#  1. create an acceptable preference for an operator.
#  2. create acceptable preferences for the substructure of the operator (its
#     name, its 'moving-block' and the 'destination).

sp {blocks-world*propose*move-block
   (state <s> ^problem-space blocks
              ^thing <thing1> {<> <thing1> <thing2>}
              ^ontop <ontop>)
   (<thing1> ^type block ^clear yes)
   (<thing2> ^clear yes)
   (<ontop> ^top-block <thing1>
            ^bottom-block <> <thing2>)
-->
   (<s> ^operator <o> +)
   (<o> ^name move-block
        ^moving-block <thing1>
        ^destination <thing2>)}

###############################################################################
# Make all acceptable move-block operators also indifferent
#
# The conditions establish that:
#  1. the state has an acceptable preference for an operator
#  2. the operator is named move-block
# The actions:
#  1. create an indifferent prefererence for the operator

sp {blocks-world*compare*move-block*indifferent
   (state <s> ^operator <o> +)
   (<o> ^name move-block)
-->
   (<s> ^operator <o> =)}



###############################################################################
# Apply a MOVE-BLOCK operator
# 
# There are two productions that are part of applying the operator.
# Both will fire in parallel.
###############################################################################

###############################################################################
# Apply a MOVE-BLOCK operator
#   (the block is no longer ontop of the thing it used to be ontop of)
#
# This production is part of the application of a move-block operator.
# The conditions establish that:
#  1. An operator has been selected for the current state
#     a. the operator is named move-block
#     b. the operator has a 'moving-block' and a 'destination'
#  2. The state has an ontop relation
#     a. the ontop relation has a 'top-block' that is the same as the
#        'moving-block' of the operator
#     b. the ontop relation has a 'bottom-block' that is different from the 
#        'destination' of the operator
# The actions:
#  1. create a reject preference for the ontop relation

sp {blocks-world*apply*move-block*remove-old-ontop
   (state <s> ^operator <o>
              ^ontop <ontop>)
   (<o> ^name move-block 
        ^moving-block <block1> 
        ^destination <block2>)
   (<ontop> ^top-block <block1> 
            ^bottom-block { <> <block2> <block3> })
-->
   (<s> ^ontop <ontop> -)}
 

###############################################################################
# Apply a MOVE-BLOCK operator
#   (the block is now ontop of the destination)
#
# This production is part of the application of a move-block operator.
# The conditions establish that:
#  1. An operator has been selected for the current state
#     a. the operator is named move-block
#     b. the operator has a 'moving-block' and a 'destination'
# The actions:
#  1. create an acceptable preference for a new ontop relation
#  2. create (acceptable preferences for) the substructure of the ontop
#     relation: the top block and the bottom block

sp {blocks-world*apply*move-block*add-new-ontop
   (state <s> ^operator <o>)
   (<o> ^name move-block
        ^moving-block <block1>
        ^destination <block2>)
-->
   (<s> ^ontop <ontop>)
   (<ontop> ^top-block <block1>
            ^bottom-block <block2>)}


###############################################################################
###############################################################################
# Detect that the goal has been achieved 
#
# The conditions establish that:
#  1. The state has a problem-space named 'blocks'
#  2. The state has three ontop relations
#     a. a block named A is ontop a block named B
#     b. a block named B is ontop a block named C
#     c. a block named C is ontop a block named TABLE
# The actions:
#  1. print a message for the user that the A,B,C tower has been built
#  2. halt Soar

sp {blocks-world*detect*goal
   (state <s> ^problem-space blocks
              ^ontop <AB> 
               { <> <AB> <BC>}
               { <> <AB> <> <BC> <CT> } )
   (<AB> ^top-block <A> ^bottom-block <B>)
   (<BC> ^top-block <B> ^bottom-block <C>)
   (<CT> ^top-block <C> ^bottom-block <T>)
   (<A> ^type block ^name A)
   (<B> ^type block ^name B)
   (<C> ^type block ^name C)
   (<T> ^type table ^name TABLE)
-->
   (write (crlf) |Achieved A, B, C|)
   (halt)}


###############################################################################
###############################################################################
# Monitor the state: Print a message every time a block is moved
#
# The conditions establish that:
#  1. An operator has been selected for the current state
#     a. the operator is named move-block
#     b. the operator has a 'moving-block' and a 'destination'
#  2. each block has a name
# The actions:
#  1. print a message for the user that the block has been moved to the
#     destination. 

sp {blocks-world*monitor*move-block
   (state <s> ^operator <o>)
   (<o> ^name move-block
        ^moving-block <block1>
        ^destination <block2>)
   (<block1> ^name <block1-name>)
   (<block2> ^name <block2-name>)   
-->
   (write (crlf) |Moving Block: | <block1-name>
                 | to: | <block2-name> ) }
\end{verbatim}
\normalsize

% ----------------------------------------------------------------------------
\typeout{--------------- appendix: GRAMMARS for productions ------------------}
\chapter{Grammars for production syntax}
\label{GRAMMARS}
\index{grammar}

This appendix contains the BNF grammars for the conditions and actions of
productions. (BNF stands for Backus-Naur form or Backus normal form; consult a
computer science book on theory, programming languages, or compilers for more
information. However, if you don't already know what a BNF grammar is, it's
unlikely that you have any need for this appendix.)

This information is provided for advanced Soar users, for example, those who
need to write their own parsers.

\comment{this section still needs a disclaimer that what you can actually do
	is less restrictive than the way we described it in the main text } 

\comment{note that grammars are no longer consistent with new rhs actions}

\nocomment{John and I decided while talking about this that we just wouldn't let
	people know that they could omit the identifier of the state

	It is legal to omit the variable test for a state when that variable is not
	tested elsewhere in the production, nor used in the action.  For
	example: 
	\begin{verbatim}
	(state ^operator <o>)
	\end{verbatim}

	is equivalent to 
	\begin{verbatim}
	(state <s> ^operator <o>)
	\end{verbatim}
	}

%-------------------------------------------------------
\section{Grammar of Soar productions}

A grammar for Soar productions is:
\begin{verbatim}
<soar-production>  ::= sp "{" <production-name> [<documentation>] [<flags>]
                     <condition-side> --> <action-side> "}"
<documentation>    ::= """ [<string>] """
<flags>            ::= ":" (o-support | i-support | chunk | default)
\end{verbatim}

% ----------------------------------------------------------------------------
\subsection{Grammar for Condition Side}
\label{SYNTAX-pm-condgrammar}
\index{condition-side grammar}
\index{grammar, condition side}


Below is a grammar for the condition sides of productions:
\begin{verbatim}
<condition-side>   ::= <state-imp-cond> <cond>*
<state-imp-cond>   ::= "(" (state | impasse) [<id_test>]
                     <attr_value_tests>+ ")"
<cond>             ::= <positive_cond> | "-" <positive_cond>
<positive_cond>    ::= <conds_for_one_id> | "{" <cond>+ "}"
<conds_for_one_id> ::= "(" [(state|impasse)] <id_test> 
                     <attr_value_tests>+ ")"
<id_test>          ::= <test>
<attr_value_tests> ::= ["-"] "^" <attr_test> ("." <attr_test>)*
                     <value_test>*
<attr_test>        ::= <test>
<value_test>       ::= <test> ["+"] | <conds_for_one_id> ["+"]  

<test>             ::= <conjunctive_test> | <simple_test>
<conjunctive_test> ::= "{" <simple_test>+ "}"
<simple_test>      ::= <disjunction_test> | <relational_test>
<disjunction_test> ::= "<<" <constant>+ ">>"
<relational_test>  ::= [<relation>] <single_test>
<relation>         ::= "<>" | "<" | ">" | "<=" | ">=" | "=" | "<=>"
<single_test>      ::= <variable> | <constant>
<variable>         ::= "<" <sym_constant> ">"
<constant>         ::= <sym_constant> | <int_constant> | <float_constant>
\end{verbatim}
\index{constant}
\index{variable}

\subsubsection*{Notes on the Condition Side}\vspace{-12pt}
\begin{itemize}
\item In an \soar{<id\_test>}, only a \soar{<variable>} may be used in a \soar{<single\_test>}.
\end{itemize}

\comment{I don't think that grammar is quite right -- e.g. should distinguish
        that acceptable preferences may appear for operators, but not other
        objects}

\comment{Grammar correctly describes Soar; it's just that you can actually do
	things that we've said can't be done. So in this section we'll mention
	that we lied before and that the grammar above is different, but
	correct.  see notes on difference on page 64 of June 7th draft}


% ----------------------------------------------------------------------------
\subsection{Grammar for Action Side}
\label{SYNTAX-pm-actgrammar}    %RHS grammar}
\index{action-side grammar}
\index{grammar, action side}

\comment{RD: this grammar is out of date}

Below is a grammar for the action sides of productions:
\begin{verbatim}
<rhs>                      ::= <rhs_action>*
<rhs_action>               ::= "(" <variable> <attr_value_make>+ ")" 
                             | <func_call>
<func_call>                ::= "(" <func_name> <rhs_value>* ")"
<func_name>                ::= <sym_constant> | "+" | "-" | "*" | "/"
<rhs_value>                ::= <constant> | <func_call> | <variable>
<attr_value_make>          ::= "^" <variable_or_sym_constant>
                             ("." <variable_or_sym_constant>)* <value_make>+
<variable_or_sym_constant> ::= <variable> | <sym_constant>
<value_make>               ::= <rhs_value> <preference_specifier>*

<preference-specifier>     ::= <unary-preference> [","]
                             | <unary-or-binary-preference> [","]
                             | <unary-or-binary-preference> <rhs_value> [","]
<unary-pref>               ::= "+" | "-" | "!" | "~" | "@"
<unary-or-binary-pref>     ::= ">" | "=" | "<" | "&"
\end{verbatim}

\comment{I don't quite understand that last bit. 
<forced-unary-pref>        ::= <binary-preference> {, | ) | ^}
       (but the parser doesn't consume the ")" or "^" here)}

\index{constant}
\index{variable}

% ----------------------------------------------------------------------------
\typeout{--------------- appendix: calculation of o-SUPPORT -----------------}
\chapter{The Calculation of O-Support}
\label{SUPPORT}
\index{support}
\index{i-support}
\index{o-support}
\index{persistence}

This appendix provides a description of when a preference is given O-support by an instantiation (a preference that is not given O-support will have I-support). Soar has four possible procedures for deciding support, which can be selected among with the o-support-mode command (see page \pageref{o-support-mode}). However, only o-support modes 3 \& 4 can be considered current to Soar 8, and o-support mode 4 should be considered an improved version of mode 3.   The default o-support mode is mode 4.

In O-support modes 3 \& 4, support is given production by production; that is, all preferences generated by the RHS of a single instantiated production will have the same support. 


In both modes, a production must meet the following two requirements to create o-supported preferences:
\begin{enumerate}
\item The RHS has no operator proposals, i.e. nothing of the form \begin{verbatim}(<s> ^operator <o> +) \end{verbatim}
\item The LHS has a condition that tests the current operator, i.e. something of the form
\footnote{Sometimes, o-support mode 3 does not notice that this condition is true. This is a bug, which is unlikely to be fixed, since users are encouraged to use mode 4.}
\begin{verbatim}(<s> ^operator <o>)\end{verbatim}
\comment{this is only true if mode 3's checks are improved}
\end{enumerate}



In condition 1, the variable \soar{<s>} must be bound to a state identifier.
In condition 2, the variable \soar{<s>} must be bound to the lowest state identifier. That is to say, each (positive) condition on the LHS takes the form \soar{(id \carat attr value)}, some of these id's match state identifiers, and the system looks for the deepest matched state identifier. The tested current operator must be on this state. For example, in the production-

\begin{verbatim}
sp {elaborate*state*operator*name
  (state <s> ^superstate <s1>)
  (<s1> ^operator <o>)
  (<o> ^name <name>)
-->
  (<s> ^name something)}
\end{verbatim}


the RHS action gets i-support. Of course, the state bound to \soar{<s>} is destroyed when \soar{(<s1> \carat operator <o>)} retracts, so o-support would make little difference. On the other hand, the production- 

\begin{verbatim}
sp {operator*superstate*application
   (state <s> ^superstate <s1>)
   (<s> ^operator <o>)
   (<o> ^name <name>)
 -->
   (<s1> ^sub-operator-name <name>)}
\end{verbatim}

gives o-support to its RHS action, which remains after the substate bound to \soar{<s>} is destroyed. 


There is a third condition that determines support, and it is in this condition that modes 3 \& 4 differ. An extension of condition 1 is that operator augmentations should always receive i-support. Soar has been written to recognize augmentations directly off the operator (ie, \soar{(<o> \carat augmentation value)}), and to attempt to give them i-support. However, there was some confusion about what to do about a production that simultaneously tests an operator, doesn't propose an operator, adds an operator augmentation, and adds a non-operator augmentation, such as-

\begin{verbatim}
sp {operator*augmentation*application
  (state <s> ^task test-support
  	      ^operator <o>)
-->
   (<o> ^new augmentation)
   (<s> ^new augmentation)}
\end{verbatim}


In o-support mode 3, both RHS actions receive o-support; in o-support mode 4, both receive i-support. In either case, Soar will print a warning on firing this production, because this is considered bad coding style.

\nocomment{Support calculations are done at run time, as each production is fired. Could these decisions be done at compile time? Much of the decision is based on the structure of the production, which could be analyzed once as the production was loaded or chunked. However, it may be impossible to guarantee that a variable will be bound to a state id just by examining production syntax. Another issue is whether the state tested in condition 2 is the lowest state - this potentially could differ from instantiation to instantiation. For instance the operator*augmentation*application production above could match against multiple states in the state stack. 

 
%-----------------------------------------------------------
\section{Possible problems with implementation of modes 3 \& 4}

\begin{enumerate}
\item Default mode is actually o-support mode 3. Do we not want 4 to be default?
\item There is still the bug Andy pointed out. In condition 1, the variable \soar{<s>} is \textit{supposed} to be bound to a state variable, but the code does not actually check for this.
\item There is one additional, strange difference between modes 3 \& 4. In condition 3, the \soar{id} of each RHS action is tested to see if it is the id of the operator. This id is represented either as a symbol or as a rete location. Mode 4 tests the id both as a symbol and as a rete location, while mode 3 does only the symbol test. The rete test should be added to mode 3.
\end{enumerate}


\section{O-support modes 1 \& 2}

In o-support modes 1 \& 2, there are some of the same calculations as in 3 \& 4 when a production is matched (which occurs when a wme is added to the rete). In particular, if it is an operator proposal, it is set as IE\_PRODS. Otherwise, if it tests the current operator, it is set as PE\_PRODS, without testing for operator  elaborations. The match is placed on the appropriate dll, according to IE\_PRODS or PE\_PRODS.

Later, when the production is instantiated and the new preferences are built, there are no support calculations for 3 \& 4. But 1 \& 2 have support calculations. I suppose that the purpose of the earlier support calculations is that it places the production on the proper list to be fired during apply or propose,that is, whether it is an IE\_PROD or a PE\_PROD.

During this instantiation process, the function calculate\_support\_for\_instantiation\_preferences() is called to redo support IF the variable need\_to\_do\_support\_calculations is set to TRUE. This variable can be true only when-

\begin{enumerate}
\item  called from chunk\_instantiation OR
\item  \#ifndef SOAR\_8\_ONLY
SOAR\_8\_ONLY is a compile option, which is not defined by default. I think that its purpose is that, when defined, there is no run-time option to switch out of Soar 8. This allows a significant portion of code to be left out. Check out function Soar\_Operand2. 
\end{enumerate}


Mode 2 computes support in what is called 'Doug Pearson's way', which is described as-

 \begin{verbatim}
 For a particular preference p=(id ^attr ...) on the RHS of an
   instantiation [LHS,RHS]:

   RULE #1 (Context pref's): If id is the match state and attr="operator", 
   then p does NOT get o-support.  This rule overrides all other rules.

   RULE #2 (O-A support):  If LHS includes (match-state ^operator ...),
   then p gets o-support.

   RULE #3 (O-M support):  If LHS includes (match-state ^operator ... +),
   then p gets o-support.

   RULE #4 (O-C support): If RHS creates (match-state ^operator ... +/!),
   and p is in TC(RHS-operators, RHS), then p gets o-support.

   Here "TC" means transitive closure; the starting points for the TC are 
   all operators the RHS creates an acceptable/require preference for (i.e., 
   if the RHS includes (match-state ^operator such-and-such +/!), then 
   "such-and-such" is one of the starting points for the TC).  The TC
   is computed only through the preferences created by the RHS, not
   through any other existing preferences or WMEs.

   If none of rules 1-4 apply, then p does NOT get o-support.

   Note that rules 1 through 3 can be handled in linear time (linear in 
   the size of the LHS and RHS); rule 4 can be handled in time quadratic 
   in the size of the RHS (and typical behavior will probably be linear).
   
   
   What is 'match state'? The match goal for the instantiation.
   Match goal - (a match goal is associated with an instantiation).
   Look through instantiated LHS conditions.
   Find the lowest goal state matched to one of the condition's ids.
 \end{verbatim}  
   
O-support mode 1 computes Doug's support and compares it to the poor cousin of mode 3 \& 4 support calculations, ie calculation without checking for operator elaboration. It prints any differences it finds.

}
   
   
 \nocomment{
   
   
   
   
   
   
   

3. the RHS has no direct elaborations of the current operator, ie no actions of the form 
(<o> ^augmentation value).
However, an indirect elaboration such as
   (<o> ^name <d>)
   -->
   (<d> ^augmentation value)
will not prevent o-support.


In mode 3, an instantiation will generate o-supported preferences iff
1. the RHS has no operator proposals (nothing of the form (<s> ^operator <o> +))
2. the LHS has a condition that tests the current operator (something of the form 
(<s> ^operator <o>))
3. 


Operator proposal - a production whose RHS has action (<s> ^operator <o> +))
Operator test - 
	LHS has condition of the form (<s> ^operator <o>)
Operator elaboration -
	o_support_mode 3:
		
	o_support_mode 4:
	

o_support_mode 4: 
1. if an operator proposal - i-support
2. if not an operator test - i-support
3. if an operator test with no elaborations - o-support
4. if an operator test with some elaborations and some non-elaboration, non-function RHS action - i-support (warns)
5. if an operator test with only elaborations - i-support

o_support_mode 3:
1. if an operator proposal - i-support
2. if not an operator test - i-support
3. if an operator test with no elaborations - o-support
4. if an operator test with some elaborations and some non-elaboration, non-function RHS action - o-support (warns)
5. if an operator test with only elaborations - i-support


o_support_mode 0:
1. if an operator proposal - i-support
2. if test operator - o-support
3. else - i-support

}


% ----------------------------------------------------------------------------
\typeout{--------------- appendix: evaluation of PREFERENCES -----------------}
\chapter{The Resolution of Operator Preferences}
\label{PREFERENCES}
\index{preferences}
% This is a technical discussion of the filtering done to evaluate preferences;
% it might belong in a different version of the manual, but not 492

\comment{what's not clear in the following discussion is what happens in the
	usual case, that is, when there's a single acceptable preference.}

During the decision phase, operator preferences are evaluated in a sequence 
of eight steps, in an effort to select a single operator. 
Each step handles a specific type of preference, as illustrated in Figure 
\ref{fig:prefsem}. (The figure should be read starting at the top
where all the operator preferences are collected and passed into the procedure. At
each step, the procedure either exits through a arrow to the right, or passes to 
the next step through an arrow to the left.)

Input to the procedure are the set of current operator preferences, and the output
consists of:
\begin{enumerate}
\item a subset of the candidate operators, either the empty set, a set consisting of a single, 
winning candidate, or a larger set of candidates that may be conflicting,
tied, or indifferent.
\item an impasse-type, possibly NONE\_IMPASSE\_TYPE.
\end{enumerate}
The procedure has several potential exit points. Some occur when the procedure
has detected a particular type of impasse. The others occur when the number of
candidates has been reduced to 
one (necessarily the winner) or zero (a no-change impasse).

\nocomment{
There are nine filter-like operations involved in evaluating the preferences
available for a particular identifier and attribute. These filters are
executed in a specific order to determine the correct values for the working
memory augmentation, as illustrated in Figure \ref{fig:prefsem}. (The figure
should be read starting at the top left where all the values for an attribute
are collected and passed to the first filter.) Each filter reduces the number
of preferences that need to be considered. If a conflict is found, then an
impasse is generated and the filtering process is halted. The impasse
generation is handled as a special exit from a filter and is indicated with a
grey line in Figure \ref{fig:prefsem}.

The preference semantics module takes as input one or more preferences for a
given identifier and attribute; its output includes: \vspace{-10pt}
\begin{enumerate}
\item a possibly empty set of candidate augmentations that may be conflicting,
	indifferent, or parallel\vspace{-10pt}
\item possibly, a flag, \soar{number\_of\_winners}, created only if there is
	not an impasse (the candidates are not conflicting)\vspace{-10pt}
\item possibly, the creation of an impasse object in working memory (if the
	candidates are conflicting)
\end{enumerate}

If a single winner is chosen or there are a set of mutually indifferent
winners, number\_of\_winners is 1. If all of the winners are mutually
parallel, then number\_of\_winners is All.  This allows the decision procedure
to distinguish a set of mutually parallel candidates that should all be
installed in working memory from a set of mutually indifferent candidates that
should have only one value installed or maintained.
}

\index{decision!procedure}

\begin{figure}
\insertfigure{newprefsem}{7in}
\insertcaption{An illustration of the preference resolution process. There are eight
	steps; only five of these provide exits from the  resolution process.}
\label{fig:prefsem}
\end{figure}

Each step in Figure \ref{fig:prefsem} is described below:

\index{preference!require}
\index{require preference}
\index{"!}
\index{constraint-failure impasse}
\begin{description}
\item[RequireTest (!)]
This test checks for required candidates in preference memory and
also constraint-failure impasses involving require preferences (see
Section \ref{ARCH-impasses} on page \pageref{ARCH-impasses}).

\begin{itemize}
\item If there is exactly one candidate operator with a require preference and
	that candidate does not have a prohibit preference, then that candidate
	is the winner and preference semantics terminates.
\item Otherwise ---
	If there is more than one required candidate, then a constraint-
	failure impasse is recognized and preference semantics terminates 
	by returning the set of required candidates.
\item Otherwise ---
	If there exists a required candidate that is also prohibited, a
	constraint-failure impasse with the required/prohibited value is
	recognized and preference semantics terminates.
\item Otherwise ---
	The candidates are passed to AcceptableCollect.
\end{itemize}

\item[AcceptableCollect (+) ] This operation builds a list of operators
	for which there is an acceptable preference in preference memory.
	This list of candidate operators is passed to the ProhibitFilter.\index{+}
\nocomment{
\begin{itemize}
\item If there are no acceptable preferences in memory for the value of an
	attribute then exit preference semantics with no items picked. 
	(This is an efficiency termination, and does not apply to other filters.)
\item Otherwise ---
	The candidates are passed to the ProhibitFilter.
\end{itemize}
}
\index{preference!acceptable}
\index{acceptable preference}


\item[ProhibitFilter ($\sim$) ] This filter removes the candidates that
	have prohibit preferences in memory. The rest of the candidates are passed to
	the RejectFilter.
\index{preference!prohibit}
\index{prohibit preference}
\index{~}

\item[RejectFilter ($-$) ] This filter removes the candidates that have
	reject preferences in memory. 
	\begin{itemize}
	\item At this point, if the set of remaining candidates is either empty or has one
	member, preference semantics terminates and this set is returned.
	\item Otherwise, the remaining candidates are passed to the
	BetterWorseFilter.
	\end{itemize}
\index{preference!reject}
\index{reject preference}
\index{-}

\item[BetterWorseFilter ($>$), ($<$) ] This filter removes any candidates that are worse
	than another candidate.
\begin{itemize}
\item If the set of remaining candidates is empty, a conflict impasse is created
	returning the set of conflicted operators (all candidates passed to this filter).
\item Otherwise, pass any remaining candidates to the BestFilter.
\end{itemize}
\index{preference!worse}
\index{worse preference}
\index{preference!better}
\index{better preference}
\index{<}
\index{>}


\item[BestFilter ($>$) ] If some remaining candidate has a best preference,
	this filter removes any candidates that do not have
	a best preference. If there are no best preferences for any of the current
	candidates, the filter has no effect. The remaining candidates are passed
	to the WorstFilter.
\index{preference!best}
\index{best preference}

\item[WorstFilter ($<$) ] If all remaining candidates have worst preferences, this filter
	has no effect. Otherwise, the filter removes any candidates that have
	a worst preference.
	\begin{itemize}
	\item Once again, if the set of remaining candidates is either empty or has one
	member, preference semantics terminates and this set is returned.
	\item Otherwise, the remaining candidates are passed to the
	IndifferentTest.
	\end{itemize}
\index{preference!worst}
\index{worst preference}

\index{=}
\item[IndifferentTest (=) ] This operation traverses the remaining candidates and marks 
	each candidate for which one of the following is true:
	\begin{itemize}
	\item the candidate has a unary indifferent preference
	\item the candidate has a numeric indifferent preference
	\item the candidate is binary indifferent to all of the remaining candidate operators
	\end{itemize}
	If some candidate is left unmarked, then the procedure signals a tie impasse and returns 
	the complete set of candidates that passed into the IndifferentTest. Otherwise, the candidates
	are mutually indifferent, in which case an operator is chosen according to the method set
	by the \textbf{indifferent-selection} command, described on page \pageref{indifferent-selection}.
\nocomment{
This filter returns them as indifferent. 
	If they are not, and this is an operator, a tie impasse is
	declared and preference semantics terminates.
\index{preference!indifferent}
\index{indifferent preference}
	\comment{that makes no sense....also, "calling routine" below}
\begin{itemize}
\item If the candidates are all mutually indifferent, terminate preference
	semantics with all of the candidates in the item set and set
	number\_of\_winners to one. The calling routine then selects one
	of the indifferent candidates.
\item Otherwise ---
	If the non-mutually indifferent candidates are for operators, generate
	a tie impasse, with all of the candidates as items, not just those
	that are not mutually indifferent.
\item Otherwise ---
	The candidates are passed to the ParallelFilter.
\end{itemize}


\index{&}
\item[ParallelFilter (\&) ] If all of the remaining candidates are mutually
	parallel, they're returned as mutually parallel. Otherwise, a tie
	impasse is returned.  A candidate is mutually parallel to the set of
	other candidates if it has a unary parallel preference, or if it has
	either of the two possible 
	relative parallels between each candidate.
\index{preference!parallel}
\index{parallel preference}
\begin{itemize}
\item If all of the candidates are mutually parallel
	terminate preference semantics with all of 
	the candidates in the item set and set number\_of\_winners to All.
\item Otherwise ---
	Generate a tie impasse with all the non-mutually parallel
	candidates in the item set.
\end{itemize}
}
\end{description}


%\include{a-using}
%\include{a-default}
%% ----------------------------------------------------------------------------
\typeout{--------------- appendix: calculation of o-SUPPORT -----------------}
\chapter{The Calculation of O-Support}
\label{SUPPORT}
\index{support}
\index{i-support}
\index{o-support}
\index{persistence}

This appendix provides a description of when a preference is given O-support by an instantiation (a preference that is not given O-support will have I-support). Soar has four possible procedures for deciding support, which can be selected among with the o-support-mode command (see page \pageref{o-support-mode}). However, only o-support modes 3 \& 4 can be considered current to Soar 8, and o-support mode 4 should be considered an improved version of mode 3.   The default o-support mode is mode 4.

In O-support modes 3 \& 4, support is given production by production; that is, all preferences generated by the RHS of a single instantiated production will have the same support. 


In both modes, a production must meet the following two requirements to create o-supported preferences:
\begin{enumerate}
\item The RHS has no operator proposals, i.e. nothing of the form \begin{verbatim}(<s> ^operator <o> +) \end{verbatim}
\item The LHS has a condition that tests the current operator, i.e. something of the form
\footnote{Sometimes, o-support mode 3 does not notice that this condition is true. This is a bug, which is unlikely to be fixed, since users are encouraged to use mode 4.}
\begin{verbatim}(<s> ^operator <o>)\end{verbatim}
\comment{this is only true if mode 3's checks are improved}
\end{enumerate}



In condition 1, the variable \soar{<s>} must be bound to a state identifier.
In condition 2, the variable \soar{<s>} must be bound to the lowest state identifier. That is to say, each (positive) condition on the LHS takes the form \soar{(id \carat attr value)}, some of these id's match state identifiers, and the system looks for the deepest matched state identifier. The tested current operator must be on this state. For example, in the production-

\begin{verbatim}
sp {elaborate*state*operator*name
  (state <s> ^superstate <s1>)
  (<s1> ^operator <o>)
  (<o> ^name <name>)
-->
  (<s> ^name something)}
\end{verbatim}


the RHS action gets i-support. Of course, the state bound to \soar{<s>} is destroyed when \soar{(<s1> \carat operator <o>)} retracts, so o-support would make little difference. On the other hand, the production- 

\begin{verbatim}
sp {operator*superstate*application
   (state <s> ^superstate <s1>)
   (<s> ^operator <o>)
   (<o> ^name <name>)
 -->
   (<s1> ^sub-operator-name <name>)}
\end{verbatim}

gives o-support to its RHS action, which remains after the substate bound to \soar{<s>} is destroyed. 


There is a third condition that determines support, and it is in this condition that modes 3 \& 4 differ. An extension of condition 1 is that operator augmentations should always receive i-support. Soar has been written to recognize augmentations directly off the operator (ie, \soar{(<o> \carat augmentation value)}), and to attempt to give them i-support. However, there was some confusion about what to do about a production that simultaneously tests an operator, doesn't propose an operator, adds an operator augmentation, and adds a non-operator augmentation, such as-

\begin{verbatim}
sp {operator*augmentation*application
  (state <s> ^task test-support
  	      ^operator <o>)
-->
   (<o> ^new augmentation)
   (<s> ^new augmentation)}
\end{verbatim}


In o-support mode 3, both RHS actions receive o-support; in o-support mode 4, both receive i-support. In either case, Soar will print a warning on firing this production, because this is considered bad coding style.

\nocomment{Support calculations are done at run time, as each production is fired. Could these decisions be done at compile time? Much of the decision is based on the structure of the production, which could be analyzed once as the production was loaded or chunked. However, it may be impossible to guarantee that a variable will be bound to a state id just by examining production syntax. Another issue is whether the state tested in condition 2 is the lowest state - this potentially could differ from instantiation to instantiation. For instance the operator*augmentation*application production above could match against multiple states in the state stack. 

 
%-----------------------------------------------------------
\section{Possible problems with implementation of modes 3 \& 4}

\begin{enumerate}
\item Default mode is actually o-support mode 3. Do we not want 4 to be default?
\item There is still the bug Andy pointed out. In condition 1, the variable \soar{<s>} is \textit{supposed} to be bound to a state variable, but the code does not actually check for this.
\item There is one additional, strange difference between modes 3 \& 4. In condition 3, the \soar{id} of each RHS action is tested to see if it is the id of the operator. This id is represented either as a symbol or as a rete location. Mode 4 tests the id both as a symbol and as a rete location, while mode 3 does only the symbol test. The rete test should be added to mode 3.
\end{enumerate}


\section{O-support modes 1 \& 2}

In o-support modes 1 \& 2, there are some of the same calculations as in 3 \& 4 when a production is matched (which occurs when a wme is added to the rete). In particular, if it is an operator proposal, it is set as IE\_PRODS. Otherwise, if it tests the current operator, it is set as PE\_PRODS, without testing for operator  elaborations. The match is placed on the appropriate dll, according to IE\_PRODS or PE\_PRODS.

Later, when the production is instantiated and the new preferences are built, there are no support calculations for 3 \& 4. But 1 \& 2 have support calculations. I suppose that the purpose of the earlier support calculations is that it places the production on the proper list to be fired during apply or propose,that is, whether it is an IE\_PROD or a PE\_PROD.

During this instantiation process, the function calculate\_support\_for\_instantiation\_preferences() is called to redo support IF the variable need\_to\_do\_support\_calculations is set to TRUE. This variable can be true only when-

\begin{enumerate}
\item  called from chunk\_instantiation OR
\item  \#ifndef SOAR\_8\_ONLY
SOAR\_8\_ONLY is a compile option, which is not defined by default. I think that its purpose is that, when defined, there is no run-time option to switch out of Soar 8. This allows a significant portion of code to be left out. Check out function Soar\_Operand2. 
\end{enumerate}


Mode 2 computes support in what is called 'Doug Pearson's way', which is described as-

 \begin{verbatim}
 For a particular preference p=(id ^attr ...) on the RHS of an
   instantiation [LHS,RHS]:

   RULE #1 (Context pref's): If id is the match state and attr="operator", 
   then p does NOT get o-support.  This rule overrides all other rules.

   RULE #2 (O-A support):  If LHS includes (match-state ^operator ...),
   then p gets o-support.

   RULE #3 (O-M support):  If LHS includes (match-state ^operator ... +),
   then p gets o-support.

   RULE #4 (O-C support): If RHS creates (match-state ^operator ... +/!),
   and p is in TC(RHS-operators, RHS), then p gets o-support.

   Here "TC" means transitive closure; the starting points for the TC are 
   all operators the RHS creates an acceptable/require preference for (i.e., 
   if the RHS includes (match-state ^operator such-and-such +/!), then 
   "such-and-such" is one of the starting points for the TC).  The TC
   is computed only through the preferences created by the RHS, not
   through any other existing preferences or WMEs.

   If none of rules 1-4 apply, then p does NOT get o-support.

   Note that rules 1 through 3 can be handled in linear time (linear in 
   the size of the LHS and RHS); rule 4 can be handled in time quadratic 
   in the size of the RHS (and typical behavior will probably be linear).
   
   
   What is 'match state'? The match goal for the instantiation.
   Match goal - (a match goal is associated with an instantiation).
   Look through instantiated LHS conditions.
   Find the lowest goal state matched to one of the condition's ids.
 \end{verbatim}  
   
O-support mode 1 computes Doug's support and compares it to the poor cousin of mode 3 \& 4 support calculations, ie calculation without checking for operator elaboration. It prints any differences it finds.

}
   
   
 \nocomment{
   
   
   
   
   
   
   

3. the RHS has no direct elaborations of the current operator, ie no actions of the form 
(<o> ^augmentation value).
However, an indirect elaboration such as
   (<o> ^name <d>)
   -->
   (<d> ^augmentation value)
will not prevent o-support.


In mode 3, an instantiation will generate o-supported preferences iff
1. the RHS has no operator proposals (nothing of the form (<s> ^operator <o> +))
2. the LHS has a condition that tests the current operator (something of the form 
(<s> ^operator <o>))
3. 


Operator proposal - a production whose RHS has action (<s> ^operator <o> +))
Operator test - 
	LHS has condition of the form (<s> ^operator <o>)
Operator elaboration -
	o_support_mode 3:
		
	o_support_mode 4:
	

o_support_mode 4: 
1. if an operator proposal - i-support
2. if not an operator test - i-support
3. if an operator test with no elaborations - o-support
4. if an operator test with some elaborations and some non-elaboration, non-function RHS action - i-support (warns)
5. if an operator test with only elaborations - i-support

o_support_mode 3:
1. if an operator proposal - i-support
2. if not an operator test - i-support
3. if an operator test with no elaborations - o-support
4. if an operator test with some elaborations and some non-elaboration, non-function RHS action - o-support (warns)
5. if an operator test with only elaborations - i-support


o_support_mode 0:
1. if an operator proposal - i-support
2. if test operator - o-support
3. else - i-support

}


%\include{SAN-preferences}
%\include{SAN-tcl-io}
% ----------------------------------------------------------------------------
\typeout{--------------- appendix: GDS ------------------}
\chapter[A Goal Dependency Set Primer]{
A Goal Dependency Set Primer\footnote{A preliminary draft by Robert Wray, contact at \texttt{wrayre@acm.org}.
}}

\label{GDS}
\index{GDS}


% a list of hyphenation points for re-occuring words in the document
\hyphenation{con-temp-or-an-e-ous}
\hyphenation{OP-ER-AND}
\hyphenation{Mich-i-gan}

%\pagestyle{myheadings}
%\markboth{GDS Primer}{DRAFT: Not for Quotation or Distribution}

%\input{macros}


      

% use optional labels to link authors explicitly to addresses:
% \author[label1,label2]{}
% \address[label1]{}
% \address[label2]{}
%\author{Robert Wray  \\  Soar Technology \\ 3600 Green Road Suite 600 \\ Ann Arbor, MI 48105 \\ (734)327-8000 \\ \texttt{wrayre@acm.org}  
%        }


%\maketitle                        %%%% To set Title and Author names.
%\thispagestyle{empty}

%%%% Replace with your Abstract.

%%%%%%%%%%%%%%%%%%%%%%%%%%%%%%%%%

This document briefly describes the Goal Dependency Set (GDS), which
was introduced with Soar~8.  There are three sections: a brief
discussion of the motivation for the GDS, a discussion of the
consequences of the GDS from a behavior developer/modeler's point of
view, and some details on the kernel implementation of the GDS, for
anyone working at the architecture level.  This document is by no
means complete, but introduces the GDS in Soar-specific terms.

\section*{Why the GDS was needed}

As a symbol system, Soar attempts to approximate the knowledge level
but will necessarily always fall short \cite{Newell90:UTC}.  We can
informally think of the way in which Soar falls short of the knowledge
level as its peculiar ``psychology.''  Those interested in using Soar
to model human psychology would like Soar's ``psychology'' to
approximate human psychology.  Those using Soar to create agent
systems would like to make Soar's processing approximate the knowledge
level as closely as possible.  However, Soar~7 had a number of
symbol-level ``quirks'' that appeared inconsistent with human
psychology and that made building large-scale, knowledge-based systems
in Soar more difficult than necessary.  Bob Wray's thesis (1998)
\nocite{Wray98:Ensuring} addressed many of these symbol-level problems
in Soar, among them logical inconsistency in symbol manipulations,
non-contemporaneous constraints in chunks \cite{Wray96:Compilation},
race conditions in rule firings and in the decision process, and
contention between original task knowledge and learned knowledge
\cite{Wray01:Resolving}.

The Goal Dependency Set implements a solution to logical
inconsistencies between persistent (o-supported) working memory
elements (WMEs) in a substate and its ``context''.  The context
consists of all the WMEs in any superstates above the local
goal/state\footnote{This report will use ``state,'' not ``goal.''  At
the kernel level, states are still called ``goals'' and ``goal'' is often
still used to refer to states.    As a result, a
confusion in terminology results, with ``\textbf{Goal} Dependency Set'' a 
specific example, even though ``goals'' have not been
an explicit, behavior-level Soar construct since Soar~6}.  In Soar, any
action (application) of an operator receives an o-support preference.
This preference makes the resulting WME persistent: it will remain in
memory until explicitly removed (or until its local state is removed),
regardless of whether it continues to be justified.

Persistent WMEs are pervasive in Soar, because operators are the main
unit of problem solving.  Persistence is necessary for taking any
non-monotonic step in a problem space.  However, persistent WMEs also
are dependent on WMEs in the superstate context.  The problem in
Soar~7, especially when trying to create large-scale systems like
TacAir-Soar \cite{Jones99:Automated}, is that the knowledge developer
must always think about which dependencies can be ``ignored'' and
which need to result in a reconsideration of the persistent WME.  For
example, imagine an exploration robot that makes a persistent decision
to travel to some distant destination based, in part, on its power
reserves.  Now suppose that the agent notices that its power reserves
have failed.  If this change is not communicated to the state where
the travel decision was made, the agent will continue to act as if its
full power reserves were still available.

Of course, for this specific example, the knowledge designer can
encode some knowledge to react to this inconsistency.  The fundamental
problem is that the knowledge designer has to consider \emph{all}
possible interactions between all o-supported WMEs and all contexts.
Soar systems often use the architecture's impasse mechanism to realize
a form of decomposition.  These potential interactions mean that the
knowledge developer cannot focus on individual problem spaces when
creating knowledge, which makes knowledge development more difficult.
Further, in all but the simplest systems, the knowledge designer will
miss some potential interactions.  The result is agents are that were
unnecessarily brittle, failing in difficult-to-understand,
difficult-to-duplicate ways.  

The GDS also solves the the problem of non-contemporaneous constraints
in chunks.  A non-contemporaneous constraint refers to two or more
conditions that never co-occur simultaneously.  An example might be a
driving robot that learned a rule that attempted to match ``red
light'' and ``green light'' simultaneously. Obviously, for functioning
traffic lights, this rule would never fire.  By ensuring that local
persistent elements are always consistent with the higher-level
context, non-contemporaneous constraints in chunks are
\emph{guaranteed} not to happen.


The GDS captures context dependencies during processing, meaning the
architecture will identify and respond to inconsistencies
automatically.  The knowledge designer then does not have to consider
potential inconsistencies between local, o-supported WMEs and the
context.  The following sections describe further how the GDS works
and how to use the GDS in behavior systems, as well as how the GDS is
implemented in the Soar kernel.


\section*{Behavior-level view of the Goal Dependency Set}

This section discusses what the GDS does, and how that impacts
production knowledge design and implementation.

\subsection*{Operation of the Goal Dependency Set}


\begin{figure}
\insertfigure{simple-ncc}{3in}
\caption{Simplified Representation of the context dependencies (above the line), local os-upported WMEs (below the line), and the generation of a result.  In Soar~7, this situation led to non-contemporaneous constraints in the chunk that generates {\bf 3}.}
\label{'ncc'}
\end{figure}

Whenever a feature is created (added to working memory) in the Soar~7
architecture, that feature will persist for some time.  The
persistence of features may differ with respect to how long the
features remain in memory, and more importantly, what circumstances
cause the feature to be removed.  The Soar~7 architecture utilizes
three primary types of persistence: i-support, o-support, and
c-support.

The weakest persistence is instantiation support.  An i-supported
feature exists in memory only as long as the production which lead to
the feature's creation remains instantiated.  Thus, the WME depends
upon this production instantiation (and, more specifically, the
features the instantiation tests).  When one of the conditions in the
production instantiation no longer matches, the instantiation is
retracted, resulting in the loss of the acceptable preference for the
WME.\footnote{Importantly, in a technical sense, the WME is only
retracted when it loses instantiation support, not when the creating
production is retracting.  For example, a WME could receive i-support
from several different instantiations and the retraction of one would
not lead to the retraction of the WME.  However, the the following
generally discusses direct dependency unmediated by preferences,
ignoring this complication for clarity.}  I-support is illustrated in
Figure~\ref{'ncc'}. A copy of {\bf A} in the subgoal, {\bf A$_s$}, is
retracted automatically when {\bf A} changes to {\bf A'}.  The
substate WME persists only as long as it remains justified by {\bf A}.
This justification is called ``instantiation support'' (I-support) in
Soar (and should not be confused with result \emph{justifications}.)

In the broadest sense, we can say that some feature $<$b$>$ is
``dependent'' upon another element $<$a$>$ if $<$a$>$ was used in the
creation of $<$b$>$, i.e., if $<$a$>$ was tested in the production
instantiation that created $<$b$>$.  Further, a dependent change with
respect to feature $<$b$>$ is a change to any of its instantiating
features.  In Figure~\ref{'ncc'}, the change from {\bf A} to {\bf A'}
is a dependent change for feature {\bf 1} because {\bf A} was used to
create {\bf 1}.

In Soar 7, some features are insensitive to dependent changes.  These
features are often referred to as ``persistent WMEs'' because, unlike
i-supported WMEs, they remain in memory until explicitly removed.
There are two different types of this stronger persistence: o-support
and c-support.  

Any feature created by the action of an operator
receives ``operator support.''  An o-supported feature remains in
memory until explicitly rejected (or until the superstructure to which
it is attached is removed).  Removal is architecturally
independent of the WME's instantiating conditions.

Context-support affects the persistence of an operator itself, rather
than its effects.  Once a unique operator has been chosen by the
decision procedure, the choice persists until explicitly re-decided
(via a reconsider preference).  C-support ensures that the WME for a
selected operator remains available even if the production that
proposed the operator is no longer instantiated.  Soar~8 eliminates
c-support, so that operators now persist only as long as they receive
instantiation support.  This change was integral to the overall
solution Soar~8 provides, but is distinct from the GDS.

The GDS provides a solution to the first problem.  When {\bf A}
changes, the persistent WME {\bf 1} may be no longer consistent with
its context (e.g., {\bf A'}).  The specific solution is inspired by
the chunking algorithm.  In Soar~8, whenever an o-supported WME is
created in the local state, the superstate dependencies of that new
feature are determined and added to the {\em goal dependency set}
(GDS) of that state. Conceptually speaking, whenever a working memory
change occurs, the dependency sets for every state in the context
hierarchy are compared to working memory changes.\footnote{The
implementation is slightly different, trading additional memory
overhead to avoid scanning all the goal dependency sets after each WM
change.  See the next section.  }  If a removed element is found in a
GDS, the state is removed from memory (along with all existing
substructure). The dependency set includes only dependencies for
o-supported features.  For example, in Figure~\ref{'gds'}, at time
$t_0$, because only i-supported features have been created in the
subgoal, the dependency set is empty.

\begin{figure}
\insertfigure{gomor-o-support}{3in}
\caption{The Dependency Set in Soar~8.}
\label{'gds'}
\end{figure}


Three types of features can be tested in the creation of an
o-supported feature.  Each requires a slightly different type of
update to the dependency set.
\begin{description}
\item [Elements in the superstate:] WMEs in the superstate are added
directly to the goal's dependency set.  In Figure~\ref{'gds'}, the
persistent subgoal item {\bf 3} is dependent upon {\bf A} and {\bf
D}. These superstate WMEs are added to the subgoal's dependency set when
{\bf 3} is added to working memory at time $t_1$.  It does not matter
that {\bf A} is i-supported and {\bf D} o-supported.\footnote{In addition,
superstate WMEs can also include context slot preferences, which 
are represented in the architecture as working memory elements.}
\item [Local I-Supported Features:] Local i-supported features are not
added to the goal dependency set.  Instead, the superstate WMEs that
led to the creation of the i-supported feature are determined and
added to the GDS.  In the example, when {\bf 4} is created, {\bf A},
{\bf B} and {\bf C} must be added to the dependency set because they
are the superstate features that led to {\bf 1}, which in turn led to
{\bf 2} and finally {\bf 4}.  However, because item {\bf A} was
previously added to the dependency set at $t_1$, it is unnecessary to
add it again.
\item [Local O-Supported Features:] The dependencies of a local
o-supported feature have already been added to the state's GDS.  Thus,
tests of local o-supported WMEs do not require additions to the
dependency set.  In Figure~\ref{'gds'}, the creation of element {\bf
5} does not change the dependency set because it is dependent only
upon persistent items {\bf 3} and {\bf 4}, whose features had been
previously added to the GDS.
\end{description}

In Soar~8, any change to the current dependency set will cause
the retraction of all subgoal structure.  Thus, any time after time
$t_1$, either the {\bf D} to {\bf D'} or {\bf A} to {\bf A'}
transition would cause the removal of the entire subgoal. The {\bf E}
to {\bf E'} transition causes no retraction because {\bf E} is not in
the goal's dependency set.

\subsection*{The role of the GDS in agent design}

The GDS places some design time constraints on operator implementation.
These constraints are:
\begin{itemize}
\item Operator actions that are used to remember a previous state/situation should be asserted in the top state
\item All operator elaborations should be i-supported
\item Any operator with local actions should be designed to be re-entrant
\end{itemize}
This section describes these issues.

Soar says any operator effect is o-supported, regardless of whether
that assertion is entailed by the current situation, or whether it
reflects an assumption about it.  The GDS adds additional (needed)
constraint.  Because any context dependencies for subgoal, o-supported
assertions will be added to the GDS, the developer must decide if an
o-supported element should be represented in a substate or the top
state.

This decision is straightforward if the functional role of the
persistent element is considered.  Four important capabilities that
require persistence are:
\begin{enumerate}

\item \textbf{Reasoning hypothetically:} ~ Some assertions may need to
reflect hypothetical states.  Such assertions are ``assumptions''
because a hypothetical inference cannot always be grounded in the
current context.  In other problem solvers with truth maintenance,
only assumptions are persistent.

\item \textbf{Reasoning non-monotonically:} ~
Sometimes the result of an inference changes one of the assertions on
which the inference is dependent.  As an example, consider the task of
counting.  Each newly counted item replaces the old value of the
count. 

\item \textbf{Remembering:} ~
Agents oftentimes need to remember an external situation or stimulus,
even when that perception is no longer available.  

\item \textbf{Avoiding Expensive Computations:} ~ In some situations,
an agent may have the information needed to assert some belief in a
new world state but the expense of performing the computation
necessary for the assertion, given what is already known, makes the
computation avoidable.  For example, in dynamic, complex domains,
determining when to make an expensive calculation is often formulated
as an explicit agent task \cite{Jones99:Automated}.
\end{enumerate}

When remembering or avoiding an expensive computation, the
agent/designer is making a commitment to retain something even though
it might not be supported in the current context.  In Soar~8, these
WMEs should be asserted in the top state.  \emph{For many Soar systems,
especially those focused on execution in a dynamic environment, 
most o-supported elements will need to be stored on the top state.} 

For any kind of local, non-monotonic reasoning about the context
(counting, projection planning), features should be stored locally.
When a dependent context change occurs, the GDS interrupts the
processing by removing the state.  While this may seem like a severe
over-reaction, formal and empirical analysis have suggested that this
solution is less computationally expensive than attempting to identify
the specific dependent assumption \cite{Wray03:Ensuring}.


\subsection*{Operator Elaborations}

Operator elaborations (i.e., placing some information on an operator
WME) should be i-supported when using Soar~8, since this information
is, by definition, temporary/not persistent (because it's located on
the non-persistent operator).  However, the kernel itself hasn't kept
up with this change.  Prior to Soar~8.5, Soar's o-support modes
computed operator elaborations as o-supported, resulting in the
context conditions being added to the GDS.  This often leads to
unwanted/unnecessary retractions.  If you are using a version prior to
Soar~8, you should declare any operator elaborations i-supported (i.e.,
using :i-support).




\section*{Kernel-level view of the Goal Dependency Set}


The actual implementation of the GDS in the Soar kernel is slightly
more complex than the conceptual description of the previous section
(but not significantly so).  

Elements are added the GDS via elaborate\_gds(), a procedure in
decide.c that mimics the chunking backtrace function.  The algorithm
is shown in Figure~\ref{tab:dhj:proc}.  When an o-supported preference
is asserted, elaborate\_gds() is called.  Conditions in a production
instantiation that are located in a higher context can be added
directly to the GDS (1).  For local conditions, elaborate\_gds() first
checks whether the tested WME is o-supported, or if it has been
previously been back traced through (2). If either of these are true,
the WME can be ignored because it's dependencies have been added to
the GDS previously.  If not, elaborate\_gds() is called recursively,
to find the context dependencies for the local, contributing WME,~$c$
(3).

\begin{figure}[h]
%\rule{\textwidth}{.5mm}
\framebox[\textwidth]{
\begin{minipage}{\textwidth}
\begin{tabbing}
xxx\=xxx\=xxx\=xxx\=xxx\=xxxxxxxxxxxxxxxxxxxx\= \kill

\textbf{PROC} $create\_new\_assertion(\ldots)$ \\
\> Whenever a new o-supported element is asserted, the GDS is updated \\ 
\> to include any new context dependencies.  \\
\> $\ldots$\\
\> $A_{inst} \leftarrow $ instantiation that asserted acceptable preference for A  \\
\> \textbf{IF} A is an o-supported WME\\
%\>\>$A_{goal_{GDS}}$ : = $append(A\rightarrow goal\rightarrow GDS$ \\
%\>\>$G \leftarrow A_{goal} \quad$ 
\>\>G is the goal/state in which A is asserted \\
\>\>$G_{GDS} \leftarrow append(G_{GDS}, elaborate\_GDS(A)) $ \\

\>$\ldots$\\
\textbf{END} \\

\\
\textbf{PROC} $elaborate\_GDS(assertion\, A)$ \\
\> $S \leftarrow \{ NIL \} $ \\
        \>\textbf{FOR} Each assertion  $c$ in $A_{inst}$, the instantiation supporting A \\
$\bigcirc \! \! \! \! 1$      
          \>\>     \textbf{IF} $\left\{ GoalLevel(c) \quad\mbox{closer to top state than}\quad GoalLevel(A) \right\}$ \\

           \>\>\>\>              $append(c,S)\quad$ (append context dependency to GDS) \\
\\
$\bigcirc \! \! \! \! 2$
             \>\>   \textbf{ELSEIF} \{ \>\>\> $GoalLevel(c) \quad$ same as $GoalLevel(A)  \quad\mbox{AND}\quad $ \\
              \>\>\> \> \>    $c$ is NOT an o-supported WME $\quad\mbox{AND}\quad $ \\
\>\>\> \> \>  $c$ has not previously been inspected \} \\
$\bigcirc \! \! \! \! 3$               \>\>\>\>          $S \leftarrow append(S,elaborate\_GDS(c))$ \\
\>\>\>\>\>(compute GDS dependencies for $c$ and add to goal's GDS) \\
$\bigcirc \! \! \! \! 4$               \>\>\>\>          $c_{inspected} \leftarrow true \quad $ \\
\>\>\>\>\>($c$'s context dependencies have been added to the GDS;  \\
\>\>\>\>\>~~ no need to consider it again for this GDS)


\\
\> return S, the list of new dependencies in the GDS \\ 
\textbf{END} \\


%         \>\>\textbf{END(IF)} \\
%\>\textbf{END(FOR)} \\
%\textbf{END(PROC)} \\

\\
\textbf{PROC} $GoalLevel(assertion \quad A)$ \\
\> Return the goal level associated with assertion A

\end{tabbing}
\end{minipage}
}
%\rule{\textwidth}{.5mm}
\caption{The algorithm for determining members of the GDS.}
\label{tab:dhj:proc}
\end{figure}


When WME changes occur, each goal/state must be checked to determine
if the WME appeared on that goal's GDS. Because WME changes occur in
nearly every Soar elaboration cycle, we chose to extend the WME data
structure to avoid this scanning.  Figure~\ref{wme} illustrates the
relationship.  Each GDS consists of a pointer to its goal and a
pointer to a WME DLL list.  The gds\_next and gds\_prev pointers on
WME define the GDS WMEs for a particular GDS and the GDS pointer
provides a link back from each GDS WME to the GDS data structure.

When a WME is removed, the GDS pointer can be checked to determine
immediately if the goal should be removed.  No scanning is necessary.


\begin{figure}
INSERT DIAGRAM HERE
%\insertfigure{NEED DIAGRAM}{3in}
\caption{The GDS and WME data structures}.
\label{wme}
\end{figure}

\subsection*{Other implementation issues}

\begin{itemize}

\item Allocating memory for the GDS \\ The GDS memory is created for
each goal when the goal is created.  The GDS is deallocated when the
goal is removed.  A NIL WME pointer for the GDS indicates a goal has
no WMEs in its GDS.

\item Updating a WME GDS pointer \\ A WME should appear in only the
GDS of the highest goal for which it is dependent.  If a WME is
determined to already be in a GDS lower than the current goal, its GDS
pointer is updated to the higher goal, it is removed from the gds\_WME
DLL of the lower goal, and added to the higher one.  If there are no
other WMEs on the gds\_WME DLL of the lower goal, its WME pointer is
set to NIL (the GDS itself is retained, because we don't want to have
to reallocate memory for the GDS if we need to add to it later.)



\end{itemize}



%\bibliography{general,personal,soar}
%\bibliographystyle{acm}
 




% ----------------------------------------------------------------------------
% References
% ----------------------------------------------------------------------------
%\addcontentsline{toc}{chapter}{Bibliography}
%\bibliography{soarmanual} 

%---------------------------------------------------------------------------
%\vspace{\fill}
%\subsection*{Colophon}
%\addcontentsline{toc}{chapter}{Colophon}
%
%This document was produced on a Sun workstation using \LaTeX 2$_\epsilon$.
%Illustrations were created using idraw.
%
% ----------------------------------------------------------------------------
% Index
%   the file manual.idx is generated by latex; run 'makeindex manual' to
%     create the file manual.ind. However, this has a number of special
%     characters in it which will have to be put in verbose mode to be readable.
%   The characters that have to be changed are mostly at the top of the file;
%     you'll also have to look for all the underscores and change _ to \_ so
%     that latex won't choke.
%   Another option might be to do the whole index in typewriter font, but I
%     haven't tried this yet (maybe for draft versions, at least?). The carat
%     symbol still won't work (replace with \carat), but everything else
%     probably will.
%   In case it isn't obvious, generating the index is one of the last things
%     to do.
% ----------------------------------------------------------------------------
%\cleardoublepage
%\label{INDEX}

\addcontentsline{toc}{chapter}{Index}
\small
\twocolumn
\label(INDEX)
\printindex
\nocomment{Yes, I know this index is lousy. I didn't have time to finish the
	indexing process, and figured that it was better than nothing. If you
	disagree, just rip these pages out of your manual. :-)}
\onecolumn

% ----------------------------------------------------------------------------
% Function Summary
% ----------------------------------------------------------------------------
%\cleardoublepage
\addcontentsline{toc}{chapter}{Summary of Soar Aliases, Variables, and Functions}
%\markboth{SUMMARY OF SOAR FUNCTIONS}{SUMMARY OF SOAR FUNCTIONS} ADD BACK IN
%\def\leftmark{\textit{SUMMARY OF SOAR FUNCTIONS}}
%\def\rightmark{\textit{SUMMARY OF SOAR FUNCTIONS}}

% ----------------------------------------------------------------------------
\typeout{--------------- FUNCTION SUMMARY AND INDEX -------------------------}
%\pagestyle{empty}
\markboth{}{}
\section*{Summary of Soar Aliases and Functions}
\label{FUNCTIONS}
\label{func-sum}

% ----------------------------------------------------------------------------
\subsection*{Predefined Aliases}\vspace{-5pt}
%\newpage
\label{predefined-aliases}

There are a number of Soar ``commands'' that are shorthand for other Soar
commands: 

\begin{small}
\begin{tabular}{ l l r }
Alias  & Command & Page \\  \hline
\soar{?}  & \soar{help} & \pageref{help}\\
\soar{a}  & \soar{alias} & \pageref{alias}\\
\soar{aw} & \soar{add-wme} & \pageref{add-wme}\\
\soar{chdir} & \soar{cd} & \pageref{cd}\\
\soar{d}  & \soar{run -d 1} & \pageref{run}\\
\soar{dir} & \soar{ls} & \pageref{ls}\\
\soar{e}  & \soar{run -e 1} & \pageref{run}\\
\soar{eb} & \soar{explain-backtraces} & \pageref{explain-backtraces}\\
\soar{ex} & \soar{excise} & \pageref{excise}\\
\soar{exit} & \soar{quit} & \pageref{quit}\\
\soar{fc} & \soar{firing-counts} & \pageref{firing-counts}\\
\soar{gds\_print} & \soar{gds-print} & \pageref{gds-print}\\
\soar{h} & \soar{help} & \pageref{help}\\
\soar{inds} & \soar{indifferent-selection} & \pageref{indifferent-selection}\\
\soar{init} & \soar{init-soar} & \pageref{init-soar}\\
\soar{interrupt} & \soar{stop-soar} & \pageref{stop-soar}\\
\soar{is} & \soar{init-soar} & \pageref{init-soar}\\
\soar{l} & \soar{learn} & \pageref{learn}\\
\soar{man} & \soar{help} & \pageref{help}\\ 
\soar{p}  & \soar{print} & \pageref{print}\\
\soar{pc} & \soar{print --chunks} & \pageref{print}\\
\soar{pr} & \soar{preferences} & \pageref{preferences}\\
\soar{pw} & \soar{pwatch} & \pageref{pwatch}\\
\soar{rn} & \soar{rete-net} & \pageref{rete-net}\\
\soar{rw} & \soar{remove-wme} & \pageref{remove-wme}\\
\soar{set-default-depth} & \soar{default-wme-depth} & \pageref{default-wme-depth}\\ 
\soar{sn} & \soar{soarnews} & \pageref{soarnews}\\
\soar{ss} & \soar{stop-soar} & \pageref{stop-soar}\\
\soar{st} & \soar{stats} & \pageref{stats}\\
\soar{step} & \soar{run 1} & \pageref{run}\\ 
\soar{stop} & \soar{stop-soar} & \pageref{stop-soar}\\ 
\soar{topd} & \soar{pwd} & \pageref{pwd}\\
\soar{un} & \soar{alias -d} & \pageref{alias}\\
\soar{unalias} & \soar{alias -d} & \pageref{alias}\\
\soar{w}  & \soar{watch} & \pageref{watch}\\
\soar{wmes} & \soar{print -i} & \pageref{print}\\
\end{tabular}
\end{small} \vspace{24pt}

\newpage
% ----------------------------------------------------------------------------
\newpage
\subsection*{Summary of Soar Functions}

The following table lists the commands in Soar. See the referenced page number
for a complete description of each command.

\begin{small}
\begin{longtable}{ l p{8cm} r }
Command  & Summary & Page \\  \hline
\input{funclist}
\end{longtable}
\end{small}





\typeout{  }
\typeout{  }
\typeout{          MAKE SURE TO UPDATE FUNCTION SUMMARY AND INDEX }
\typeout{          BY EDITING manual.glo FILE INTO functions.glo }
\typeout{          AND RUN LATEX AGAIN.}
\typeout{  }
\typeout{  	   ALSO, OFTEN HAVE TO EDIT THE .toc FILE BY HAND TO }
\typeout{          FIX PROBLEMS WITH THE APPENDICES, AND TO TAKE OUT THE }
\typeout{          ARGUMENTS TO USER-INTERFACE COMMANDS.}
\typeout{  }

\end{document}
