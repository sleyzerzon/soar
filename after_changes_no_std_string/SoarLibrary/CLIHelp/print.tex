\subsection{\soarb{print}}
\label{print}
\index{print}
Print items in working memory or production memory. 
\subsubsection*{Synopsis}
print [-fFin] production_name
print -[a|c|D|j|u][fFin]
print [-i] [-d <depth>] \emph{identifier}
|\emph{timetag}
|\emph{pattern}
print -s[oS]
\end{verbatim}
\subsubsection*{Options}
\subsection*{Printing items in production memory}
\hline
\soar{\soar{\soar{ -a, --all }}} & print the names of all productions currently loaded  \\
\hline
\soar{\soar{\soar{ -c, --chunks }}} & print the names of all chunks currently loaded  \\
\hline
\soar{\soar{\soar{ -D, --defaults }}} & print the names of all default productions currently loaded  \\
\hline
\soar{\soar{\soar{ -f, --full }}} & When printing productions, print the whole production. This is the default when printing a named production.  \\
\hline
\soar{\soar{\soar{ -F, --filename }}} & also prints the name of the file that contains the production.  \\
\hline
\soar{\soar{\soar{ -i, --internal }}} & items should be printed in their internal form. For productions, this means leaving conditions in their reordered (rete net) form.  \\
\hline
\soar{\soar{\soar{ -j, --justifications }}} & print the names of all justifications currently loaded.  \\
\hline
\soar{\soar{\soar{ -n, --name }}} & When printing productions, print only the name and not the whole production. This is the default when printing any category of productions, as opposed to a named production.  \\
\hline
\soar{\soar{\soar{ -u, --user }}} & print the names of all user productions currently loaded  \\
\hline
\soar{\soar{\soar{production\_name}}} & print the production named production-name \\
\hline
\end{tabular}
\subsection*{Printing items in working memory}
\hline
 -d, --depth \emph{n}
 & This option overrides the default printing depth (see the default-wme-depth command for more detail).  \\
\hline
\soar{\soar{\soar{ -i, --internal }}} & items should be printed in their internal form. For working memory, this means printing the individual elements with their timetags, rather than the objects.  \\
\hline
\soar{\soar{\soar{ -v, --varprint }}} & Print identifiers enclosed in angle brackets.  \\
\hline
\emph{identifier}
 & print the object \emph{identifier}
. \emph{identifier}
 must be a valid Soar symbol such as \textbf{S1 }
\hline
\emph{pattern}
 & print the object whose working memory elements matching the given pattern. See Description for more information on printing objects matching a specific pattern.  \\
\hline
\emph{timetag}
 & print the object in working memory with the given \emph{timetag}
\hline
\end{tabular}
\subsection*{Printing the current subgoal stack}
\hline
\soar{\soar{\soar{ -s, --stack }}} & Specifies that the Soar goal stack should be printed. By default this includes both states and operators.  \\
\hline
\soar{\soar{\soar{ -o, --operators }}} & When printing the stack, print only \textbf{operators}
.  \\
\hline
\soar{\soar{\soar{ -S, --states }}} & When printing the stack, print only \textbf{states}
.  \\
\hline
\end{tabular}
\subsubsection*{Description}
 The \textbf{print}
(\emph{identifier}
 ^\emph{attribute value}
 [+])
\end{verbatim}
 The pattern is surrounded by parentheses. The \emph{identifier}
, \emph{attribute}
, and \emph{value}
 must be valid Soar symbols or the wildcard symbol * which matches all occurences. The optional \textbf{+}
 symbol restricts pattern matches to acceptable preferences. 
\subsubsection*{Examples}
print --internal (s1 ^* v2)
\end{verbatim}
print --stack
\end{verbatim}
print -if prodname
\end{verbatim}
print -u
\end{verbatim}
\subsubsection*{Default Aliases}
\hline
\soar{\soar{\soar{ Alias }}} & Maps to  \\
\hline
\soar{\soar{\soar{ p }}} & print  \\
\hline
\soar{\soar{\soar{ pc }}} & print --chunks  \\
\hline
\soar{\soar{\soar{ wmes }}} & print -i  \\
\hline
\end{tabular}
\subsubsection*{See Also}
\hyperref[default-wme-depth]{default-wme-depth} \hyperref[predefined-aliases]{predefined-aliases} 