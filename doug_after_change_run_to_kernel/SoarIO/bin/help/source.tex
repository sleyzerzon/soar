\documentclass[10pt]{article}
\usepackage{fullpage, graphicx, url}
\title{Source - Soar Wiki}
\begin{document}
\section*{Source}
\subsubsection*{From Soar Wiki}


 This is part of the Soar Command Line Interface. 
\section*{ Name }


 \textbf{source}
 - Load and evaluate the contents of a file. 


 Status: Complete
\section*{ Synopsis }
\begin{verbatim}
source filename

\end{verbatim}
\section*{ Options }


\begin{tabular}{|p{1in}|p{5in}|}
\hline 
filename & The file of Soar productions and commands to load.  \\
 \hline 

\end{tabular}



 \\ 

\section*{ Description }


 Load and evaluate the contents of a file. The \emph{filename}
 can be a relative path or a fully qualified path. \textbf{source}
 will generate an implicit push to the new directory, execute the command, and then pop back to the current working directory from which the command was issued. 
\section*{ See Also }
\begin{description}
cd dirs home ls pushd popd \textbf{source}
 topd

\end{description}
\section*{ Structured Output }
\subsection*{ On Success }
\begin{verbatim}
 <result output="raw">true</result>

\end{verbatim}
\subsection*{ Notes }
\begin{itemize}
\item  source executes the 'pwd' command if the pushd's and popd's don't match up after parsing the file. Type 'helpex pwd' for details. 

\end{itemize}
\section*{ Error Values }
\subsection*{ During Parsing }


 kTooFewArgs, kSourceOnlyOneFile
\subsection*{ During Execution }


 kAgentRequired, kOpenFileFail, kUnmatchedBrace, kExtraClosingBrace

\end{document}
