
\section{Low Level Control}

The design philosophy of SORTS requires that Soar agents only issue high
level commands comparable to those that would be expected from a human
player. The reason human players of commercial RTS games do not have to
worry (too much) about micromanagement is because much of it is handled
for them by the game engine. Likewise, in SORTS, we handle low-level
micromanagement with our middleware so that Soar does not have to. This
is achieved by assigning each controllable unit a behavior specified
as an finite state machine. Soar only has to specify which behavior a
unit should take on, and the FSM will control the unit in predictable
ways. We have already programmed a basic library of behaviors, listed in
figure~\ref{fig:implementedBehaviors}.

\subsection{Behaviors as Finite State Machines}

The main functionality of a behavior resides in its {\tt update}
function. This function is called for all living, controllable units
whenever the higher level function \verb|updateGroups()| is called in
the ORTS event handler (see section~\ref{sec:OrtsEventHandler}). This
means that the {\tt update} function for every active behavior is called
once everytime SORTS receives a viewframe from the ORTS server, which is
the smallest unit of discernable state change.

During the {\tt update} call, the behavior decides which low-level action
provided by the ORTS API, or lack thereof, to execute for the game
object which it controls. This decision is made based on the behavior's
current state and also the state of the game object, which it as direct
access to, unlike the Soar agent. The behavior also decides which state
to transition to for the next {\tt update} call.

\subsubsection{Coordination Managers}

During preparations for the first ORTS tournament, it was decided that
certain behaviors that required high levels of coordination amongst
each other needed to have access to more information than just game
object state if they were to perform well enough to be competitive in
the tournament. Therefore, two behaviors, one that controls mining and
one that controls attacking, were written to have access to a bigger
picture of the overall game state via communication with coordinating
mechanisms called we called "managers."

The result was that the behaviors themselves were relatively simple.
During every {\tt update} call, a managed behavior just asks the manager
it was assigned to what it should do, and then does it. The advantage to
using such an approach is that the manager can keep track of what each
behavior it controls is doing, and coordinate sophisticated strategies
such as optimal assignment of miners to mineral patches, or focusing the
firepower of many units on one enemy at a time.

\subsubsection{Default Behaviors}

There are some actions that are always advantageous to do. For example,
if an offensive unit passes within firing range of an enemy unit, it is
almost always better to attack the enemy unit given that there is not a
more important target to attack. We don't want to burden a Soar agent
with having to detect every opportunity to perform such actions, so
game units can be assigned default behaviors. The desired behavior in
the situation described above is implemented by the {\tt AttackNearFSM}
default behavior, which fires on nearby targets whenever they are in
range and the controlled unit was not explicitly ordered to fire on
something else.

As opposed to assigned behaviors, default behaviors are specified
when units are created rather than when they are first desired, and
a single in-game unit can have several default behaviors at once.
It is important that a default behavior not override the actions
of an assigned behavior. For example, the default behavior for a
unit to run away when being attacked should not be manifested if the
unit was assigned the move behavior in the opposite direction of the
retreat. However, there is no built-in mechanism to ensure this kind of
suppression in SORTS yet, so it is the responsibility of the default
behavior to make sure it doesn't obstruct assigned behaviors.

\begin{figure}
\begin{center}
\begin{minipage}{5in}
\begin{tabular}{|l|p{1.0in}|p{1.0in}|}
Behavior name     & Parameters     & Description\\ 
\hline\hline
AttackFSM         & id             & Attack group with specified id \\ 
\hline
AttackNearFSM     &                & Attack any enemy target within range,
                                     except when already attacking another
                                     target                         \\
\hline
BuildFSM          & t, x, y        & Build a building of type t at (x, y) \\ 
\hline
MineFSM           &                & Mine for minerals              \\
\hline
MoveFSM           & x, y, speed    & Object tries to pathfind and move
                                     to coordinate (x, y)           \\
\hline
TrainFSM          & t              & Train a unit of type t         \\
\hline

\end{tabular}
\end{minipage}
\end{center}
\caption{A list of implemented behaviors.}
\label{fig:implementedBehaviors}
\end{figure}

