% ----------------------------------------------------------------------------
\typeout{--------------- appendix: GRAMMARS for productions ------------------}
\chapter{Grammars for production syntax}
\label{GRAMMARS}
\index{grammar}

This appendix contains the BNF grammars for the conditions and actions of
productions. (BNF stands for Backus-Naur form or Backus normal form; consult a
computer science book on theory, programming languages, or compilers for more
information. However, if you don't already know what a BNF grammar is, it's
unlikely that you have any need for this appendix.)

This information is provided for advanced Soar users, for example, those who
need to write their own parsers.

\comment{this section still needs a disclaimer that what you can actually do
	is less restrictive than the way we described it in the main text } 

\comment{note that grammars are no longer consistent with new rhs actions}

\nocomment{John and I decided while talking about this that we just wouldn't let
	people know that they could omit the identifier of the state

	It is legal to omit the variable test for a state when that variable is not
	tested elsewhere in the production, nor used in the action.  For
	example: 
	\begin{verbatim}
	(state ^operator <o>)
	\end{verbatim}

	is equivalent to 
	\begin{verbatim}
	(state <s> ^operator <o>)
	\end{verbatim}
	}

%-------------------------------------------------------
\section{Grammar of Soar productions}

A grammar for Soar productions is:
\begin{verbatim}
<soar-production>  ::= sp "{" <production-name> [<documentation>] [<flags>]
                     <condition-side> --> <action-side> "}"
<documentation>    ::= """ [<string>] """
<flags>            ::= ":" (o-support | i-support | chunk | default)
\end{verbatim}

% ----------------------------------------------------------------------------
\subsection{Grammar for Condition Side}
\label{SYNTAX-pm-condgrammar}
\index{condition-side grammar}
\index{grammar, condition side}


Below is a grammar for the condition sides of productions:
\begin{verbatim}
<condition-side>   ::= <state-imp-cond> <cond>*
<state-imp-cond>   ::= "(" (state | impasse) [<id_test>]
                     <attr_value_tests>+ ")"
<cond>             ::= <positive_cond> | "-" <positive_cond>
<positive_cond>    ::= <conds_for_one_id> | "{" <cond>+ "}"
<conds_for_one_id> ::= "(" [(state|impasse)] <id_test> 
                     <attr_value_tests>+ ")"
<id_test>          ::= <test>
<attr_value_tests> ::= ["-"] "^" <attr_test> ("." <attr_test>)*
                     <value_test>*
<attr_test>        ::= <test>
<value_test>       ::= <test> ["+"] | <conds_for_one_id> ["+"]  

<test>             ::= <conjunctive_test> | <simple_test>
<conjunctive_test> ::= "{" <simple_test>+ "}"
<simple_test>      ::= <disjunction_test> | <relational_test>
<disjunction_test> ::= "<<" <constant>+ ">>"
<relational_test>  ::= [<relation>] <single_test>
<relation>         ::= "<>" | "<" | ">" | "<=" | ">=" | "=" | "<=>"
<single_test>      ::= <variable> | <constant>
<variable>         ::= "<" <sym_constant> ">"
<constant>         ::= <sym_constant> | <int_constant> | <float_constant>
\end{verbatim}
\index{constant}
\index{variable}

\subsubsection*{Notes on the Condition Side}\vspace{-12pt}
\begin{itemize}
\item In an \soar{<id\_test>}, only a \soar{<variable>} may be used in a \soar{<single\_test>}.
\end{itemize}

\comment{I don't think that grammar is quite right -- e.g. should distinguish
        that acceptable preferences may appear for operators, but not other
        objects}

\comment{Grammar correctly describes Soar; it's just that you can actually do
	things that we've said can't be done. So in this section we'll mention
	that we lied before and that the grammar above is different, but
	correct.  see notes on difference on page 64 of June 7th draft}


% ----------------------------------------------------------------------------
\subsection{Grammar for Action Side}
\label{SYNTAX-pm-actgrammar}    %RHS grammar}
\index{action-side grammar}
\index{grammar, action side}

\comment{RD: this grammar is out of date}

Below is a grammar for the action sides of productions:
\begin{verbatim}
<rhs>                      ::= <rhs_action>*
<rhs_action>               ::= "(" <variable> <attr_value_make>+ ")" 
                             | <func_call>
<func_call>                ::= "(" <func_name> <rhs_value>* ")"
<func_name>                ::= <sym_constant> | "+" | "-" | "*" | "/"
<rhs_value>                ::= <constant> | <func_call> | <variable>
<attr_value_make>          ::= "^" <variable_or_sym_constant>
                             ("." <variable_or_sym_constant>)* <value_make>+
<variable_or_sym_constant> ::= <variable> | <sym_constant>
<value_make>               ::= <rhs_value> <preference_specifier>*

<preference-specifier>     ::= <unary-preference> [","]
                             | <unary-or-binary-preference> [","]
                             | <unary-or-binary-preference> <rhs_value> [","]
<unary-pref>               ::= "+" | "-" | "!" | "~" | "@"
<unary-or-binary-pref>     ::= ">" | "=" | "<" | "&"
\end{verbatim}

\comment{I don't quite understand that last bit. 
<forced-unary-pref>        ::= <binary-preference> {, | ) | ^}
       (but the parser doesn't consume the ")" or "^" here)}

\index{constant}
\index{variable}
