\subsection{\soarb{memories}}
\label{memories}
\index{memories}
Print memory usage for partial matches. 
 Status: Complete
\subsubsection*{Synopsis}
\begin{verbatim}
memories [-cdju] [\emph{n}
]
memories production_name 
\end{verbatim}
\subsubsection*{Options}
\begin{tabular}{|l|l|}
\hline 
 -c, --chunks  & Print memory usage of chunks.  \\
 \hline 
 -d, --default  & Print memory usage of default productions.  \\
 \hline 
 -j, --justifications  & Print memory usage of justifications.  \\
 \hline 
 -u, --user  & Print memory usage of user-defined productions.  \\
 \hline 
production\_name & Print memory usage for a specific production.  \\
 \hline 
\emph{n}
 & Number of productions to print, sorted by those that use the most memory.  \\
 \hline 
\end{tabular}
\subsubsection*{Description}
 is given, only \emph{n}
 productions will be printed: the \emph{n}
 productions that use the most memory. Output may be restricted to print memory usage for particular types of productions using the command options. 
 Memory usage is recorded according to the tokens that are allocated in the rete network for the given production(s). This number is a function of the number of elements in working memory that match each production. Therefore, this command will not provide useful information at the beginning of a Soar run (when working memory is empty) and should be called in the middle (or at the end) of a Soar run. 
 As a rule of thumb, numbers less than 100 mean that the production is using a small amount of memory, numbers above 1000 mean that the production is using a large amount of memory, and numbers above 10,000 mean that the production is using a \emph{very}
 large amount of memory. 
\subsubsection*{Examples}
 To show how to use the command in context, do this: \begin{verbatim}
command --option arg
\end{verbatim}
 and possibly explain the results. 
\subsubsection*{See Also}
 matches
