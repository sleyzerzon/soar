\subsection{\soarb{stop-soar}}
\label{stop-soar}
\index{stop-soar}
Pause Soar. 
 Status: Complete\\ 
Reason for stopping currently ignored, not sure what this is for/why this is here.--Jonathan 13:59, 4 Feb 2005 (EST) 
\subsubsection*{Synopsis}
\begin{verbatim}
stop-soar [-s] [reason string]
\end{verbatim}
\subsubsection*{Options}
\begin{tabular}{|l|l|}
\hline 
 -s, --self  & Stop only the soar agent where the command is issued. All other agents continue running as previously specified.  \\
 \hline 
 reason\_string  & An optional string which will be printed when Soar is stopped, to indicate why it was stopped. If left blank, no message will be printed when Soar is stopped.  \\
 \hline 
\end{tabular}
\subsubsection*{Description}
 This command is usually not issued at the command line prompt. A more common use of this command is as a side-effect of pressing a button on a Graphical User Interface (GUI), or as a monitor to be executed at a specific Soar Event. For example, a user may wish to examine an agent's ``matches'' after the Soar Decision Phase. In order to do this in Soar 8, the user must register a monitor, or callback, to issue the ``stop-soar -self'' command for the event, after-decision-phase-cycle. 
\subsubsection*{Examples}
\subsubsection*{Default Aliases}
\begin{tabular}{|l|l|}
\hline 
 Alias  & Maps to  \\
 \hline 
 stop  & stop-soar  \\
 \hline 
 interrupt  & stop-soar  \\
 \hline 
\end{tabular}
\subsubsection*{See Also}
 run monitor matches
\subsubsection*{Warnings}
 If the graphical interface doesn't periodically do an ``update'' of flush the pending I/O, then it may not be possible to interrupt a Soar agent from the command line. 
