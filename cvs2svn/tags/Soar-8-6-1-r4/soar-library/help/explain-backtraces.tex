\subsection{\soarb{explain-backtraces}}
\label{explain-backtraces}
\index{explain-backtraces}
Print information about chunk and justification backtraces. 
 3 Incomplete EvilBackDoor ResultByKernel
\subsubsection*{Synopsis}
\begin{verbatim}
explain-backtraces -f prod_name
explain-backtraces [-c <n>] prod_name
\end{verbatim}
\subsubsection*{Options}
\begin{tabular}{|l|l|}
\hline 
 (no args)  & List all productions that can be ``explained''  \\
 \hline 
 prod\_name  & List all conditions and grounds for the chunk or justification.  \\
 \hline 
 -c, --condition  & Explain why condition number \emph{n}
 is in the chunk or justification.  \\
 \hline 
 -f, --full  & Print the full backtrace for the named production  \\
 \hline 
\end{tabular}
\subsubsection*{Description}
 This command provides some interpretation of backtraces generated during chunking. 
 The two most useful variants are: \begin{verbatim}
explain-backtraces prodname 
explain-backtraces -c n prodname
\end{verbatim}
 The first variant prints a numbered list of all the conditions for the named chunk or justification, and the ground which resulted in inclusion in the chunk/justification. A \emph{ground}
 is a working memory element (WME) which was tested in the supergoal. Just knowing which WME was tested may be enough to explain why the chunk/justification exists. If not, the second variant, \textbf{explain-backtraces -c n prodname}
, where \emph{n}
 is the condition of interest, can be used to obtain a list of the productions which fired to obtain this condition in the chunk/justification (and the crucial WMEs tested along the way). 
 \textbf{save-backtraces}
 mode must be on when a chunk or justification is created or no explanation will be available. Calling \textbf{explain-backtraces}
 with no argument prints a list of all chunks and justifications for which backtracing information is available. 
\subsubsection*{Examples}
 Examining the chunk \textbf{chunk-65*d13*tie*2}
 generated in a water-jug task:  \begin{verbatim}
soar> explain-backtraces chunk-65*d13*tie*2
 (sp chunk-65*d13*tie*2
  (state <s2> ^name water-jug ^jug <n4> ^jug <n3>)
  (state <s1> ^name water-jug ^desired <d1> ^operator <o1> + ^jug <n1>
        ^jug <n2>)
  (<s2> ^desired <d1>)
  (<o1> ^name pour ^into <n1> ^jug <n2>)
  (<n1> ^volume 3 ^contents 0)
  (<s1> ^problem-space <p1>)
  (<p1> ^name water-jug)
  (<n4> ^volume 3 ^contents 3)
  (<n3> ^volume 5 ^contents 0)
  (<n2> ^volume 5 ^contents 3)
-->
  (<s3> ^operator <o1> -))
 1�:  (state <s2> ^name water-jug)     Ground�: (S3 ^name water-jug)
 2�:  (state <s1> ^name water-jug)     Ground�: (S5 ^name water-jug)
 3�:  (<s1> ^desired <d1>)             Ground�: (S5 ^desired D1)
 4�:  (<s2> ^desired <d1>)             Ground�: (S3 ^desired D1)
 5�:  (<s1> ^operator <o1> +)          Ground�: (S5 ^operator O18 +)
 6�:  (<o1> ^name pour)                Ground�: (O18 ^name pour)
 7�:  (<o1> ^into <n1>)                Ground�: (O18 ^into N3)
 8�:  (<n1> ^volume 3)                 Ground�: (N3 ^volume 3)
 9�:  (<n1> ^contents 0)               Ground�: (N3 ^contents 0)
10�:  (<s1> ^jug <n1>)                 Ground�: (S5 ^jug N3)
11�:  (<s1> ^problem-space <p1>)       Ground�: (S5 ^problem-space P3)
12�:  (<p1> ^name water-jug)           Ground�: (P3 ^name water-jug)
13�:  (<s2> ^jug <n4>)                 Ground�: (S3 ^jug N1)
14�:  (<n4> ^volume 3)                 Ground�: (N1 ^volume 3)
15�:  (<n4> ^contents 3)               Ground�: (N1 ^contents 3)
16�:  (<s2> ^jug <n3>)                 Ground�: (S3 ^jug N2)
17�:  (<n3> ^volume 5)                 Ground�: (N2 ^volume 5)
18�:  (<n3> ^contents 0)               Ground�: (N2 ^contents 0)
19�:  (<s1> ^jug <n2>)                 Ground�: (S5 ^jug N4)
20�:  (<n2> ^volume 5)                 Ground�: (N4 ^volume 5)
21�:  (<n2> ^contents 3)               Ground�: (N4 ^contents 3)
22�:  (<o1> ^jug <n2>)                 Ground�: (O18 ^jug N4)
\end{verbatim}
 Further examining condition 21:  \begin{verbatim}
soar> explain-backtraces -c 21 chunk-65*d13*tie*2
Explanation of why condition  (N4 ^contents 3) was included in chunk-65*d13*tie*2
Production chunk-64*d13*opnochange*1 matched
    (N4 ^contents 3) which caused
production selection*select*failure-evaluation-becomes-reject-preference to match
    (E3 ^symbolic-value failure) which caused
A result to be generated.
\end{verbatim}
\subsubsection*{See Also}
\hyperref[save-backtraces]{save-backtraces} 