\subsection{\soarb{max-chunks}}
\label{max-chunks}
\index{max-chunks}
Limit the number of chunks created during a decision cycle. 
 Priority: 3�; Status: Complete
\subsubsection*{Synopsis}
\begin{verbatim}
max-chunks [n]
\end{verbatim}
\subsubsection*{Options}
\begin{tabular}{|l|l|}
\hline 
 n  & Maximum number of chunks allowed during a decision cycle.  \\
 \hline 
\end{tabular}
\subsubsection*{Description}
 The max-chunks command is used to limit the maximum number of chunks that may be created during a decision cycle. The initial value of this variable is 50; allowable settings are any integer greater than 0. 
\subsubsection*{Structured Output:}
\paragraph*{On Query}
\begin{verbatim}
<result>
  <arg name="value" type="int">max_chunks</arg>
</result>
\end{verbatim}
\paragraph*{Otherwise}
\begin{verbatim}
<result output="raw">true</result>
\end{verbatim}
\subsubsection*{Error Values:}
\paragraph*{During Parsing}
 kTooManyArgs, kIntegerMustBePositive
\paragraph*{During Execution}
 kAgentRequired
