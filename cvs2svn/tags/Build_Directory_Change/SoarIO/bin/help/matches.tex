\subsection{\soarb{matches}}
\label{matches}
\index{matches}
Prints information about partial matches and the match set. 
 Priority: 1�; Status: Incomplete, EvilBackDoor\\ 
Result generated by kernel.--Jonathan 12:18, 7 Feb 2005 (EST) 
\subsubsection*{Synopsis}
\begin{verbatim}
matches [-nc0t1w2] production name
matches -[a|r] [-nc0t1w2]
\end{verbatim}
\subsubsection*{Options}
\begin{tabular}{|l|l|}
\hline 
production\_name & Print partial match information for the named production.  \\
 \hline 
 -0, -n, --names, -c, --count  & For the match set, print only the names of the productions that are about to fire or retract (the default). If printing partial matches for a production, just list the partial match counts.  \\
 \hline 
 -1, -t, --timetags  & Also print the timetags of the wmes at the first failing condition  \\
 \hline 
 -2, -w, --wmes  & Also print the full wmes, not just the timetags, at the first failing condition.  \\
 \hline 
 -a, --assertions  & List only productions about to fire.  \\
 \hline 
 -r, --retractions  & List only productions about to retractions.  \\
 \hline 
\end{tabular}
\subsubsection*{Description}
 The matches command prints a list of productions that have instantiations in the match set, i.e., those productions that will retract or fire in the next Propose or Apply phase. It also will preint partial match information for a single, named production. 
\section*{ Description from Soar 8.3 }
 This command prints partial match information for a selected production. If issued at the end of the Decision Phase, it also prints the current set of productions that are about to fire or retract in the next APPLY phase. This information is useful for determining what changes to working memory are necessary in order to achieve a match of the left-hand sides of productions. If no production name is given, then it is assumed that the user desires information about the entire match set, including both assertions and retractions. In Soar 8, the Decision Phase occurs in the middle of the Decision Cycle, followed by the Apply Phase. Therefore, at the end of the Decision Cycle, productions will have already fired and the ``matches'' command will not be very helpful. To examine the match information in immediately after the Decision Phase in Soar 8, users can implement either of the following callbacks: \begin{verbatim}
         monitor -add {matches -wmes} after-decision-phase-cycle
\end{verbatim}
 \begin{verbatim}
         monitor -add {stop-soar -self} after-decision-phase-cycle
\end{verbatim}
 The first example can use any options to the matches command, whatever the user finds most helpful. The second option, which stops Soar at the end of the Decision Phase, offers the most flexibility for debugging. It is recommended that users issue this callback for all agents, so they don't get out of synch when running. Once the productions have been debugged, the monitors can be deleted. 
\subsubsection*{Examples}
 This example prints the productions which are about to fire and the wmes that match the productions on their left-hand sides: \begin{verbatim}
matches --assertions --wmes
\end{verbatim}
 This example prints the wme timetags for a single production. \begin{verbatim}
matches -t my*first*production</code.
\end{verbatim}
\subsubsection*{See Also}
 monitor
\subsubsection*{Structured Output:}
 Result is a formatted string generated by kernel, this needs to be re-done.--Jonathan 12:18, 7 Feb 2005 (EST) 
\paragraph*{On Success}
\begin{verbatim}
<result>
  <arg param="message" type="string">output_string</arg>
</result>
\end{verbatim}
\subsubsection*{Error Values:}
\paragraph*{During Parsing}
 kUnrecognizedOption, kGetOptError, kTooManyArgs
\paragraph*{During Execution}
 kInvalidWMEDetail, kProductionNotFound
