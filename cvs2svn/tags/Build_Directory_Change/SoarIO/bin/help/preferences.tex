\subsection{\soarb{preferences}}
\label{preferences}
\index{preferences}
Examine details about the preferences that support the specified \emph{id}
 and \emph{attribute}
. 
 Priority: 2; Status: Incomplete, EvilBackDoor\\ 
Result generated by kernel.--Jonathan 15:45, 18 Feb 2005 (EST) 
\subsubsection*{Synopsis}
\begin{verbatim}
preferences [-0123nNtw] [id] [[^]attribute]
\end{verbatim}
\subsubsection*{Options}
\begin{tabular}{|l|l|}
\hline 
 -0, -n, --none  & Print just the preferences themselves  \\
 \hline 
 -1, -N, --names  & Print the preferences and the names of the productions that generated them  \\
 \hline 
 -2, -t, --timetags  & Print the information for the --names option above plus the timetags of the wmes matched by the indicated productions  \\
 \hline 
 -3, -w, --wmes  & Print the information for the --timetags option above plus the entire wme.  \\
 \hline 
id & Must be an existing Soar object identifier.  \\
 \hline 
attribute & Must be an existing \emph{\^{}attribute}
 of the specified identifier.  \\
 \hline 
\end{tabular}
\subsubsection*{Description}
 This command prints all the preferences for the given object id and attribute. If \emph{id}
 and \emph{attribute}
 are not specified, they default to the current state and the current operator. The '\^{}' is optional when specifying the attribute. The optional arguments indicates the level of detail to print about each preference. 
\subsubsection*{Examples}
 This example prints the preferences on the (S1 \^{}operator) and the production names which created the preferences: \begin{verbatim}
preferences S1 operator --names
\end{verbatim}
 if the current state is S1, then the above syntax is equivalent to: \begin{verbatim}
 preferences -n
\end{verbatim}
\subsubsection*{See Also}
\subsubsection*{Structured Output:}
 preferences returns formatted output in a string, this needs to be re-done.--Jonathan 15:44, 18 Feb 2005 (EST) 
\paragraph*{On Success}
\begin{verbatim}
<result>
  <arg param="message" type="string">output_string</arg>
</result>
\end{verbatim}
\paragraph*{Notes}
\subsubsection*{Error Values:}
\paragraph*{During Parsing}
 kUnrecognizedOption, kGetOptError, kTooManyArgs
\paragraph*{During Execution}
 kAgentRequired, kKernelRequired, kgSKIError
