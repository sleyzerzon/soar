\documentclass[10pt]{article}
\usepackage{fullpage, graphicx, url}
\title{Add-wme - Soar Wiki}
\begin{document}
\section*{Add-wme}
\subsubsection*{From Soar Wiki}


 This is part of the Soar Command Line Interface. 
\section*{ Name }


 \textbf{add-wme}
 - Manually add an element to working memory. 


 Status: Complete, EvilBackDoor
\section*{ Synopsis }


  \begin{verbatim}
add-wme id [^]attribute value [+]

\end{verbatim}



 
\section*{ Options }


\begin{tabular}{|p{1in}|p{5in}|}
\hline 
 id  & Must be an existing identifier.  \\
 \hline 
 \^{}  & Leading \^{} on attribute is optional.  \\
 \hline 
 attribute  & Attribute can be any Soar symbol. Use * to have Soar create a new identifier.  \\
 \hline 
 value  & Value can be any soar symbol. Use * to have Soar create a new identifier.  \\
 \hline 
 +  & If the optional preference is specified, its value must be + (acceptable).  \\
 \hline 

\end{tabular}



 \\ 

\section*{ Description }


 Manually add an element to working memory. add-wme is often used by an input function to update Soar's information about the state of the external world. 


 add-wme adds a new wme with the given id, attribute, value and optional preference. The given id must be an existing identifier. The attribute and value fields can be any Soar symbol. If * is given in the attribute or value field, Soar creates a new identifier (symbol) for that field. If the preference is given, it can only have the value + to indicate that an acceptable preference should be created for this wme. 
\section*{ Examples }


 This example adds the attribute/value pair ``message-status received'' to the identifier (symbol) S1: \begin{verbatim}
 add-wme S1 ^message-status received

\end{verbatim}



 \\ 
 This example adds an attribute/value pair with an acceptable preference to the identifier (symbol) Z2. The attribute is ``message'' and the value is a unique identifier generated by Soar. Note that since the \^{} is optional, it has been left off in this case. \begin{verbatim}
 add-wme Z2 message * + 

\end{verbatim}

\section*{ Warnings }


 Be careful how you use this command. It may have weird side effects (possibly even including system crashes). For example, the chunker can't backtrace through wmes created via add-wme, nor will such wmes ever be removed thru Soar's garbage collection. Manually removing context/impasse wmes may have unexpected side effects. 
\section*{ See Also }


 remove-wme


 \\ 

\section*{ Structured Output }
\subsection*{ On Success }


 param = value, type = int, the value is the timetag of the new wme 


 Perhaps this should be param = timetag? type = unsigned long?--Jonathan 13:53, 22 Feb 2005 (EST) 
\subsection*{ Notes }
\section*{ Error Values }
\subsection*{ During Parsing }


 kTooFewArgs, kTooManyArgs, kCustomError (``Expected acceptable preference (+) or nothing'') 
\subsection*{ During Execution }


 kAgentRequired, kKernelRequired, kInvalidID, kInvalidAttribute, kInvalidValue

\end{document}
