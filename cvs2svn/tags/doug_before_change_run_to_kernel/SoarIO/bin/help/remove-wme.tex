\documentclass[10pt]{article}
\usepackage{fullpage, graphicx, url}
\title{Remove-wme - Soar Wiki}
\begin{document}
\section*{Remove-wme}
\subsubsection*{From Soar Wiki}


 This is part of the Soar Command Line Interface. 
\section*{ Name }


 \textbf{remove-wme}
 - Manually remove an element from working memory. 


 Status: Complete
\section*{ Synopsis }
\begin{verbatim}
remove-wme \emph{timetag}


\end{verbatim}
\section*{ Options }


\begin{tabular}{|p{1in}|p{5in}|}
\hline 
 timetag  & A positive integer matching the timetag of an existing working memory element.  \\
 \hline 

\end{tabular}



 \\ 

\section*{ Description }


 The remove-wme command removes the working memory element with the given timetag. This command is provided primarily for use in Soar input functions; although there is no programming enforcement, add-wme should only be called from registered input functions to create working memory elements on Soar's input link. 


 Beware of weird side effects, including system crashes. 
\section*{ Examples }
\section*{ See Also }


 add-wme io
\section*{ Warnings }


 remove-wme should never be called from the RHS: if you try to match a wme on the LHS of a production, and then remove the matched wme on the RHS, Soar will crash. 


 If used other than by input and output functions interfaced with Soar, this command may have weird side effects (possibly even including system crashes). Removing input wmes or context/impasse wmes may have unexpected side effects. You've been warned. 
\section*{ Structured Output }


 Returns true on success 
\section*{ Error Values }
\subsection*{ During Parsing }


 kTooFewArgs, kTooManyArgs, kIntegerMustBePositive
\subsection*{ During Execution }


 kAgentRequired, kKernelRequired, kRemoveWMEFailed

\end{document}
