\documentclass[10pt]{article}
\usepackage{fullpage, graphicx, url}
\title{Watch - Soar Wiki}
\begin{document}
\section*{Watch}
\subsubsection*{From Soar Wiki}


 This is part of the Soar Command Line Interface. 
\section*{ Name }


 \textbf{watch}
 - Control the run-time tracing of Soar. 


 Status: Complete, EvilBackDoor
\section*{ Synopsis }
\begin{verbatim}
watch
watch [--level] [0|1|2|3|4|5]
watch -N
watch -[dpPwrDujcbi] [<remove>] -[n|t|f]
watch --learning [<print|noprint|fullprint>]

\end{verbatim}
\section*{ Options }


\begin{tabular}{|p{1in}|p{1in}|p{4in}|}
\hline 
\emph{Option Flag}
 &\emph{Argument to Option}
 &\emph{Description}
 \\
 \hline 
 -l, --level  & 0 to 5 (see \textbf{Watch Levels}
 below)  & This flag is optional but recommended. Set a specific watch level using an integer 0 to 5, this is an inclusive operation  \\
 \hline 
 -N, --none  & No argument  & Turns off all printing about Soar's internals, equivalent to --level 0  \\
 \hline 
 -d, --decisions  & remove (optional, see \textbf{Remove}
 below)  & Controls whether state and operator decisions are printed as they are made  \\
 \hline 
 -p, --phases  & remove (optional, see \textbf{Remove}
 below)  & Controls whether decisions cycle phase names are printed as Soar executes  \\
 \hline 
 -P, --productions  & remove (optional, see \textbf{Remove}
 below)  & Controls whether the names of productions are printed as they fire and retract, equivalent to -Dujc  \\
 \hline 
 -w, --wmes  & remove (optional, see \textbf{Remove}
 below)  & Controls the printing of working memory elements that are added and deleted as productions are fired and retracted  \\
 \hline 
 -r, --preferences  & remove (optional, see \textbf{Remove}
 below)  & Controls whether the preferences generated by the traced productions are printed when those productions fire or retract  \\
 \hline 
 -D, --default  & remove (optional, see \textbf{Remove}
 below)  & Control only default-productions as they fire and retract  \\
 \hline 
 -u, --user  & remove (optional, see \textbf{Remove}
 below)  & Control only user-productions as they fire and retract  \\
 \hline 
 -c, --chunks  & remove (optional, see \textbf{Remove}
 below)  & Control only chunks as they fire and retract  \\
 \hline 
 -j, --justifications  & remove (optional, see \textbf{Remove}
 below)  & Control only justifications as they fire and retract  \\
 \hline 
 -n, --nowmes  & No argument  & When watching productions, do not print any information about wmes as they are added or retracted  \\
 \hline 
 -t, --timetags  & No argument  & When watching productions, print only the timetags for wmes as they are added or retracted  \\
 \hline 
 -f, --fullwmes  & No argument  & When watching productions, print the full wmes as they are added or retracted  \\
 \hline 
 -b, --backtracing  & remove (optional, see \textbf{Remove}
 below)  & Controls the printing of backtracing information when a chunk or justification is created  \\
 \hline 
 -i, --indifferent-selection  & remove (optional, see \textbf{Remove}
 below)  & Controls the printing of the scores for tied operators in random indifferent selection mode  \\
 \hline 
 -L, --learning  & noprint, print, or fullprint (see \textbf{Learning}
 below)  & Controls the printing of chunks/justifications as they are created  \\
 \hline 

\end{tabular}

\subsection*{ Watch Levels }


 Use of the --level (-l) flag is optional but recommended. 

\begin{tabular}{|p{1in}|p{5in}|}
\hline 
 0  & watch nothing; equivalent to \^a��N  \\
 \hline 
 1  & watch decisions; equivalent to -d  \\
 \hline 
 2  & watch phases and decisions; equivalent to -dp  \\
 \hline 
 3  & watch productions, phases, and decisions; equivalent to -dpP  \\
 \hline 
 4  & watch wmes, productions, phases, and decisions; equivalent to -dpPw  \\
 \hline 
 5  & watch preferences, wmes, productions, phases, and decisions; equivalent to -dpPwr  \\
 \hline 

\end{tabular}




 \\ 
 It is important to note that watch level 0 turns off ALL watch options, including backtracing, indifferent selection and learning. However, the other watch levels do not change these settings. That is, if any of these settings is changed from its default, it will retain its new setting until it is either explicitly changed again or the watch level is set to 0. 
\subsection*{ Remove }


 The remove argument has a numeric alias; you can use 0 for remove. A mix of formats is acceptable, even in the same command line. 

\begin{tabular}{|p{1in}|p{1in}|p{4in}|}
\hline 
 remove  & 0  & Turn watching off only for the specified option  \\
 \hline 

\end{tabular}


\subsection*{ Learning }


 The learning options have numeric aliases; you can use 0 for noprint, 1 for print, and 2 for fullprint. A mix of formats is acceptable, even in the same command line. 

\begin{tabular}{|p{1in}|p{1in}|p{4in}|}
\hline 
 noprint  & 0  & Print nothing about new chunks or justifications (default)  \\
 \hline 
 print  & 1  & Print the names of new chunks and justifications when created  \\
 \hline 
 fullprint  & 2  & Print entire chunks and justifications when created  \\
 \hline 

\end{tabular}




 \\ 

\section*{ Description }


 The watch command controls run-time tracing of Soar. With no arguments, this command prints out the current watch status. The various levels are used to modify the current watch settings. Each level can be indicated with either a number or a series of flags as follows: \begin{verbatim}
0 or --none
1 or --decisions
2 or --decisions --phases
3 or --decisions --phases --productions
4 or --decisions --phases --productions --wmes
5 or --decisions --phases --productions --wmes --preferences

\end{verbatim}



 The numerical arguments \emph{inclusively}
 turn on all levels up to the number specified. To use numerical arguments to turn off a level, specify a number which is less than the level to be turned off. For instance, to turn off watching of productions, specify ``watch --level 2'' (or 1 or 0). Numerical arguments are provided for shorthand convenience. For more detailed control over the watch settings, the named arguments should be used. 


 For the named arguments, including the named argument turns on only that setting. To turn off a specific setting, follow the named argument with \emph{remove}
 or \emph{0}
. 


 The named argument --productions is shorthand for the four arguments --default, --user, --justifications, and --chunks. 


 The pwatch command is used to watch individual productions specified by name rather than watch a type of productions, such as --user. 
\section*{ Examples }


 The most common uses of watch are by using the numeric arguments which indicate watch levels. To turn off all printing of Soar internals, do any one of the following (not all possibilities listed): \begin{verbatim}
watch --level 0
watch -l 0
watch -N

\end{verbatim}



 Although the --level flag is optional, its use is recommended: \begin{verbatim}
watch --level 5 \emph{... OK}

watch 5         \emph{... OK, but try to avoid}


\end{verbatim}



 Be careful of where the level is on the command line, for example, if you want level 2 and preferences: \begin{verbatim}
watch -r -l 2 \emph{... Incorrect: -r flag ignored, level 2 parsed after it and overrides the setting}

watch -r 2    \emph{... Syntax error: 0 or remove expected as optional argument to -r}

watch -r -l 2 \emph{... Incorrect: -r flag ignored, level 2 parsed after it and overrides the setting}

watch 2 -r    \emph{... OK, but try to avoid}

watch -l 2 -r \emph{... OK}


\end{verbatim}



 To turn on printing of decisions, phases and productions, do any one of the following (not all possibilities listed): \begin{verbatim}
watch --level 3
watch -l 3
watch --decisions --phases --productions
watch -d -p -P

\end{verbatim}



 Individual options can be changed as well. To turn on printing of decisions and wmes, but not phases and productions, do any one of the following (not all possibilities listed): \begin{verbatim}
watch --level 1 --wmes
watch -l 1 -w
watch --decisions --wmes
watch -d --wmes
watch -w --decisions
watch -w -d

\end{verbatim}



 To turn on printing of decisions, productions and wmes, and turns phases off, do any one of the following (not all possibilities listed): \begin{verbatim}
watch --level 4 --phases remove
watch -l 4 -p remove
watch -l 4 -p 0
watch -d -P -w -p remove

\end{verbatim}



 To watch the firing and retraction of decisions and \emph{only}
 user productions, do any one of the following (not all possibilities listed): \begin{verbatim}
watch -l 1 -u
watch -d -u

\end{verbatim}



 To watch decisions, phases and all productions \emph{except}
 user productions and justifications, and to see full wmes, do any one of the following (not all possibilities listed): \begin{verbatim}
watch --decisions --phases --productions --user remove --justifications remove --fullwmes
watch -d -p -P -f -u remove -j 0 
watch -f -l 3 -u 0 -j 0

\end{verbatim}

\section*{ See Also }


 pwatch print run watch-wmes
\section*{ Structured Output }
\subsection*{ On Query }


 The following arg parameters are returned: \begin{verbatim}
kParamWatchDecisions, kTypeBoolean
kParamWatchPhases, kTypeBoolean
kParamWatchProductionDefault, kTypeBoolean
kParamWatchProductionUser, kTypeBoolean
kParamWatchProductionChunks, kTypeBoolean
kParamWatchProductionJustifications, kTypeBoolean
kParamWatchWMEDetail, kTypeInt
kParamWatchWorkingMemoryChanges, kTypeBoolean
kParamWatchPreferences, kTypeBoolean
kParamWatchLearning, kTypeInt
kParamWatchBacktracing, kTypeBoolean
kParamWatchIndifferentSelection, kTypeBoolean

\end{verbatim}

\subsection*{ Otherwise }
\begin{verbatim}
<result output="raw">true</result>

\end{verbatim}


 \\ 

\section*{ Error Values }
\subsection*{ During Parsing }


 kMissingOptionArg, kUnrecognizedOption, kGetOptError, kTooManyArgs, kIntegerExpected, kIntegerMustBeNonNegative, kIntegerOutOfRange, kInvalidLearnSetting, kRemoveOrZeroExpected
\subsection*{ During Execution }


 kAgentRequired, kKernelRequired

\end{document}
