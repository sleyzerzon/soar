\documentclass[10pt]{article}
\usepackage{fullpage, graphicx, url}
\setlength{\parskip}{1ex}
\setlength{\parindent}{0ex}
\title{Max-chunks - Soar Wiki}
\begin{document}
\section*{Max-chunks}
\subsubsection*{From Soar Wiki}


 This is part of the Soar Command Line Interface. 
\section*{ Name }


 \textbf{max-chunks}
 - Limit the number of chunks created during a decision cycle. 


 Priority: 3�; Status: Complete
\section*{ Synopsis }
\begin{verbatim}
max-chunks [n]

\end{verbatim}
\section*{ Options }


\begin{tabular}{|c|c|}
\hline 
 n  & Maximum number of chunks allowed during a decision cycle.  \\
 \hline 

\end{tabular}



 \\ 

\section*{ Description }


 The max-chunks command is used to limit the maximum number of chunks that may be created during a decision cycle. The initial value of this variable is 50; allowable settings are any integer greater than 0. 
\section*{ Structured Output }
\subsection*{ On Query }
\begin{verbatim}
<result>
  <arg name="value" type="int">max_chunks</arg>
</result>

\end{verbatim}
\subsection*{ Otherwise }
\begin{verbatim}
<result output="raw">true</result>

\end{verbatim}
\section*{ Error Values }
\subsection*{ During Parsing }


 kTooManyArgs, kIntegerMustBePositive
\subsection*{ During Execution }


 kAgentRequired Retrieved from ``\url{http://winter.eecs.umich.edu/soarwiki/Max-chunks}``

\end{document}
