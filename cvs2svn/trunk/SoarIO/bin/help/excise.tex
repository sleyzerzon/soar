\subsection{\soarb{excise}}
\label{excise}
\index{excise}
Delete Soar productions from production memory. 
 Status: Complete
\subsubsection*{Synopsis}
  \begin{verbatim}
excise production_name [production_name ...]
excise -[acdtu]
\end{verbatim}
\subsubsection*{Options}
\begin{tabular}{|l|l|}
\hline 
 -a, --all  & Remove all productions from memory and perform an init-soar command  \\
 \hline 
 -c, --chunks  & Remove all chunks (learned productions) and justifications from memory  \\
 \hline 
 -d, --default  & Remove all default productions (:default) from memory  \\
 \hline 
 -t, --task  & Remove chunks, justifications, and user productions from memory  \\
 \hline 
 -u, --user  & Remove all user productions (but not chunks or default rules) from memory  \\
 \hline 
production\_name & Remove the specific production with this name.  \\
 \hline 
\end{tabular}
\subsubsection*{Description}
 This command removes productions from Soar's memory. The command must be called with either a specific production name or with a flag that indicates a particular group of productions to be removed. Using the flag -a or --all also causes an init-soar. 
\subsubsection*{Examples}
 This command removes the production my*first*production and all chunks: \begin{verbatim}
excise my*first*production --chunks
\end{verbatim}
 This removes all productions and does an init-soar: \begin{verbatim}
excise --all
\end{verbatim}
\subsubsection*{See Also}
 init-soar
\subsubsection*{Structured Output:}
\paragraph*{On Success}
\begin{verbatim}
<result>
  <arg param="count" type="int">number_of_productions_excised</arg>
  <arg param="name" type="string">production_name</arg>
</result>
\end{verbatim}
\paragraph*{Notes}
\begin{itemize}
\item  Each excised production will have its name returned in a separate arg. 
\end{itemize}
\subsubsection*{Error Values:}
\paragraph*{During Parsing}
 kUnrecognizedOption, kGetOptError, kTooManyArgs, kTooFewArgs
\paragraph*{During Execution}
 kAgentRequired, kgSKIError, kProductionNotFound
