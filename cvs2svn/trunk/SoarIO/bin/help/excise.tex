\documentclass[10pt]{article}
\usepackage{fullpage, graphicx, url}
\title{Excise - Soar Wiki}
\begin{document}
\section*{Excise}
\subsubsection*{From Soar Wiki}


 This is part of the Soar Command Line Interface. 
\section*{ Name }


 \textbf{excise}
 - Delete Soar productions from production memory. 


 Status: Complete
\section*{ Synopsis }


  \begin{verbatim}
excise production_name [production_name ...]
excise -[acdtu]

\end{verbatim}



 
\section*{ Options }


\begin{tabular}{|p{1in}|p{5in}|}
\hline 
 -a, --all  & Remove all productions from memory and perform an init-soar command  \\
 \hline 
 -c, --chunks  & Remove all chunks (learned productions) and justifications from memory  \\
 \hline 
 -d, --default  & Remove all default productions (:default) from memory  \\
 \hline 
 -t, --task  & Remove chunks, justifications, and user productions from memory  \\
 \hline 
 -u, --user  & Remove all user productions (but not chunks or default rules) from memory  \\
 \hline 
production\_name & Remove the specific production with this name.  \\
 \hline 

\end{tabular}



 \\ 

\section*{ Description }


 This command removes productions from Soar's memory. The command must be called with either a specific production name or with a flag that indicates a particular group of productions to be removed. Using the flag -a or --all also causes an init-soar. 
\section*{ Examples }


 This command removes the production my*first*production and all chunks: \begin{verbatim}
excise my*first*production --chunks

\end{verbatim}



 This removes all productions and does an init-soar: \begin{verbatim}
excise --all

\end{verbatim}

\section*{ See Also }
\begin{description}
init-soar

\end{description}


 \\ 

\section*{ Structured Output }
\subsection*{ On Success }
\begin{verbatim}
<result>
  <arg param="count" type="int">number_of_productions_excised</arg>
  <arg param="name" type="string">production_name</arg>
</result>

\end{verbatim}
\subsection*{ Notes }
\begin{itemize}
\item  Each excised production will have its name returned in a separate arg. 

\end{itemize}
\section*{ Error Values }
\subsection*{ During Parsing }


 kUnrecognizedOption, kGetOptError, kTooManyArgs, kTooFewArgs
\subsection*{ During Execution }


 kAgentRequired, kgSKIError, kProductionNotFound

\end{document}
