\subsection{\soarb{print}}
\label{print}
\index{print}
Print items in working memory or production memory. 
 Priority: 1�; Status: Incomplete, EvilBackDoor\\ 
Result generated by kernel.--Jonathan 14:26, 3 Feb 2005 (EST) \\ 
This command does too much \^a�� too overloaded. --Jonathan 13:59, 23 Mar 2005 (EST) 
\subsubsection*{Synopsis}
\begin{verbatim}
print [-fFin] production_name
print -[a|c|D|j|u][fFin]
print [-i] [-d <depth>] \emph{identifier}
|\emph{timetag}
|\emph{pattern}
print -s[oS]
\end{verbatim}
\subsubsection*{Options}
\begin{tabular}{|l|l|}
\hline 
 -a, --all  & print the names of all productions currently loaded  \\
 \hline 
 -c, --chunks  & print the names of all chunks currently loaded  \\
 \hline 
 -d, --depth \emph{n}
 & This option overrides the default printing depth (see the default-wme-depth command for more detail).  \\
 \hline 
 -D, --defaults  & print the names of all default productions currently loaded  \\
 \hline 
 -f, --full  & When printing productions, print the whole production. This is the default when printing a named production.  \\
 \hline 
 -F, --filename  & also prints the name of the file that contains the production.  \\
 \hline 
 -i, --internal  & items should be printed in their internal form. For productions, this means leaving conditions in their reordered (rete net) form. For working memory, this means printing the individual elements with their timetags, rather than the objects.  \\
 \hline 
 -j, --justifications  & print the names of all justifications currently loaded.  \\
 \hline 
 -n, --name  & When printing productions, print only the name and not the whole production. This is the default when printing any catorgory of productions, as opposed to a named production.  \\
 \hline 
 -o, --operators  & When printing the stack, print only \textbf{operators}
.  \\
 \hline 
 -s, --stack  & Specifies that the Soar goal stack should be printed. By default this includes both states and operators.  \\
 \hline 
 -S, --states  & When printing the stack, print only \textbf{states}
.  \\
 \hline 
 -u, --user  & print the names of all user productions currently loaded  \\
 \hline 
\emph{identifier}
 & print the object \emph{identifier}
. \emph{identifier}
 must be a valid Soar symbol such as \textbf{S1 }
 \hline 
\emph{pattern}
 & print the object whose working memory elements matching the given pattern. See Description for more information on printing objects matching a specific pattern.  \\
 \hline 
production\_name & print the production named production-name \\
 \hline 
\emph{timetag}
 & print the object in working memory with the given \emph{timetag}
 \hline 
\end{tabular}
\subsubsection*{Description}
(\emph{identifier}
 ^\emph{attribute value}
 [+])
\end{verbatim}
 The pattern is surrounded by parentheses. The \emph{identifier}
, \emph{attribute}
, and \emph{value}
 must be valid Soar symbols or the wildcard symbol * which matches all occurences. The optional \textbf{+}
 symbol restricts pattern matches to acceptable preferences. 
\subsubsection*{Examples}
 Print the working memory elements (and their timetags) which have the identifier s1 as object and v2 as value: \begin{verbatim}
print --internal (s1 ^* v2)
\end{verbatim}
 Print the Soar stack which includes states and operators: \begin{verbatim}
print --stack
\end{verbatim}
 Print the named production in its RETE form: \begin{verbatim}
print -if prodname
\end{verbatim}
 Print the names of all user productions currently loaded: \begin{verbatim}
print -u
\end{verbatim}
\subsubsection*{See Also}
 default-wme-depth, predefined-aliases
\subsubsection*{Structured Output:}
 print returns formatted output in a string, this needs to be re-done.--Jonathan 14:26, 3 Feb 2005 (EST) 
\paragraph*{On Success}
\begin{verbatim}
<result>
  <arg param="message" type="string">output_string</arg>
</result>
\end{verbatim}
\paragraph*{Notes}
\subsubsection*{Error Values:}
\paragraph*{During Parsing}
 kIntegerExpected, kIntegerMustBeNonNegative, kMissingOptionArg, kUnrecognizedOption, kGetOptError, kTooManyArgs
\paragraph*{During Execution}
 kAgentRequired, kKernelRequired
