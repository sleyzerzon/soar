\documentclass[10pt]{article}
\usepackage{fullpage, graphicx, url}
\setlength{\parskip}{1ex}
\setlength{\parindent}{0ex}
\title{Echo - Soar Wiki}
\begin{document}
\section*{Echo}
\subsubsection*{From Soar Wiki}


 This is part of the Soar Command Line Interface. 
\section*{ Name }


 \textbf{echo}
 - Print a string to the current output device. 


 Status: Complete
\section*{ Synopsis }
\begin{verbatim}
echo string

\end{verbatim}
\section*{ Options }


\begin{tabular}{|c|c|}
\hline 
 string  & The string to print.  \\
 \hline 

\end{tabular}



 \\ 

\section*{ Description }


 This command echos the args to the current output stream. This is normally stdout but can be set to a variety of channels. If an arg is -nonewline then no newline is printed at the end of the printed strings. Otherwise a newline is printed after printing all the given args. Echo is the easiest way to add user comments or identification strings in a log file. 
\section*{ Examples }


 This example will add these comments to the screen and any open log file. \begin{verbatim}
echo This is the first run with disks = 12

\end{verbatim}

\section*{ See Also }
\begin{description}
log

\end{description}


 \\ 

\section*{ Structured Output }
\subsection*{ On Success }
\begin{verbatim}
<result>
  <arg param="message" type="string">message</arg>
</result>

\end{verbatim}
\subsection*{ Notes }
\section*{ Error Values }
\subsection*{ During Parsing }
\subsection*{ During Execution }


 No errors.  Retrieved from ``\url{http://winter.eecs.umich.edu/soarwiki/Echo}``

\end{document}
