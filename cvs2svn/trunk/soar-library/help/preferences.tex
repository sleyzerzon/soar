\subsection{\soarb{preferences}}
\label{preferences}
\index{preferences}
Examine details about the preferences that support the specified \emph{id}
 and \emph{attribute}
. 
\subsubsection*{Synopsis}
\begin{verbatim}
preferences [-0123nNtw] [id] [[^]attribute]
\end{verbatim}
\subsubsection*{Options}
\begin{tabular}{|l|l|}
\hline 
 -0, -n, --none  & Print just the preferences themselves  \\
 \hline 
 -1, -N, --names  & Print the preferences and the names of the productions that generated them  \\
 \hline 
 -2, -t, --timetags  & Print the information for the --names option above plus the timetags of the wmes matched by the indicated productions  \\
 \hline 
 -3, -w, --wmes  & Print the information for the --timetags option above plus the entire wme.  \\
 \hline 
id & Must be an existing Soar object identifier.  \\
 \hline 
attribute & Must be an existing \emph{\^{}attribute}
 of the specified identifier.  \\
 \hline 
\end{tabular}
\subsubsection*{Description}
 The \textbf{preferences}
 command prints all the preferences for the given object id and attribute. If \emph{id}
 and \emph{attribute}
 are not specified, they default to the current state and the current operator. The '\^{}' is optional when specifying the attribute. The optional arguments indicates the level of detail to print about each preference. 
 This command is useful for examining which candidate operators have been proposed and what relationships, if any, exist among them. If a preference has O-support, the string, ``:O'' will also be printed. 
\subsection*{Note}
 For the time being, \textbf{numeric-indifferent}
 preferences are listed under the heading ``binary indifferents:''. 
\subsubsection*{Examples}
 This example prints the preferences on (S1 \^{}operator) and the production names which created the preferences: \begin{verbatim}
soar> preferences S1 operator --names
Preferences for S1 ^operator:
acceptables:
 O2 (fill) +
   From waterjug*propose*fill
 O3 (fill) +
   From waterjug*propose*fill
unary indifferents:
 O2 (fill) =
   From waterjug*propose*fill
 O3 (fill) =
   From waterjug*propose*fill
\end{verbatim}
 If the current state is S1, then the above syntax is equivalent to: \begin{verbatim}
 preferences -n
\end{verbatim}
 This example shows the support for the WMEs with the \^{}jug attribute: \begin{verbatim}
soar> preferences s1 jug
Preferences for S1 ^jug:
acceptables:
 I5  +�:O 
 J1  +�:O
\end{verbatim}
\subsubsection*{Default Aliases}
\begin{tabular}{|l|l|}
\hline 
 Alias  & Maps to  \\
 \hline 
 pr  & preferences  \\
 \hline 
\end{tabular}
\subsubsection*{See Also}
