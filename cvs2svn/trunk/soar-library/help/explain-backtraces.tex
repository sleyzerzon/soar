\subsection{\soarb{explain-backtraces}}
\label{explain-backtraces}
\index{explain-backtraces}
Print information about chunk and justification backtraces. 
 Priority: 3; Status: Incomplete, EvilBackDoor\\ 
Result generated by kernel.--Jonathan 18:16, 25 Feb 2005 (EST) 
\subsubsection*{Synopsis}
\begin{verbatim}
explain-backtraces -f prod_name
explain-backtraces [-c <n>] prod_name
\end{verbatim}
\subsubsection*{Options}
\begin{tabular}{|l|l|}
\hline 
 (no args)  & List all productions that can be ``explained''  \\
 \hline 
 prod\_name  & List all conditions and grounds for the chunk or justification.  \\
 \hline 
 -c, --condition  & Explain why condition number \emph{n}
 is in the chunk or justification.  \\
 \hline 
 -f, --full  & Print the full backtrace for the named production  \\
 \hline 
\end{tabular}
\subsubsection*{Description}
 This command provides some interpretation of backtraces generated during chunking. 
 The two most useful variants are: \begin{verbatim}
explain-backtraces prodname 
explain-backtraces -c n prodname
\end{verbatim}
 The first variant prints a numbered list of all the conditions for the named chunk or justification, and the ground which resulted in inclusion in the chunk/justification. A \emph{ground}
 is a working memory element (WME) which was tested in the supergoal. Just knowing which WME was tested may be enough to explain why the chunk/justification exists. If not, the second variant, . This value can be used in \textbf{explain-backtraces prodname n}
 to obtain a list of the productions which fired to obtain this condition in the chunk/justification (and the crucial WMEs tested along the way). 
 \textbf{save\_backtraces}
\subsubsection*{Examples}
 (fix this) 
 Could someone generate an example output from 
 \textbf{explain-backtraces \emph{prodname}
}
 and 
 \textbf{explain-backtraces \emph{prodname n}
}
? 
\subsubsection*{See Also}
 save-backtraces
